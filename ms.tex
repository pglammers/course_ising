% \input{preamble/final}
\input{preamble/draft}
\mathtoolsset{showonlyrefs}

\makeatletter
\@namedef{subjclassname@2020}{\textup{2020} Mathematics Subject Classification}
\makeatother


\title{A course on the Ising model}
\subjclass[2020]{[Mathematics classification]}


\author{Piet Lammers}
\address{CNRS and Sorbonne Université, LPSM}
\email{piet.lammers@cnrs.fr}

\date{\today}
% \keywords{%
%     [keyword 1],
%     [keyword 2]
% }

\newcommand\n{\mathbf{n}}
\newcommand\m{\mathbf{m}}
\renewcommand\a{\mathbf{a}}
\renewcommand\b{\mathbf{b}}

\thanks{This work is licensed under CC BY-NC-SA 4.0. This work is licensed under CC BY-NC-SA 4.0. To view a copy of this license, visit \url{https://creativecommons.org/licenses/by-nc-sa/4.0/}}

\begin{document}


\maketitle

\tableofcontents
% \begin{abstract}
%     [Abstract text]
% \end{abstract}


\section*{Preface}
These lecture notes are progressively written during the 2025 spring semester,
as the course is taught at Sorbonne university in the M2 (second-year masters) programme.
Its purpose is to give a broad introduction to the rigorous analysis of the Ising model.
The main focus is on four techniques and their applications:
\begin{itemize}
    \item The Peierls argument,
    \item The random-currents representation,
    \item The FKG inequality for the Ising spins,
    \item The FKG inequality for the random-cluster (FK) representation.
\end{itemize}

A basic understanding of analysis and probability theory is essential for following this course.
Experience with other models in statistical mechanics (such as the Bernoulli percolation model)
is a plus but by no means essential.

\section{Introduction}
\label{sec:intro}

The Ising model is the archetypal model for the study of phase transitions in
mathematical physics.
It was first introduced by Wilhelm Lenz in 1920 and later solved by Ernst Ising
in 1924 in the one-dimensional case.
The model consists of a lattice of spins,
each of which can be in one of two states,
up or down.
Informally,
one may think of these spins as the magnetic moments of atoms in a ferromagnetic material.
The behaviour of this probabilistic model depends strongly on a few different parameters:
\begin{itemize}
    \item The dimension $d$ of the lattice $\Z^d$ on which the spins are placed,
    \item The interaction strength $\beta$ between neighbouring spins,
    \item The way that boundary conditions are imposed,
    \item The strength of external magnetic field $h$.
\end{itemize}
In fact, we shall start by defining the Ising model on arbitrary finite graphs.
We shall now give a definition of the Ising model, although
we keep boundary conditions and external magnetic fields for later.

\begin{definition}[Ising model on a finite graph]
    The Ising model on a finite graph \( G = (V, E) \) with \emph{inverse temperature} \( \beta \in [0,\infty) \) is defined as follows.
    Let $\Omega:=\{\pm1\}^V$ denote the set of spin configurations on the vertices of the graph;
    a typical element of $\Omega$ is denoted by $\sigma=(\sigma_u)_{u\in V}$.
    Elements $\sigma\in\Omega$ are called \emph{spin configurations};
    elements $\sigma_u$ are called \emph{spins}.
    The \emph{energy} or \emph{Hamiltonian} of a spin configuration $\sigma$ is given by
    \[
        H_{G,\beta}^{\operatorname{Ising}}(\sigma) := -\beta \sum_{uv \in E} \sigma_u \sigma_v.
    \]
    We write $\P_{G,\beta}^{\operatorname{Ising}}$ for the associated \emph{Boltzmann distribution} or \emph{Gibbs measure}:
    \[
        \P_{G,\beta}^{\operatorname{Ising}}(\sigma) := \frac{1}{Z_{G,\beta}^{\operatorname{Ising}}} e^{-H^{\operatorname{Ising}}_{G,\beta}(\sigma)},
    \]
    where \(Z_{G,\beta}^{\operatorname{Ising}}\) is normalisation constant or \emph{partition function} defined by
    \[
        Z_{G,\beta}^{\operatorname{Ising}}:= \sum_{\sigma\in\Omega} e^{-H^{\operatorname{Ising}}_{G,\beta}(\sigma)}.
    \]
    We shall write $\langle\blank\rangle_{G,\beta}^{\operatorname{Ising}}$ for the expectation functional associated to this probability measure.
\end{definition}

\begin{remark}[Flip-symmetry]
    The Ising model is \emph{flip-symmetric} in the sense that the distribution of the spins is invariant under the transformation $\sigma\mapsto-\sigma$.
    This is because the Hamiltonian is invariant under this transformation.
\end{remark}

\begin{remark}
    We shall often suppress subscripts and superscripts when they are clear from the context.
\end{remark}

\begin{remark}
    Adding a constant to the Hamiltonian does not change the distribution of the Ising model,
    even though it affects the partition function.
\end{remark}

\begin{remark}
    The mathematical community has widely adopted the terminology coming from the physics
    literature.
\end{remark}

\input{chapters/02_definitions_examples.tex}
\input{chapters/03_ising.tex}
\section{Early developments. 1936: Peierls' argument}
\label{sec:peierls}

\begin{theorem}[Peierls, 1936]
    \label{thm:peierls}
    The Ising model exhibits magnetisation in two dimensions.
\end{theorem}

We shall discuss a slight variation of Peierls' original setup,
so that we can fully focus the proof on the core idea.
Let $\T$ denote the triangular lattice graph,
comprised of vertices of the form
\[
    \T:=\left\{n+m e^{\pi i/3}:n,m\in\Z \right\}\subset\C,
\]
and such that each vertex is connected to the six
vertices at distance one.
Let $\Lambda_n\subset\T$ denote the set of vertices at a graph
distance at most $n-1$ from $0\in\T$.
We consider the Ising model on the infinite graph $\T$.
We shall prove the following version of Peierls' result.

\begin{theorem}[Peierls, 1936]
    \label{thm:peierls_triangles}
    Consider the Ising model on the two-dimensional
    triangular lattice graph $\T$.
    For sufficiently large $\beta$,
    we have
    \[
        \inf_{n}\langle\sigma_0\rangle_{\Lambda_n,\beta}^+
        >0.
    \]
\end{theorem}

Let $\H:=\T^*$ denote the hexagonal lattice
that is dual to the triangular lattice.
For a fixed configuration $\sigma\in\Omega$,
we let $\calI(\sigma)\subset E(\H)$ denote the set of
hexagonal lattice edges separating hexagons with different spins.
The set $\calI(\sigma)$ is called the
\emph{interface} between the spins valued $+1$
and those valued $-1$.
\todo{Add figure}
Notice that $\calI(\sigma)$ has a partition into
loops and bi-infinite paths.
If only finitely many spins of $\sigma$ are valued $-1$,
then there are no bi-infinite paths,
and all connected components of $\calI(\sigma)$
are loops.
This happens almost surely when sampling from $\langle\blank\rangle_{\Lambda_n,\beta}^+$.

The core of Peierls' argument is the following lemma.

\begin{lemma}[Exponential decay of loop lengths]
    \label{lem:exp_decay_ising_loops}
    Consider the Ising model on the two-dimensional triangular lattice $\T$
    at inverse temperature $\beta$.
    Suppose that $e^{-2\beta}<\frac12$.
    Then for any hexagonal lattice edge $e\in E(\H)$
    and for any minimal loop length $\ell\in\Z_{\geq 1}$,
    we get
    \[
        \P_{\Lambda_n,\beta}^+
        (\{\text{$\calI(\sigma)$ has a loop of length at least $\ell$ through $e$}\})
        \leq \frac{(2e^{-2\beta})^\ell}{1-2e^{-2\beta}},
    \]
    uniformly in $n$.
\end{lemma}

\begin{proof}
    Fix $\beta$, $e$, and $n$.
    Let $\calL$ denote a loop through $e$,
    and consider the event $\{\calL\subset\calI\}$.
    We claim that
    \[
        \P_{\Lambda_n,\beta}^+
        (\{\calL\subset\calI\})
        \leq e^{-2\beta|\calL|}.
    \]

    To prove the claim,
    we introduce the injective ``loop erasure map''
    \[
        \calE_\calL:\{\calL\subset\calI\}
        \to \Omega\setminus \{\calL\subset\calI\},
    \]
    which is defined such that it flips all the spins inside the loop $\calL$.
    As a consequence, $\calI(\calE_\calL(\sigma))=\calI(\sigma)\setminus\calL$.
    For any $\sigma\in\{\calL\subset\calI\}$, we have
    \[
        \P_{\Lambda_n,\beta}^+
        (\sigma)
        =
        e^{-2\beta|\calL|}
        \cdot
        \P_{\Lambda_n,\beta}^+
        (\calE_\calL(\sigma))
        .
    \]
    We can write down this identity because we know that the loop erasure map
    decreases the Hamiltonian by $2\beta|\calL|$.
    Since $\calE_\calL$ is injective, we get
    \[
        \P_{\Lambda_n,\beta}^+
        (\{\calL\subset\calI\})
        = e^{-2\beta|\calL|}\cdot \P_{\Lambda_n,\beta}^+(\operatorname{Image}(\calE_\calL))
        \leq e^{-2\beta|\calL|},
    \]
    which proves the claim.

    To prove the lemma, observe simply that the number of loops of length $k$
    through $e$ is bounded by $2^k$, so that 
    \begin{align}
        &\P_{\Lambda_n,\beta}^+
        (\{\text{$\calI(\sigma)$ has a loop of length at least $\ell$ through $e$}\})
        \\&\qquad=
        \sum\nolimits_{\text{$\calL$ is a loop of length at least $\ell$ through $e$}}
        \P_{\Lambda_n,\beta}^+
        (\{\calL\subset\calI\})
        \\&\qquad\leq 
        \sum\nolimits_{k\geq \ell} 2^k\cdot e^{-2\beta k}.
    \end{align}
    The final expression is a geometric series converging to the upper bound
    in the lemma.
\end{proof}

\begin{remark}
    In the previous proof,
    the interplay between entropy and energy is quite transparent.
    The entropy in the argument comes from the number of loops of length
    $\ell$, which we upper bounded by $2^\ell$.
    Such a loop contributes a total of $2\beta\ell$ to the Hamiltonian.
    When $2e^{-2\beta}<1$, the energy term dominates,
    forcing the loops to be small.
\end{remark}

\begin{proof}[Proof of Theorem~\ref{thm:peierls_triangles}]
    For a fixed configuration $\sigma\in\Omega$
    sampled from $\P_{\Lambda_n,\beta}^+$,
    we may express $\sigma_0$ as the parity of the number
    of loops in $\calI(\sigma)$ surrounding $0$.
    In particular, if no loop surrounds $0$,
    then $\sigma_0=+1$.
    Thus, for Theorem~\ref{thm:peierls_triangles},
    it suffices to prove that
    \begin{equation}
        \label{eq:peierls_target_equation}
        \P_{\Lambda_n,\beta}^+
        (\{\text{$\calI(\sigma)$ contains a loop surrounding $0$}\})
        <\frac12,
    \end{equation}
    for sufficiently large $\beta$,
    and uniformly in $n$.

    Suppose given some loop $\calL\subset E(\H)$ surrounding $0$.
    Then $\calL$ must intersect the half-line $\R_{\geq 0}\subset\C$.
    More precisely, $\calL$ must contain some edge $e$ whose midpoint
    lies precisely in the set of half-integers $-\frac12+\Z_{\geq 1}$.
    If the endpoint of $e$ is $k-\frac12$, then
    $|\calL|\geq k$, otherwise it cannot surround $0$.
    We are now ready to complete Peierls' argument,
    using exponential decay of the loop lengths (Lemma~\ref{lem:exp_decay_ising_loops}).
    
    Let us perform a union bound over the intersection point,
    in order to obtain
    \begin{align}
        &\P_{\Lambda_n,\beta}^+
        (\{\text{$\calI(\sigma)$ contains a loop surrounding $0$}\})
        \\
        &\qquad\leq
        \sum_{k=1}^\infty
        \P_{\Lambda_n,\beta}^+
        (\{\text{$\calI(\sigma)$ contains a loop surrounding $0$ and hitting $k-\tfrac12$}\})
        \\
        &\qquad\leq
        \sum_{k=1}^\infty
        \frac{(2e^{-2\beta})^k}{1-2e^{-2\beta}}
        =
        \frac{2e^{-2\beta}}{(1-2e^{-2\beta})^2}.
    \end{align}
    This upper bound is independent of $n$
    and tends to $0$ with $\beta\to\infty$,
    thus establishing Equation~\eqref{eq:peierls_target_equation}.
\end{proof}

\begin{exercise}[The Peierls argument on other graphs]
    \label{exo:peierls_general}
    \begin{enumerate}
        \item     Consider the Ising model on the two-dimensional square lattice graph $\mathbb{Z}^2$.
        In this case, the interface $\calI(\sigma)$ does not consist of loops, but of even subgraphs
        of the dual lattice.
        How can Peierls' argument be adapted to this case?
        \item  Now consider the $d$-dimensional square lattice for $d\geq 3$.
        What is the structure of the interface in this case?
        Can we adapt Peierls' to prove magnetisation for sufficiently large $\beta$?
    \end{enumerate}
\end{exercise}

\begin{remark}
    Peierls' is robust,
    in the sense that it can be adapted to many other models in statistical mechanics.
\end{remark}


\section{The Markov property}

Recall the definition of the Ising model on a finite graph
(Definition~\ref{def:ising_finite})
and on general graphs with boundary conditions (Definition~\ref{def:ising_bc}).
The second definition includes the first, since we may simply 
choose our domain $\Lambda$ to be the full vertex set whenever the
graph $G=(V,E)$ is finite.
That is why we state our results for general graphs with boundary conditions
in this section.

One important property of the definition with boundary conditions is that
it in fact encodes \emph{conditional probability measures}.

\begin{lemma}[Boundary conditions as conditional measures]
    \label{lemma:boundary_conditions_conditional_measures}
    Let $G$ denote a locally finite graph and $\beta\in[0,\infty)$
    an inverse temperature.
    Let $\Lambda\subset\Delta$ denote two finite domains
    and fix $\xi\in\{\pm1\}^{\Delta^c}$
    and $\zeta\in\{\pm1\}^{\Delta\setminus\Lambda}$.
    Then
    \[
        \P_{\Delta,\beta}^\xi(\blank|\{\sigma|_{\Delta\setminus\Lambda}=\zeta\})
        =
        \P_{\Lambda,\beta}^{\xi\zeta}.
    \]
\end{lemma}

\begin{proof}
    For the two measures, we get
    \begin{align}
        \P_{\Delta,\beta}^\xi(\sigma|\{\sigma|_{\Delta\setminus\Lambda}=\zeta\})
        &\propto\true{\sigma|_{\Lambda^c}=\xi\zeta}\cdot e^{-H_{\Delta,\beta}(\sigma)};
        \\\P_{\Lambda,\beta}^{\xi\zeta}(\sigma)
        &\propto\true{\sigma|_{\Lambda^c}=\xi\zeta}\cdot e^{-H_{\Lambda,\beta}(\sigma)}.
    \end{align}
    But $H_{\Delta,\beta}-H_{\Lambda,\beta}$ is constant on the
    event $\{\sigma|_{\Lambda^c}=\xi\zeta\}$,
    which means that the two probability measures are the same.
\end{proof}

The Ising model is a \emph{nearest-neighbour model},
meaning that the interactions are associated with the edges of the graph.
A consequence of this is the so-called \emph{Markov property}.
There are several ways to state it.
We shall first state and prove the following lemma.
For any domain $\Lambda\subset V$,
we let $\partial\Lambda\subset V$ denote the set of vertices
at graph distance one from $\Lambda$.
This is called the \emph{boundary} of $\Lambda$.

\begin{lemma}[Markov property]
    \label{lemma:markov_property_general}
    Consider a locally finite graph $G$,
    an inverse temperature $\beta$,
    a domain $\Lambda$,
    and a boundary condition $\zeta$.
    Let $(\Lambda_i)_i$ denote the partition of $\Lambda$
    into connected components.
    Then in the measure
    $\P_{\Lambda,\beta}^\zeta$,
    the family $(\sigma|_{\Lambda_i})_i$
    is a family of independent random variables.
    Moreover, the distribution of $\sigma|_{\Lambda_i}$ only depends on
    $\zeta|_{\partial\Lambda_i}$.
\end{lemma}

\begin{proof}
    We have
    $\P_{\Lambda,\beta}^\zeta(\sigma)\propto \true{\sigma|_{\Lambda^c}=\zeta}\cdot e^{-H_{\Lambda,\beta}(\sigma)}$.
    The Hamiltonian may be written
    \[
        H_{\Lambda,\beta}(\sigma)
        =
        \sum_i H_{\Lambda_i,\beta}(\sigma).
    \]
    But each term $H_{\Lambda_i,\beta}(\sigma)$
    is a function of $\sigma|_{\Lambda_i}$ and $\zeta|_{\partial\Lambda_i}$.
    This means that $e^{-H_{\Lambda,\beta}(\sigma)}$ 
    may be written as a product of factors, where the factor corresponding
    to $\Lambda_i$ only depends on $\sigma|_{\Lambda_i}$ and $\zeta|_{\partial\Lambda_i}$.
    This implies the desired independence.
\end{proof}

The Markov property is often phrased in a slightly different fashion.

\begin{theorem}[Markov property]
    Consider a locally finite graph $G$,
    and inverse temperature $\beta$,
    and two domains $\Lambda\subset\Delta$.
    Let $\zeta\in\{\pm1\}^{\Delta^c}$ denote a boundary condition,
    and fix $\xi\in\{\pm1\}^{\partial\Lambda}$.
    If $\P_{\Delta,\beta}^\zeta(\{\sigma|_{\partial\Lambda=\xi}\})>0$,
    then in the conditional probability measure
    \[
        \P_{\Delta,\beta}^\zeta(\blank|\{\sigma|_{\partial\Lambda=\xi}\})
        =
        \P_{\Delta\cup\partial\Lambda,\beta}^{\zeta\xi}
        ,
    \]
    the random variables $\sigma|_{\Lambda}$
    and $\sigma|_{\Lambda^c}$ are independent.
    Moreover, the distribution of $\sigma|_{\Lambda}$ only depends on $\xi$.
\end{theorem}

\begin{proof}
The two measures in the display in this theorem are equal because of Lemma~\ref{lemma:boundary_conditions_conditional_measures}.
The theorem is then a mere corollary of the previous lemma (Lemma~\ref{lemma:markov_property_general}).
\end{proof}



\input{chapters/06_correlation.tex}
\section{Random currents}

\begin{definition}[Currents]
    Let $G=(V,E)$ denote a graph.
    A \emph{current} is a map $\n:E\to\Z_{\geq 0}$.
    If $G$ is finite and $\beta\in[0,\infty)$, then the
    \emph{weight} of a current is defined as
    \[
        w_\beta(\n):=\prod_{xy\in E}
        \frac{\beta^{\n_{xy}}}{\n_{xy}!}.
    \]
\end{definition}

Consider the Ising model on some finite graph $G$ at inverse temperature $\beta$.
Fix $A\subset V$, and consider the non-normalised correlation function
\[
    Z_{G,\beta}\langle\sigma_A\rangle_{G,\beta}.
\]


\section{Double random currents}

The previous section explained how correlation functions are expressed
in terms of random currents.
We also proved a basic result, namely the existence of a demagnetised phase
of the Ising model on graphs of bounded degree.

All results discussed so far concern the behaviour of the Ising model
in the off-critical regime (very large values of $\beta$, very small values of $\beta$).
Our main interest is however in the \emph{critical regimes}:
the values for $\beta$ where the model undergoes a qualitative change,
such as values in the topological boundary of the set
\[
    \{\beta\in[0,\infty):\lim_{n\to\infty}\langle\sigma_u\rangle_{\Lambda_n,\beta}^+=0\}
\]
for a given infinite graph $G$ with a reference point $u$
(as per usual, $\Lambda_n$ refers to the graph metric ball around $u$).

We shall now introduce a new tool to study correlation functions
and random currents: the \emph{switching lemma}.
In recent years this tool has proved to be instrumental in the derivation
of rigorous results on the critical
behaviour of the Ising model, especially in graph dimensions $3$ and $4$.

\begin{lemma}[Switching lemma]
    Consider the Ising model on a finite graph $G$ at inverse temperature $\beta$.
    Let $A,B,S\subset V$.
    Then for any bounded function $F:(\Z_{\geq 0})^E\to\C$,
    the following identities hold true:
    \[
        \M^A\times \M^B[F(\n+\m)\true{\n+\m\in\calE_S}]
        =
        \M^{A\Delta S}\times \M^{B\Delta S}[F(\n+\m)\true{\n+\m\in\calE_S}],
    \]
    where $A\Delta S$ denotes the symmetric difference of $A$ and $S$.

    In terms of weights, this is equivalent to
    \begin{multline}
        \sum_{\substack{\n:\:\partial\n=A\\\m:\:\partial\m=B}}
        w_\beta(\n)w_\beta(\m)
        F(\n+\m)\true{\n+\m\in\calE_S}
        \\
        =
        \sum_{\substack{\n:\:\partial\n=A\Delta S\\\m:\:\partial\m=B\Delta S}}
        w_\beta(\n)w_\beta(\m)
        F(\n+\m)\true{\n+\m\in\calE_S}.
    \end{multline}
\end{lemma}

\begin{remark}
    While the switching lemma is an extremely powerful tool,
    its statement may appear daunting at first sight.
    Let us quickly see how switching may be applied to a simple example.
    Suppose that we record cars traversing a bridge on a road.
    Blue cars appear according to a Poisson process with rate $\lambda$.
    Let $X_B$ denote the number of blue cars that passed after recording for one hour.
    Can we easily prove, without a calculation, that
    \[
        \P[X_B\in 2\Z]\geq \P[X_B\in 2\Z+1]?
    \]

    Suppose that there are also yellow cars,
    which arrive according to an independent Poisson process with rate $\lambda$.
    Let $X_Y$ denote the number of yellow cars that passed.
    Suppose that, after waiting for one hour, $X_B+X_Y=N>0$ cars passed.
    What is the \emph{conditional} probability that $X_B$ is even?

    Well, we must have $\P[X_B\in2\Z|\{X_B+X_Y=N\}]=1/2$. Indeed, by the properties of the Poisson process,
    the distribution of the cars is invariant under \emph{switching} the colour
    of the last car. If it was blue before, then we paint it yellow,
    and vice versa.
    This operation changes the parity of $X_B$
    but leaves the conditional distribution invariant: 
    hence the symmetric probability $1/2$.

    But we cannot always do the switch.
    If $X_B+X_Y=0$
    then there is no car to repaint, and also
    $X_B=0$.
    Thus, we conclude that
    \[
        \P[X_B\in 2\Z]-\P[X_B\in 2\Z+1]=\P[\{X_B+X_Y=0\}]\geq 0.
    \]
    
    Notice that we originally asked a question about blue cars,
    but introducing yellow cars allowed us to answer it.
    This is the essence of the switching lemma.

    The switching lemma is analogous to the above example:
    \begin{itemize}
        \item The product measure corresponds to the joint distribution of blue and yellow cars,
        \item The weights correspond to the rates of the Poisson processes,
        \item The function $F$ corresponds to the conditioning event $\{X_B+X_Y=N\}$,
        \item The event $\{\n+\m\in\calE_S\}$ corresponds to the event $\{X_B+X_Y>0\}$.
    \end{itemize}
\end{remark}

\begin{proof}[Proof of the switching lemma]
    By linearity of expectation,
    it suffices to consider the case that $F(\b):=\true{\b=\a}$
    for some fixed $\a\in\calE_S$.
    Our objective is then to derive the equality
    \[
        \M^A\times \M^B[\{\n+\m=\a\}]
        =
        \M^{A\Delta S}\times \M^{B\Delta S}[\{\n+\m=\a\}]
    \]
    or
    \begin{align}
        \M^2[\{\partial\n=A,\,\partial\m=B,\,\n+\m=\a\}]
        =
        \M^2[\{\partial\n=A\Delta S,\,\partial\m=B\Delta S,\,\n+\m=\a\}].
    \end{align}
    Define the probability measure
    \[
        \P:\propto\M^2[(\blank)\true{\n+\m=\a}].
    \]
    It suffices to prove that
    \begin{equation}
        \label{eq:switching_target_equation}
    \P[\{\partial\n=A,\,\partial\m=B\}]
    =
    \P[\{\partial\n=A\Delta S,\,\partial\m=B\Delta S\}].
    \end{equation}

    By going back to the definition of $\M$ in terms of $w_\beta$,
    it is straightforward to see that the pair $(\n,\m)$ has the following probability distribution under $\P$:
    \begin{itemize}
        \item The family $(\n_{uv})_{uv}$ is a family of independent random variables,
        \item The distribution of $\n_{uv}$ is $\operatorname{Binomial}(\a_{uv},1/2)$,
        \item We have $\n+\m=\a$ almost surely, which fixes the joint distribution of $(\n,\m)$.
    \end{itemize}
    In fact, we may interpret $\P$ in a different way.
    Define the \emph{multigraph}
    \[
        \calM_\a:=\{(uv,k)\in E\times\Z_{\geq 0}:\a_{uv}<k\}
    \]
    on the vertex set $V$.
    Then $\P$ is interpreted as follows:
    \begin{itemize}
        \item We let $\calK$ denote a uniformly random subset of $\calM_\a$,
        \item We let $\n_{uv}$ denote the number of multiedges in $\calK$ between $u$ and $v$,
        \item We let $\m_{uv}$ denote the number of multiedges in $\calM_\a\setminus\calK$ between $u$ and $v$.
    \end{itemize}
    Indeed, this definition of $\P$ is consistent with our previous one.

    Proving Equation~\eqref{eq:switching_target_equation} now comes down to
    proving that the number of submultigraphs $\calK\subset\calM_\a$ contributing
    to the events on the left and right, is the same.
    Let $E_S\subset E(\a)$ denote an arbitrary subset such that $\partial E_S=S$,
    and write $E_{S,0}:=E_S\times\{0\}\subset\calM_\a$.
    The existence of the set $E_S$ follows from the fact that $\a\in\calE_S$.
    The reader may now verify that the map
    \[
        \{\partial\n=A,\,\partial\m=B\}
        \to
        \{\partial\n=A\Delta S,\,\partial\m=B\Delta S\}
        ,\,
        \calK\mapsto \calK\Delta E_{S,0}
    \]
    is a bijection.
    This proves that the two sets have the same cardinality,
    and thus the same probability under the measure $\P$.
    We have now established Equation~\eqref{eq:switching_target_equation}
    and therefore the lemma.
\end{proof}

\begin{corollary}[Second Griffiths inequality]
    Consider the Ising model on a finite graph $G$ at inverse temperature $\beta$.
    Then for any $A,B\subset V$, we have
    $\langle\sigma_{A\Delta B}\rangle_{G,\beta}-\langle\sigma_A\rangle_{G,\beta}\langle\sigma_B\rangle_{G,\beta}\geq 0$.
\end{corollary}

\begin{proof}
    We have $1=\langle1\rangle=\langle\sigma_\emptyset\rangle$.
    By the previous section (for example Corollary~\ref{cor:current_representation_of_correlation_functions}),
    \[
    Z^2
    (\langle\sigma_{A\Delta B}\rangle\langle\sigma_\emptyset\rangle-\langle\sigma_A\rangle\langle\sigma_B\rangle)
    =
    2^{2|V|}
    (
    \M^{A\Delta B}\times\M^\emptyset[1]
    -
    \M^A\times\M^B[1]
    ).
    \]
    Claim that the quantity on the right is nonnegative.
    Notice that if $\partial\m= B$, then $\m\in\calE_S$,
    and therefore
    \begin{multline}
        \M^A\times\M^B[1]
        =
        \M^A\times\M^B[\true{\n+\m\in\calE_B}]
        \stackrel{\text{switch}}=
        M^{A\Delta B}\times\M^\emptyset[\true{\n+\m\in\calE_B}]
        \\
        \leq
        M^{A\Delta B}\times\M^\emptyset[1].
    \end{multline}
    This inequality implies the claim, and therefore the second Griffiths inequality.
\end{proof}

\begin{exercise}[Conditioning on equality increases the correlation functions]
    \label{exo:conditioning_equality}
    Consider the Ising model on a finite graph $G$
    at inverse temperature $\beta$, and fix some subset $A\subset V$.
    \begin{itemize}
        \item     Prove that for any two distinct vertices $u,v\in V$,
        we have
        \[
            \E_{G,\beta}[\sigma_A|\{\sigma_u=\sigma_v\}]
            \geq
            \E_{G,\beta}[\sigma_A]=\langle\sigma_A\rangle_{G,\beta}.
        \]
        \item Prove for any $X\subset Y\subset V$, we have
        \[
            \E_{G,\beta}[\sigma_A|\{\text{$\sigma$ is constant on $X$}\}]
            \leq
            \E_{G,\beta}[\sigma_A|\{\text{$\sigma$ is constant on $Y$}\}]
            .
        \]
    \end{itemize}
\end{exercise}

\begin{exercise}[The two-point function as a metric]
    Consider the Ising model on a finite graph $G$
    at inverse temperature $\beta>0$.
    Prove that $V\times V\to [0,\infty],\,
    (u,v)\mapsto-\log\langle\sigma_u\sigma_v\rangle_{G,\beta}$
    defines a metric on $V$.
\end{exercise}

\begin{definition}[Probability measures on currents]
    Consider the Ising model on a finite graph $G$
    at inverse temperature $\beta$.
    For any $A\subset V$,
    define the probability measure
    \[
        \P^A_{G,\beta}:=\frac{2^{|V|}}{Z_{G,\beta}\langle\sigma_A\rangle_{G,\beta}}\M^A_{G,\beta}.
    \]
    For any $A_1,\ldots,A_n$,
    write $\P^{A_1,\ldots,A_n}:=\P^{A_1}\times\cdots\times\P^{A_n}$.
\end{definition}

\begin{exercise}[Correlation functions in terms of sourceless currents]
    Consider the Ising model on a finite graph $G$
    at inverse temperature $\beta$.
    Prove that for any $A\subset V$,
    \[
        \langle\sigma_A\rangle^2
        =
        \P^{\emptyset,\emptyset}[\{\n+\m\in\calE_A\}].
    \]

    Observe that we can now express all correlation functions in terms of a single
    fixed probability measure on sourceless random currents.
\end{exercise}


\section{Monotonicity in the temperature}

\begin{theorem}[Monotonicity in the temperature]
    Let $G$ denote a finite graph and $A\subset V$ a finite set.
    Then the function $\beta\mapsto\langle\sigma_A\rangle_{G,\beta}$
    is non-decreasing.
\end{theorem}

\begin{proof}
    We want to prove that
    \[
        \frac{\partial}{\partial\beta}
        \langle\sigma_A\rangle_{G,\beta}
        =
        \frac{\partial}{\partial\beta}
        \left(
            \frac{
                \sum_\sigma\sigma_A\prod_{uv}e^{\beta\sigma_u\sigma_v}
            }{
                \sum_\sigma\prod_{uv}e^{\beta\sigma_u\sigma_v}
            }
        \right)
        \geq 0.
    \]
    Since we are differentiating a fraction,
    it suffices to show that the numerator grows at a faster rate
    than the denominator,
    that is,
    \[
        \frac{\frac{\partial}{\partial\beta}
            \sum_\sigma\sigma_A\prod_{uv}e^{\beta\sigma_u\sigma_v}
        }{
            Z\langle\sigma_A\rangle
        }
        \geq
        \frac{\frac{\partial}{\partial\beta}
            \sum_\sigma\prod_{uv}e^{\beta\sigma_u\sigma_v}
        }{
            Z
        }.
    \]
    By multiplying either side by $\langle\sigma_A\rangle$
    and differentiating each side,
    we see that this inequality is equivalent to
    \[
       \sum_{xy}
        \frac{\sum_\sigma
            \sigma_x\sigma_y\sigma_A\prod_{uv}e^{\beta\sigma_u\sigma_v}
        }{Z}
        \geq
        \langle\sigma_A\rangle
        \sum_{xy}
        \frac{
            \sum_\sigma
            \sigma_x\sigma_y
            \prod_{uv}e^{\beta\sigma_u\sigma_v}
        }{Z}.
    \]
    Each fraction may now be reinterpreted as a correlation function,
    so that the previous inequality is equivalent to
    \[
        \sum_{xy}\langle\sigma_x\sigma_y\sigma_A\rangle
        \geq
        \langle\sigma_A\rangle
        \sum_{xy}\langle\sigma_x\sigma_y\rangle.
    \]
    But this is just the second Griffiths inequality.
\end{proof}

\begin{exercise}
    What are some other properties of the function $\beta\mapsto\langle\sigma_A\rangle_{G,\beta}$
    in the context of the theorem above?
\end{exercise}

\section{The thermodynamical limit. Wired boundary, existence}
\label{sec:infinite_volume}

Consider the Ising model on the infinite graph $\Z^d$ for $d\geq 2$.
Let $u=0\in\Z^d$ and let $\Lambda_n$ denote the graph metric ball around $u$.
We have already derived the following results.
\begin{itemize}
    \item For large $\beta$,
    the Ising model exhibits magnetisation
    in the sense that
    \[
        \inf_{n}\langle\sigma_0\rangle_{\Lambda_n,\beta}^+
        >0.
    \]
    This was proved via the Peierls argument, see Theorem~\ref{thm:peierls}
    and Exercise~\ref{exo:peierls_general}.
    \item For small $\beta$,
    the Ising model \emph{does not} exhibit magnetisation:
    \[
        \lim_{n\to\infty}\langle\sigma_0\rangle_{\Lambda_n,\beta}^+
        =0.
    \]
    This was proved via a Peierls argument for random currents,
    see Exercise~\ref{exercise:currents_peierls}.
\end{itemize}
At the time moment of stating the Peierls argument (Section~\ref{sec:peierls}),
we knew almost nothing about the Ising model.
Our understanding is now advancing.
We already used the first Griffiths inequality to show that $\langle\sigma_0\rangle_{\Lambda_n,\beta}^+\geq 0$
(Corollary~\ref{cor:griffiths_1}
and Exercise~\ref{exo:correlation_functions_with_odd_sets}).
Our first objective is now to prove the following result.
To state it, we write
\[
    \lim_{\Lambda\uparrow V}f(\Lambda)
    \qquad\text{for}\qquad
    \lim_{n\to\infty}f(\Lambda_n),
\]
where $(\Lambda_n)_n$ is any increasing sequence of domains 
with $\cup_n\Lambda_n=V$.
This notation makes sense only when the limit is independent
of the precise choice of the sequence $(\Lambda_n)_n$,
and is called the \emph{thermodynamical limit} or \emph{infinite-volume limit}.

\begin{lemma}[Correlation functions are monotone in the domain (wired boundary)]
    \label{lemma:correlation_functions_monotone}
    Consider the Ising model on a locally finite graph
    $G$ at inverse temperature $\beta$.
    Let $A\subset V$ denote any finite subset.
    Then the function
    \[
        \Lambda\mapsto
        \langle\sigma_A\rangle_{\Lambda,\beta}^+
    \]
    is a nonincreasing function of the domain $\Lambda$.

    In particular, we have well-definedness of the thermodynamical limit
    \[
        \lim_{\Lambda\uparrow V}
        \langle\sigma_A\rangle_{\Lambda,\beta}^+.
    \]
\end{lemma}

\begin{proof}
    Consider two domains $\Lambda\subset\bar\Lambda$.
    We want to show that
    \[
        \langle\sigma_A\rangle_{\Lambda,\beta}^+
        \geq
        \langle\sigma_A\rangle_{\bar\Lambda,\beta}^+.
    \]
    Without loss of generality,
    $A\subset\bar\Lambda$ and
    $\bar\Lambda\setminus\Lambda=\{u\}$ for some
    vertex $u\in V$.

    Let $G'=(\bar\Lambda\cup\{\bar\Lambda^c\},E(\bar\Lambda))$ denote the graph obtained from $\bar\Lambda$
    as in Remark~\ref{remark:infinite_graphs_as_finite_graphs}
    and Exercise~\ref{exercise:infinite_graphs_as_finite_graphs}.
    We refer to the Ising model on $G'$ when subscripts are submitted from now on.
    Assume that $|A|$ is even for now.
    Then
    \begin{equation}
        \langle\sigma_A\rangle_{\bar\Lambda,\beta}^+
        =
        \E[\sigma_A];\qquad
        \langle\sigma_A\rangle_{\Lambda,\beta}^+
        =
        \E[\sigma_A|\{\sigma_u=\sigma_{\bar\Lambda^c}\}].
    \end{equation}
    It suffices to show that the conditioning increases the expectation.
    This is just Exercise~\ref{exo:conditioning_equality}.
    
    % Define the event $Q_\pm:=\{\sigma_u\sigma_{\bar\Lambda^c}=\pm 1\}$.
    % It suffices to prove that
    % \[
    %     \E[\sigma_A|Q_+]
    %     \geq
    %     \E[\sigma_A|Q_-].
    % \]
    % Notice that
    % \begin{multline}
    %     \P[Q_+]\E[\sigma_A|Q_+]
    %     -
    %     \P[Q_-]\E[\sigma_A|Q_-]
    %     =
    %     \E[\sigma_A\sigma_{\{u,\bar\Lambda^c\}}]
    %     \\
    %     \stackrel{\text{second Griffiths}}\geq
    %     \E[\sigma_A]\E[\sigma_{\{u,\bar\Lambda^c\}}]
    %     =(\P[Q_+]
    %     -
    %     \P[Q_-])\E[\sigma_A].
    % \end{multline}
    % This is the desired inequality.

    If $|A|$ is odd then we just need to replace the set $A$
    by $A':=A\cup\{\bar\Lambda^c\}$.
    More precisely,
    we have
    \begin{equation}
        \langle\sigma_A\rangle_{\bar\Lambda,\beta}^+
        =
        \E[\sigma_{A'}];\qquad
        \langle\sigma_A\rangle_{\Lambda,\beta}^+
        =
        \E[\sigma_{A'}|\{\sigma_u=\sigma_{\bar\Lambda^c}\}].
    \end{equation}
    One may then simply apply Exercise~\ref{exo:conditioning_equality} as for the even case.
\end{proof}

Perhaps we were wondering if $\langle\sigma_0\rangle_{\Lambda_n,\beta}^+$
was decreasing in $n$ in the statement of the Peierls argument
(Theorem~\ref{thm:peierls_triangles}),
but the result we proved just now is much stronger:
we proved that the thermodynamical limit of any ``local Fourier coefficient''
is well-defined.
Rather than taking a thermodynamical limit of observables,
we would however like to make sense of the thermodynamical limit
of the family of measures $\langle\blank\rangle_{\Lambda,\beta}^+$.
The previous lemma enables us to do this;
we only need to set up the definitions to make formal sense of our limit.

\begin{definition}[The compact ``local convergence'' topology]
    Let $G$ denote a locally finite graph.
    Recall that $(\Omega,\calF)$ is the measurable space $\Omega:=\{\pm1\}^V$
    endowed with the product $\sigma$-algebra.
    For a domain $\Lambda$, we write $\calF_\Lambda$
    for the $\sigma$-algebra generated by spins in $\Lambda$.
    An observable $X:\Omega\to\C$ is called \emph{local} if it is measurable
    with respect to $\calF_\Lambda$ for some domain $\Lambda$.

    Let $\calP(\Omega,\calF)$ denote the set of all probability measures
    on this measurable space.
    We endow this set with the \emph{local convergence topology},
    which is defined as the topology making the map
    \[
        \calP(\Omega,\calF)\to\C,\,\mu\mapsto\mu[X]
    \]
    continuous for any local observable $X$.
\end{definition}

\begin{remark}
    This topology is sometimes known under different names in the literature
    (such as the \emph{weak topology}).
    I like the name \emph{local convergence topology} because it captures the essence quite literally:
    if the statistics of the measures within a fixed domain $\Lambda$ converge,
    then we have local convergence.
\end{remark}

\begin{exercise}
    Prove that $\calP(\Omega,\calF)$ is a compact space in this topology.
\end{exercise}

\begin{theorem}[Existence of the thermodynamical limit (wired boundary)]
    Consider the Ising model on a locally finite graph $G$
    at inverse temperature $\beta$.
    Then there exists a unique probability measure
    $\langle\blank\rangle_{G,\beta}^+\in\calP(\Omega,\calF)$
    such that
    \[
        \lim_{\Lambda\uparrow V}\langle X\rangle_{\Lambda,\beta}^+=\langle X\rangle_{G,\beta}^+
    \]
    for any local observable $X:\Omega\to\R$.
    In other words,
    \[
        \lim_{\Lambda\uparrow V}\langle \blank\rangle_{\Lambda,\beta}^+
        =:
        \langle \blank\rangle_{G,\beta}^+.
    \]
    The measure $\langle\blank\rangle_{G,\beta}^+$ is called
    the \emph{thermodynamical limit} or \emph{infinite-volume limit}
    with $+$ boundary conditions.
\end{theorem}

\begin{proof}
    Any local observable may be written as a finite linear conbination
    of observables of the form $\sigma_A$ where $A$ is a finite subset of
    $V$.
    The theorem then follows by compactness and Lemma~\ref{lemma:correlation_functions_monotone}.
\end{proof}

\begin{exercise}[Continuity properties in $\beta$ (wired boundary)]
    Consider the Ising model on a locally finite graph $G$.
    Prove all of the following statements.
    \begin{itemize}
        \item For any finite sets $A,\Lambda\subset V$,
        the function $[0,\infty)\to\R,\,\beta\mapsto\langle\sigma_A\rangle_{\Lambda,\beta}^+$
        is non-decreasing and continuous.
        \item For any finite set $A\subset V$,
        the function $[0,\infty)\to\R,\,\beta\mapsto\langle\sigma_A\rangle_{G,\beta}^+$
        is non-decreasing and right-continuous.
        \item The function $[0,\infty)\to\calP(\Omega,\calF),\,\beta\mapsto\langle\blank\rangle_{G,\beta}^+$
        is a right-continuous function.
        \item The points of discontinuity form a countable subset of $[0,\infty)$.
    \end{itemize}
\end{exercise}

\begin{definition}[Magnetisation and critical temperature]
    Let $G$ denote a locally finite graph and $u$ some distinguished reference vertex.
    The function
    \[
        m=m_G:[0,\infty)\to\R,\,\beta\mapsto\langle\sigma_u\rangle_{G,\beta}^+
    \]
    is called the \emph{magnetisation}.

    The \emph{critical (inverse) temperature} is defined via
    \[
        \beta_c:=\inf\{\beta\in[0,\infty):m(\beta)>0\}.
    \]
\end{definition}

\begin{remark}
    The exercise above proves that $m$ is a non-decreasing right-continuous function.
    We have already seen that:
    \begin{itemize}
        \item If $G$ has max-degree $d$,
        then $m(\beta)=0$ for $\beta<1/d$ (Exercise~\ref{exercise:currents_peierls}),
        \item If $G$ is the graph $\Z^d$ for $d\geq 2$,
        then $\lim_{\beta\to\infty}m(\beta)>0$ (Exercise~\ref{exo:peierls_general}).
    \end{itemize}
    This implies in particular that on the square lattice graph $\Z^d$ in dimension
    $d\geq 2$,
    the critical inverse temperature $\beta_c$ is a strictly positive real number.
    A key objective of our field is to understand the behaviour of the Ising model
    at $\beta=\beta_c$ and at $\beta\approx\beta_c$.
    As a very first question we can ask:
    is the function $m$ continuous?
    We will see the answer in Theorem~\ref{}\todo{Add ref to theorem}.
\end{remark}


% \input{chapters/xx_correlation_fkg.tex}
% \input{chapters/xx_correlation_fkg_ising.tex}

\bibliographystyle{amsalpha}
\bibliography{}
\end{document}
