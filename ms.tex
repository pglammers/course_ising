% \documentclass[dvipsnames,11pt,reqno,twoside,final]{amsart}
\usepackage[a4paper,left=3cm,right=3cm,top=3cm,bottom=3cm]{geometry}
\usepackage[T1]{fontenc}
\usepackage[utf8]{inputenc}
\usepackage{xcolor}
\usepackage{amssymb,mathtools}  % Also loads "amsmath"
\usepackage{dsfont}
\usepackage{enumitem}
\usepackage{subcaption}
\usepackage{booktabs}
\newcommand\numberthis{\addtocounter{equation}{1}\tag{\theequation}}
\usepackage{preamble/mhequ}

\usepackage{amsthm}


\newcounter{counterEnvMain}
\newcounter{counterEnvDefault}
\numberwithin{counterEnvDefault}{section}


% ====================
\theoremstyle{plain}

\newtheorem{mainlemma}[counterEnvMain]{Lemma}
\newtheorem{lemma}[counterEnvDefault]{Lemma}
\newtheorem*{lemma*}{Lemma}

\newtheorem{maintheorem}[counterEnvMain]{Theorem}
\newtheorem{theorem}[counterEnvDefault]{Theorem}

\newtheorem{proposition}[counterEnvDefault]{Proposition}
\newtheorem{corollary}[counterEnvDefault]{Corollary}
\newtheorem{assumption}[counterEnvDefault]{Assumption}


% ====================
\theoremstyle{definition}

\newtheorem{definition}[counterEnvDefault]{Definition}
\newtheorem*{definition*}{Definition}

\newtheorem{example}[counterEnvDefault]{Example}
\newtheorem*{example*}{Example}


\newtheorem{exercise}[counterEnvDefault]{Exercise}
\newtheorem*{exercise*}{Exercise}

\newtheorem{remark}[counterEnvDefault]{Remark}

\newtheorem*{claim*}{Claim}

\newtheorem*{assertion*}{Assertion}

\newtheorem*{proposition*}{Proposition}

\usepackage{microtype}
\renewcommand\phi\varphi
\renewcommand\epsilon\varepsilon

\usepackage{hyperref}
\definecolor{colorlinks}{RGB}{0, 24, 168}
\definecolor{colorcites}{RGB}{124, 10, 2}
\hypersetup{
    colorlinks=true,
    linkcolor=colorlinks,
    citecolor=colorcites,
    urlcolor=colorlinks,
    pdfborder={0 0 0}
}

\usepackage{xargs}
\usepackage[colorinlistoftodos,prependcaption,textsize=tiny]{todonotes}

\newcommandx\work[2][1=]{\todo[linecolor=RoyalBlue,backgroundcolor=RoyalBlue!25,bordercolor=RoyalBlue,#1]{\textsc{todo} #2}}
\newcommandx\comment[2][1=]{\todo[linecolor=OliveGreen,backgroundcolor=OliveGreen!25,bordercolor=OliveGreen,#1]{\textsc{comment} #2}}
\newcommandx\mistake[2][1=]{\todo[linecolor=red,backgroundcolor=red!25,bordercolor=red,#1]{\textsc{mistake} #2}}
\newcommandx\improve[2][1=]{\todo[linecolor=orange,backgroundcolor=orange!25,bordercolor=orange,#1]{\textsc{improve} #2}}
\newcommandx\change[2][1=]{\todo[linecolor=yellow,backgroundcolor=yellow!25,bordercolor=yellow,#1]{\textsc{change} #2}}
\newcommandx\mem[2][1=]{\todo[linecolor=orange,backgroundcolor=orange!25,bordercolor=orange,#1]{\textsc{mem} #2}}
\newcommandx\status[2][1=]{\todo[linecolor=Blue,backgroundcolor=Blue!25,bordercolor=Blue,#1]{\textsc{Status} #2}}

\newcommand\hidetodos{
    \renewcommandx\todo[2][1=]{}
    \renewcommandx\work[2][1=]{}
    \renewcommandx\comment[2][1=]{}
    \renewcommandx\mistake[2][1=]{}
    \renewcommandx\improve[2][1=]{}
    \renewcommandx\change[2][1=]{}
    \renewcommandx\mem[2][1=]{}
    \renewcommandx\status[2][1=]{}
}

\newcommand\blank{\,\cdot\,}
\newcommand\ssubset{\Subset}

\newcommand\diam{\operatorname{diam}}
\newcommand\Var{\operatorname{Var}}
\newcommand\Cov{\operatorname{Cov}}

\newcommand\ind[1]{\mathds{1}_{#1}}
\newcommand\true[1]{\mathds{1}({#1})}
\newcommand\diffi{{\,\mathrm{d}}}
\newcommand\diff{{\mathrm{d}}}

\newcommand\A{\mathbb A}
\newcommand\B{\mathbb B}
\newcommand\C{\mathbb C}
\newcommand\D{\mathbb D}
\newcommand\E{\mathbb E}
\newcommand\F{\mathbb F}
\newcommand\G{\mathbb G}
\renewcommand\H{\mathbb H}
\newcommand\I{\mathbb I}
\newcommand\J{\mathbb J}
\newcommand\K{\mathbb K}
\renewcommand\L{\mathbb L}
\newcommand\M{\mathbb M}
\newcommand\N{\mathbb N}
\renewcommand\O{\mathbb O}
\renewcommand\P{\mathbb P}
\newcommand\Q{\mathbb Q}
\newcommand\R{\mathbb R}
\renewcommand\S{\mathbb S}
\newcommand\T{\mathbb T}
\newcommand\U{\mathbb U}
\newcommand\V{\mathbb V}
\newcommand\W{\mathbb W}
\newcommand\X{\mathbb X}
\newcommand\Y{\mathbb Y}
\newcommand\Z{\mathbb Z}

\newcommand\bbA{\mathbb A}
\newcommand\bbB{\mathbb B}
\newcommand\bbC{\mathbb C}
\newcommand\bbD{\mathbb D}
\newcommand\bbE{\mathbb E}
\newcommand\bbF{\mathbb F}
\newcommand\bbG{\mathbb G}
\newcommand\bbH{\mathbb H}
\newcommand\bbI{\mathbb I}
\newcommand\bbJ{\mathbb J}
\newcommand\bbK{\mathbb K}
\newcommand\bbL{\mathbb L}
\newcommand\bbM{\mathbb M}
\newcommand\bbN{\mathbb N}
\newcommand\bbO{\mathbb O}
\newcommand\bbP{\mathbb P}
\newcommand\bbQ{\mathbb Q}
\newcommand\bbR{\mathbb R}
\newcommand\bbS{\mathbb S}
\newcommand\bbT{\mathbb T}
\newcommand\bbU{\mathbb U}
\newcommand\bbV{\mathbb V}
\newcommand\bbW{\mathbb W}
\newcommand\bbX{\mathbb X}
\newcommand\bbY{\mathbb Y}
\newcommand\bbZ{\mathbb Z}

\newcommand\CA{\mathcal A}
\newcommand\CB{\mathcal B}
\newcommand\CC{\mathcal C}
\newcommand\CD{\mathcal D}
\newcommand\CE{\mathcal E}
\newcommand\CF{\mathcal F}
\newcommand\CG{\mathcal G}
\newcommand\CH{\mathcal H}
\newcommand\CI{\mathcal I}
\newcommand\CJ{\mathcal J}
\newcommand\CK{\mathcal K}
\newcommand\CL{\mathcal L}
\newcommand\CM{\mathcal M}
\newcommand\CN{\mathcal N}
\newcommand\CO{\mathcal O}
\newcommand\CP{\mathcal P}
\newcommand\CQ{\mathcal Q}
\newcommand\CR{\mathcal R}
\newcommand\CS{\mathcal S}
\newcommand\CT{\mathcal T}
\newcommand\CU{\mathcal U}
\newcommand\CV{\mathcal V}
\newcommand\CW{\mathcal W}
\newcommand\CX{\mathcal X}
\newcommand\CY{\mathcal Y}
\newcommand\CZ{\mathcal Z}

\newcommand\calA{\mathcal A}
\newcommand\calB{\mathcal B}
\newcommand\calC{\mathcal C}
\newcommand\calD{\mathcal D}
\newcommand\calE{\mathcal E}
\newcommand\calF{\mathcal F}
\newcommand\calG{\mathcal G}
\newcommand\calH{\mathcal H}
\newcommand\calI{\mathcal I}
\newcommand\calJ{\mathcal J}
\newcommand\calK{\mathcal K}
\newcommand\calL{\mathcal L}
\newcommand\calM{\mathcal M}
\newcommand\calN{\mathcal N}
\newcommand\calO{\mathcal O}
\newcommand\calP{\mathcal P}
\newcommand\calQ{\mathcal Q}
\newcommand\calR{\mathcal R}
\newcommand\calS{\mathcal S}
\newcommand\calT{\mathcal T}
\newcommand\calU{\mathcal U}
\newcommand\calV{\mathcal V}
\newcommand\calW{\mathcal W}
\newcommand\calX{\mathcal X}
\newcommand\calY{\mathcal Y}
\newcommand\calZ{\mathcal Z}

\newcommand\FA{\mathfrak A}
\newcommand\FB{\mathfrak B}
\newcommand\FC{\mathfrak C}
\newcommand\FD{\mathfrak D}
\newcommand\FE{\mathfrak E}
\newcommand\FF{\mathfrak F}
\newcommand\FG{\mathfrak G}
\newcommand\FH{\mathfrak H}
\newcommand\FI{\mathfrak I}
\newcommand\FJ{\mathfrak J}
\newcommand\FK{\mathfrak K}
\newcommand\FL{\mathfrak L}
\newcommand\FM{\mathfrak M}
\newcommand\FN{\mathfrak N}
\newcommand\FO{\mathfrak O}
\newcommand\FP{\mathfrak P}
\newcommand\FQ{\mathfrak Q}
\newcommand\FR{\mathfrak R}
\newcommand\FS{\mathfrak S}
\newcommand\FT{\mathfrak T}
\newcommand\FU{\mathfrak U}
\newcommand\FV{\mathfrak V}
\newcommand\FW{\mathfrak W}
\newcommand\FX{\mathfrak X}
\newcommand\FY{\mathfrak Y}
\newcommand\FZ{\mathfrak Z}

\newcommand\Fa{\mathfrak a}
\newcommand\Fb{\mathfrak b}
\newcommand\Fc{\mathfrak c}
\newcommand\Fd{\mathfrak d}
\newcommand\Fe{\mathfrak e}
\newcommand\Ff{\mathfrak f}
\newcommand\Fg{\mathfrak g}
\newcommand\Fh{\mathfrak h}
\newcommand\Fi{\mathfrak i}
\newcommand\Fj{\mathfrak j}
\newcommand\Fk{\mathfrak k}
\newcommand\Fl{\mathfrak l}
\newcommand\Fm{\mathfrak m}
\newcommand\Fn{\mathfrak n}
\newcommand\Fo{\mathfrak o}
\newcommand\Fp{\mathfrak p}
\newcommand\Fq{\mathfrak q}
\newcommand\Fr{\mathfrak r}
\newcommand\Fs{\mathfrak s}
\newcommand\Ft{\mathfrak t}
\newcommand\Fu{\mathfrak u}
\newcommand\Fv{\mathfrak v}
\newcommand\Fw{\mathfrak w}
\newcommand\Fx{\mathfrak x}
\newcommand\Fy{\mathfrak y}
\newcommand\Fz{\mathfrak z}

\newcommand\frakA{\mathfrak A}
\newcommand\frakB{\mathfrak B}
\newcommand\frakC{\mathfrak C}
\newcommand\frakD{\mathfrak D}
\newcommand\frakE{\mathfrak E}
\newcommand\frakF{\mathfrak F}
\newcommand\frakG{\mathfrak G}
\newcommand\frakH{\mathfrak H}
\newcommand\frakI{\mathfrak I}
\newcommand\frakJ{\mathfrak J}
\newcommand\frakK{\mathfrak K}
\newcommand\frakL{\mathfrak L}
\newcommand\frakM{\mathfrak M}
\newcommand\frakN{\mathfrak N}
\newcommand\frakO{\mathfrak O}
\newcommand\frakP{\mathfrak P}
\newcommand\frakQ{\mathfrak Q}
\newcommand\frakR{\mathfrak R}
\newcommand\frakS{\mathfrak S}
\newcommand\frakT{\mathfrak T}
\newcommand\frakU{\mathfrak U}
\newcommand\frakV{\mathfrak V}
\newcommand\frakW{\mathfrak W}
\newcommand\frakX{\mathfrak X}
\newcommand\frakY{\mathfrak Y}
\newcommand\frakZ{\mathfrak Z}

\newcommand\fraka{\mathfrak a}
\newcommand\frakb{\mathfrak b}
\newcommand\frakc{\mathfrak c}
\newcommand\frakd{\mathfrak d}
\newcommand\frake{\mathfrak e}
\newcommand\frakf{\mathfrak f}
\newcommand\frakg{\mathfrak g}
\newcommand\frakh{\mathfrak h}
\newcommand\fraki{\mathfrak i}
\newcommand\frakj{\mathfrak j}
\newcommand\frakk{\mathfrak k}
\newcommand\frakl{\mathfrak l}
\newcommand\frakm{\mathfrak m}
\newcommand\frakn{\mathfrak n}
\newcommand\frako{\mathfrak o}
\newcommand\frakp{\mathfrak p}
\newcommand\frakq{\mathfrak q}
\newcommand\frakr{\mathfrak r}
\newcommand\fraks{\mathfrak s}
\newcommand\frakt{\mathfrak t}
\newcommand\fraku{\mathfrak u}
\newcommand\frakv{\mathfrak v}
\newcommand\frakw{\mathfrak w}
\newcommand\frakx{\mathfrak x}
\newcommand\fraky{\mathfrak y}
\newcommand\frakz{\mathfrak z}

\newcommand\figleft{{\scshape{Left}}}
\newcommand\figmiddle{{\scshape{Middle}}}
\newcommand\figright{{\scshape{Right}}}



\documentclass[dvipsnames,11pt,reqno,oneside,draft]{amsart}
\usepackage[a4paper,left=1cm,right=5cm,top=3cm,bottom=3cm]{geometry}
\usepackage[T1]{fontenc}
\usepackage[utf8]{inputenc}
\usepackage{xcolor}
\usepackage{amssymb,mathtools}  % Also loads "amsmath"
\usepackage{dsfont}
\usepackage{enumitem}
\usepackage{subcaption}
\usepackage{booktabs}
\newcommand\numberthis{\addtocounter{equation}{1}\tag{\theequation}}
\usepackage{preamble/mhequ}

\usepackage{amsthm}


\newcounter{counterEnvMain}
\newcounter{counterEnvDefault}
\numberwithin{counterEnvDefault}{section}


% ====================
\theoremstyle{plain}

\newtheorem{mainlemma}[counterEnvMain]{Lemma}
\newtheorem{lemma}[counterEnvDefault]{Lemma}
\newtheorem*{lemma*}{Lemma}

\newtheorem{maintheorem}[counterEnvMain]{Theorem}
\newtheorem{theorem}[counterEnvDefault]{Theorem}

\newtheorem{proposition}[counterEnvDefault]{Proposition}
\newtheorem{corollary}[counterEnvDefault]{Corollary}
\newtheorem{assumption}[counterEnvDefault]{Assumption}


% ====================
\theoremstyle{definition}

\newtheorem{definition}[counterEnvDefault]{Definition}
\newtheorem*{definition*}{Definition}

\newtheorem{example}[counterEnvDefault]{Example}
\newtheorem*{example*}{Example}


\newtheorem{exercise}[counterEnvDefault]{Exercise}
\newtheorem*{exercise*}{Exercise}

\newtheorem{remark}[counterEnvDefault]{Remark}

\newtheorem*{claim*}{Claim}

\newtheorem*{assertion*}{Assertion}

\newtheorem*{proposition*}{Proposition}

\usepackage{microtype}
\renewcommand\phi\varphi
\renewcommand\epsilon\varepsilon

\usepackage{hyperref}
\definecolor{colorlinks}{RGB}{0, 24, 168}
\definecolor{colorcites}{RGB}{124, 10, 2}
\hypersetup{
    colorlinks=true,
    linkcolor=colorlinks,
    citecolor=colorcites,
    urlcolor=colorlinks,
    pdfborder={0 0 0}
}

\usepackage{xargs}
\usepackage[colorinlistoftodos,prependcaption,textsize=tiny]{todonotes}

\newcommandx\work[2][1=]{\todo[linecolor=RoyalBlue,backgroundcolor=RoyalBlue!25,bordercolor=RoyalBlue,#1]{\textsc{todo} #2}}
\newcommandx\comment[2][1=]{\todo[linecolor=OliveGreen,backgroundcolor=OliveGreen!25,bordercolor=OliveGreen,#1]{\textsc{comment} #2}}
\newcommandx\mistake[2][1=]{\todo[linecolor=red,backgroundcolor=red!25,bordercolor=red,#1]{\textsc{mistake} #2}}
\newcommandx\improve[2][1=]{\todo[linecolor=orange,backgroundcolor=orange!25,bordercolor=orange,#1]{\textsc{improve} #2}}
\newcommandx\change[2][1=]{\todo[linecolor=yellow,backgroundcolor=yellow!25,bordercolor=yellow,#1]{\textsc{change} #2}}
\newcommandx\mem[2][1=]{\todo[linecolor=orange,backgroundcolor=orange!25,bordercolor=orange,#1]{\textsc{mem} #2}}
\newcommandx\status[2][1=]{\todo[linecolor=Blue,backgroundcolor=Blue!25,bordercolor=Blue,#1]{\textsc{Status} #2}}

\newcommand\hidetodos{
    \renewcommandx\todo[2][1=]{}
    \renewcommandx\work[2][1=]{}
    \renewcommandx\comment[2][1=]{}
    \renewcommandx\mistake[2][1=]{}
    \renewcommandx\improve[2][1=]{}
    \renewcommandx\change[2][1=]{}
    \renewcommandx\mem[2][1=]{}
    \renewcommandx\status[2][1=]{}
}

\newcommand\blank{\,\cdot\,}
\newcommand\ssubset{\Subset}

\newcommand\diam{\operatorname{diam}}
\newcommand\Var{\operatorname{Var}}
\newcommand\Cov{\operatorname{Cov}}

\newcommand\ind[1]{\mathds{1}_{#1}}
\newcommand\true[1]{\mathds{1}({#1})}
\newcommand\diffi{{\,\mathrm{d}}}
\newcommand\diff{{\mathrm{d}}}

\newcommand\A{\mathbb A}
\newcommand\B{\mathbb B}
\newcommand\C{\mathbb C}
\newcommand\D{\mathbb D}
\newcommand\E{\mathbb E}
\newcommand\F{\mathbb F}
\newcommand\G{\mathbb G}
\renewcommand\H{\mathbb H}
\newcommand\I{\mathbb I}
\newcommand\J{\mathbb J}
\newcommand\K{\mathbb K}
\renewcommand\L{\mathbb L}
\newcommand\M{\mathbb M}
\newcommand\N{\mathbb N}
\renewcommand\O{\mathbb O}
\renewcommand\P{\mathbb P}
\newcommand\Q{\mathbb Q}
\newcommand\R{\mathbb R}
\renewcommand\S{\mathbb S}
\newcommand\T{\mathbb T}
\newcommand\U{\mathbb U}
\newcommand\V{\mathbb V}
\newcommand\W{\mathbb W}
\newcommand\X{\mathbb X}
\newcommand\Y{\mathbb Y}
\newcommand\Z{\mathbb Z}

\newcommand\bbA{\mathbb A}
\newcommand\bbB{\mathbb B}
\newcommand\bbC{\mathbb C}
\newcommand\bbD{\mathbb D}
\newcommand\bbE{\mathbb E}
\newcommand\bbF{\mathbb F}
\newcommand\bbG{\mathbb G}
\newcommand\bbH{\mathbb H}
\newcommand\bbI{\mathbb I}
\newcommand\bbJ{\mathbb J}
\newcommand\bbK{\mathbb K}
\newcommand\bbL{\mathbb L}
\newcommand\bbM{\mathbb M}
\newcommand\bbN{\mathbb N}
\newcommand\bbO{\mathbb O}
\newcommand\bbP{\mathbb P}
\newcommand\bbQ{\mathbb Q}
\newcommand\bbR{\mathbb R}
\newcommand\bbS{\mathbb S}
\newcommand\bbT{\mathbb T}
\newcommand\bbU{\mathbb U}
\newcommand\bbV{\mathbb V}
\newcommand\bbW{\mathbb W}
\newcommand\bbX{\mathbb X}
\newcommand\bbY{\mathbb Y}
\newcommand\bbZ{\mathbb Z}

\newcommand\CA{\mathcal A}
\newcommand\CB{\mathcal B}
\newcommand\CC{\mathcal C}
\newcommand\CD{\mathcal D}
\newcommand\CE{\mathcal E}
\newcommand\CF{\mathcal F}
\newcommand\CG{\mathcal G}
\newcommand\CH{\mathcal H}
\newcommand\CI{\mathcal I}
\newcommand\CJ{\mathcal J}
\newcommand\CK{\mathcal K}
\newcommand\CL{\mathcal L}
\newcommand\CM{\mathcal M}
\newcommand\CN{\mathcal N}
\newcommand\CO{\mathcal O}
\newcommand\CP{\mathcal P}
\newcommand\CQ{\mathcal Q}
\newcommand\CR{\mathcal R}
\newcommand\CS{\mathcal S}
\newcommand\CT{\mathcal T}
\newcommand\CU{\mathcal U}
\newcommand\CV{\mathcal V}
\newcommand\CW{\mathcal W}
\newcommand\CX{\mathcal X}
\newcommand\CY{\mathcal Y}
\newcommand\CZ{\mathcal Z}

\newcommand\calA{\mathcal A}
\newcommand\calB{\mathcal B}
\newcommand\calC{\mathcal C}
\newcommand\calD{\mathcal D}
\newcommand\calE{\mathcal E}
\newcommand\calF{\mathcal F}
\newcommand\calG{\mathcal G}
\newcommand\calH{\mathcal H}
\newcommand\calI{\mathcal I}
\newcommand\calJ{\mathcal J}
\newcommand\calK{\mathcal K}
\newcommand\calL{\mathcal L}
\newcommand\calM{\mathcal M}
\newcommand\calN{\mathcal N}
\newcommand\calO{\mathcal O}
\newcommand\calP{\mathcal P}
\newcommand\calQ{\mathcal Q}
\newcommand\calR{\mathcal R}
\newcommand\calS{\mathcal S}
\newcommand\calT{\mathcal T}
\newcommand\calU{\mathcal U}
\newcommand\calV{\mathcal V}
\newcommand\calW{\mathcal W}
\newcommand\calX{\mathcal X}
\newcommand\calY{\mathcal Y}
\newcommand\calZ{\mathcal Z}

\newcommand\FA{\mathfrak A}
\newcommand\FB{\mathfrak B}
\newcommand\FC{\mathfrak C}
\newcommand\FD{\mathfrak D}
\newcommand\FE{\mathfrak E}
\newcommand\FF{\mathfrak F}
\newcommand\FG{\mathfrak G}
\newcommand\FH{\mathfrak H}
\newcommand\FI{\mathfrak I}
\newcommand\FJ{\mathfrak J}
\newcommand\FK{\mathfrak K}
\newcommand\FL{\mathfrak L}
\newcommand\FM{\mathfrak M}
\newcommand\FN{\mathfrak N}
\newcommand\FO{\mathfrak O}
\newcommand\FP{\mathfrak P}
\newcommand\FQ{\mathfrak Q}
\newcommand\FR{\mathfrak R}
\newcommand\FS{\mathfrak S}
\newcommand\FT{\mathfrak T}
\newcommand\FU{\mathfrak U}
\newcommand\FV{\mathfrak V}
\newcommand\FW{\mathfrak W}
\newcommand\FX{\mathfrak X}
\newcommand\FY{\mathfrak Y}
\newcommand\FZ{\mathfrak Z}

\newcommand\Fa{\mathfrak a}
\newcommand\Fb{\mathfrak b}
\newcommand\Fc{\mathfrak c}
\newcommand\Fd{\mathfrak d}
\newcommand\Fe{\mathfrak e}
\newcommand\Ff{\mathfrak f}
\newcommand\Fg{\mathfrak g}
\newcommand\Fh{\mathfrak h}
\newcommand\Fi{\mathfrak i}
\newcommand\Fj{\mathfrak j}
\newcommand\Fk{\mathfrak k}
\newcommand\Fl{\mathfrak l}
\newcommand\Fm{\mathfrak m}
\newcommand\Fn{\mathfrak n}
\newcommand\Fo{\mathfrak o}
\newcommand\Fp{\mathfrak p}
\newcommand\Fq{\mathfrak q}
\newcommand\Fr{\mathfrak r}
\newcommand\Fs{\mathfrak s}
\newcommand\Ft{\mathfrak t}
\newcommand\Fu{\mathfrak u}
\newcommand\Fv{\mathfrak v}
\newcommand\Fw{\mathfrak w}
\newcommand\Fx{\mathfrak x}
\newcommand\Fy{\mathfrak y}
\newcommand\Fz{\mathfrak z}

\newcommand\frakA{\mathfrak A}
\newcommand\frakB{\mathfrak B}
\newcommand\frakC{\mathfrak C}
\newcommand\frakD{\mathfrak D}
\newcommand\frakE{\mathfrak E}
\newcommand\frakF{\mathfrak F}
\newcommand\frakG{\mathfrak G}
\newcommand\frakH{\mathfrak H}
\newcommand\frakI{\mathfrak I}
\newcommand\frakJ{\mathfrak J}
\newcommand\frakK{\mathfrak K}
\newcommand\frakL{\mathfrak L}
\newcommand\frakM{\mathfrak M}
\newcommand\frakN{\mathfrak N}
\newcommand\frakO{\mathfrak O}
\newcommand\frakP{\mathfrak P}
\newcommand\frakQ{\mathfrak Q}
\newcommand\frakR{\mathfrak R}
\newcommand\frakS{\mathfrak S}
\newcommand\frakT{\mathfrak T}
\newcommand\frakU{\mathfrak U}
\newcommand\frakV{\mathfrak V}
\newcommand\frakW{\mathfrak W}
\newcommand\frakX{\mathfrak X}
\newcommand\frakY{\mathfrak Y}
\newcommand\frakZ{\mathfrak Z}

\newcommand\fraka{\mathfrak a}
\newcommand\frakb{\mathfrak b}
\newcommand\frakc{\mathfrak c}
\newcommand\frakd{\mathfrak d}
\newcommand\frake{\mathfrak e}
\newcommand\frakf{\mathfrak f}
\newcommand\frakg{\mathfrak g}
\newcommand\frakh{\mathfrak h}
\newcommand\fraki{\mathfrak i}
\newcommand\frakj{\mathfrak j}
\newcommand\frakk{\mathfrak k}
\newcommand\frakl{\mathfrak l}
\newcommand\frakm{\mathfrak m}
\newcommand\frakn{\mathfrak n}
\newcommand\frako{\mathfrak o}
\newcommand\frakp{\mathfrak p}
\newcommand\frakq{\mathfrak q}
\newcommand\frakr{\mathfrak r}
\newcommand\fraks{\mathfrak s}
\newcommand\frakt{\mathfrak t}
\newcommand\fraku{\mathfrak u}
\newcommand\frakv{\mathfrak v}
\newcommand\frakw{\mathfrak w}
\newcommand\frakx{\mathfrak x}
\newcommand\fraky{\mathfrak y}
\newcommand\frakz{\mathfrak z}

\newcommand\figleft{{\scshape{Left}}}
\newcommand\figmiddle{{\scshape{Middle}}}
\newcommand\figright{{\scshape{Right}}}



\mathtoolsset{showonlyrefs}

\makeatletter
\@namedef{subjclassname@2020}{\textup{2020} Mathematics Subject Classification}
\makeatother


\title{A course on the Ising model}
\subjclass[2020]{82-01, 82B20}


\author{Piet Lammers}
\address{CNRS and Sorbonne Université, LPSM}
\email{piet.lammers@cnrs.fr}

\date{\today}
% \keywords{%
%     [keyword 1],
%     [keyword 2]
% }

\newcommand\n{\mathbf{n}}
\newcommand\m{\mathbf{m}}
\newcommand\s{\mathbf{s}}
\renewcommand\a{\mathbf{a}}
\renewcommand\b{\mathbf{b}}
\newcommand\f{{\operatorname{f}}}
\newcommand\bfM{\mathbf{M}}
\newcommand\bfP{\mathbf{P}}

\thanks{This work is licensed under CC BY-NC-SA 4.0.\\\indent To view a copy of this license, visit \url{https://creativecommons.org/licenses/by-nc-sa/4.0/}}

\begin{document}


\maketitle

\tableofcontents
% \begin{abstract}
%     [Abstract text]
% \end{abstract}


\section*{Preface}
These lecture notes are written progressively during the 2025 spring semester,
as the course is taught at Sorbonne university in the M2 (second-year masters) programme.
Its purpose is to give a broad introduction to the rigorous analysis of the Ising model.
The main focus is on four techniques and their applications:
\begin{itemize}
    \item The Peierls argument,
    \item The random-currents representation,
    \item The FKG inequality for the Ising spins,
    \item The FKG inequality for the random-cluster (FK) representation.
\end{itemize}

A basic understanding of analysis and probability theory is essential for following this course.
Experience with other models in statistical mechanics (such as the Bernoulli percolation model)
is a plus but by no means essential.

The appendices contain overviews of the main definitions, expansions, and inequalities in this text.

These notes are inspired by the lecture notes \emph{Lectures on the Ising and Potts models on the hypercubic lattice}
and the overview \emph{100 Years of the (Critical) Ising Model on the Hypercubic Lattice},
both due to Hugo Duminil-Copin.
The main text does not contain references at this stage; they will be added at a later time.


\section{The Curie--Weiss model}
\label{sec:definitions_examples}

At the end of the 19th century, Curie published his experimental results on 
\emph{ferromagnetism}:
the
magnetic properties of metals.
He made three striking observations.
\begin{itemize}
    \item The magnetic strength of a metal varies with the temperature.
    Increasing the temperature decreases the magnetic strength.
    \item Each metal has a certain temperature, specific to that metal,
    at which the magnetic properties disappear entirely.
    We call this temperature the \emph{Curie temperature}.
    \item Around the Curie temperature, the magnetic strength drops continuously
    to zero. In other words, the magnetic strength does not ``jump'' to zero.
\end{itemize}
The first observation singles out the temperature as the driving parameter of the system.
This is good news for us, since the temperature may be regarded informally as the amount of 
``randomness'' or ``entropy'' in the system, justifying a probabilistic analysis of the situation.
The second observation implies that there is a \emph{phase transition}:
there is a special temperature (in this case the Curie temperature) at which
the system's behaviour undergoes a qualitative change.
The third observation entails an important property of this phase transition.

The first mathematical explanation for Curie's experimental results came from
Weiss.
He proposed the following mathematical axioms for studying the magnetic properties of metals.
\begin{itemize}
    \item The metal consists of $n$ atoms.
    \item Each atom acts like a small magnet in itself.
    It is in one of two states, denoted $\pm$.
    \item The total strength of the metal is obtained by summing the states of all atoms.
    \item Each atom interacts with all other atoms.
    The atoms prefer to \emph{align}, that is, to be in the same state.
    The temperature regulates the strength of the interaction.
\end{itemize}
Physically, it makes sense that the temperature regulates the interaction strength.
When atoms move slowly, they will stabilise, oriented in alignment with the magnetic field imposed by the other atoms.
When atoms move fast, they will not bother with the states of the other atoms, and simply align themselves randomly.
It is thus natural to think of the interaction strength as the \emph{inverse temperature}.

\begin{definition}[Curie-Weiss model]
    The Curie-Weiss model is the probability measure $\P^{\operatorname{CW}}_{n,\beta}$
    on $\sigma\in\Omega:=\{+,-\}^n$
    defined via
    \[
        \P(\sigma):=\P^{\operatorname{CW}}_{n,\beta}(\sigma)
        \propto
        e^{-H^{\operatorname{CW}}_{n,\beta}(\sigma)}
        ;
        \qquad
        H(\sigma):=H^{\operatorname{CW}}_{n,\beta}(\sigma):=
        -\frac\beta{n} \sum_{i<j}\sigma_i\sigma_j,
    \]
    where $n\in\Z_{\geq 1}$ and $\beta\in[0,\infty)$.
    The parameter $\beta$ is called the \emph{interaction strength} or \emph{inverse temperature}.
    The function $H$ is called the \emph{Hamiltonian} and captures the \emph{energy} in the system.
    The probability measure $\P$ is also called the \emph{Boltzmann distribution}.
\end{definition}

Let $n_+=n_+(\sigma)$ denote the number of vertices with spin $+$ in a configuration $\sigma\in\Omega$.
This is a random variable.
Let us try to calculate the probability of the event $\{n_+=k\}$,
without worrying about the partition function (the normalising constant).
One may easily check that
the Hamiltonian satisfies
\begin{align}
    H(\sigma)=2\frac\beta{n} n_+(n-n_+) + \text{const}(n).
\end{align}
The distribution of $n_+$ can then be calculated as follows:
\begin{align}
    \label{eq:CurieWeissDistribution}
    \P(\{n_+ = k\}) &\propto \binom{n}{k} e^{-2\frac\beta{n}  k(n-k)}
    \propto \frac1{k! (n-k)!} e^{-2\frac\beta{n} k (n-k)}.
\end{align}
Using Stirling's approximation for the factorials, we find that
\begin{gather}
    \log\P(\{n_+ = k\})
    \stackrel{\text{Stirling}}{\approx}
    -n f_{\beta}(k/n) + \text{const}(n);
    \\
    f_{\beta}:[0,1]\to\R,\,
    x\mapsto x \log x + (1-x)\log (1-x) + 2 \beta x (1-x).
\end{gather}
If we fix $\beta$ and send $n$ to infinity, then 
we discover a large deviations principle for the random variable $n_+/n$
with rate function $f_{\beta}$ and speed $n$.
In particular, the random variable $n_+/n$ concentrates
around the minimisers of the function $f_{\beta}$.

\begin{exercise}[The rate function of the Curie--Weiss model]
    \begin{enumerate}
        \item Show that for small $\beta$, the function \( f_{\beta} \) has a single minimum at $x=1/2$, which means that the random variable \( n_+/n \) concentrates around the value \( 1/2 \).
        \item Show that for large $\beta$, the function \( f_{\beta} \) has two minima at \( (1 \pm m)/2 \) for some $m>0$, which means that the random variable \( n_+/n \) is concentrated around these minima.
        The value of $m$ is called the \emph{magnetisation}.
        \item Calculate the critical value for $\beta$. At this value, the second derivative of \( f_{\beta} \) vanishes at \( x=1/2 \). What does this mean for the distribution of \( n_+/n \)?
        Estimate the order of magnitude of $\Var\frac{n_+}{n}$ as $n\to\infty$ for this value of $\beta$.
    \end{enumerate}
\end{exercise}

\begin{remark}[Entropy versus energy in the Curie--Weiss model]
    Reconsider Equation~\eqref{eq:CurieWeissDistribution}.
    In this equation, the competition between the two factors is extremely transparent.
    \begin{itemize}
        \item     First, there is a combinatorial term or \emph{entropy}, which favours values $k$ for the random variable
        $n_+$ such that the cardinality of the set $\{n_+=k\}$ is large.
        This means that values $k\approx n/2$ are preferred.
        \item     Second, there is the \emph{energy} term, which favours values such that the energy 
        is minimised. This favours configurations where as many spins as possible align.    
    \end{itemize}
    The interaction parameter $\beta$ allows us to put more emphasis
    on the entropy term or on the energy term.
    In the $n\to\infty$ limit, there is a precise value for $\beta$
    where the behaviour of the random system undergoes a qualitative change:
    a rudimentary example of a \emph{phase transition}.
\end{remark}

\section{Ising's model and basic notions}
\label{sec:ising_1d}

While the competition between entropy and energy is transparent in the Curie--Weiss model,
the model does not encode any kind of geometry.
Indeed, all atoms interact equally with all other atoms.
It would perhaps be more realistic to place the atoms on a Euclidean grid,
and let the interactions strength between two atoms depend on their distance.
In the simplest case, we could simply let each atom interact only with the atoms
closest to it. This is called the \emph{nearest-neighbour interaction}.
We mainly focus on this setup in these lecture notes.

Wilhelm Lenz challenged his doctoral student Ernst Ising
to solve this nearest-neighbour model for magnetism on the one-dimensional line graph $\Z$.
Lenz was not entirely precise when posing this question,
and it was Ising who first formulated a definition for the model under consideration.
The model is therefore called the \emph{Ising model} in his honour.
We shall later derive Ising's result from a broader theorem (Theorem~\ref{thm:Exponential decay at high temperature}).

\begin{definition}[Ising model]
    \label{def:ising_finite}
    The Ising model on a finite graph \( G = (V, E) \) with \emph{inverse temperature} \( \beta \in [0,\infty) \) is defined as follows.
    Let $\Omega:=\{\pm1\}^V$ denote the set of spin configurations on the vertices of the graph;
    a typical element of $\Omega$ is denoted by $\sigma=(\sigma_u)_{u\in V}$.
    Elements $\sigma\in\Omega$ are called \emph{spin configurations};
    elements $\sigma_u$ are called \emph{spins}.
    The \emph{energy} or \emph{Hamiltonian} of a spin configuration $\sigma$ is given by
    \[
        H_{G,\beta}^{\operatorname{Ising}}(\sigma) := -\beta \sum_{uv \in E} \sigma_u \sigma_v.
    \]
    We write $\P_{G,\beta}^{\operatorname{Ising}}$ for the associated \emph{Boltzmann distribution} or \emph{Gibbs measure}:
    \[
        \P_{G,\beta}^{\operatorname{Ising}}(\sigma) := \frac{1}{Z_{G,\beta}^{\operatorname{Ising}}} e^{-H^{\operatorname{Ising}}_{G,\beta}(\sigma)},
    \]
    where \(Z_{G,\beta}^{\operatorname{Ising}}\) is normalisation constant or \emph{partition function} defined by
    \[
        Z_{G,\beta}^{\operatorname{Ising}}:= \sum_{\sigma\in\Omega} e^{-H^{\operatorname{Ising}}_{G,\beta}(\sigma)}.
    \]
    We shall write $\langle\blank\rangle_{G,\beta}^{\operatorname{Ising}}$ for the expectation functional associated to this probability measure.
\end{definition}

\begin{remark*}
    We shall often suppress subscripts and superscripts when they are clear from the context.
\end{remark*}

\begin{remark*}
    The mathematical community has widely adopted the terminology coming from the physics
    literature.
    We often prefer the symbol $\langle\blank\rangle$ over $\E[\blank]$ when taking expecations,
    expect when considering \emph{conditional} expectations.
\end{remark*}

\begin{exercise}[The edge graph]
    Consider the Ising model on the complete graph on the two vertices $V:=\{x,y\}$
    at inverse temperature $\beta\in[0,\infty)$.
    \begin{itemize}
        \item Calculate $\langle\sigma_x\rangle_\beta$.
        \item Calculate $\langle\sigma_x\sigma_y\rangle_\beta$.
    \end{itemize}
\end{exercise}

\begin{definition}[Correlation functions]
    Consider $\Omega:=\{\pm\}^V$.
    Then for any finite subset $A\subset V$,
    we define $\sigma_A:\Omega\to\{\pm\},\,\sigma\mapsto\prod_{x\in A}\sigma_x$.
    Its expectation $\langle\sigma_A\rangle$ in any probability measure $\langle\blank\rangle$
    on $\Omega$ is called a \emph{correlation function}.
    If $|A|=n$ then $\langle\sigma_A\rangle$ is also called an \emph{$n$-point correlation function}.
\end{definition}

\begin{remark*}[Flip-symmetry]
    The Ising model is \emph{flip-symmetric} in the sense that the distribution of the spins is invariant under the transformation $\sigma\mapsto-\sigma$.
    This is because the Hamiltonian is invariant under this transformation.
\end{remark*}

\begin{exercise}[Flip-symmetry]
    \label{exercise:flip-symmetry}
    Consider the Ising model on a finite graph $G=(V,E)$.
    \begin{itemize}
        \item Prove that if $A\subset V$ contains an odd number of vertices,
        then $\langle\sigma_A\rangle=0$.
        \item Prove that if $A\subset V$ contains an odd number of vertices
        and $x\in V$,
        then \[\E[\sigma_A|\{\sigma_x=+\}]=\langle \sigma_{A}\sigma_x\rangle.\]
        \item Prove that if $A\subset V$ contains an even number of vertices
        and $x\in V$,
        then \[\E[\sigma_A|\{\sigma_x=+\}]=\langle \sigma_{A}\rangle.\]
    \end{itemize}
\end{exercise}

In practice, we are interested in the Ising model on finite portions of the square
lattice $\Z^d$ endowed with nearest-neighbour connectivity.
We now provide the definitions for this setup.

\begin{definition}[Free boundary conditions]
    Let $G=(V,E)$ denote a locally finite graph and $\Lambda\subset V$ a finite set.
    Write $\Lambda^\f$ for the subgraph of $G$ induced by $\Lambda$.
    Write $\langle\blank\rangle_{\Lambda,\beta}^\f:=\langle\blank\rangle_{\Lambda^\f,\beta}$ for the \emph{free-boundary
    Ising model} in $\Lambda$ at inverse temperature $\beta\in[0,\infty)$.    
\end{definition}

\begin{definition}[Fixed boundary conditions]
    Let $G=(V,E)$ denote a locally finite graph and $\Lambda\subset V$ a finite set.
    Let $\partial\Lambda\subset V\setminus\Lambda$ denote the set of vertices adjacent to $\Lambda$.
    Write $\bar\Lambda$ for the graph defined by
    \[
        V(\bar\Lambda):=\Lambda\cup\partial\Lambda;\qquad
        E(\bar\Lambda):=\{\{x,y\}\in E:\{x,y\}\cap\Lambda\neq\emptyset\}.
    \]
    For any $\zeta\in\{\pm\}^{\partial\Lambda}$,
    we shall write $\langle\blank\rangle_{\Lambda,\beta}^\zeta$
    for the measure
    \[
        \langle\blank\rangle_{\Lambda,\beta}^\zeta:=\E_{\bar\Lambda,\beta}[\blank|\{\sigma|_{\partial\Lambda}=\zeta\}].
    \]
    This is called the \emph{fixed-boundary Ising model} with boundary conditions $\zeta$.
    The boundary condition $\zeta\equiv\pm$ is of particular interest,
    and it is denoted $\langle\blank\rangle_{\Lambda,\beta}^\pm$.
\end{definition}

\begin{exercise}[Markov property]
    Consider the Ising model on some finite graph $G=(V,E)$ at inverse temperature $\beta$.
    Fix some $\Lambda\subset V$ and let $(\Lambda_i)_i$ denote the partition of $\Lambda$ into connected components.
    Let $\zeta\in\{\pm\}^{\Lambda^c}$,
    and consider the conditional probability measure
    $\P[\blank|\{\sigma|_{\Lambda^c}=\zeta\}]$.
    \begin{itemize}
        \item Prove that $(\sigma|_{\Lambda_i})_i$ is a family of independent random variables in this measure.
        \item Prove that the law of $\sigma|_{\Lambda_i}$ is $\langle\blank\rangle_{\Lambda_i}^{\zeta|_{\partial\Lambda_i}}$.
    \end{itemize}
    \emph{Hint.}
    Decompose the Hamiltonian according to $H(\sigma)=C+\sum_i H_i(\sigma)$,
    where each $H_i$ is measurable in terms of $\zeta|_{\partial\Lambda_i}$ and $\sigma|_{\Lambda_i}$.
\end{exercise}

Ising proved that in one dimension, the Ising model exhibits exponential decay of correlations at all temperatures.
In other words, there is no phase transition.
We now state his result, without a proof.
While the proof is quite straightforward even with elementary methods,
its proof becomes entirely trivial after the introduction of more recent methods.

\begin{theorem}[Ising, 1924]
    \label{thm:Ising}
    Consider the finite domains $\Lambda_n:=\{-n,\dots,n\}$ of the graph $\Z$.
    Then for any $\beta\in[0,\infty)$,
    there exists a constant $c=c_\beta>0$ such that
    \[
        \langle\sigma_0\rangle_{\Lambda_n,\beta}^+
        \leq \frac1ce^{-c_\beta\cdot n}.
    \]
\end{theorem}

Unfortunately, Ising wrongly conjectured that the same would be true in higher dimension.
Disappointed with this prediction, he left academia.

\section{Early developments. 1936: Peierls' argument}
\label{sec:peierls}

\begin{theorem}[Peierls, 1936]
    \label{thm:peierls}
    The Ising model exhibits phase transition in two dimensions.
\end{theorem}

We shall prove a slight variation of Peierls' original setup,
so that we can fully focus the proof on the core idea.
Let $\T$ denote the triangular lattice graph,
comprised of vertices of the form
\[
    \T:=\left\{n+m e^{\pi i/3}:n,m\in\Z \right\}\subset\C,
\]
and such that each vertex is connected to the six
vertices at distance one.
Let $\Lambda_n\subset\T$ denote the set of vertices at a graph
distance at most $n-1$ from $0\in\T$.
We consider the Ising model on the infinite graph $\T$.
We shall prove the following version of Peierls' result.

\begin{theorem}[Peierls, 1936]
    \label{thm:peierls_triangles}
    Consider the Ising model on the two-dimensional
    triangular lattice graph $\T$.
    For sufficiently large $\beta$,
    we have
    \[
        \inf_{n}\langle\sigma_0\rangle_{\Lambda_n,\beta}^+
        >0.
    \]
\end{theorem}

Let $\H:=\T^*$ denote the hexagonal lattice
that is dual to the triangular lattice.
For a fixed configuration $\sigma\in\Omega$,
we let $\calI(\sigma)\subset E(\H)$ denote the set of
hexagonal lattice edges separating hexagons with different spins.
The set $\calI(\sigma)$ is called the
\emph{interface} between the spins valued $+1$
and those valued $-1$.
\todo{Add figure}
Notice that $\calI(\sigma)$ has a partition into
loops and bi-infinite paths.
If only finitely many spins of $\sigma$ are valued $-1$,
then there are no bi-infinite paths,
and all connected components of $\calI(\sigma)$
are loops.
This happens almost surely when sampling from $\langle\blank\rangle_{\Lambda_n,\beta}^+$.

The core of Peierls' argument is the following lemma.

\begin{lemma}[Exponential decay of loop lengths]
    \label{lem:exp_decay_ising_loops}
    Consider the Ising model on the two-dimensional triangular lattice $\T$
    at inverse temperature $\beta$.
    Suppose that $e^{-2\beta}<\frac12$.
    Then for any hexagonal lattice edge $e\in E(\H)$
    and for any minimal loop length $\ell\in\Z_{\geq 1}$,
    we get
    \[
        \P_{\Lambda_n,\beta}^+
        (\{\text{$\calI(\sigma)$ has a loop of length at least $\ell$ through $e$}\})
        \leq \frac{(2e^{-2\beta})^\ell}{1-2e^{-2\beta}},
    \]
    uniformly in $n$.
\end{lemma}

\begin{proof}
    Fix $\beta$, $e$, and $n$.
    Let $\calL$ denote a loop through $e$,
    and consider the event $\{\calL\subset\calI\}$.
    We claim that
    \[
        \P_{\Lambda_n,\beta}^+
        (\{\calL\subset\calI\})
        \leq e^{-2\beta|\calL|}.
    \]

    To prove the claim,
    we introduce the injective ``loop erasure map''
    \[
        \calE_\calL:\{\calL\subset\calI\}
        \to \Omega\setminus \{\calL\subset\calI\},
    \]
    which is defined such that it flips all the spins inside the loop $\calL$.
    As a consequence, $\calI(\calE_\calL(\sigma))=\calI(\sigma)\setminus\calL$.
    For any $\sigma\in\{\calL\subset\calI\}$, we have
    \[
        \P_{\Lambda_n,\beta}^+
        (\sigma)
        =
        e^{-2\beta|\calL|}
        \cdot
        \P_{\Lambda_n,\beta}^+
        (\calE_\calL(\sigma))
        .
    \]
    We can write down this identity because we know that the loop erasure map
    decreases the Hamiltonian by $2\beta|\calL|$.
    Since $\calE_\calL$ is injective, we get
    \[
        \P_{\Lambda_n,\beta}^+
        (\{\calL\subset\calI\})
        = e^{-2\beta|\calL|}\cdot \P_{\Lambda_n,\beta}^+(\operatorname{Image}(\calE_\calL))
        \leq e^{-2\beta|\calL|},
    \]
    which proves the claim.

    To prove the lemma, observe simply that the number of loops of length $k$
    through $e$ is bounded by $2^k$, so that 
    \begin{align}
        &\P_{\Lambda_n,\beta}^+
        (\{\text{$\calI(\sigma)$ has a loop of length at least $\ell$ through $e$}\})
        \\&\qquad=
        \sum\nolimits_{\text{$\calL$ is a loop of length at least $\ell$ through $e$}}
        \P_{\Lambda_n,\beta}^+
        (\{\calL\subset\calI\})
        \\&\qquad\leq 
        \sum\nolimits_{k\geq \ell} 2^k\cdot e^{-2\beta k}.
    \end{align}
    The final expression is a geometric series converging to the upper bound
    in the lemma.
\end{proof}

\begin{remark}
    In the previous proof,
    the interplay between entropy and energy is quite transparent.
    The entropy in the argument comes from the number of loops of length
    $\ell$, which we upper bounded by $2^\ell$.
    Such a loop contributes a total of $2\beta\ell$ to the Hamiltonian.
    When $2e^{-2\beta}<1$, the energy term dominates,
    forcing the loops to be small.
\end{remark}

\begin{proof}[Proof of Theorem~\ref{thm:peierls_triangles}]
    For a fixed configuration $\sigma\in\Omega$
    sampled from $\P_{\Lambda_n,\beta}^+$,
    we may express $\sigma_0$ as the parity of the number
    of loops in $\calI(\sigma)$ surrounding $0$.
    In particular, if no loop surrounds $0$,
    then $\sigma_0=+1$.
    Thus, for Theorem~\ref{thm:peierls_triangles},
    it suffices to prove that
    \begin{equation}
        \label{eq:peierls_target_equation}
        \P_{\Lambda_n,\beta}^+
        (\{\text{$\calI(\sigma)$ contains a loop surrounding $0$}\})
        <\frac12,
    \end{equation}
    for sufficiently large $\beta$,
    and uniformly in $n$.

    Suppose given some loop $\calL\subset E(\H)$ surrounding $0$.
    Then $\calL$ must intersect the half-line $\R_{\geq 0}\subset\C$.
    More precisely, $\calL$ must contain some edge $e$ whose midpoint
    lies precisely in the set of half-integers $-\frac12+\Z_{\geq 1}$.
    If the endpoint of $e$ is $k-\frac12$, then
    $|\calL|\geq k$, otherwise it cannot surround $0$.
    We are now ready to complete Peierls' argument,
    using exponential decay of the loop lengths (Lemma~\ref{lem:exp_decay_ising_loops}).
    
    Let us perform a union bound over the intersection point,
    in order to obtain
    \begin{align}
        &\P_{\Lambda_n,\beta}^+
        (\{\text{$\calI(\sigma)$ contains a loop surrounding $0$}\})
        \\
        &\qquad\leq
        \sum_{k=1}^\infty
        \P_{\Lambda_n,\beta}^+
        (\{\text{$\calI(\sigma)$ contains a loop surrounding $0$ and hitting $k-\tfrac12$}\})
        \\
        &\qquad\leq
        \sum_{k=1}^\infty
        \frac{(2e^{-2\beta})^k}{1-2e^{-2\beta}}
        =
        \frac{2e^{-2\beta}}{(1-2e^{-2\beta})^2}.
    \end{align}
    This upper bound is independent of $n$
    and tends to $0$ with $\beta\to\infty$,
    thus establishing Equation~\eqref{eq:peierls_target_equation}.
\end{proof}

\begin{exercise}
    \begin{enumerate}
        \item     Consider the Ising model on the two-dimensional square lattice graph $\mathbb{Z}^2$.
        In this case, the interface $\calI(\sigma)$ does not consist of loops, but of even subgraphs
        of the dual lattice.
        How can Peierls' argument be adapted to this case?
        \item  Now consider the $d$-dimensional square lattice for $d\geq 3$.
        What is the structure of the interface in this case?
        Can we adapt Peierls' to prove magnetisation for sufficiently large $\beta$?
    \end{enumerate}
\end{exercise}

\begin{remark}
    Peierls' is robust,
    in the sense that it can be adapted to many other models in statistical mechanics.
\end{remark}

\section{The high-temperature expansion}

The previous section proved the Peierls argument.
An essential ingredient was to view the Ising model in two dimensions
through the \emph{interfaces} of the spins.
Such a transformation of the model may be viewed as a rudimentary version of
an \emph{expansion}.
The interface perspective is sometimes called the \emph{low-temperature expansion}
because it works well in the low-temperature regime (when $\beta$ is large).
There are several useful expansions for the Ising model;
each one of them is adapted to a different setting.
In this section we discuss another expansion: the \emph{high-temperature expansion}.
As the name suggests, this expansion is well-adapted to situations where $\beta$
is small, even though we can also use it to prove useful results in other regimes.
Appendix~\ref{}\todo{Add Appendix and reference} contains an overview of the expansions
discussed in these notes, and may serve as a reference.

For a streamlined presentation, we will henceforth present all our expansions
for the Ising model on \emph{finite graphs without boundary conditions}.
This obviously includes the free boundary conditions.
It is straightforward to see that fixed boundary conditions
also fit into this framework, see Definition~\ref{definition:ghost} and Lemma~\ref{lemma:ghost} below.

Consider the Ising model on a finite graph $G$.
We are typically interested in the correlation functions,
defined via
\[
    \langle\sigma_A\rangle
    =
    \frac{\sum_{\sigma\in\Omega}\sigma_A e^{-H(\sigma)}}{\sum_{\sigma\in\Omega}e^{-H(\sigma)}}
    =
    \frac{\sum_{\sigma\in\Omega}\sigma_A \prod_{xy\in E} e^{\beta\sigma_x\sigma_y} }{\sum_{\sigma\in\Omega}\prod_{xy\in E} e^{\beta\sigma_x\sigma_y} }.
\]
An \emph{expansion} of the Ising model involves rewriting the sum
\[
    \sum_{\sigma\in\Omega}\sigma_A \prod_{xy\in E} e^{\beta\sigma_x\sigma_y}.
\]
A typical expansion comes down to rewriting the exponential, for example:
\begin{itemize}
    \item We may write $e^{\beta\sigma_x\sigma_y}=\cosh \beta + \sigma_x\sigma_y \sinh \beta$,
    \item We may write $e^{\beta\sigma_x\sigma_y}=\sum_{k=0}^\infty (\beta\sigma_x\sigma_y)^k/k!$,
    \item We may write $e^{\beta\sigma_x\sigma_y}=e^{-2\beta} + 2\cdot\true{\sigma_x=\sigma_y}\sinh\beta$.
\end{itemize}
Every expansion comes with its own advantages and disadvantages.
The high-temperature expansion derives from the first identity.

\begin{definition}[High-temperature expansion]
    Consider the Ising model on a finite graph $G=(V,E)$ at inverse temperature $\beta$.
    We consider \emph{percolation configurations} $\omega\in\{0,1\}^E$;
    each $\omega$ is also regarded a (random) set of edges.
    We write $\partial\omega\subset V$ for the set of vertices having
    \emph{odd} degree in the graph $(V,\omega)$.

    The \emph{high-temperature expansion} is the measure $\bfM_{G,\beta}$
    on $\omega\in\{0,1\}^E$ defined by
    \[
        \bfM[\omega]:=\bfM_{G,\beta}[\omega]:=(\cosh\beta)^{|E\setminus\omega|}(\sinh\beta)^{|\omega|}.
    \]
\end{definition}

\begin{theorem}[High-temperature expansion for correlation functions]
    \label{thm:High-temperature expansion for correlation functions}
    Consider the Ising model on a finite graph $G=(V,E)$.
    Then for any $A\subset V$, we have
    \[
        Z\langle\sigma_A\rangle=2^{|V|}\bfM[\{\partial\omega=A\}].
    \]
    In particular, $Z=2^{|V|}\bfM[\{\partial\omega=\emptyset\}]$.
\end{theorem}

\begin{proof}
    We claim that
    \begin{align}
        \label{eq:hight:1}
        Z\langle\sigma_A\rangle
        &=
        \sum_{\sigma\in\Omega}
        \sigma_A
        \prod_{xy\in E}
        e^{\beta\sigma_x\sigma_y}
        \\
        \label{eq:hight:2}
        &=
        \sum_{\sigma\in\Omega}
        \sigma_A
        \prod_{xy\in E}
        (\cosh\beta+\sigma_x\sigma_y\sinh\beta)
        \\
        \label{eq:hight:3}
        &=
        \sum_{\sigma\in\Omega}
        \sigma_A
        \sum_{\omega\in\{0,1\}^E}
        \prod_{xy\in E}
        (\cosh\beta)^{1-\omega_{xy}}(\sigma_x\sigma_y\sinh\beta)^{\omega_{xy}}
        \\
        \label{eq:hight:4}
        &=
        \sum_{\omega\in\{0,1\}^E}
        (\cosh\beta)^{|E\setminus\omega|}(\sinh\beta)^{|\omega|}
        \sum_{\sigma\in\Omega}
        \sigma_A
        \sigma_{\partial\omega}
        \\
        \label{eq:hight:5}
        &=
        \sum_{\omega\in\{0,1\}^E}
        (\cosh\beta)^{|E\setminus\omega|}(\sinh\beta)^{|\omega|}
        2^{|V|}\true{A=\partial\omega}
        \\
        &=
        2^{|V|}\bfM[\{\partial\omega=A\}].
    \end{align}
    Equations~\eqref{eq:hight:1} and~\eqref{eq:hight:2}
    come down to definitions
    and the identity for $e^{\beta\sigma_x\sigma_y}$.
    Swapping the sum and the product yields Equation~\eqref{eq:hight:3}.
    Equation~\eqref{eq:hight:4} is a rearrangement of the terms,
    noting that $\prod_{xy}(\sigma_x\sigma_y)^{\omega_{xy}}=\sigma_{\partial\omega}$.
    Equation~\eqref{eq:hight:5} is obtained by resolving the sum over $\sigma$.
    The final equation is the definition of $\bfM$.
\end{proof}

We can use this theorem to state our first important \emph{correlation inequality}.

\begin{theorem}[First Griffiths inequality]
    Consider the Ising model on a finite graph $G=(V,E)$.
    Then for any $A\subset V$, we have
    \(
        \langle\sigma_A\rangle\geq 0
    \).
\end{theorem}

\begin{proof}
    The previous theorem yields a nonnegative number for $Z\langle\sigma_A\rangle$.
\end{proof}

One advantage of the high temperature expansion is that it yields a straightforward
proof of exponential decay of the correlation functions at high temperature.

\begin{theorem}[Exponential decay at high temperature]
    \label{thm:Exponential decay at high temperature}
    Consider the Ising model with $+$ boundary conditions 
    on the graph $\Z^d$ for $d\in\Z_{\geq 1}$.
    Then for any $\beta\in[0,\infty)$
    such that $(2d-1)\tanh\beta < 1$,
    there exists a constant $c=c_{d,\beta}>0$
    such that
    \[
        \langle\sigma_x\rangle_{\Lambda,\beta}^+
        \leq \tfrac1ce^{-c \operatorname{Distance}(x,\Lambda^c)}
    \]
    for any $x\in\Z^d$ and any domain $\Lambda\subset\Z^d$.

    In particular, in dimension $d=1$, there is exponential decay at all temperatures.
\end{theorem}

We would like to use the high-temperature expansion,
but for this we must first write $\langle\blank\rangle^+_\Lambda$
as an Ising model on a finite graph without boundary condition.

\begin{definition}[Ghost vertex]
    \label{definition:ghost}
    Let $G=(V,E)$ denote a locally finite graph,
    and $\Lambda\subset V$ a finite domain.
    We already defined the graphs $\Lambda^\f$ and $\bar\Lambda$.
    Now define the graph $\Lambda^\frakg$ as follows:
    it is obtained from the graph $\bar\Lambda$
    by replacing all vertices in $\partial\Lambda$ by a single distinguished
    vertex $\frakg$, called the \emph{ghost vertex}.
    Its vertex set is given by
    \(V(\Lambda^\frakg):=\Lambda\cup\{\frakg\}\),
    and there is a natural bijection from $E(\bar\Lambda)$
    to $E(\Lambda^\frakg)$.

    Notice that $\Lambda^\frakg$ is a multigraph when some $x\in\Lambda$
    is connected to multiple vertices in $\partial\Lambda$ in the graph $\bar\Lambda$,
    but this does not really affect our setup.
\end{definition}

It is easy to see that the following lemma holds true.

\begin{lemma}
    \label{lemma:ghost}
    Let $G=(V,E)$ denote a locally finite graph,
    and $\Lambda\subset V$ a finite domain.
    Then the distribution of $\sigma|_{\Lambda}$ is the same in the following
    two measures:
    \[
        \langle\blank\rangle_{\Lambda}^+
        \qquad\text{and}\qquad
        \E_{\Lambda^\frakg}[\blank|\sigma_\frakg=+].
    \]
    Correlation functions can thus be expressed in terms of correlation functions
    on finite graphs via Exercise~\ref{exercise:flip-symmetry}.
\end{lemma}

\begin{proof}[Proof overview of Theorem~\ref{thm:Exponential decay at high temperature}]
    We have
    \[
        \langle\sigma_x\rangle_{\Lambda}^+
        =
        \langle\sigma_x\sigma_\frakg\rangle_{\Lambda^\frakg}
        =
        \frac{
        \bfM_{\Lambda^\frakg}[\{\partial\omega=\{x,\frakg\}\}]
        }{
        \bfM_{\Lambda^\frakg}[\{\partial\omega=\emptyset\}]
        }.
    \]
    If $\partial\omega=\{x,\frakg\}$, then $\omega$ contains a self-avoiding
    walk $\gamma$ from $x$ to $\frakg$.
    A union bound yields
    \[
        \langle\sigma_x\rangle_{\Lambda}^+
        \leq
        \sum_{\gamma}\frac{\bfM_{\Lambda^\frakg}[\{\partial\omega=\{x,\frakg\}\}\cap\{\gamma\subset\omega\}]}{\bfM_{\Lambda^\frakg}[\{\partial\omega=\emptyset\}]}.
    \]
    The proof is now completed after performing the two steps of the Peierls argument:
    \begin{itemize}
        \item One bounds each term by $(\tanh\beta)^{|\gamma|}$,
        \item One bounds the number of walks $\gamma$ of length $n$ from $x$ by $2d(2d-1)^{|\gamma|-1}$.
    \end{itemize} 
    \qedhere
\end{proof}

\begin{exercise}
    Fill in the details of the previous proof overview.
\end{exercise}

Let us summarise what we have proved so far.
\begin{itemize}
    \item Theorem~\ref{thm:Exponential decay at high temperature}
    implies that there is exponential decay of correlations when $\beta$ is sufficiently small.
    \item In dimension $d=1$, Theorem~\ref{thm:Exponential decay at high temperature} also implies Ising's result (Theorem~\ref{thm:Ising}),
    since there was no requirement on $\beta$ when $d=1$.
    \item In dimension $d\geq 2$, we proved that there is \emph{magnetisation} via the
    Peierls argument (Theorem~\ref{thm:peierls}).
    Thus, in dimension $d\geq 2$, there must be a phase transition,
    and we aim to investigate further.
\end{itemize}

The high-temperature expansion is typically used to find upper bounds on correlation
functions.
However, it is also possible to use it to find lower bounds.
Let $G=(V,E)$ denote a finite graph.
For any fixed set $Q\subset E$ of edges,
we define the \emph{XOR map}
\[
    \Xi_Q:\{0,1\}^E\to\{0,1\}^E,\,\omega\mapsto\omega\Delta Q,
\]
where $\Delta$ denotes the symmetric difference of two sets.
This map is an involution.
Moreover, for any $A\subset V$, it restricts to a bijection
\begin{equation}
    \label{eq:XiQ}
\Xi_Q:
\{\partial\omega=A\}\to\{\partial\omega=A\Delta\partial Q\}.
\end{equation}
The measure $\bfM$ is not invariant under the involution $\Xi_Q$,
but it is easy to see how the map affects the measure.
More precisely, for any $\eta\in\{0,1\}^E$, we have
\begin{equation}
    \label{eq:reweighting}
    \bfM[\{\omega = \Xi_Q(\eta)\}]
    =
    (\tanh\beta)^{|Q\setminus\eta|-|Q\cap\eta|}
    \cdot
    \bfM[\{\omega = \eta\}].
\end{equation}
The prefactor is upper bounded by $(\tanh\beta)^{-|Q|}$.
Thus, writing $(\Xi_Q)_*$ for the pushforward map,
we obtain
\[
    (\Xi_Q)_*\bfM\leq (\tanh\beta)^{-|Q|}
    \cdot
    \bfM.
\]
For example, using the bijection in Equation~\eqref{eq:XiQ},
we obtain
\begin{equation}
    \label{eq:M-Xi-Q-high-temperature}
    \bfM[\{\partial\omega=A\}]
    \leq
    (\tanh\beta)^{-|Q|}
    \cdot
    \bfM[\{\partial\omega=A\Delta\partial Q\}].
\end{equation}
We have now proved the following result.

\begin{lemma}
    \label{lemma:high-temperature}
    Consider the Ising model on a finite graph $G=(V,E)$.
    Then for any $A\subset V$ and any $Q\subset E$, we have
    \[
        \langle\sigma_A\sigma_{\partial Q}\rangle
        \geq
        (\tanh\beta)^{|Q|}\cdot\langle\sigma_A\rangle.
    \]
    In particular,
    \[
        \langle\sigma_x\sigma_y\rangle
        \geq
        (\tanh\beta)^{\operatorname{Distance}(x,y)}.
    \]
\end{lemma}

\begin{proof}
    The first inequality is Equation~\eqref{eq:M-Xi-Q-high-temperature}.
    For the second inequality, simply set $A=\emptyset$
    and let $Q$ denote a shortest path from $x$ to $y$.
\end{proof}

This lemma complements Theorem~\ref{thm:Exponential decay at high temperature}
at high temperature, as the lemma asserts that the correlation functions cannot
decay \emph{faster} than exponentially at any finite temperature
(that is, when $\beta>0$).


\section{The high-temperature expansion and switching}

We already used the high-temperature expansion to prove one correlation inequality:
the first Griffiths inequality, which asserts that $\langle\sigma_A\rangle\geq 0$.
This was an immediate consequence of the fact that the high-temperature expansion is a sum of positive terms.

There are many other interested inequalities.
Many of those are obtained via the \emph{switching lemma}.
The switching lemma is traditionally stated for the random-current expansion
(which is a refinement of the high-temperature expansion introduced in the next section),
but we shall first state it in the context of the high-temperature expansion
because the setup is a little bit simpler.
We can already use it to prove two interesting inequalities:
\begin{itemize}
    \item The \emph{pairing bound}, which relates multi-point and two-point correlation functions,
    \item The \emph{Simon--Lieb inequality}, which yields a finite-size criterion for exponential decay.
\end{itemize}

We first prove the following switching lemma.
We state it in its most general form,
namely for overlapping graphs $G$ and $G'$.
In practice, we often care about the special case that $G=G'$, or the slightly more general
case that $G\subset G'$.

\begin{lemma}[Switching lemma for the high-temperature expansion]
    Let $G=(V,E)$ and $G'=(V',E')$ denote two finite graphs and fix $\beta\in[0,\infty)$.
    For $A\subset V$ and $A'\subset V'$,
    write $S_{A,A'}:=\{\partial\omega=A,\,\partial\omega'=A'\}\subset \{0,1\}^E\times\{0,1\}^{E'}$.

    Fix $\eta\subset E\cup E'$
    and $Q\subset \eta\cap E\cap E'$.
    Then for any $A\subset V$ and $A'\subset V'$,
    we get
    \begin{equation}
        \bfM_{G,\beta}\times\bfM_{G',\beta}
        [\{\omega\Delta\omega'=\eta\}\cap S_{A,A'}]
        =
        \bfM_{G,\beta}\times\bfM_{G',\beta}
        [\{\omega\Delta\omega'=\eta\}\cap S_{A\Delta\partial Q,A'\Delta\partial Q}].
    \end{equation}
\end{lemma}

\begin{proof}
    First, assume simply that $G=G'$.
    Write $\Xi_Q^2$ for the map
    \[
        \Xi_Q^2:
        (\{0,1\}^E)^2
        \to
        (\{0,1\}^E)^2
        ,\,(\omega,\omega')\mapsto (\omega\Delta Q,\omega'\Delta Q).
    \]
    We make two important observations:
    \begin{itemize}
        \item $\Xi_Q^2$ restricts to a involution on $\{\omega\Delta\omega'=\eta\}$,
        \item On $\{\omega\Delta\omega'=\eta\}$, the map $\Xi_Q^2$ does not modify the total number $|\omega|+|\omega'|$
        of edges.
    \end{itemize} 
    Since the weight of each configuration $(\omega,\omega')$ is a function of $|\omega|+|\omega'|$,
    the two observations imply that
    the measure \[\bfM_{G,\beta}\times\bfM_{G',\beta}
        [\{\omega\Delta\omega'=\eta\}\cap (\blank)]\] is preserved by the involution
    $\Xi_Q^2$.
    The result follows since $\Xi_Q^2$ is is also a bijection from $S_{A,A'}$
    to $S_{A\Delta\partial Q,A'\Delta\partial Q}$.

    If $G\neq G'$ then we simply view $\Xi_Q^2$
    as an involution on $\{0,1\}^E\times\{0,1\}^{E'}$,
    and the rest of the proof works in the same way.
\end{proof}

To apply this switching lemma,
it is useful to have some simple terminology for graphs and percolations.

\begin{definition}[Percolation events]
    Let $G=(V,E)$ denote a graph and $\omega\subset E$ a percolation configuration.
    Write \[\{u\xleftrightarrow{\omega}v\}\]
    for the event there is an open path from $u$ to $v$
    ($u$ and $v$ may represent vertices or sets of vertices).
    For fixed $A\subset V$, we shall also write $\calE_A$ for the set
    \[
        \{\omega\subset E:\text{$|C\cap A|$ is even for any connected component $C\subset V$ of $(V,\omega)$}\}.
    \]
\end{definition}

\begin{exercise}
    Let $G$ denote a graph and 
    $x,y\in V$  distinct vertices. Prove that:
    \begin{itemize}
        \item If $A=\{x,y\}$, then $\{\omega\in\calE_A\}=\{x\xleftrightarrow{\omega}y\}$,
        \item If $\omega\in\calE_A$, then we may find a finite subset $\eta\subset\omega$ with $\partial\eta=A$,
        \item If $G$ is a finite graph and $\partial\omega=A$, then $\omega\in\calE_A$,
        \item For any $A\subset V$, the event $\{\omega\in\calE_A\}$ is an increasing event of the percolation $\omega$.
    \end{itemize}
\end{exercise}

We can now prove some interesting bounds.
The first bound is the pairing bound.
This bound suggests that multi-correlation functions may be viewed
as interacting paths which pair up the sources in our source set.

\begin{theorem}[Pairing bound]
    Let $G=(V,E)$ denote a finite graph and $\beta\in[0,\infty)$.
    For any $x\in A\subset V$ we have
    \begin{equation}
        \langle\sigma_A\rangle
        \leq
        \sum_{y\in A\setminus\{x\}}
        \langle\sigma_x\sigma_y\rangle
        \langle \sigma_{A\setminus\{x,y\}}\rangle.
    \end{equation}
    In particular, iterating yields
    \[
        \langle\sigma_A\rangle
        \leq
        \sum_{\pi}
        \prod_{xy\in\pi}
        \langle\sigma_x\sigma_y\rangle,
    \]
    where $\pi$ runs over all \emph{pairings} of $A$,
    that is, over all partitions of $A$ into pairs.
\end{theorem}

\begin{proof}
    By the high-temperature expansion,
    we get
    \[
        (2^{-|V|}Z)^2 \langle\sigma_A\rangle
        =
        \bfM^2[\{\partial\omega = A,\,\partial\omega'=\emptyset\}].
    \]
    But on this event we have $\partial(\omega\Delta\omega')=A$,
    which means that $\omega\Delta\omega'$ contains a path
    from $x$ to at least one other vertex in $A$ (see the exercise).
    In other words,
    \[
        \true{\partial\omega = A,\,\partial\omega'=\emptyset}
        \leq
        \sum\nolimits_{y\in A\setminus\{x\},\,\eta\in\{0,1\}^E,\,\{x\xleftrightarrow{\eta}y\}}
        \true{\omega\Delta\omega'=\eta,\,\partial\omega = A,\,\partial\omega'=\emptyset}.
    \]
    We now claim that
    \begin{align}
        &\bfM^2[\{\partial\omega = A,\,\partial\omega'=\emptyset\}]
        \\&\leq 
            \sum\nolimits_{y\in A\setminus\{x\},\,\eta\in\{0,1\}^E,\,\{x\xleftrightarrow{\eta}y\}}
        \bfM^2[
        \{\omega\Delta\omega'=\eta,\,\partial\omega = A,\,\partial\omega'=\emptyset\}
        ]
        \\&=
            \sum\nolimits_{y\in A\setminus\{x\},\,\eta\in\{0,1\}^E,\,\{x\xleftrightarrow{\eta}y\}}
        \bfM^2[
        \{\omega\Delta\omega'=\eta,\,\partial\omega = A\setminus\{x,y\},\,\partial\omega'=\{x,y\}\}
        ].
    \end{align}
    The inequality is the previous inequality,
    the equality is the switching lemma for the high-temperature expansion
    applied to each term $(y,\eta)$,
    where $Q$ is simply some path in $\eta$ from $x$ to $y$.

    The final expression in the claim is equal to
    \[
        \sum\nolimits_{y\in A\setminus\{x\}}
        \bfM^2[
        \{\partial\omega = A\setminus\{x,y\},\,\partial\omega'=\{x,y\}\}
        ]
        =
        (2^{-|V|}Z)^2 \langle\sigma_{A\setminus\{x,y\}}\rangle\langle\sigma_x\sigma_y\rangle,
    \]
    which finishes the proof.
\end{proof}

Next, we focus on Simon's inequality.
We already proved that there is exponential decay at high temperature.
Simon's inequality says that \emph{if} there is exponential decay of correlations,
then it can be detected within a finite volume.

Recall that if $G=(V,E)$ is a graph and $\Lambda\subset V$
a subset, then $\partial\Lambda$ denotes the set of vertices in $V\setminus\Lambda$
which are adjacent to $\Lambda$.
Write $\partial_\circ\Lambda$ for the \emph{interior boundary},
that is, the set of vertices in $\Lambda$ adjacent to $V\setminus\Lambda$.
Write $\partial_e\Lambda$ for the \emph{edge boundary},
that is, the set of edges connecting $\Lambda$ and $V\setminus\Lambda$.

\begin{theorem}[Simon's inequality]
    Let $G=(V,E)$ denote a finite graph,
    $\Lambda\subset V$ some domain,
    and let $\beta\in[0,\infty)$.
    Fix $x\in \Lambda$ and $y\in V\setminus\Lambda$.
    \begin{itemize}
        \item We have
        \[
            \langle\sigma_x\sigma_y\rangle_{G,\beta}
            \leq
            \sum_{z\in\partial_\circ\Lambda}
            \langle\sigma_x\sigma_z\rangle_{\Lambda,\beta}^\f
            \langle\sigma_y\sigma_z\rangle_{G,\beta}.
        \]
        \item We have
        \[
            \langle\sigma_x\sigma_y\rangle_{G,\beta}
            \leq
            (\tanh\beta)
            \sum_{zz'\in\partial_e\Lambda}
            \langle\sigma_x\sigma_z\rangle_{\Lambda,\beta}^\f
            \langle\sigma_y\sigma_{z'}\rangle_{G,\beta}.
        \]
    \end{itemize}
    In fact, the converse inequalities are also true if we 
    multiply the right hand sides by $1/|\partial_\circ\Lambda|$
    and $1/|\partial_e\Lambda|$ respectively.
\end{theorem}

\begin{proof}
    Focus on the first inequality.
    Let $\bfM':=\bfM_{G,\beta}\times\bfM_{\Lambda^\f,\beta}$.
    Expanding the left hand side yields
    \[
        \frac{Z_GZ_{\Lambda^\f}}{2^{|V|}2^{|\Lambda|}}
        \langle\sigma_x\sigma_y\rangle_{G,\beta}
        =
        \bfM'[\{\partial\omega=\{x,y\},\,\partial\omega'=\emptyset\}].
    \]
    But on the event on the right,
    we have $\partial(\omega\Delta\omega')=\{x,y\}$,
    which means that $\omega\Delta\omega'$ contains a path
    which remains in $\Lambda$
    and which connects $x$ to some vertex in $\partial_\circ\Lambda$.
    In other words,
    \begin{multline}
        \label{eq:converse}
        \true{\partial\omega=\{x,y\},\,\partial\omega'=\emptyset}
        \\
        \leq
        \sum\nolimits_{z\in\partial_\circ\Lambda,\,\eta\in\{0,1\}^E,\,x\xleftrightarrow{\eta\cap E(\Lambda^\f)}z}
        \true{\omega\Delta\omega'=\eta,\,\partial\omega=\{x,y\},\,\partial\omega'=\emptyset}.
    \end{multline}
    Using the switching lemma like for the pairing bound,
    we obtain
    \begin{align}
        &\bfM'[\{\partial\omega=\{x,y\},\,\partial\omega'=\emptyset\}]
        \\&
        \leq 
        \sum\nolimits_{z\in\partial_\circ\Lambda,\,\eta\in\{0,1\}^E,\,x\xleftrightarrow{\eta\cap E(\Lambda^\f)}z}
        \bfM'[
        \{\omega\Delta\omega'=\eta,\,\partial\omega=\{x,y\},\,\partial\omega'=\emptyset\}]
        \\&
        =
        \sum\nolimits_{z\in\partial_\circ\Lambda,\,\eta\in\{0,1\}^E,\,x\xleftrightarrow{\eta\cap E(\Lambda^\f)}z}
        \bfM'[
        \{\omega\Delta\omega'=\eta,\,\partial\omega=\{y,z\},\,\partial\omega'=\{x,z\}\}]
        \\&
        =
        \sum\nolimits_{z\in\partial_\circ\Lambda}
        \bfM'[
        \{\partial\omega=\{y,z\},\,\partial\omega'=\{x,z\}\}]
        \\&=
        \frac{Z_GZ_{\Lambda^\f}}{2^{|V|}2^{|\Lambda|}}
        \sum\nolimits_{z\in\partial_\circ\Lambda}
        \langle\sigma_y\sigma_z\rangle_{G,\beta}
        \langle\sigma_x\sigma_z\rangle_{\Lambda^\f,\beta}.
    \end{align}
    We use the switching lemma for the first equality;
    we choose $Q$ to be a path from $x$ to $z$ in $\eta\cap E(\Lambda^\f)$
    to get an equality for each term.
    This proves the first inequality in the statement of the theorem.

    The converse inequality is true simply because we may reverse the inequality in
    Equation~\eqref{eq:converse} if we multiply the right hand side by $1/|\partial_\circ\Lambda|$.

    For the second inequality we only give a proof outline.
    It is obtained in a similar fashion,
    noticing that if $\omega\Delta\omega'$ connects
    $x$ and $y$,
    then there must be some edge $zz'\in\partial_e\Lambda$
    such that $\omega\Delta\omega'$ contains a self-avoiding path
    from $x$ to $y$, which passes through $zz'$ 
    and which does not leave $\Lambda$ before using this edge.
    The switching lemma may then be applied in a similar fashion.
    By switching the edge $zz'$, which only appears in the bigger
    graph,
    we make the extra factor $\tanh\beta$ appear.
\end{proof}

\begin{corollary}[Finite size criterion]
    Consider the Ising model on $\Z^d$ at inverse temperature $\beta$.
    For any finite $\Lambda\subset\Z^d$ containing $0\in\Z^d$,
    we define
    \[
        \phi_\beta(\Lambda):=
        (\tanh\beta)
        \sum_{zz'\in\partial_e\Lambda}
            \langle\sigma_0\sigma_z\rangle_{\Lambda,\beta}^\f.
    \]
    If $\phi_\beta(\Lambda)<1$ for some $\Lambda$, then
    there exists a constant $c=c_{d,\beta}>0$
    such that
    \[
        \langle\sigma_x\rangle_{\Delta,\beta}^+
        \leq \tfrac1ce^{-c \operatorname{Distance}(x,\Delta^c)}
    \]
    for any $x\in\Z^d$ and any domain $\Delta\subset\Z^d$.
\end{corollary}

\begin{proof}
    Fix $\Lambda$ with $\phi_\beta(\Lambda)<1$,
    and fix $\Delta$.
    For any $x\in\Z^d$, define
    \[
        a(x):=\left\lfloor
        \frac{
            \operatorname{Distance}(x,\Delta^c)
        }{
            \operatorname{Diameter}(\Lambda)+1
        }
    \right\rfloor.
    \]
    It suffices to prove that for any $x\in\Z^d$ and $n\in\Z_{\geq 0}$, we have
    \[
        a(x)\geq n
        \qquad\implies\qquad
        \langle\sigma_x\rangle_{\Delta,\beta}^+ \leq \phi_\beta(\Lambda)^n.
    \]
    The induction basis $n=0$ is obvious; we limit this proof to the induction step.

    Using Simon's inequality, we get
    \[
        \langle\sigma_x\rangle_{\Delta,\beta}^+
        =
        \langle\sigma_x\sigma_\frakg\rangle_{\Delta^\frakg,\beta}
        \leq 
        (\tanh\beta)
            \sum_{zz'\in\partial_e(\Lambda+x)}
            \langle\sigma_x\sigma_z\rangle_{\Lambda+x,\beta}^\f
            \langle\sigma_\frakg\sigma_{z'}\rangle_{\Delta^\frakg,\beta}.
    \]
    By induction, we may bound $\langle\sigma_\frakg\sigma_{z'}\rangle_{\Delta^\frakg,\beta}=\langle\sigma_{z'}\rangle_{\Delta,\beta}^+\leq \phi_\beta(\Lambda)^{a(x)-1}$,
    so that the previous line is upper bounded by
    \[
        \langle\sigma_x\rangle_{\Delta,\beta}^+
        \leq
        \left(\textstyle
        (\tanh\beta)
            \sum_{zz'\in\partial_e(\Lambda+x)}
            \langle\sigma_x\sigma_z\rangle_{\Lambda+x,\beta}^\f
            \right)
            \phi_\beta(\Lambda)^{a(x)-1}
            =
            \phi_\beta(\Lambda)^{a(x)}.
    \]
    This is the desired bound.
\end{proof}

\begin{remark}
    At this point, we have proved that
    \[
        \inf_\Lambda \phi_\beta(\Lambda) < 1
        \qquad
        \implies
        \qquad
        \text{exponential decay of correlations},
    \]
    but not the converse implication.

    It is easy to see that the converse implication is also true.
    Indeed, the converse version of the Simon inequality yields
    \[
        \langle \sigma_0\rangle_{\Lambda,\beta}^+
        \geq \frac{\tanh\beta}{|\partial_e\Lambda|} 
        \sum_{zz'\in\partial_e\Lambda}
            \langle\sigma_0\sigma_z\rangle_{\Lambda,\beta}^\f
            =\frac{\phi_\beta(S)}{|\partial_e\Lambda|}.
    \]
    If the left hand side decays exponentially fast in the distance from $0$ to the boundary,
    then one may clearly choose $\Lambda$ so large that 
    $\langle \sigma_0\rangle_{\Lambda,\beta}^+\leq 1/|\partial_e\Lambda|$, in which case $\phi_\beta(\Lambda)<1$.
\end{remark}


\section{The random-currents expansion}

Next, we introduce random currents.
Random currents are a refinement of the high-temperature expansion.
Let us make precise what we mean, before turning to the details.
A significant drawback of Lemma~\ref{lemma:switching_high_t}
is the fact that we can switch subsets of the \emph{symmetric difference}
$\omega\Delta\omega'$, but not of the \emph{union} $\omega\cup\omega'$.
Random currents carry slightly more information than high-temperature expansion,
which enables this ``upgrade''.
Switching over the union of the two percolations is important for several
new correlation inequalities, and eventually the proof of continuity of the phase transition.

\begin{definition}[Currents]
    Let $G=(V,E)$ denote a graph.
    A \emph{current} is a map $\n:E\to\Z_{\geq 0}$.
    We think of $(V,\n)$ as a multigraph,
    where for each edge $uv\in E$ we have $\n_{uv}$ multi-edges between $u$ and $v$.
    The set of \emph{sources} $\partial\n\subset V$ of a current
    $\n$ is defined as the set of vertices $u\in V$ with an odd degree in the multigraph.
    We let $\hat\n:=(\n\wedge 1)\in\{0,1\}^E$ denote the associated percolation,
    which simply contains the edges carrying at least one current.

    If $G$ is finite and $\beta\in[0,\infty)$, then the
    \emph{weight} of a current is defined as
    \[
        w(\n):=w_{G,\beta}(\n):=\prod_{xy\in E}
        \frac{\beta^{\n_{xy}}}{\n_{xy}!}.
    \]
    The \emph{random-currents measure} is the measure $\M_{G,\beta}$ on $(\Z_{\geq 0})^E$
    defined by
    \[
        \M[\n]:=\M_{G,\beta}[\n]:=w_{G,\beta}(\n).
    \]
\end{definition}

\begin{remark}
    Notice that $e^{-\beta|E|}\M_{G,\beta}$ is a probability measure
    in which $(\n_{xy})_{xy\in E}$ is a family of independent
    random variables with distribution $\operatorname{Poisson}(\beta)$.
\end{remark}

The random-currents measure is a richer object than the high-temperature expansion.
To see this, consider a single fixed edge $xy\in E$.
Then we have the following correspondence between the high-temperature weights
and the random-currents weights:
\[
    \cosh\beta = \sum_{\n\in 2\Z_{\geq 0}} \frac{\beta^\n}{\n!}
    ;
    \qquad
    \sinh\beta = \sum_{\n\in 2\Z_{\geq 0}+1} \frac{\beta^\n}{\n!}.
\]
We may thus interpret the relation between $e^{-\beta|E|}\bfM$ and $e^{-\beta|E|}\M$ as follows:
\begin{itemize}
    \item $\M$ is a nonnormalised family of independent $\operatorname{Poisson}(\beta)$-variables $(\n_{xy})_{xy\in E}$,
    \item $\bfM$ is obtained from $\M$ by writing $\omega_{xy}$ for the \emph{parity} of $\n_{xy}$,
    \item In particular, $\partial\n\sim\M$ and $\partial\omega\sim\bfM$ have the same distribution,
    \item Moreover, the percolation $\hat\n$ may be viewed as
    \[
        \hat\n = \omega \cup\{xy\in E: \n_{xy}\in\{2,4,8,\ldots\}\}.
    \]
    In particular, the following distributions are the same:
    \begin{equation}
        \label{eq:equivalence}
        \text{$\hat\n$ in the measure $\M$}
        \qquad
        \text{and}
        \qquad
        \text{$\omega\cup\eta$ in the measure $\bfM\times\P_{p}$,}
    \end{equation}
    where $\eta\sim\P_p$ is an independent bond percolation on $G$
    with parameter
    \[
        p
        =\frac{\cosh\beta-1}{\cosh\beta}
        =\frac{
            \sum_{\n\in 2\Z_{\geq 1}} \frac{\beta^\n}{\n!}
        }{
            \sum_{\n\in 2\Z_{\geq 0}} \frac{\beta^\n}{\n!}
        }
        .
    \]
    This is called \emph{sprinkling};
    $\hat\n$ is a \emph{sprinkled} version of $\omega$.
\end{itemize}

The last observation arises from the simple fact that edges carring an even current
still have a probability $p$ of carrying a strictly positive even current.

Theorem~\ref{thm:High-temperature expansion for correlation functions}
translates to random currents as follows.

\begin{theorem}[Current representation of correlation functions]
    \label{thm:current_representation_of_correlation_functions}
    Consider the Ising model on a finite graph $G$ at inverse temperature $\beta$.
    Let $A\subset V$ be a subset of vertices.
    Then
    \[
        Z\langle\sigma_A\rangle
        =2^{|V|}\sum_{\n:\:\partial\n=A}
        w(\n)
        =
        2^{|V|}\M[\{\partial\n=A\}].
    \]
    In particular, the partition function is given by
    \[
        Z=2^{|V|}\sum_{\n:\:\partial\n=\emptyset}
        w(\n)
        =
        2^{|V|}\M[\{\partial\n=\emptyset\}].
    \]
\end{theorem}

We will now explain how random currents allow us to ``upgrade'' from switching
over subsets of $\omega\Delta\omega'$ to subsets of $\hat\n\cup\hat\m$.

\begin{exercise}[Poisson switching]
    Suppose that we record cars traversing a bridge on a road.
    Blue cars pass according to a Poisson point process with rate $1$ (per second).
    Let $X_B$ denote the number of blue cars that pass after recording $\beta$ seconds.
    Can we easily prove, without a calculation, that
    \[
        \P[\{\text{$X_B$ is even}\}]\geq \P[\{\text{$X_B$ is odd}\}]?
    \]

    Suppose that there are also yellow cars,
    which arrive according to an independent Poisson process with the same rate.
    Let $X_Y$ denote the number of yellow cars that passed.

    By considering the colour of the \emph{last} car that passed the bridge, prove that:
    \begin{enumerate}
        \item $\P[\{\text{$X_B$ is even}\}|\{X_B+X_Y= N \}]=\frac12$ whenever $N>0$,
        \item $\P[\{\text{$X_B$ is even}\}|\{X_B+X_Y= N \}]=0$ whenever $N=0$,
        \item $\P[\{\text{$X_B$ is even}\}]-\P[\{\text{$X_B$ is odd}\}]=\P[\{\text{$X_B+X_Y=0$}\}]\geq 0$.
    \end{enumerate}
\end{exercise}

\begin{lemma}[Switching lemma]
    \label{lem:explicit_switching_lemma}
    Let $G$ and $G'$ denote two finite graphs and fix $\beta\in[0,\infty)$.
    If $A\subset V$, $\s\in(\Z_{\geq 0})^{E\cup E'}$, and $Q\subset \hat\s\cap E\cap E'$,
    then 
    \begin{equation}
        \M_{G,\beta}\times\M_{G',\beta}
        [\{\n+\m=\s\}\cap \{\partial\n=A\}]
        =
        \M_{G,\beta}\times\M_{G',\beta}
        [\{\n+\m=\s\}\cap \{\partial\n= A\Delta\partial Q\}].
    \end{equation}
    Notice also that if $\n+\m=\s$, then $\partial\s=(\partial\n)\Delta(\partial\m)$.
\end{lemma}

\begin{proof}
    By induction, we may simply suppose that $|Q|=1$, say $Q=\{xy\}$.
    Introduce the probability measure
    \[
        \P:\propto  \M_{G,\beta}\times\M_{G',\beta}
        [\{\n+\m=\s,\,\n|_{Q^c}=\a,\,\m|_{Q^c}=\b\}\cap(\blank)];
    \]
    for the claimed identity it suffices to derive, for fixed $\a$ and $\b$, the stronger equality
    \[
        \P[\{\partial\n=A\}]=\P[\{\partial\n=A\Delta\partial Q\}].
    \]
    But the only randomness that is left in the measure $\P$,
    are the values of $\n_{xy}$ and $\m_{xy}$,
    which are independent Poisson random variables conditioned to sum to $\s_{xy}>0$.
    By the previous exercise, the parity of $\n_{xy}$ has the distribution of a fair coin flip,
    which proves the previous identity.
\end{proof}


% \begin{lemma}[Switching lemma]
%     \label{lem:switching_lemma}
%     Let $G$ denote a finite graph and $\beta\in[0,\infty)$.
%     Consider the measurable pair $(\n,\m)\sim \M^2=\M_{G,\beta}^2$.
%     Then for any $A,B,S\subset V$ and for any bounded function $F:(\Z_{\geq 0})^E\to\C$,
%     we have
%     \begin{align}
%         &\M^2[F(\n+\m)\true{\widehat{\n+\m} \in \calE_S}\true{\partial\n=A}\true{\partial\m=B}]
%         \\={}&
%         \M^2[F(\n+\m)\true{\widehat{\n+\m} \in \calE_S}\true{\partial\n=A\Delta S}\true{\partial\m=B\Delta S}].
%     \end{align}
% \end{lemma}

% \begin{proof}
%     By linearity of integration, it suffices to consider the case that $F(\n+\m):=\true{\n+\m=\s}$
%     for some fixed current $\s$ with $\hat\s\in\calE_S$.
%     By the previous exercise, we may find some $\eta\subset\hat\s$
%     such that $\partial\eta=S$.
%     The desired equality
%     \begin{align}
%     &
%     \M^2[\true{\n+\m =\s}\true{\partial\n=A}\true{\partial\m=B}]
%     \\={}&
%     \M^2[\true{\n+\m =\s}\true{\partial\n=A\Delta S}\true{\partial\m=B\Delta S}]
%     \end{align}
%     then follows by the explicit switching lemma.
% \end{proof}

% In practice, we do not care so much about the function $F$, and simply set it to $F\equiv 1$.

An important corollary of the switching lemma is the \emph{second Griffiths inequality}.

\begin{lemma}[Second Griffiths inequality]
    Consider the Ising model on a finite graph $G$ at inverse temperature $\beta$.
    Then for any $A,B\subset V$, we have
    $\langle\sigma_{A}\sigma_B\rangle-\langle\sigma_A\rangle\langle\sigma_B\rangle\geq 0$.
\end{lemma}

The second Griffiths inequality is more subtle than the first,
as it bounds a \emph{difference} of correlation functions.
This is typical for the switching lemma.

\begin{proof}
    Claim that
    \begin{align}
        &\M^2[\{\partial\n=A\}\cap\{\partial\m=B\}]
        \\
        &\qquad=
        \M^2[\{\widehat{\n+\m} \in \calE_B\}\cap\{\partial\n=A\}\cap\{\partial\m=B\}]
        \\
        &\qquad\stackrel{\text{switch}}=
        \M^2[\{\widehat{\n+\m} \in \calE_B\}\cap\{\partial\n=A\Delta B\}\cap\{\partial\m=\emptyset\}]
        \\
        &\qquad\leq
        \M^2[\{\partial\n=A\Delta B\}\cap\{\partial\m=\emptyset\}].
    \end{align}
    For the first equality, we simply observe that $\{\partial\m=B\}\subset \{\widehat{\n+\m} \in \calE_B\}$
    (see Exercice~\ref{exo:basic_perco}).
    The switch is the switching lemma, where we switch over a subset of $\widehat{\n+\m}$ 
    having $B$ as source set.
    The inequality is inclusion of events.

    By the random currents expansion of correlation functions (Theorem~\ref{thm:current_representation_of_correlation_functions}),
    the left- and rightmost expressions are given by
    \[
        Z^2\langle\sigma_A\rangle\langle\sigma_B\rangle/4^{|V|}
        \leq
        Z^2\langle\sigma_A\sigma_B\rangle\langle\sigma_\emptyset\rangle/4^{|V|}.
    \]
    Since $\langle\sigma_\emptyset\rangle=1$, this is the desired inequality.
\end{proof}

\begin{exercise}[The two-point function as a metric]
    Consider the Ising model on a finite graph $G$
    at inverse temperature $\beta>0$.
    Prove that $V\times V\to [0,\infty],\,
    (u,v)\mapsto-\log\langle\sigma_u\sigma_v\rangle_{G,\beta}$
    defines a metric on $V$.
    Use directly the switching lemma, and not the second Griffiths inequality.
    What does the percolation event $\calE_S$ look like in this case?
\end{exercise}

% \begin{definition}[Probability measures on currents]
%     Consider the Ising model on a finite graph $G$
%     at inverse temperature $\beta$.
%     For any $A\subset V$,
%     define the probability measure $\P^A_{G,\beta}$ by
%     \[
%         \P^A:=\P^A_{G,\beta}:=\frac{2^{|V|}}{Z_{G,\beta}\langle\sigma_A\rangle_{G,\beta}}\M_{G,\beta}[\true{\partial\n=A}(\blank)].
%     \]
%     For any $A_1,\ldots,A_n$,
%     write $\P^{A_1,\ldots,A_n}:=\P^{A_1}\times\cdots\times\P^{A_n}$.
% \end{definition}

\begin{exercise}[Correlation functions in terms of sourceless currents]
    Consider the Ising model on a finite graph $G$
    at inverse temperature $\beta$.
    Prove that for any $A\subset V$,
    \[
        Z^2\langle\sigma_A\rangle^2/4^{|V|}
        =
        \M^2[\{\partial\n=\partial\m=\emptyset\}\cap\{\widehat{\n+\m}\in\calE_A\}].
    \]

    Observe that we can now express all correlation functions in terms of a single
    fixed probability measure $4^{|V|}\M^2[\{\partial\n=\partial\m=\emptyset\}\cap(\blank)]/Z^2$ on sourceless random currents.
\end{exercise}

\section{Monotonicity via the second Griffiths inequality}

\begin{theorem}[Monotonicity in the temperature]
    Let $G$ denote a finite graph and $A\subset V$ a finite set.
    Then the function $\beta\mapsto\langle\sigma_A\rangle_{G,\beta}$
    is non-decreasing.
\end{theorem}

\begin{proof}
    We want to prove that
    \[
        \frac{\partial}{\partial\beta}
        \langle\sigma_A\rangle_{G,\beta}
        =
        \frac{\partial}{\partial\beta}
        \left(
            \frac{
                \sum_\sigma\sigma_A\prod_{uv}e^{\beta\sigma_u\sigma_v}
            }{
                \sum_\sigma\prod_{uv}e^{\beta\sigma_u\sigma_v}
            }
        \right)
        \geq 0.
    \]
    Since we are differentiating a fraction,
    it suffices to show that the numerator grows at a faster rate
    than the denominator,
    that is,
    \[
        \frac{\frac{\partial}{\partial\beta}
            \sum_\sigma\sigma_A\prod_{uv}e^{\beta\sigma_u\sigma_v}
        }{
            Z\langle\sigma_A\rangle
        }
        \geq
        \frac{\frac{\partial}{\partial\beta}
            \sum_\sigma\prod_{uv}e^{\beta\sigma_u\sigma_v}
        }{
            Z
        }.
    \]
    We perform the differential and then multiply each side by $\langle\sigma_A\rangle$,
    to see that this inequality is equivalent to
    \[
       \sum_{xy}
        \frac{\sum_\sigma
            \sigma_x\sigma_y\sigma_A\prod_{uv}e^{\beta\sigma_u\sigma_v}
        }{Z}
        \geq
        \langle\sigma_A\rangle
        \sum_{xy}
        \frac{
            \sum_\sigma
            \sigma_x\sigma_y
            \prod_{uv}e^{\beta\sigma_u\sigma_v}
        }{Z}.
    \]
    Each fraction may now be reinterpreted as a correlation function,
    so that the previous inequality is equivalent to
    \[
        \sum_{xy}\langle\sigma_x\sigma_y\sigma_A\rangle
        \geq
        \langle\sigma_A\rangle
        \sum_{xy}\langle\sigma_x\sigma_y\rangle.
    \]
    But this is just the second Griffiths inequality.
\end{proof}

\begin{exercise}[Regularity properties of the correlation functions in $\beta$]
    Prove that the function $[0,\infty)\to\R,\,\beta\mapsto\langle\sigma_A\rangle_{G,\beta}$
    in the above context is an analytic function.
\end{exercise}

Next, we want to prove monotonicity in domains.
We first challenge the reader to prove the following exercise.


\begin{exercise}[Conditioning on equality increases the correlation functions]
    \label{exo:conditioning_equality}
    Consider the Ising model on a finite graph $G$
    at inverse temperature $\beta$, and fix some subset $A\subset V$.
    \begin{itemize}
        \item     Prove that for any two distinct vertices $u,v\in V$,
        we have
        \[
            \E_{G,\beta}[\sigma_A|\{\sigma_u=\sigma_v\}]
            \geq
            \E_{G,\beta}[\sigma_A]=\langle\sigma_A\rangle_{G,\beta}.
        \]
        \item Prove for any $X\subset Y\subset V$, we have
        \[
            \E_{G,\beta}[\sigma_A|\{\text{$\sigma$ is constant on $X$}\}]
            \leq
            \E_{G,\beta}[\sigma_A|\{\text{$\sigma$ is constant on $Y$}\}]
            .
        \]
    \end{itemize}
\end{exercise}

\begin{lemma}[Monotonicity in domains]
    \label{lemma:correlation_functions_monotone_both}
    Consider the Ising model on a locally finite graph
    $G=(V,E)$ at inverse temperature $\beta$.
    Consider two finite domains $\Lambda\subset\Lambda'\subset V$
    and a subset $A\subset \Lambda$.
    \begin{itemize}
        \item \textbf{Free boundary.}
        We have $\langle\sigma_A\rangle^\f_{\Lambda,\beta}\leq\langle\sigma_A\rangle^\f_{\Lambda',\beta}$.
        \item \textbf{Wired boundary.}
        We have $\langle\sigma_A\rangle^+_{\Lambda,\beta}\geq\langle\sigma_A\rangle^+_{\Lambda',\beta}$.
    \end{itemize}
\end{lemma}

\begin{proof}[Proof for $\langle\blank\rangle^\f$]
    We first prove the following claim:
    if $G'$ and $G''$ are finite graphs on the same vertex set,
    and such that $E(G'')=E(G')\cup\{xy\}$,
    then
    \[
        \langle\sigma_A\rangle_{G',\beta}
        \leq
        \langle\sigma_A\rangle_{G'',\beta}
    \]
    for any $A\subset V(G')$.
    To prove the claim, we simply expand 
    \[
        \langle\sigma_A\rangle_{G'',\beta}
        =
        \frac{
            \langle   e^{\beta\sigma_x\sigma_y}\sigma_A\rangle_{G'}
        }{
            \langle e^{\beta\sigma_x\sigma_y}\rangle_{G'}
        }.
    \]
    Thus, we want to show that
    \[
            \langle e^{\beta\sigma_x\sigma_y} \sigma_A \rangle_{G'}
            \geq 
            \langle e^{\beta\sigma_x\sigma_y}\rangle_{G'}
            \langle\sigma_A\rangle_{G'}.
    \]
    This follows from the second Griffiths inequality.
    We have now proved the claim.

    Recall the definition of the finite graph $\Lambda^\f$.
    Let $\tilde\Lambda^\f:=((\Lambda')^\f,E(\Lambda^\f))$;
    this is just the graph $\Lambda^\f$
    supplemented with some isolated vertices $\Lambda'\setminus\Lambda$.
    The law of $\sigma$ in $\langle\blank\rangle_{\tilde\Lambda^\f}$
    is just given by $\langle\blank\rangle_{\Lambda^\f}$,
    with independent fair coin flips for the isolated vertices in $\Lambda'\setminus\Lambda$.
    Thus, it suffices to prove that
    \[
        \langle\sigma_A\rangle^\f_{\Lambda}
        =
        \langle\sigma_A\rangle_{\tilde\Lambda^\f}
        \leq
        \langle\sigma_A\rangle_{(\Lambda')^\f}
        =
        \langle\sigma_A\rangle^\f_{\Lambda'}
        .
    \]
    This follows from the claim.
\end{proof}


\begin{proof}[Proof for $\langle\blank\rangle^+$]
    Without loss of generality,
    $\Lambda'\setminus\Lambda=\{u\}$ for some
    vertex $u\in V$.
    We make all calculations in the graph $(\Lambda')^\frakg$ with the ghost vertex:
    we get
    \[
        \langle\sigma_A\rangle_{\Lambda'}^+
        =
        \E_{(\Lambda')^\frakg}[\sigma_A|\{\sigma_\frakg=+\}];
        \qquad
        \langle\sigma_A\rangle_{\Lambda}^+
        =
        \E_{(\Lambda')^\frakg}[\sigma_A|\{\sigma_\frakg=+\}\cap\{\sigma_u=\sigma_\frakg\}].
    \]

    Assume first that $|A|$ is even for now.
    Then
    \begin{equation}
        \langle\sigma_A\rangle_{\Lambda}^+
        =
        \E_{(\Lambda')^\frakg}[\sigma_A|\{\sigma_u=\sigma_\frakg\}]
        \geq
        \E_{(\Lambda')^\frakg}[\sigma_A]
        =
        \langle\sigma_A\rangle_{\Lambda'}^+,
    \end{equation}
    due to Exercise~\ref{exo:conditioning_equality}.

    If $|A|$ is odd, we just need to replace the set $A$
    by $A':=A\cup\{\frakg\}$.
    More precisely,
    \begin{equation}
        \langle\sigma_A\rangle_{\Lambda}^+
        =
        \E_{(\Lambda')^\frakg}[\sigma_{A'}|\{\sigma_u=\sigma_\frakg\}]
        \geq
        \E_{(\Lambda')^\frakg}[\sigma_{A'}]
        =
        \langle\sigma_A\rangle_{\Lambda'}^+,
    \end{equation}
    where the inequality uses the same exercise.

    Those are the desired inequalities.
\end{proof}

\begin{definition}[Infinite-volume limit]
    Let $G=(V,E)$ denote a locally finite graph.
    Write
    \[
        \lim_{\Lambda\uparrow V}f(\Lambda)
        \qquad\text{for}\qquad
        \lim_{n\to\infty}f(\Lambda_n),
    \]
where $(\Lambda_n)_n$ is any increasing sequence of finite domains 
with $\cup_n\Lambda_n=V$.
This notation makes sense only when the limit is independent
of the precise choice of the sequence $(\Lambda_n)_n$,
and is called the \emph{thermodynamical limit} or \emph{infinite-volume limit}.

Let $(\Omega,\calF)$ denote the measurable space $\Omega:=\{\pm1\}^V$
endowed with the product $\sigma$-algebra.
For a domain $\Lambda$, we write $\calF_\Lambda$
for the $\sigma$-algebra generated by spins in $\Lambda$.
An observable $X:\Omega\to\C$ is called \emph{local} if it is measurable
with respect to $\calF_\Lambda$ for some domain $\Lambda$.

Let $\calP(\Omega,\calF)$ denote the set of all probability measures
on this measurable space.
We endow this set with the \emph{local convergence topology},
which is defined as the topology making the map
\[
    \calP(\Omega,\calF)\to\C,\,\mu\mapsto\mu[X]
\]
continuous for any local observable $X$.
\end{definition}

\begin{remark}
    This topology is sometimes known under different names in the literature
    (such as the \emph{weak topology}).
    I like the name \emph{local convergence topology} because it captures the essence quite literally:
    if the statistics of the measures within a fixed domain $\Lambda$ converge,
    then we have local convergence.
\end{remark}

\begin{exercise}
    Prove that $\calP(\Omega,\calF)$ is a compact space in this topology.
\end{exercise}

\begin{theorem}[Existence of the thermodynamical limit]
    Consider the Ising model on a locally finite graph $G$
    at inverse temperature $\beta$.
    Then there exists unique probability measures
    $\langle\blank\rangle_{G,\beta}^\f,\langle\blank\rangle^+_{G,\beta}\in\calP(\Omega,\calF)$
    such that
    \[
        \lim_{\Lambda\uparrow V}\langle X\rangle_{\Lambda,\beta}^*=\langle X\rangle_{G,\beta}^*
    \]
    for $*\in\{\f,+\}$ and
    for any local observable $X:\Omega\to\R$.
    In other words,
    \[
        \lim_{\Lambda\uparrow V}\langle \blank\rangle_{\Lambda,\beta}^*
        =:
        \langle \blank\rangle_{G,\beta}^*.
    \]
    The measures $\langle\blank\rangle_{G,\beta}^*$ are called
    the \emph{thermodynamical limits} or \emph{infinite-volume limits}.
\end{theorem}

\begin{proof}
    Any local observable may be written as a finite linear conbination
    of observables of the form $\sigma_A$ where $A$ is a finite subset of
    $V$.
    The theorem then follows by compactness and Lemma~\ref{lemma:correlation_functions_monotone}.
\end{proof}

\begin{exercise}[Continuity properties in $\beta$]
    Consider the Ising model on a locally finite graph $G=(V,E)$.
    Fix $A\subset V$ finite.
    \begin{itemize}
        \item The function $\beta\mapsto \langle\sigma_A\rangle_{G,\beta}^*$ is non-decreasing for $*\in\{\f,+\}$.
        \item The function $\beta\mapsto \langle\sigma_A\rangle_{G,\beta}^\f$ is left continuous.
        \item The function $\beta\mapsto \langle\sigma_A\rangle_{G,\beta}^+$ is right continuous.
    \end{itemize}
    \emph{Hint.} Argue that $\beta\mapsto \langle\sigma_A\rangle_{G,\beta}^\f$
    is a limit of a non-decreasing sequence of non-decreasing functions.
\end{exercise}

\begin{definition}[Magnetisation and critical temperature]
    Let $G$ be a vertex-transitive locally finite graph and $u$ some distinguished reference vertex.
    The non-decreasing right-continuous function
    \[
        m=m_G:[0,\infty)\to\R,\,\beta\mapsto\langle\sigma_u\rangle_{G,\beta}^+
    \]
    is called the \emph{magnetisation}.

    The \emph{critical (inverse) temperature} is defined via
    \[
        \beta_c:=\beta_c(G):=\inf\{\beta\in[0,\infty):m(\beta)>0\}.
    \]
\end{definition}

We have already proved that $\beta_c\in(0,\infty)$ for $G=\Z^d$
in dimension $d\geq 2$,
and that $\beta_c=\infty$ for $G=\Z$.

\begin{definition}[Shift-invariance]
    Let $G=\Z^d$.
    Consider a measure $\langle\blank\rangle\in\calP(\Omega,\calF)$.
    \begin{itemize}
        \item For any $u\in\Z^d$, we define the \emph{shift operator} $\tau_u:\Omega\to\Omega$ by
        \[
            (\tau_u\sigma)_x = \sigma_{x-u}.
        \]
        An event $A$ is \emph{shift-invariant} if $\tau_uA:=\{\tau_u\sigma:\sigma\in A\}$ for any $u\in\Z^d$.
        \item The measure is called \emph{shift-invariant} if
        \[
            \langle X\circ\tau_u \rangle = \langle X\rangle
        \]
        for any vertex $u\in\Z^d$ and
        for any bounded local observable $X$.
    \end{itemize}
\end{definition}

\begin{theorem}
    Let $G=\Z^d$.
    The measures $\langle\blank\rangle^\f_{\Z^d,\beta}$ and $\langle\blank\rangle^+_{\Z^d,\beta}$
    are shift-invariant.
\end{theorem}

\begin{proof}
    The desired symmetry simply follows from the symmetry in the definitions.
\end{proof}


\section{Infinite-volume limits}
\label{sec:vanishing_magnetisation}

\begin{definition}[Infinite-volume limit]
    Let $G=(V,E)$ denote a locally finite graph.
    Write
    \[
        \lim_{\Lambda\uparrow V}f(\Lambda)
        \qquad\text{for}\qquad
        \lim_{n\to\infty}f(\Lambda_n),
    \]
where $(\Lambda_n)_n$ is any increasing sequence of finite domains 
with $\cup_n\Lambda_n=V$.
This notation makes sense only when the limit is independent
of the precise choice of the sequence $(\Lambda_n)_n$,
and is called the \emph{thermodynamical limit} or \emph{infinite-volume limit}.

Let $(\Omega,\calF)$ denote the measurable space $\Omega:=\{\pm1\}^V$
endowed with the product $\sigma$-algebra.
For a domain $\Lambda$, we write $\calF_\Lambda$
for the $\sigma$-algebra generated by spins in $\Lambda$.
An observable $X:\Omega\to\C$ is called \emph{local} if it is measurable
with respect to $\calF_\Lambda$ for some domain $\Lambda$.

Let $\calP(\Omega,\calF)$ denote the set of all probability measures
on this measurable space.
We endow this set with the \emph{local convergence topology},
which is defined as the topology making the map
\[
    \calP(\Omega,\calF)\to\C,\,\mu\mapsto\mu[X]
\]
continuous for any local observable $X$.
\end{definition}

\begin{remark*}
    This topology is sometimes known under different names in the literature,
    such as the \emph{weak topology}.
    The name \emph{local convergence topology} is quite explicit:
    if the statistics of the measures within a fixed domain $\Lambda$ converge,
    then we have local convergence.
\end{remark*}

\begin{exercise}
    Prove that $\calP(\Omega,\calF)$ is a compact space in this topology.
\end{exercise}

\begin{theorem}[Existence of the thermodynamical limit]
    Consider the Ising model on a locally finite graph $G$
    at inverse temperature $\beta$.
    Then there exists unique probability measures
    $\langle\blank\rangle_{G,\beta}^\f,\langle\blank\rangle^+_{G,\beta}\in\calP(\Omega,\calF)$
    such that
    \[
        \lim_{\Lambda\uparrow V}\langle X\rangle_{\Lambda,\beta}^*=\langle X\rangle_{G,\beta}^*
    \]
    for $*\in\{\f,+\}$ and
    for any local observable $X:\Omega\to\R$.
    In other words,
    \[
        \lim_{\Lambda\uparrow V}\langle \blank\rangle_{\Lambda,\beta}^*
        =:
        \langle \blank\rangle_{G,\beta}^*.
    \]
    The measures $\langle\blank\rangle_{G,\beta}^*$ are called
    the \emph{thermodynamical limits} or \emph{infinite-volume limits}.
\end{theorem}

\begin{proof}
    Any local observable may be written as a finite linear combination
    of observables of the form $\sigma_A$ where $A$ is a finite subset of
    $V$.
    The theorem then follows by compactness and Lemma~\ref{lemma:correlation_functions_monotone_both}.
\end{proof}

\begin{definition}[Shift operator]
    Let $G=\Z^d$.
    Consider a measure $\langle\blank\rangle\in\calP(\Omega,\calF)$.
    \begin{itemize}
        \item For any $u\in\Z^d$, we define the \emph{shift operator} $\tau_u:\Omega\to\Omega$ by
        \[
            (\tau_u\sigma)_x = \sigma_{x-u}.
        \]
        An event $A$ is \emph{shift-invariant} if $\tau_uA:=\{\tau_u\sigma:\sigma\in A\}$ for any $u\in\Z^d$.
        \item The measure is called \emph{shift-invariant} if
        \[
            \langle X\circ\tau_u \rangle = \langle X\rangle
        \]
        for any vertex $u\in\Z^d$ and
        for any bounded local observable $X$.
    \end{itemize}
\end{definition}

\begin{theorem}[Shift-invariance]
    Let $G=\Z^d$.
    The measures $\langle\blank\rangle^\f_{\Z^d,\beta}$ and $\langle\blank\rangle^+_{\Z^d,\beta}$
    are shift-invariant.
\end{theorem}

\begin{proof}
    The desired symmetry simply follows from the symmetry in the definitions.
\end{proof}


\begin{exercise}[Continuity properties in $\beta$]
    Consider the Ising model on a locally finite graph $G=(V,E)$.
    Fix $A\subset V$ finite.
    \begin{itemize}
        \item The function $\beta\mapsto \langle\sigma_A\rangle_{G,\beta}^*$ is non-decreasing for $*\in\{\f,+\}$.
        \item The function $\beta\mapsto \langle\sigma_A\rangle_{G,\beta}^\f$ is left continuous.
        \item The function $\beta\mapsto \langle\sigma_A\rangle_{G,\beta}^+$ is right continuous.
    \end{itemize}
    \emph{Hint.} Argue that $\beta\mapsto \langle\sigma_A\rangle_{G,\beta}^\f$
    is a limit of a non-decreasing sequence of non-decreasing functions.
\end{exercise}

\begin{definition}[Magnetisation and critical temperature]
    Let $G$ be a vertex-transitive locally finite graph.
    The non-decreasing right-continuous function
    \[
        m=m_G:[0,\infty)\to\R,\,\beta\mapsto\langle\sigma_u\rangle_{G,\beta}^+
    \]
    is called the \emph{magnetisation} ($u$ is an arbitrary reference vertex).

    The \emph{critical (inverse) temperature} is defined via
    \[
        \beta_c:=\beta_c(G):=\inf\{\beta\in[0,\infty):m(\beta)>0\}.
    \]
\end{definition}

We have already proved that $\beta_c\in(0,\infty)$ for $G=\Z^d$
in dimension $d\geq 2$,
and that $\beta_c=\infty$ for $G=\Z$.

It is easy to derive the following result when the $m(\beta)=0$.

\begin{theorem}[$+$ and $-$ boundary conditions coincide
    when the magnetisation vanishes] 
    \label{thm:vanishing_magnetisation}
    Let $G$ denote a connected locally finite graph,
    endowed with some reference vertex $u$.
    Then 
    \[
        \langle\blank\rangle_{G,\beta}^+=\langle\blank\rangle_{G,\beta}^-
        \qquad
        \iff 
        \qquad
        m_G(\beta)=0.
    \]
\end{theorem}



\begin{proof}
    Notice that $\langle\blank\rangle_{G,\beta}^+$ and $\langle\blank\rangle_{G,\beta}^-$ are related
    by a global spin flip (the pushforward map corresponding to $\sigma\mapsto-\sigma$).
    Therefore all of the following are equivalent:
    \begin{itemize}
        \item $\langle\blank\rangle_{G,\beta}^+=\langle\blank\rangle_{G,\beta}^-$,
        \item $\langle\blank\rangle_{G,\beta}^+$ is invariant under the map $\sigma\mapsto-\sigma$,
        \item $\langle\sigma_A\rangle_{G,\beta}^+=0$ whenever $A\subset V$ has odd cardinal.
    \end{itemize}

    The implication ``$\implies$'' is now obvious, and we focus on  ``$\impliedby$''.
    Suppose that $m(\beta)=0$,
    that is, $\langle\sigma_x\rangle^+=0$ for any $x\in V$.
    Fix $A\subset V$ with $|A|$ odd.
    It suffices to prove that $\langle\sigma_A\rangle^+=0$.
    But we simply observe that
    \[
        \langle\sigma_A\rangle^+
        =
        \lim_{\Lambda\uparrow V}
        \langle\sigma_A\sigma_\frakg\rangle_{\Lambda^\frakg}
        \leq
        \lim_{\Lambda\uparrow V}
        \sum_{x\in A}
        \langle\sigma_{A\setminus\{x\}}\rangle_{\Lambda^\frakg}
        \langle\sigma_x\sigma_\frakg\rangle_{\Lambda^\frakg}
        \leq
        \sum_{x\in A}
        \langle\sigma_x\rangle_{\Lambda}^+= |A|\cdot m(\beta)=0.
    \]
    The first inequality is the pairing bound (Theorem~\ref{thm:pairing}).
\end{proof}

\todo{Add: Messager--Miracle-Solé}

\section{Continuity of the magnetisation in dimension $d\geq 3$}

The objective of this section is to prove the following deep theorems.

\begin{theorem}[Continuity in dimension $d\geq 3$]
    \label{thm:continuity}
    Consider the Ising model on the square lattice graph $G=\Z^d$
    in dimension $d\in\Z_{\geq 3}$.
    Then $m(\beta_c)=0$,
    that is, the magnetisation is continuous at $\beta=\beta_c$.
    Moreover, for $\beta\in[0,\beta_c]$,
    we have $\langle\blank\rangle^\f_{\Z^d,\beta}=\langle\blank\rangle^+_{\Z^d,\beta}$.
\end{theorem}

The proof presented here works only in dimension $d\geq 3$,
because we use an essential input called the \emph{infrared bound}.
The infrared bound is a classical tool in the analysis of spin systems.
Unfortunately, its proof is beyond the scope of these lecture notes.

\begin{theorem}[Infrared bound]
    Consider the Ising model on the square lattice graph $\Z^d$ for fixed $d\in\Z_{\geq 1}$.
    Then there exists a constant $C\in\R_{\geq 0}$ such that
    \[
        \langle\sigma_x\sigma_y\rangle_{\Z^d,\beta}^\f
        \leq C\frac1{\|y-x\|_2^{d-2}}
    \]
    for any $\beta\in[0,\beta_c]$.
    In particular, if $d\geq 3$,
    then
    \begin{equation}
        \lim_{\|y-x\|_2\to\infty}\langle\sigma_x\sigma_y\rangle_{\Z^d,\beta}^\f=0.
    \end{equation}    
\end{theorem}

Thus, we aim to prove that the infrared bound implies continuity (Theorem~\ref{thm:continuity}).
In fact, once we proved that $m(\beta_c)=0$,
it is quite easy to deduce the last part of Theorem~\ref{thm:continuity}.
We focus on proving that $m(\beta_c)=0$ for now.
Globally, the proof consists of the following two lemmas.

\begin{lemma}[Continuity, Step~1]
    Consider the Ising model on $\Z^d$ for $d\in\Z_{\geq 1}$ at $\beta\in[0,\infty)$.
    Then
    \[
        m(\beta)^2=\inf_{x,y}\langle\sigma_x\sigma_y\rangle_{\Z^d,\beta}^+.
    \]
\end{lemma}

\begin{lemma}[Continuity, Step~2]
    Consider the Ising model on $\Z^d$ for $d\in\Z_{\geq 3}$ at $\beta\in[0,\beta_c]$.
    Then
    \[
        \langle\sigma_x\sigma_y\rangle_{\Z^d,\beta}^+
        =
        \langle\sigma_x\sigma_y\rangle_{\Z^d,\beta}^\f.
    \]
    for any $x,y\in\Z^d$.
    More generally, for any subset $A\subset\Z^d$ of even cardinal, we have 
    \[
        \langle\sigma_A\rangle_{\Z^d,\beta}^+
        =
        \langle\sigma_A\rangle_{\Z^d,\beta}^\f.
    \]
\end{lemma}

Suppose that we have proved these two lemmas.
The infrared bound then tells us that at $\beta_c$ the two point function tends
to zero with the distance (for both free and wired boundary conditions, due to Step~2).
Step~1 then tells us that the magnetisation vanishes.
Step~2 is the hard step; we start with a proof of Step~1.

\begin{proof}[Proof of Continuity, Step~1]
    Fix $x,y\in\Z^d$.
    For any finite domain $\Lambda\ni x,y$,
    we have
    \[
        \langle\sigma_x\rangle_{\Lambda,\beta}^+
        \langle\sigma_y\rangle_{\Lambda,\beta}^+
        =
        \langle\sigma_x\sigma_\frakg\rangle_{\Lambda^\frakg,\beta}
        \langle\sigma_y\sigma_\frakg\rangle_{\Lambda^\frakg,\beta}
        \leq
        \langle\sigma_x\sigma_y\rangle_{\Lambda,\beta}^+
    \]
    by the second Griffiths inequality.
    Sending $\Lambda\uparrow\Z^d$ yields
    \[
        m(\beta)^2\leq \langle\sigma_x\sigma_y\rangle_{\Z^d,\beta}^+.
    \]
    It suffices to prove the other bound.

    Fix $x=0\in\Z^d$, and let $\Lambda\ni x$ denote a large finite domain.
    For any $y\in\Z^d$, let $\Lambda_y:=\Lambda\cup(\Lambda+y)$.
    Then
    \[
        \limsup_{\|y\|_2\to\infty}\langle\sigma_x\sigma_y\rangle_{\Z^d,\beta}^+
        \leq
        \limsup_{\|y\|_2\to\infty}\langle\sigma_x\sigma_y\rangle_{\Lambda_y,\beta}^+
        =
        (\langle\sigma_x\rangle_{\Lambda,\beta}^+)^2
        \to_{\Lambda\uparrow\Z^d}m(\beta)^2.
    \]
    The equality holds true because for $\|y\|_2$ sufficiently large,
    $\Lambda$ and $\Lambda+y$ are no longer adjancent,
    and therefore the restrictions $\sigma|_\Lambda$
    and $\sigma|_{\Lambda+y}$ behave like independent Ising models.
\end{proof}



\begin{proof}[Proof of continuity in dimension $d\geq 3$]
    We have already seen that $m(\beta_c)=0$
    (and also $m(\beta)=0$ for $\beta\leq\beta_c$).
    It suffices to prove that $\langle\blank\rangle^+_{\Z^d,\beta}=\langle\blank\rangle^\f_{\Z^d,\beta}$.
    We shall prove that
    \[
        \langle\sigma_A\rangle^+_{\Z^d,\beta}=\langle\sigma_A\rangle^\f_{\Z^d,\beta}
    \]
    for any finite $A\subset\Z^d$.
    If $|A|$ is even then this also follows from Step~2.
    If $|A|$ is odd then we must simply show that $\langle\sigma_A\rangle^+_{\Z^d,\beta}=0$.
    For $|A|=1$ this is just the statement that $m(\beta)=0$.
    For $|A|>1$ we can deduce this from the pairing bound (Lemma~\ref{???}).
\end{proof}

\newpage 

\section{Coupling from the past}

Consider the following practical objective:
use a computer to sample a configuration of the Ising model on a finite graph.
Computers generally have access to a source of i.i.d.\ randomness,
but cannot immediately sample from complicated distributions.
We must therefore transform the i.i.d.\ randomness
into a sample from the Ising model.

Let us first consider \emph{Glauber dynamics},
which is a standard strategy for reaching this objective.
One uses the i.i.d.\ randomness
to construct a Markov chain whose invariant distribution is the Ising model
(for example, via the Metropolis--Hastings algorithm).
To take a sample, one starts at a deterministic configuration, runs the algorithm for a deterministic number of steps (say $N$),
and then takes the final state as the sample.
This strategy has two drawbacks:
\begin{itemize}
    \item The Markov chain approximates its invariant distribution;
    the sample is not ``perfect'',
    \item The number $N$ required for the desired precision is often hard to calculate.
\end{itemize}

\emph{Coupling from the past} circumvents these problems.
This section explains the basic algorithm,
and the second section discusses theoretical implications (such as ergodicity, mixing, and uniqueness of Gibbs measures).
Throughout this section, we consider the Ising model
on a finite graph $G$ at inverse temperature $\beta$.

\begin{definition}[Treshold value at a single spin]
    The distribution of $\sigma_x$ conditional on $\{\sigma_{V\setminus\{x\}}=\zeta\}$ is given by
    \[
        \P_{G,\beta}[\{\sigma_x=-\}|\{\sigma_{V\setminus\{x\}}=\zeta\}]
        =
        \tau_\beta({\textstyle\sum_{y\sim x}\zeta_x});
        \qquad
        \tau_\beta(a):=\frac{e^{-\beta a}}{2\cosh\beta a}.
    \]
    The value $\tau_\beta$ is called the \emph{treshold value}.
\end{definition}

This means that the value of $\sigma_x$ may be resampled as follows:
first sample $U\sim U([0,1])$ (the uniform distribution on the unit interval),
then set
\[
    \sigma_x:=\begin{cases}
        - &\text{if $U\leq \tau_\beta(\sum_{y\sim x}\sigma_x)$,}\\
        + &\text{if $U> \tau_\beta(\sum_{y\sim x}\sigma_x)$.}
    \end{cases}
\]

\begin{definition}[Local update map]
    For any $x\in V$ and $U\in[0,1]$, define the \emph{local update map}
    \[
        R_{x,U}:\Omega\to\Omega,\,\sigma\mapsto
        \left(
            z\mapsto \begin{cases}
                \sigma_z&\text{if $z\neq x$}\\
                - &\text{if $z=x$ and $U\leq \tau_\beta(\sum_{y\sim x}\sigma_y)$}\\
                + &\text{if $z=x$ and $U>\tau_\beta(\sum_{y\sim x}\sigma_y)$}
            \end{cases} 
        \right).
    \]
\end{definition}

Let $\P$ denote the uniform probability measure on $(x,U)\in V\times[0,1]$.
Consider the Markov chain on $\Omega$ with transition matrix
\[
    A_{\sigma',\sigma}:=\P[\{R_{x,U}(\sigma)=\sigma'\}].
\]

\begin{exercise}[Glauber dynamics]
    \begin{enumerate}
        \item     Prove that $\langle\blank\rangle_{G,\beta}$ is a probability distribution on
        $\Omega$ solving the detailed balance equations for the transition matrix $A$.
        \item Argue that the Markov chain corresponding to $A$ is acyclic and irreducible.
        \item Conclude that $\langle\blank\rangle_{G,\beta}$ is the unique limit distribution
        of the finite state Markov chain $A$.
    \end{enumerate}
\end{exercise}

Now let $((x_i,U_i))_{i\in\Z_{\leq 0}}$ denote a sequence of i.i.d.\ copies of
the uniform random variable $(x,U)\in V\times[0,1]$,
and write $\P$ for the corresponding measure.
Write $R^{-n}$ for the random composition
\[
    R^{-n}:=R_{x_0,U_0}\circ R_{x_{-1},U_{-1}}\circ\cdots\circ R_{x_{-n+1},U_{-n+1}}.
\]
Then it is easy to see that the distribution of $R^{-n}(\sigma)$
is given by
\[
    A^n\delta_\sigma.
\]
In particular, $A^n\delta_\sigma$ converges to $\langle\blank\rangle$ as $n\to\infty$
(notice that $\R^\Omega$ is a finite dimensional vector space and therefore all reasonable topologies coincide).

\begin{theorem}[Coupling from the past]
    \label{thm:coupling_from_the_past}
    Consider the sequence $((x_i,U_i))_{i\in\Z_{\leq 0}}$ in the measure $\P$ as defined above.
    Let $T\in\Z_{\geq 0}\cup\{\infty\}$ denote the random stopping time
    defined via
    \[
        T:=\inf\{n\in\Z_{\geq 0}:\text{the function $R^{-n}:\Omega\to\Omega$ is constant as a function on $\Omega$}\}.
    \]
    Suppose that $\P(\{T<\infty\})=1$.
    Then all of the following are true:
    \begin{itemize}
        \item For any $S\geq T$, the function $R^{-S}:\Omega\to\Omega$ is also constant on $\Omega$, and $R^{-S}=R^{-T}$,
        \item The distribution of function $R^{-T}$ is $\langle\blank\rangle$.
    \end{itemize}
    Here we simply write $R^{-T}$ for $R^{-T}(\sigma)$ (with $\sigma\in\Omega$ arbitrary)
    whenever $R^{-T}$ is constant.
\end{theorem}

\begin{proof}
    Fix $\sigma\in\Omega$.
    Fix a sequence $(x_i,U_i)_i$.
    If $R^{-n}$ is constant for some $n$, then
    \[
            R^{-(n+1)}=R^{-n}\circ R_{x_{-n},U_{-n}}=R^{-n}.
    \]
    In that case, we simply have
    \[
        \lim_{m\to\infty}R^{-m}(\sigma)=R^{-n}(\sigma).
    \]
    In particular, if $T<\infty$ almost surely,
    then almost surely
    \[
        \lim_{m\to\infty} R^{-m}(\sigma)=R^{-T}.
    \]
    Since the distribution of $R^{-m}(\sigma)$ converges to $\langle\blank\rangle$
    as $m\to\infty$,
    we also have $R^{-T}\sim\langle\blank\rangle$.
\end{proof}

It is clearly important that $\P(\{T<\infty\})=1$.
We invite the reader to prove this in the following exercise.
Another proof is provided later.

\begin{exercise}[Coupling from the past: convergence (generic)]
    Show that $T$ is almost surely finite
    in the context of the above theorem.
    Hint: observe that almost surely,
    there are $|\Lambda|$ consequitive entries in the sequence $((x_i,U_i))_{i\in\Z_{\leq 0}}$
    for which $x_i$ enumerates $\Lambda$ and such that $U_i\approx 0$.
\end{exercise}

The algorithm can practically be implemented in a computer as follows.
\begin{enumerate}
    \item Fix a strictly increasing sequence $(n_k)_{k\geq 0}$ of integers with $n_0=0$.
    \item Set $k=0$, and repeat the following procedure as long as $R^{-n_k}$ is \emph{not} constant:
    \begin{enumerate}
        \item Add $1$ to the counter $k$,
        \item Draw $((x_i,U_i))_{-n_k< i\leq -n_{k-1}}$ from the independent source of randomness,
        \item Calculate the map $R^{-n_k}$.
    \end{enumerate}
    \item Output the constant value $R^{-n_k}$ as our sample from $\langle\blank\rangle$.
\end{enumerate}

The algorithm has two problems:
\begin{itemize}
    \item It requires memory to record $R^{-n_k}$ or $((x_i,U_i))_{-n_k< i\leq 0}$ between iterations,
    \item It requires time to determine if $R^{-n_k}$ is constant on $\Omega$ (which has cardinal $2^{|V|}$).
\end{itemize}

It turns out that there is an easy way to get around the second problem,
thanks to \emph{monotonicity} in the model.
This monotonicity is relative to the partial ordering $\leq$
on the sample space $\Omega=\{\pm\}^V$.

\begin{lemma}[Monotonicity of Glauber dynamics]
    Consider $((x_i,U_i))_{i\in\Z_{\leq 0}}$ fixed. Then:
    \begin{enumerate}
        \item Each map $R_{x_i,U_i}:\Omega\to\Omega$ is increasing, that is, it preserves $\leq$,
        \item Each map $R^{-n}:\Omega\to\Omega$ is increasing, that is, it preserves $\leq$,
        \item $(R^{-n}(\sigma^{\max}))_n$ is decreasing in $n$, where $\sigma^{\max}\equiv+\in\Omega$ is the maximal element,
        \item $(R^{-n}(\sigma^{\min}))_n$ is increasing in $n$, where $\sigma^{\min}\equiv+\in\Omega$ is the minimal element,
        \item We have $R^{-\infty}(\sigma^{\min})\leq R^{-\infty}(\sigma^{\max})$,
        where $R^{-\infty}(\cdots)$ denotes the $n\to\infty$ limit,
        \item If  $R^{-n}(\sigma^{\min})= R^{-n}(\sigma^{\max})$, then the map $R^{-n}$ is constant.
    \end{enumerate}
\end{lemma}

\begin{proof}
    The first item can be proved by simply inspecting the definition of $R_{x_i,U_i}$.
    The second item then follows: indeed, a composition of increasing functions is increasing.
    For the third item (and similarly, the fourth item), observe that monotonicity of $R^{-n}$ implies:
    \[
        R^{-(n+1)}(\sigma^{\max})
        =
        R^{-n}(R_{x_{-n},U_{-n}}(\sigma^{\max}))
        \leq
        R^{-n}(\sigma^{\max})
    \]
    The last two items follow from monotonicity of the map $R^{-n}$ (the second item).
\end{proof}


\begin{lemma}[Coupling from the past: convergence]
    Recall the context of Theorem~\ref{thm:coupling_from_the_past}.
    Then we have
    \[
        \{T<\infty\}=\{R^{-\infty}(\sigma^{\max})=R^{-\infty}(\sigma^{\min})\},
    \]
    and this event occurs $\P$-almost surely.
\end{lemma}

\begin{proof}
    The reformulation of the event is valid thanks to the previous lemma.
    It suffices to prove that it occurs almost surely.
    Notice that the distribution of both $R^{-\infty}(\sigma^{\max})$ and $R^{-\infty}(\sigma^{\min})$
    is given by $\langle\blank\rangle$.
    Since they have the same distribution and are almost surely $\leq$-ordered in $\P$,
    they must be almost surely equal.
\end{proof}

% \section{Coupling from the past: theoretical implications}


% \section{Mixing properties of the infinite-volume limit}

In this section, we consider the Ising model on the infinite square lattice $\Z^d$.

\begin{definition}
    Consider a measure $\langle\blank\rangle\in\calP(\Omega,\calF)$.
    \begin{itemize}
        \item For any $u\in\Z^d$, we define the \emph{shift operator} $\tau_u:\Omega\to\Omega$ by
        \[
            (\tau_u\sigma)_x = \sigma_{x-u}.
        \]
        An event $A$ is \emph{shift-invariant} if $\tau_uA:=\{\tau_u\sigma:\sigma\in A\}$ for any $u\in\Z^d$.
        \item The measure is called \emph{shift-invariant} if
        \[
            \langle X\circ\tau_u \rangle = \langle X\rangle
        \]
        for any vertex $u\in\Z^d$ and
        for any bounded local observable $X$.
        \item The measure is called \emph{mixing} if
        \[
            \lim_{\|u\|_2\to\infty}
            \left(\langle X(Y\circ\tau_u)\rangle-\langle X\rangle\langle Y\circ\tau_u\rangle\right)
            =0
        \]
        for any bounded local observables $X$ and $Y$.
        \item The measure is called \emph{ergodic}
        if it is shift-invariant
        and
        \[\langle\ind{A}\rangle\in\{0,1\}\]
        for any shift-invariant event $A\in\calF$.
    \end{itemize}
\end{definition}

\begin{lemma}[Mixing implies ergodicity]
    If a shift-invariant measure is mixing, then it is ergodic.
\end{lemma}

\begin{proof}
    Let $\langle\blank\rangle$ denote a shift-invariant measure that is mixing, but not ergodic.
    We aim to derive a contradiction.
    Let $A$ denote a shift-invariant event with $p:=\langle\ind{A}\rangle\in(0,1)$.
    We shall derive a contradiction by constructing two events which
    are both extremely correlated with $A$ (using ergodicity),
    while also being almost independent (using mixing).

    Fix $\epsilon>0$.
    By the martingale convergence theorem,
    there exists a finite domain $\Lambda\subset\Z^d$
    and an $\calF_\Lambda$-measurable event
    $A_\Lambda$
    such that
    $\langle \ind{A\Delta A_\Lambda}\rangle<\epsilon$.
    Write $p':=\langle \ind{A_\Lambda}\rangle$;
    notice that $|p'-p|<\epsilon$.
    Define $A_{\Lambda+u}:=\tau_uA_\Lambda\in\calF_{\Lambda+u}$.
    We claim that there exists some $u\in\Z^d$ such that:
    \[
        \langle \ind{A_\Lambda\Delta A_{\Lambda+u}}\rangle
        <2\epsilon
        \qquad
        \text{and}
        \qquad
        \langle \ind{A_\Lambda\Delta A_{\Lambda+u}}\rangle
        \geq 2p'(1-p')-\epsilon.
    \]
    This yields the desired contradiction when $\epsilon$ is small enough.

    The inequality on the left is easy to obtain:
    for any $u\in\Z^d$,
    the event $A_{\Lambda+u}$
    also satisfies
    $\langle \ind{A\Delta A_{\Lambda+u}}\rangle=\langle \ind{A\Delta A_\Lambda}\rangle<\epsilon$.
    By the triangular inequality, we have
    \(
        \langle \ind{A_\Lambda\Delta A_{\Lambda+u}}\rangle
        <2\epsilon.
    \)

    For the inequality on the right, we use mixing:
    we get
    \[
        \langle \ind{A_\Lambda\Delta A_{\Lambda+u}}\rangle
        =
        \langle \ind{A_\Lambda}+\ind{A_{\Lambda+u}}
        -
        2 \ind{A_\Lambda}\ind{A_{\Lambda+u}}\rangle
        \to_{\|u\|_2\to\infty}
        p'+p'-2p'p'
        =2p'(1-p').
    \]
    We then simply choose $u$ such that $\|u\|_2$ is sufficiently large.
\end{proof}

\begin{theorem}
    Consider the Ising model on $\Z^d$ at inverse temperature $\beta$.
    \begin{itemize}
        \item \textbf{Wired boundary.}
        The measure $\langle\blank\rangle_{\Z^d,\beta}^+$ is mixing.
        \item \textbf{Free boundary.}
        The measure $\langle\blank\rangle_{\Z^d,\beta}^\f$ is mixing if
        $\lim_{\|y\|_2\to\infty}\langle\sigma_x\sigma_y\rangle_{\Z^d,\beta}^\f=0$.
    \end{itemize}
\end{theorem}

\begin{proof}[Proof for $\langle\blank\rangle^+$]
    It suffices to prove that for any finite sets $A,B\subset\Z^d$,
    we have
    \[
        \lim_{\|u\|_2\to\infty}
        \langle\sigma_A\sigma_{B+u}\rangle_{\Z^d,\beta}^+
        =
        \langle\sigma_A\rangle_{\Z^d,\beta}^+
        \langle\sigma_B\rangle_{\Z^d,\beta}^+.
    \]
    By the second Griffiths inequality,
    we get
    \[
        \langle\sigma_A\sigma_{B+u}\rangle_{\Z^d,\beta}^+
        \geq
        \langle\sigma_A\rangle_{\Z^d,\beta}^+
        \langle\sigma_B\rangle_{\Z^d,\beta}^+.
    \]
    It suffices to prove the other inequality.
    Consider an extremely large finite domain $\Lambda\supset A\cup B$.
    Define $\Lambda_u:=\Lambda\cup(\Lambda+u)$.
    If $\|u\|_2$ is sufficiently large,
    then $\Lambda$ is not connected to $\Lambda_u$
    in the set $\Lambda_u$, and so we get
    \[
        \limsup_{\|u\|_2\to\infty}
        \langle\sigma_A\sigma_{B+u}\rangle_{\Z^d,\beta}^+
        \leq
        \limsup_{\|u\|_2\to\infty}
        \langle\sigma_A\sigma_{B+u}\rangle_{\Lambda_u,\beta}^+
        =
        \langle\sigma_A\rangle_{\Lambda,\beta}^+
        \langle\sigma_B\rangle_{\Lambda,\beta}^+
        .
    \]
    Sending $\Lambda\uparrow\Z^d$ yields the desired inequality.
\end{proof}

\begin{proof}[Proof for $\langle\blank\rangle^\f$]
    As for $+$ boundary conditions,
    it suffices to fix two finite sets $A,B\subset\Z^d$,
    and prove that
    \[
        \limsup_{\|u\|_2\to\infty}
        \langle\sigma_A\sigma_{B+u}\rangle_{\Z^d,\beta}^\f
        \leq
        \langle\sigma_A\rangle_{\Z^d,\beta}^\f
        \langle\sigma_B\rangle_{\Z^d,\beta}^\f
        .
    \]
    Fix $u$, and
    consider an extremely large finite domain $\Lambda\supset A\cup (B+u)$.
    By the switching lemma, it is straightforward to deduce that
    \[
        \langle\sigma_A\sigma_{B+u}\rangle_{\Lambda,\beta}^\f
        \leq
        \langle\sigma_A\rangle_{\Lambda,\beta}^\f\langle\sigma_{B+u}\rangle_{\Lambda,\beta}^\f
        +\sum_{x\in A,\,y\in B}\langle\sigma_x\sigma_{y+u}\rangle_{\Lambda,\beta}^\f.
    \]
    By sending $\Lambda\uparrow\Z^d$, we get
    \[
        \langle\sigma_A\sigma_{B+u}\rangle_{\Z^d,\beta}^\f
        \leq
        \langle\sigma_A\rangle_{\Z^d,\beta}^\f\langle\sigma_B\rangle_{\Z^d,\beta}^\f
        +\sum_{x\in A,\,y\in B}\langle\sigma_x\sigma_{y+u}\rangle_{\Z^d,\beta}^\f.
    \]
    The error term vanishes by our additional assumption in the statement of the theorem.
\end{proof}


% \section{The FKG inequality. Proof}

Cees Fortuin, Pieter Kasteleyn, and Jean Ginibre discovered a general
way to prove that increasing observables are positive correlated.
This inequality already known in the context of percolation
theory (independent randomness) as the \emph{Harris inequality},
after Theodore Harris.
In these lecture notes, we shall state and prove the FKG inequality in a simplified
context, which will be sufficient for our purposes.
The interested reader may consult the original paper, which is an accessible classic in statistical mechanics.

\begin{definition}[FKG inequality]
    Let $(\Omega,\preceq)$ denote a partially ordered set
    and $\mu$ a probability measure on $\Omega$.
    We say that $\mu$ satisfies the \emph{FKG inequality}
    if
    \begin{equation}
        \label{eq:FKG}
        \mu[fg]\geq \mu[f]\mu[g]
    \end{equation}
    for any bounded $\preceq$-nondecreasing functions
    $f,g:\Omega\to\R$.
\end{definition}

For simplicity, we often call $\preceq$-nondecreasing functions \emph{increasing}
and $\preceq$-nonincreasing functions \emph{decreasing}.
Notice that $f$ is increasing if and only if $-f$ is decreasing.
The FKG inequality may therefore be formulated in terms of decreasing functions,
or in terms of a mixture of increasing and decreasing functions.

\begin{remark}[The FKG inequality and increasing events]
    We stated the FKG inequality in terms of observables.
    Examples of observables are functions of the form $\ind{A}$.
    In that case, the FKG inequality states that, if $\mu(A)>0$,
    then
    \[
        \mu(B|A)\geq \mu(B)
    \]
    whenever $\ind{A}$ and $\ind{B}$ are increasing functions.
    Such events are called \emph{increasing events}.
\end{remark}

\begin{exercise}[Iterating the FKG inequality]
    Let $\mu$ denote a probability measure satisfying the FKG inequality and
     $(A_i)_i$ a finite family of increasing events.
    Prove that
    \[
        \mu(\cap_i A_i)\geq \prod_i\mu(A_i).
    \]
    
    If $(f_i)_i$ is a finite family of increasing functions,
    does it hold true that $\mu[\prod_i f_i]\geq \prod_i \mu[f_i]$?
    Why (not)?
\end{exercise}

We have now defined the FKG inequality,
but to derive it, we require the notion of a distributive lattice.

\begin{definition}[Distributive lattices]
    A \emph{distributive lattice} is a tuple $(\Omega,\preceq,\vee,\wedge)$
    where $(\Omega,\preceq)$ is a partially ordered set
    and where $\vee,\wedge:\Omega\times\Omega\to\Omega$
    are binary operators satisfying the following properties
    for any $x,y,z\in\Omega$:
    \begin{enumerate}
        \item $x\vee y$ equals the least upper bound of $x$ and $y$ with respect to $\preceq$,
        \item $x\wedge y$ equals the greatest lower bound of $x$ and $y$ with respect to $\preceq$,
        \item The following two \emph{distribution equations}:
        \begin{itemize}
            \item $x\wedge (y\vee z)=(x\wedge y)\vee(x\wedge z)$,
            \item $x\vee (y\wedge z)=(x\vee y)\wedge(x\vee z)$.
        \end{itemize}
    \end{enumerate}
    A distributive lattice is called \emph{finite} or \emph{countable}
    whenever $\Omega$ has these respective properties.
    It is called a \emph{binary lattice} if it is isomorphic
    to $\{0,1\}^I$ for some index set $I$.
\end{definition}


\begin{definition}[FKG lattice condition]
    Let $X:\Omega\to[0,\infty)$ denote a
    function defined on some distributive lattice $(\Omega,\preceq,\vee,\wedge)$.
    We say that $X$ satisfies the \emph{FKG lattice condition}
    if 
    \begin{equation}
        X(\omega\vee\eta)\cdot X(\omega\wedge\eta)
        \geq
        X(\omega)\cdot X(\eta)
        \qquad
        \forall\omega,\eta\in\Omega.
    \end{equation}
\end{definition}

\begin{theorem}[FKG, 1971]
    \label{thm:original_FKG}
    Let $(\Omega,\preceq,\vee,\wedge)$ denote a finite binary lattice,
    and let $X:\Omega\to[0,\infty)$ denote a strictly positive function
    satisfying the FKG lattice condition.
    Then the probability measure $\mu$ defined by its expectation functional
    \[
        \mu[f]:=\frac1Z\sum_{\omega\in\Omega}X(\omega)f(\omega);
        \qquad Z:=\sum_{\omega\in\Omega}X(\omega)
    \]
    satisfies the FKG inequality on $(\Omega,\preceq)$.
\end{theorem}

\begin{proof}
    Without loss of generality, $\Omega=\{0,1\}^n$ for some $n\in\Z_{\geq 0}$.
    We induct on $n$.
    The case $n=0$ is trivial.
    The case $n=1$ is elementary and left as an exercise for the interested reader.
    Notice that the measures
    \[
        \mu_\pm:=\mu[\blank|\Omega_\pm];
        \qquad
        \Omega_-:=\{\omega_n=0\};
        \qquad
        \Omega_+:=\{\omega_n=1\}
    \]
    satisfy the FKG inequality due to the induction hypothesis.

    \begin{claim*}
    For any increasing function $f$  on $(\Omega,\preceq)$ we have \(
            \mu_-[f]\leq\mu_+[f]
        \).
    \end{claim*}

    We shall first see how the claim implies the theorem,
    then prove the claim.
    Let $f,g:\Omega\to\R$ denote increasing functions.
    We then simply assert that
    \[
        \mu[fg] = \mu[\mu[fg|\omega_n]]
        \geq
        \mu[\mu[f|\omega_n]\mu[g|\omega_n]]
        \geq
        \mu[\mu[f|\omega_n]]\mu[\mu[g|\omega_n]]
        =
        \mu[f]\mu[g].
    \]
    The two equalities are just the tower property.
    The first inequality is the FKG inequality applied
    to the measures $\mu_\pm$.
    For the second inequality, notice that $\mu[f|\omega_n]$
    and $\mu[g|\omega_n]$ are increasing functions
    of the bit $\omega_n$,
    so that we may simply apply the FKG inequality
    coming from the $n=1$ case already discussed above.

    We now prove the claim.
    Remark that we have not yet used the FKG lattice condition;
    this will be crucial in the proof of the claim.
    Write $\omega\mapsto \omega^+$ for the obvious bijection
    from $\Omega_-$ to $\Omega_+$ (which flips the last bit).
    Then
    \[
        \mu_+[ f]
        =
        \frac{\sum_{\omega\in\Omega_-}X(\omega^+)f(\omega^+)}{\sum_{\omega\in\Omega_-}X(\omega^+)}.
    \]
    Writing $X(\omega^+)=X(\omega)X'(\omega)$
    where
    $X'(\omega):=\frac{X(\omega^+)}{X(\omega)}$,
    we get 
    \[
        \mu_+[ f]
        =
        \frac{\mu_-[ f(\omega^+) X']}{\mu_-[ X']}
        \geq
        \frac{\mu_-[ f X']}{\mu_-[ X']}
        .
    \]
    For the inequality in this display we just used that $f(\omega^+)\geq f(\omega)$.
    To conclude that the right hand side equals at least $\mu_-[ f]$
    we apply the FKG inequality to $\mu_-[\blank]$,
    observing that $f$ is increasing by assumption and that $X'$ is increasing
    due to the FKG lattice condition.
    This establishes the claim, and thus the theorem.
\end{proof}

\begin{remark}[No FKG after conditioning]
    Suppose that $\mu$ is a probability measure satisfying the FKG inequality.
    Then we have $\mu(B|A)\geq\mu(B)$ whenever $A$ and $B$ are increasing events
    with $\mu(A)>0$,
    but we \emph{do not} know if the conditional probability measure $\mu(\blank|A)$
    satisfies the FKG inequality.
\end{remark}

\begin{exercise}[No FKG after conditioning]
    Consider the following example of two independent coin flips:
    $\Omega=\{\pm1\}^2$ and $X\equiv 1$,
    so that $\mu$ is the uniform distribution on $\Omega$.
    Since $X$ satisfies the FKG lattice condition,
    the measure $\mu$ satisfies the FKG inequality.
    Let $A$ denote the increasing event that \emph{at least}
    one of the coins is valued $+1$.
    What is the correlation of the two coins in the conditional measure
    $\mu(\blank|A)$?
    Argue that this conditional measure does not satisfy the FKG inequality.
\end{exercise}

\begin{remark}[Preservation of FKG after conditioning on binary sublattices]
    Consider the setting of Theorem~\ref{thm:original_FKG}.
    That theorem does imply that the FKG inequality is preserved
    under conditioning on an event $A\subset\Omega$
    which is itself a binary lattice.
\end{remark}

% \section{The FKG inequality. Application to the Ising spins}

\begin{lemma}
    Consider the Ising model on a finite graph $G$ at inverse
    temperature $\beta\geq 0$.
    Then the map
    \[
        \sigma\mapsto e^{-H_{G,\beta}(\sigma)}
    \]
    satisfies the FKG lattice condition.
    In particular, the Ising measure
    $\langle\blank\rangle_{G,\beta}$ satisfies the FKG inequality.
\end{lemma}

\begin{proof}
    Let $\sigma,\eta\in\Omega$ denote two spin configurations.
    It suffices to show that
    \[
        H(\sigma\vee\eta)+H(\sigma\wedge\eta)
        \leq
        H(\sigma)+H(\eta).
    \]
    This is immediate from the definition of the Hamiltonian:
    it is a sum of terms of the form
    \[
        -\beta\sigma_u\sigma_v,
    \]
    while these terms satisfy the obvious inequality
    \[
        (\sigma_u\vee \eta_u)(\sigma_v\vee \eta_v)
        +
        (\sigma_u\wedge \eta_u)(\sigma_v\wedge \eta_v)
        \geq
        \sigma_u\sigma_v+\eta_u\eta_v.
    \]
    This finishes the proof.
\end{proof}

We can now already prove the first Griffiths inequality for the case of
two vertices.

\begin{corollary}
    Let $G$ be a finite graph and let $\beta\geq 0$.
    Then the associated Ising model satisfies $\langle\sigma_u\sigma_v\rangle_{G,\beta}\geq 0$.
\end{corollary}

\begin{proof}
    Note that $\sigma_u$ and $\sigma_v$ are increasing functions
    of the spin configuration $\sigma$,
    while they have zero expectation due to flip-symmetry.
    The result then follows from the FKG inequality.
\end{proof}


\bibliographystyle{amsalpha}
\bibliography{}
\end{document}
