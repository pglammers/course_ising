% \documentclass[dvipsnames,11pt,reqno,twoside,final]{amsart}
\usepackage[a4paper,left=3cm,right=3cm,top=3cm,bottom=3cm]{geometry}
\usepackage[T1]{fontenc}
\usepackage[utf8]{inputenc}
\usepackage{xcolor}
\usepackage{amssymb,mathtools}  % Also loads "amsmath"
\usepackage{dsfont}
\usepackage{enumitem}
\usepackage{subcaption}
\usepackage{booktabs}
\newcommand\numberthis{\addtocounter{equation}{1}\tag{\theequation}}
\usepackage{preamble/mhequ}

\usepackage{amsthm}


\newcounter{counterEnvMain}
\newcounter{counterEnvDefault}
\numberwithin{counterEnvDefault}{section}


% ====================
\theoremstyle{plain}

\newtheorem{mainlemma}[counterEnvMain]{Lemma}
\newtheorem{lemma}[counterEnvDefault]{Lemma}
\newtheorem*{lemma*}{Lemma}

\newtheorem{maintheorem}[counterEnvMain]{Theorem}
\newtheorem{theorem}[counterEnvDefault]{Theorem}

\newtheorem{proposition}[counterEnvDefault]{Proposition}
\newtheorem{corollary}[counterEnvDefault]{Corollary}
\newtheorem{assumption}[counterEnvDefault]{Assumption}


% ====================
\theoremstyle{definition}

\newtheorem{definition}[counterEnvDefault]{Definition}
\newtheorem*{definition*}{Definition}

\newtheorem{example}[counterEnvDefault]{Example}
\newtheorem*{example*}{Example}


\newtheorem{exercise}[counterEnvDefault]{Exercise}
\newtheorem*{exercise*}{Exercise}

\newtheorem{remark}[counterEnvDefault]{Remark}

\newtheorem*{claim*}{Claim}

\newtheorem*{assertion*}{Assertion}

\newtheorem*{proposition*}{Proposition}

\usepackage{microtype}
\renewcommand\phi\varphi
\renewcommand\epsilon\varepsilon

\usepackage{hyperref}
\definecolor{colorlinks}{RGB}{0, 24, 168}
\definecolor{colorcites}{RGB}{124, 10, 2}
\hypersetup{
    colorlinks=true,
    linkcolor=colorlinks,
    citecolor=colorcites,
    urlcolor=colorlinks,
    pdfborder={0 0 0}
}

\usepackage{xargs}
\usepackage[colorinlistoftodos,prependcaption,textsize=tiny]{todonotes}

\newcommandx\work[2][1=]{\todo[linecolor=RoyalBlue,backgroundcolor=RoyalBlue!25,bordercolor=RoyalBlue,#1]{\textsc{todo} #2}}
\newcommandx\comment[2][1=]{\todo[linecolor=OliveGreen,backgroundcolor=OliveGreen!25,bordercolor=OliveGreen,#1]{\textsc{comment} #2}}
\newcommandx\mistake[2][1=]{\todo[linecolor=red,backgroundcolor=red!25,bordercolor=red,#1]{\textsc{mistake} #2}}
\newcommandx\improve[2][1=]{\todo[linecolor=orange,backgroundcolor=orange!25,bordercolor=orange,#1]{\textsc{improve} #2}}
\newcommandx\change[2][1=]{\todo[linecolor=yellow,backgroundcolor=yellow!25,bordercolor=yellow,#1]{\textsc{change} #2}}
\newcommandx\mem[2][1=]{\todo[linecolor=orange,backgroundcolor=orange!25,bordercolor=orange,#1]{\textsc{mem} #2}}
\newcommandx\status[2][1=]{\todo[linecolor=Blue,backgroundcolor=Blue!25,bordercolor=Blue,#1]{\textsc{Status} #2}}

\newcommand\hidetodos{
    \renewcommandx\todo[2][1=]{}
    \renewcommandx\work[2][1=]{}
    \renewcommandx\comment[2][1=]{}
    \renewcommandx\mistake[2][1=]{}
    \renewcommandx\improve[2][1=]{}
    \renewcommandx\change[2][1=]{}
    \renewcommandx\mem[2][1=]{}
    \renewcommandx\status[2][1=]{}
}

\newcommand\blank{\,\cdot\,}
\newcommand\ssubset{\Subset}

\newcommand\diam{\operatorname{diam}}
\newcommand\Var{\operatorname{Var}}
\newcommand\Cov{\operatorname{Cov}}

\newcommand\ind[1]{\mathds{1}_{#1}}
\newcommand\true[1]{\mathds{1}({#1})}
\newcommand\diffi{{\,\mathrm{d}}}
\newcommand\diff{{\mathrm{d}}}

\newcommand\A{\mathbb A}
\newcommand\B{\mathbb B}
\newcommand\C{\mathbb C}
\newcommand\D{\mathbb D}
\newcommand\E{\mathbb E}
\newcommand\F{\mathbb F}
\newcommand\G{\mathbb G}
\renewcommand\H{\mathbb H}
\newcommand\I{\mathbb I}
\newcommand\J{\mathbb J}
\newcommand\K{\mathbb K}
\renewcommand\L{\mathbb L}
\newcommand\M{\mathbb M}
\newcommand\N{\mathbb N}
\renewcommand\O{\mathbb O}
\renewcommand\P{\mathbb P}
\newcommand\Q{\mathbb Q}
\newcommand\R{\mathbb R}
\renewcommand\S{\mathbb S}
\newcommand\T{\mathbb T}
\newcommand\U{\mathbb U}
\newcommand\V{\mathbb V}
\newcommand\W{\mathbb W}
\newcommand\X{\mathbb X}
\newcommand\Y{\mathbb Y}
\newcommand\Z{\mathbb Z}

\newcommand\bbA{\mathbb A}
\newcommand\bbB{\mathbb B}
\newcommand\bbC{\mathbb C}
\newcommand\bbD{\mathbb D}
\newcommand\bbE{\mathbb E}
\newcommand\bbF{\mathbb F}
\newcommand\bbG{\mathbb G}
\newcommand\bbH{\mathbb H}
\newcommand\bbI{\mathbb I}
\newcommand\bbJ{\mathbb J}
\newcommand\bbK{\mathbb K}
\newcommand\bbL{\mathbb L}
\newcommand\bbM{\mathbb M}
\newcommand\bbN{\mathbb N}
\newcommand\bbO{\mathbb O}
\newcommand\bbP{\mathbb P}
\newcommand\bbQ{\mathbb Q}
\newcommand\bbR{\mathbb R}
\newcommand\bbS{\mathbb S}
\newcommand\bbT{\mathbb T}
\newcommand\bbU{\mathbb U}
\newcommand\bbV{\mathbb V}
\newcommand\bbW{\mathbb W}
\newcommand\bbX{\mathbb X}
\newcommand\bbY{\mathbb Y}
\newcommand\bbZ{\mathbb Z}

\newcommand\CA{\mathcal A}
\newcommand\CB{\mathcal B}
\newcommand\CC{\mathcal C}
\newcommand\CD{\mathcal D}
\newcommand\CE{\mathcal E}
\newcommand\CF{\mathcal F}
\newcommand\CG{\mathcal G}
\newcommand\CH{\mathcal H}
\newcommand\CI{\mathcal I}
\newcommand\CJ{\mathcal J}
\newcommand\CK{\mathcal K}
\newcommand\CL{\mathcal L}
\newcommand\CM{\mathcal M}
\newcommand\CN{\mathcal N}
\newcommand\CO{\mathcal O}
\newcommand\CP{\mathcal P}
\newcommand\CQ{\mathcal Q}
\newcommand\CR{\mathcal R}
\newcommand\CS{\mathcal S}
\newcommand\CT{\mathcal T}
\newcommand\CU{\mathcal U}
\newcommand\CV{\mathcal V}
\newcommand\CW{\mathcal W}
\newcommand\CX{\mathcal X}
\newcommand\CY{\mathcal Y}
\newcommand\CZ{\mathcal Z}

\newcommand\calA{\mathcal A}
\newcommand\calB{\mathcal B}
\newcommand\calC{\mathcal C}
\newcommand\calD{\mathcal D}
\newcommand\calE{\mathcal E}
\newcommand\calF{\mathcal F}
\newcommand\calG{\mathcal G}
\newcommand\calH{\mathcal H}
\newcommand\calI{\mathcal I}
\newcommand\calJ{\mathcal J}
\newcommand\calK{\mathcal K}
\newcommand\calL{\mathcal L}
\newcommand\calM{\mathcal M}
\newcommand\calN{\mathcal N}
\newcommand\calO{\mathcal O}
\newcommand\calP{\mathcal P}
\newcommand\calQ{\mathcal Q}
\newcommand\calR{\mathcal R}
\newcommand\calS{\mathcal S}
\newcommand\calT{\mathcal T}
\newcommand\calU{\mathcal U}
\newcommand\calV{\mathcal V}
\newcommand\calW{\mathcal W}
\newcommand\calX{\mathcal X}
\newcommand\calY{\mathcal Y}
\newcommand\calZ{\mathcal Z}

\newcommand\FA{\mathfrak A}
\newcommand\FB{\mathfrak B}
\newcommand\FC{\mathfrak C}
\newcommand\FD{\mathfrak D}
\newcommand\FE{\mathfrak E}
\newcommand\FF{\mathfrak F}
\newcommand\FG{\mathfrak G}
\newcommand\FH{\mathfrak H}
\newcommand\FI{\mathfrak I}
\newcommand\FJ{\mathfrak J}
\newcommand\FK{\mathfrak K}
\newcommand\FL{\mathfrak L}
\newcommand\FM{\mathfrak M}
\newcommand\FN{\mathfrak N}
\newcommand\FO{\mathfrak O}
\newcommand\FP{\mathfrak P}
\newcommand\FQ{\mathfrak Q}
\newcommand\FR{\mathfrak R}
\newcommand\FS{\mathfrak S}
\newcommand\FT{\mathfrak T}
\newcommand\FU{\mathfrak U}
\newcommand\FV{\mathfrak V}
\newcommand\FW{\mathfrak W}
\newcommand\FX{\mathfrak X}
\newcommand\FY{\mathfrak Y}
\newcommand\FZ{\mathfrak Z}

\newcommand\Fa{\mathfrak a}
\newcommand\Fb{\mathfrak b}
\newcommand\Fc{\mathfrak c}
\newcommand\Fd{\mathfrak d}
\newcommand\Fe{\mathfrak e}
\newcommand\Ff{\mathfrak f}
\newcommand\Fg{\mathfrak g}
\newcommand\Fh{\mathfrak h}
\newcommand\Fi{\mathfrak i}
\newcommand\Fj{\mathfrak j}
\newcommand\Fk{\mathfrak k}
\newcommand\Fl{\mathfrak l}
\newcommand\Fm{\mathfrak m}
\newcommand\Fn{\mathfrak n}
\newcommand\Fo{\mathfrak o}
\newcommand\Fp{\mathfrak p}
\newcommand\Fq{\mathfrak q}
\newcommand\Fr{\mathfrak r}
\newcommand\Fs{\mathfrak s}
\newcommand\Ft{\mathfrak t}
\newcommand\Fu{\mathfrak u}
\newcommand\Fv{\mathfrak v}
\newcommand\Fw{\mathfrak w}
\newcommand\Fx{\mathfrak x}
\newcommand\Fy{\mathfrak y}
\newcommand\Fz{\mathfrak z}

\newcommand\frakA{\mathfrak A}
\newcommand\frakB{\mathfrak B}
\newcommand\frakC{\mathfrak C}
\newcommand\frakD{\mathfrak D}
\newcommand\frakE{\mathfrak E}
\newcommand\frakF{\mathfrak F}
\newcommand\frakG{\mathfrak G}
\newcommand\frakH{\mathfrak H}
\newcommand\frakI{\mathfrak I}
\newcommand\frakJ{\mathfrak J}
\newcommand\frakK{\mathfrak K}
\newcommand\frakL{\mathfrak L}
\newcommand\frakM{\mathfrak M}
\newcommand\frakN{\mathfrak N}
\newcommand\frakO{\mathfrak O}
\newcommand\frakP{\mathfrak P}
\newcommand\frakQ{\mathfrak Q}
\newcommand\frakR{\mathfrak R}
\newcommand\frakS{\mathfrak S}
\newcommand\frakT{\mathfrak T}
\newcommand\frakU{\mathfrak U}
\newcommand\frakV{\mathfrak V}
\newcommand\frakW{\mathfrak W}
\newcommand\frakX{\mathfrak X}
\newcommand\frakY{\mathfrak Y}
\newcommand\frakZ{\mathfrak Z}

\newcommand\fraka{\mathfrak a}
\newcommand\frakb{\mathfrak b}
\newcommand\frakc{\mathfrak c}
\newcommand\frakd{\mathfrak d}
\newcommand\frake{\mathfrak e}
\newcommand\frakf{\mathfrak f}
\newcommand\frakg{\mathfrak g}
\newcommand\frakh{\mathfrak h}
\newcommand\fraki{\mathfrak i}
\newcommand\frakj{\mathfrak j}
\newcommand\frakk{\mathfrak k}
\newcommand\frakl{\mathfrak l}
\newcommand\frakm{\mathfrak m}
\newcommand\frakn{\mathfrak n}
\newcommand\frako{\mathfrak o}
\newcommand\frakp{\mathfrak p}
\newcommand\frakq{\mathfrak q}
\newcommand\frakr{\mathfrak r}
\newcommand\fraks{\mathfrak s}
\newcommand\frakt{\mathfrak t}
\newcommand\fraku{\mathfrak u}
\newcommand\frakv{\mathfrak v}
\newcommand\frakw{\mathfrak w}
\newcommand\frakx{\mathfrak x}
\newcommand\fraky{\mathfrak y}
\newcommand\frakz{\mathfrak z}

\newcommand\figleft{{\scshape{Left}}}
\newcommand\figmiddle{{\scshape{Middle}}}
\newcommand\figright{{\scshape{Right}}}



\documentclass[dvipsnames,11pt,reqno,oneside,draft]{amsart}
\usepackage[a4paper,left=1cm,right=5cm,top=3cm,bottom=3cm]{geometry}
\usepackage[T1]{fontenc}
\usepackage[utf8]{inputenc}
\usepackage{xcolor}
\usepackage{amssymb,mathtools}  % Also loads "amsmath"
\usepackage{dsfont}
\usepackage{enumitem}
\usepackage{subcaption}
\usepackage{booktabs}
\newcommand\numberthis{\addtocounter{equation}{1}\tag{\theequation}}
\usepackage{preamble/mhequ}

\usepackage{amsthm}


\newcounter{counterEnvMain}
\newcounter{counterEnvDefault}
\numberwithin{counterEnvDefault}{section}


% ====================
\theoremstyle{plain}

\newtheorem{mainlemma}[counterEnvMain]{Lemma}
\newtheorem{lemma}[counterEnvDefault]{Lemma}
\newtheorem*{lemma*}{Lemma}

\newtheorem{maintheorem}[counterEnvMain]{Theorem}
\newtheorem{theorem}[counterEnvDefault]{Theorem}

\newtheorem{proposition}[counterEnvDefault]{Proposition}
\newtheorem{corollary}[counterEnvDefault]{Corollary}
\newtheorem{assumption}[counterEnvDefault]{Assumption}


% ====================
\theoremstyle{definition}

\newtheorem{definition}[counterEnvDefault]{Definition}
\newtheorem*{definition*}{Definition}

\newtheorem{example}[counterEnvDefault]{Example}
\newtheorem*{example*}{Example}


\newtheorem{exercise}[counterEnvDefault]{Exercise}
\newtheorem*{exercise*}{Exercise}

\newtheorem{remark}[counterEnvDefault]{Remark}

\newtheorem*{claim*}{Claim}

\newtheorem*{assertion*}{Assertion}

\newtheorem*{proposition*}{Proposition}

\usepackage{microtype}
\renewcommand\phi\varphi
\renewcommand\epsilon\varepsilon

\usepackage{hyperref}
\definecolor{colorlinks}{RGB}{0, 24, 168}
\definecolor{colorcites}{RGB}{124, 10, 2}
\hypersetup{
    colorlinks=true,
    linkcolor=colorlinks,
    citecolor=colorcites,
    urlcolor=colorlinks,
    pdfborder={0 0 0}
}

\usepackage{xargs}
\usepackage[colorinlistoftodos,prependcaption,textsize=tiny]{todonotes}

\newcommandx\work[2][1=]{\todo[linecolor=RoyalBlue,backgroundcolor=RoyalBlue!25,bordercolor=RoyalBlue,#1]{\textsc{todo} #2}}
\newcommandx\comment[2][1=]{\todo[linecolor=OliveGreen,backgroundcolor=OliveGreen!25,bordercolor=OliveGreen,#1]{\textsc{comment} #2}}
\newcommandx\mistake[2][1=]{\todo[linecolor=red,backgroundcolor=red!25,bordercolor=red,#1]{\textsc{mistake} #2}}
\newcommandx\improve[2][1=]{\todo[linecolor=orange,backgroundcolor=orange!25,bordercolor=orange,#1]{\textsc{improve} #2}}
\newcommandx\change[2][1=]{\todo[linecolor=yellow,backgroundcolor=yellow!25,bordercolor=yellow,#1]{\textsc{change} #2}}
\newcommandx\mem[2][1=]{\todo[linecolor=orange,backgroundcolor=orange!25,bordercolor=orange,#1]{\textsc{mem} #2}}
\newcommandx\status[2][1=]{\todo[linecolor=Blue,backgroundcolor=Blue!25,bordercolor=Blue,#1]{\textsc{Status} #2}}

\newcommand\hidetodos{
    \renewcommandx\todo[2][1=]{}
    \renewcommandx\work[2][1=]{}
    \renewcommandx\comment[2][1=]{}
    \renewcommandx\mistake[2][1=]{}
    \renewcommandx\improve[2][1=]{}
    \renewcommandx\change[2][1=]{}
    \renewcommandx\mem[2][1=]{}
    \renewcommandx\status[2][1=]{}
}

\newcommand\blank{\,\cdot\,}
\newcommand\ssubset{\Subset}

\newcommand\diam{\operatorname{diam}}
\newcommand\Var{\operatorname{Var}}
\newcommand\Cov{\operatorname{Cov}}

\newcommand\ind[1]{\mathds{1}_{#1}}
\newcommand\true[1]{\mathds{1}({#1})}
\newcommand\diffi{{\,\mathrm{d}}}
\newcommand\diff{{\mathrm{d}}}

\newcommand\A{\mathbb A}
\newcommand\B{\mathbb B}
\newcommand\C{\mathbb C}
\newcommand\D{\mathbb D}
\newcommand\E{\mathbb E}
\newcommand\F{\mathbb F}
\newcommand\G{\mathbb G}
\renewcommand\H{\mathbb H}
\newcommand\I{\mathbb I}
\newcommand\J{\mathbb J}
\newcommand\K{\mathbb K}
\renewcommand\L{\mathbb L}
\newcommand\M{\mathbb M}
\newcommand\N{\mathbb N}
\renewcommand\O{\mathbb O}
\renewcommand\P{\mathbb P}
\newcommand\Q{\mathbb Q}
\newcommand\R{\mathbb R}
\renewcommand\S{\mathbb S}
\newcommand\T{\mathbb T}
\newcommand\U{\mathbb U}
\newcommand\V{\mathbb V}
\newcommand\W{\mathbb W}
\newcommand\X{\mathbb X}
\newcommand\Y{\mathbb Y}
\newcommand\Z{\mathbb Z}

\newcommand\bbA{\mathbb A}
\newcommand\bbB{\mathbb B}
\newcommand\bbC{\mathbb C}
\newcommand\bbD{\mathbb D}
\newcommand\bbE{\mathbb E}
\newcommand\bbF{\mathbb F}
\newcommand\bbG{\mathbb G}
\newcommand\bbH{\mathbb H}
\newcommand\bbI{\mathbb I}
\newcommand\bbJ{\mathbb J}
\newcommand\bbK{\mathbb K}
\newcommand\bbL{\mathbb L}
\newcommand\bbM{\mathbb M}
\newcommand\bbN{\mathbb N}
\newcommand\bbO{\mathbb O}
\newcommand\bbP{\mathbb P}
\newcommand\bbQ{\mathbb Q}
\newcommand\bbR{\mathbb R}
\newcommand\bbS{\mathbb S}
\newcommand\bbT{\mathbb T}
\newcommand\bbU{\mathbb U}
\newcommand\bbV{\mathbb V}
\newcommand\bbW{\mathbb W}
\newcommand\bbX{\mathbb X}
\newcommand\bbY{\mathbb Y}
\newcommand\bbZ{\mathbb Z}

\newcommand\CA{\mathcal A}
\newcommand\CB{\mathcal B}
\newcommand\CC{\mathcal C}
\newcommand\CD{\mathcal D}
\newcommand\CE{\mathcal E}
\newcommand\CF{\mathcal F}
\newcommand\CG{\mathcal G}
\newcommand\CH{\mathcal H}
\newcommand\CI{\mathcal I}
\newcommand\CJ{\mathcal J}
\newcommand\CK{\mathcal K}
\newcommand\CL{\mathcal L}
\newcommand\CM{\mathcal M}
\newcommand\CN{\mathcal N}
\newcommand\CO{\mathcal O}
\newcommand\CP{\mathcal P}
\newcommand\CQ{\mathcal Q}
\newcommand\CR{\mathcal R}
\newcommand\CS{\mathcal S}
\newcommand\CT{\mathcal T}
\newcommand\CU{\mathcal U}
\newcommand\CV{\mathcal V}
\newcommand\CW{\mathcal W}
\newcommand\CX{\mathcal X}
\newcommand\CY{\mathcal Y}
\newcommand\CZ{\mathcal Z}

\newcommand\calA{\mathcal A}
\newcommand\calB{\mathcal B}
\newcommand\calC{\mathcal C}
\newcommand\calD{\mathcal D}
\newcommand\calE{\mathcal E}
\newcommand\calF{\mathcal F}
\newcommand\calG{\mathcal G}
\newcommand\calH{\mathcal H}
\newcommand\calI{\mathcal I}
\newcommand\calJ{\mathcal J}
\newcommand\calK{\mathcal K}
\newcommand\calL{\mathcal L}
\newcommand\calM{\mathcal M}
\newcommand\calN{\mathcal N}
\newcommand\calO{\mathcal O}
\newcommand\calP{\mathcal P}
\newcommand\calQ{\mathcal Q}
\newcommand\calR{\mathcal R}
\newcommand\calS{\mathcal S}
\newcommand\calT{\mathcal T}
\newcommand\calU{\mathcal U}
\newcommand\calV{\mathcal V}
\newcommand\calW{\mathcal W}
\newcommand\calX{\mathcal X}
\newcommand\calY{\mathcal Y}
\newcommand\calZ{\mathcal Z}

\newcommand\FA{\mathfrak A}
\newcommand\FB{\mathfrak B}
\newcommand\FC{\mathfrak C}
\newcommand\FD{\mathfrak D}
\newcommand\FE{\mathfrak E}
\newcommand\FF{\mathfrak F}
\newcommand\FG{\mathfrak G}
\newcommand\FH{\mathfrak H}
\newcommand\FI{\mathfrak I}
\newcommand\FJ{\mathfrak J}
\newcommand\FK{\mathfrak K}
\newcommand\FL{\mathfrak L}
\newcommand\FM{\mathfrak M}
\newcommand\FN{\mathfrak N}
\newcommand\FO{\mathfrak O}
\newcommand\FP{\mathfrak P}
\newcommand\FQ{\mathfrak Q}
\newcommand\FR{\mathfrak R}
\newcommand\FS{\mathfrak S}
\newcommand\FT{\mathfrak T}
\newcommand\FU{\mathfrak U}
\newcommand\FV{\mathfrak V}
\newcommand\FW{\mathfrak W}
\newcommand\FX{\mathfrak X}
\newcommand\FY{\mathfrak Y}
\newcommand\FZ{\mathfrak Z}

\newcommand\Fa{\mathfrak a}
\newcommand\Fb{\mathfrak b}
\newcommand\Fc{\mathfrak c}
\newcommand\Fd{\mathfrak d}
\newcommand\Fe{\mathfrak e}
\newcommand\Ff{\mathfrak f}
\newcommand\Fg{\mathfrak g}
\newcommand\Fh{\mathfrak h}
\newcommand\Fi{\mathfrak i}
\newcommand\Fj{\mathfrak j}
\newcommand\Fk{\mathfrak k}
\newcommand\Fl{\mathfrak l}
\newcommand\Fm{\mathfrak m}
\newcommand\Fn{\mathfrak n}
\newcommand\Fo{\mathfrak o}
\newcommand\Fp{\mathfrak p}
\newcommand\Fq{\mathfrak q}
\newcommand\Fr{\mathfrak r}
\newcommand\Fs{\mathfrak s}
\newcommand\Ft{\mathfrak t}
\newcommand\Fu{\mathfrak u}
\newcommand\Fv{\mathfrak v}
\newcommand\Fw{\mathfrak w}
\newcommand\Fx{\mathfrak x}
\newcommand\Fy{\mathfrak y}
\newcommand\Fz{\mathfrak z}

\newcommand\frakA{\mathfrak A}
\newcommand\frakB{\mathfrak B}
\newcommand\frakC{\mathfrak C}
\newcommand\frakD{\mathfrak D}
\newcommand\frakE{\mathfrak E}
\newcommand\frakF{\mathfrak F}
\newcommand\frakG{\mathfrak G}
\newcommand\frakH{\mathfrak H}
\newcommand\frakI{\mathfrak I}
\newcommand\frakJ{\mathfrak J}
\newcommand\frakK{\mathfrak K}
\newcommand\frakL{\mathfrak L}
\newcommand\frakM{\mathfrak M}
\newcommand\frakN{\mathfrak N}
\newcommand\frakO{\mathfrak O}
\newcommand\frakP{\mathfrak P}
\newcommand\frakQ{\mathfrak Q}
\newcommand\frakR{\mathfrak R}
\newcommand\frakS{\mathfrak S}
\newcommand\frakT{\mathfrak T}
\newcommand\frakU{\mathfrak U}
\newcommand\frakV{\mathfrak V}
\newcommand\frakW{\mathfrak W}
\newcommand\frakX{\mathfrak X}
\newcommand\frakY{\mathfrak Y}
\newcommand\frakZ{\mathfrak Z}

\newcommand\fraka{\mathfrak a}
\newcommand\frakb{\mathfrak b}
\newcommand\frakc{\mathfrak c}
\newcommand\frakd{\mathfrak d}
\newcommand\frake{\mathfrak e}
\newcommand\frakf{\mathfrak f}
\newcommand\frakg{\mathfrak g}
\newcommand\frakh{\mathfrak h}
\newcommand\fraki{\mathfrak i}
\newcommand\frakj{\mathfrak j}
\newcommand\frakk{\mathfrak k}
\newcommand\frakl{\mathfrak l}
\newcommand\frakm{\mathfrak m}
\newcommand\frakn{\mathfrak n}
\newcommand\frako{\mathfrak o}
\newcommand\frakp{\mathfrak p}
\newcommand\frakq{\mathfrak q}
\newcommand\frakr{\mathfrak r}
\newcommand\fraks{\mathfrak s}
\newcommand\frakt{\mathfrak t}
\newcommand\fraku{\mathfrak u}
\newcommand\frakv{\mathfrak v}
\newcommand\frakw{\mathfrak w}
\newcommand\frakx{\mathfrak x}
\newcommand\fraky{\mathfrak y}
\newcommand\frakz{\mathfrak z}

\newcommand\figleft{{\scshape{Left}}}
\newcommand\figmiddle{{\scshape{Middle}}}
\newcommand\figright{{\scshape{Right}}}



\mathtoolsset{showonlyrefs}

\makeatletter
\@namedef{subjclassname@2020}{\textup{2020} Mathematics Subject Classification}
\makeatother


\title{A course on the Ising model}
\subjclass[2020]{[Mathematics classification]}


\author{Piet Lammers}
\address{CNRS and Sorbonne Université, LPSM}
\email{piet.lammers@cnrs.fr}

\date{\today}
% \keywords{%
%     [keyword 1],
%     [keyword 2]
% }

\newcommand\n{\mathbf{n}}
\newcommand\m{\mathbf{m}}
\renewcommand\a{\mathbf{a}}
\renewcommand\b{\mathbf{b}}

\thanks{This work is licensed under CC BY-NC-SA 4.0. This work is licensed under CC BY-NC-SA 4.0. To view a copy of this license, visit \url{https://creativecommons.org/licenses/by-nc-sa/4.0/}}

\begin{document}


\maketitle

\tableofcontents
% \begin{abstract}
%     [Abstract text]
% \end{abstract}


\section*{Preface}
These lecture notes are progressively written during the 2025 spring semester,
as the course is taught at Sorbonne university in the M2 (second-year masters) programme.
Its purpose is to give a broad introduction to the rigorous analysis of the Ising model.
The main focus is on four techniques and their applications:
\begin{itemize}
    \item The Peierls argument,
    \item The random-currents representation,
    \item The FKG inequality for the Ising spins,
    \item The FKG inequality for the random-cluster (FK) representation.
\end{itemize}

A basic understanding of analysis and probability theory is essential for following this course.
Experience with other models in statistical mechanics (such as the Bernoulli percolation model)
is a plus but by no means essential.

\section{Introduction}
\label{sec:intro}

The Ising model is the archetypal model for the study of phase transitions in
mathematical physics.
It was first introduced by Wilhelm Lenz in 1920 and later solved by Ernst Ising
in 1924 in the one-dimensional case.
The model consists of a lattice of spins,
each of which can be in one of two states,
up or down.
Informally,
one may think of these spins as the magnetic moments of atoms in a ferromagnetic material.
The behaviour of this probabilistic model depends strongly on a few different parameters:
\begin{itemize}
    \item The dimension $d$ of the lattice $\Z^d$ on which the spins are placed,
    \item The interaction strength $\beta$ between neighbouring spins,
    \item The way that boundary conditions are imposed,
    \item The strength of external magnetic field $h$.
\end{itemize}
In fact, we shall start by defining the Ising model on arbitrary finite graphs.
We shall now give a definition of the Ising model, although
we keep boundary conditions and external magnetic fields for later.

\begin{definition}[Ising model on a finite graph]
    The Ising model on a finite graph \( G = (V, E) \) with \emph{inverse temperature} \( \beta \in [0,\infty) \) is defined as follows.
    Let $\Omega:=\{\pm1\}^V$ denote the set of spin configurations on the vertices of the graph;
    a typical element of $\Omega$ is denoted by $\sigma=(\sigma_u)_{u\in V}$.
    Elements $\sigma\in\Omega$ are called \emph{spin configurations};
    elements $\sigma_u$ are called \emph{spins}.
    The \emph{energy} or \emph{Hamiltonian} of a spin configuration $\sigma$ is given by
    \[
        H_{G,\beta}^{\operatorname{Ising}}(\sigma) := -\beta \sum_{uv \in E} \sigma_u \sigma_v.
    \]
    We write $\P_{G,\beta}^{\operatorname{Ising}}$ for the associated \emph{Boltzmann distribution} or \emph{Gibbs measure}:
    \[
        \P_{G,\beta}^{\operatorname{Ising}}(\sigma) := \frac{1}{Z_{G,\beta}^{\operatorname{Ising}}} e^{-H^{\operatorname{Ising}}_{G,\beta}(\sigma)},
    \]
    where \(Z_{G,\beta}^{\operatorname{Ising}}\) is normalisation constant or \emph{partition function} defined by
    \[
        Z_{G,\beta}^{\operatorname{Ising}}:= \sum_{\sigma\in\Omega} e^{-H^{\operatorname{Ising}}_{G,\beta}(\sigma)}.
    \]
    We shall write $\langle\blank\rangle_{G,\beta}^{\operatorname{Ising}}$ for the expectation functional associated to this probability measure.
\end{definition}

\begin{remark}[Flip-symmetry]
    The Ising model is \emph{flip-symmetric} in the sense that the distribution of the spins is invariant under the transformation $\sigma\mapsto-\sigma$.
    This is because the Hamiltonian is invariant under this transformation.
\end{remark}

\begin{remark}
    We shall often suppress subscripts and superscripts when they are clear from the context.
\end{remark}

\begin{remark}
    Adding a constant to the Hamiltonian does not change the distribution of the Ising model,
    even though it affects the partition function.
\end{remark}

\begin{remark}
    The mathematical community has widely adopted the terminology coming from the physics
    literature.
\end{remark}

\section{The Curie--Weiss model}
\label{sec:definitions_examples}

At the end of the 19th century, Curie published his experimental results on 
\emph{ferromagnetism}:
the
magnetic properties of metals.
He made three striking observations.
\begin{itemize}
    \item The magnetic strength of a metal varies with the temperature.
    Increasing the temperature decreases the magnetic strength.
    \item Each metal has a certain temperature, specific to that metal,
    at which the magnetic properties disappear entirely.
    We call this temperature the \emph{Curie temperature}.
    \item Around the Curie temperature, the magnetic strength drops continuously
    to zero. In other words, the magnetic strength does not ``jump'' to zero.
\end{itemize}
The first observation singles out the temperature as the driving parameter of the system.
This is good news for us, since the temperature may be regarded informally as the amount of 
``randomness'' or ``entropy'' in the system, justifying a probabilistic analysis of the situation.
The second observation implies that there is a \emph{phase transition}:
there is a special temperature (in this case the Curie temperature) at which
the system's behaviour undergoes a qualitative change.
The third observation entails an important property of this phase transition.

The first mathematical explanation for Curie's experimental results came from
Weiss.
He proposed the following mathematical axioms for studying the magnetic properties of metals.
\begin{itemize}
    \item The metal consists of $n$ atoms.
    \item Each atom acts like a small magnet in itself.
    It is in one of two states, denoted $\pm$.
    \item The total strength of the metal is obtained by summing the states of all atoms.
    \item Each atom interacts with all other atoms.
    The atoms prefer to \emph{align}, that is, to be in the same state.
    The temperature regulates the strength of the interaction.
\end{itemize}
Physically, it makes sense that the temperature regulates the interaction strength.
When atoms move slowly, they will stabilise, oriented in alignment with the magnetic field imposed by the other atoms.
When atoms move fast, they will not bother with the states of the other atoms, and simply align themselves randomly.
It is thus natural to think of the interaction strength as the \emph{inverse temperature}.

\begin{definition}[Curie-Weiss model]
    The Curie-Weiss model is the probability measure $\P^{\operatorname{CW}}_{n,\beta}$
    on $\sigma\in\Omega:=\{+,-\}^n$
    defined via
    \[
        \P(\sigma):=\P^{\operatorname{CW}}_{n,\beta}(\sigma)
        \propto
        e^{-H^{\operatorname{CW}}_{n,\beta}(\sigma)}
        ;
        \qquad
        H(\sigma):=H^{\operatorname{CW}}_{n,\beta}(\sigma):=
        -\frac\beta{n} \sum_{i<j}\sigma_i\sigma_j,
    \]
    where $n\in\Z_{\geq 1}$ and $\beta\in[0,\infty)$.
    The parameter $\beta$ is called the \emph{interaction strength} or \emph{inverse temperature}.
    The function $H$ is called the \emph{Hamiltonian} and captures the \emph{energy} in the system.
    The probability measure $\P$ is also called the \emph{Boltzmann distribution}.
\end{definition}

Let $n_+=n_+(\sigma)$ denote the number of vertices with spin $+$ in a configuration $\sigma\in\Omega$.
This is a random variable.
Let us try to calculate the probability of the event $\{n_+=k\}$,
without worrying about the partition function (the normalising constant).
One may easily check that
the Hamiltonian satisfies
\begin{align}
    H(\sigma)=2\frac\beta{n} n_+(n-n_+) + \text{const}(n).
\end{align}
The distribution of $n_+$ can then be calculated as follows:
\begin{align}
    \label{eq:CurieWeissDistribution}
    \P(\{n_+ = k\}) &\propto \binom{n}{k} e^{-2\frac\beta{n}  k(n-k)}
    \propto \frac1{k! (n-k)!} e^{-2\frac\beta{n} k (n-k)}.
\end{align}
Using Stirling's approximation for the factorials, we find that
\begin{gather}
    \log\P(\{n_+ = k\})
    \stackrel{\text{Stirling}}{\approx}
    -n f_{\beta}(k/n) + \text{const}(n);
    \\
    f_{\beta}:[0,1]\to\R,\,
    x\mapsto x \log x + (1-x)\log (1-x) + 2 \beta x (1-x).
\end{gather}
If we fix $\beta$ and send $n$ to infinity, then 
we discover a large deviations principle for the random variable $n_+/n$
with rate function $f_{\beta}$ and speed $n$.
In particular, the random variable $n_+/n$ concentrates
around the minimisers of the function $f_{\beta}$.

\begin{exercise}[The rate function of the Curie--Weiss model]
    \begin{enumerate}
        \item Show that for small $\beta$, the function \( f_{\beta} \) has a single minimum at $x=1/2$, which means that the random variable \( n_+/n \) concentrates around the value \( 1/2 \).
        \item Show that for large $\beta$, the function \( f_{\beta} \) has two minima at \( (1 \pm m)/2 \) for some $m>0$, which means that the random variable \( n_+/n \) is concentrated around these minima.
        The value of $m$ is called the \emph{magnetisation}.
        \item Calculate the critical value for $\beta$. At this value, the second derivative of \( f_{\beta} \) vanishes at \( x=1/2 \). What does this mean for the distribution of \( n_+/n \)?
        Estimate the order of magnitude of $\Var\frac{n_+}{n}$ as $n\to\infty$ for this value of $\beta$.
    \end{enumerate}
\end{exercise}

\begin{remark}[Entropy versus energy in the Curie--Weiss model]
    Reconsider Equation~\eqref{eq:CurieWeissDistribution}.
    In this equation, the competition between the two factors is extremely transparent.
    \begin{itemize}
        \item     First, there is a combinatorial term or \emph{entropy}, which favours values $k$ for the random variable
        $n_+$ such that the cardinality of the set $\{n_+=k\}$ is large.
        This means that values $k\approx n/2$ are preferred.
        \item     Second, there is the \emph{energy} term, which favours values such that the energy 
        is minimised. This favours configurations where as many spins as possible align.    
    \end{itemize}
    The interaction parameter $\beta$ allows us to put more emphasis
    on the entropy term or on the energy term.
    In the $n\to\infty$ limit, there is a precise value for $\beta$
    where the behaviour of the random system undergoes a qualitative change:
    a rudimentary example of a \emph{phase transition}.
\end{remark}

\section{Early developments. 1924: Ising's analysis}
\label{sec:ising_1d}

Wilhelm Lenz challenged his doctoral student Ernst Ising
to solve the model of interest on the one-dimensional line graph $\Z$.
Readers acquainted with percolation theory
will suspect that such a simple model is unlikely to exhibit
a phase transition.
This suspicion is correct,
but we stress that the percolation model had not yet
been described at the time that Ising undertook his doctoral research.

We defined the Ising model in the previous section,
but only on finite graphs.
Let us extend this definition to infinite graphs.

\begin{definition}[The Ising model with boundary conditions]
    Let $G=(V,E)$ denote a fixed locally finite graph.
    We consider the measurable space 
    $(\Omega,\calF)$ where $\Omega=\{\pm1\}^V$
    and where $\calF$ is the product-$\sigma$-algebra.

    Let $\Lambda$ denote a \emph{domain},
    that is, a finite subset of $V$.
    The Ising model in the domain $\Lambda$
    with inverse temperature $\beta \geq 0$ and boundary conditions
    $\zeta\in\{\pm1\}^{V\setminus\Lambda}$
    is the probability measure $\P_{\Lambda,\beta}^{\operatorname{Ising},\zeta}$
    on $(\Omega,\calF)$ defined via:
    \[
        \P_{\Lambda,\beta}^{\operatorname{Ising},\zeta}(\sigma)
        =
        \frac1{Z_{\Lambda,\beta}^{\operatorname{Ising},\zeta}}
        \cdot
        \true{\sigma|_{V\setminus\Lambda}=\zeta}\cdot e^{-H_{\Lambda,\beta}^{\operatorname{Ising}}(\sigma)}
        ,
    \]
    where $H_{\Lambda,\beta}^{\operatorname{Ising}}(\sigma)$ is the Hamiltonian given by
    \[
        H_{\Lambda,\beta}^{\operatorname{Ising}}(\sigma)
        =
        -\beta\sum_{uv\in E(\Lambda)}\sigma_u\sigma_v
        ,
    \]
    and where the partition function $Z_{\Lambda,\beta}^{\operatorname{Ising},\zeta}$ is given by
    \[
        Z_{\Lambda,\beta}^{\operatorname{Ising},\zeta}
        =
        \sum_{\sigma \in \Omega} \true{\sigma|_{V\setminus\Lambda}=\zeta} \cdot e^{-H_{\Lambda,\beta}^{\operatorname{Ising}}(\sigma)}
        .
    \]
    The set $E(\Lambda)\subset E$ denotes 
    the set of edges with at least one endpoint in $\Lambda$.
    Indeed, adding a constant to the Hamiltonian does not affect the measure,
    and edges which do not intersect $\Lambda$ contribute with a constant.

    We write $+$ and $-$ for the boundary conditions
    $+1\in\Omega$ and $-1\in\Omega$ respectively.
\end{definition}

\begin{theorem}[Ising, 1924]
    The one-dimensional Ising model is demagnetised at all temperatures.
    This means the following.
    Let $G=(V,E)$ denote the one-dimensional lattice $\Z$,
    and define $\Lambda_n:=\{-n+1,\ldots,n-1\}$.
    Then, for any $\beta\geq 0$, we have
    \[
        \lim_{n\to\infty}\langle\sigma_0\rangle_{\Lambda_n,\beta}^+
        =
        0
        .
    \]
\end{theorem}

\begin{proof}
    Write $T$ for the matrix
    \[
        T:=
        \begin{pmatrix}
            e^{\beta} & e^{-\beta} \\
            e^{-\beta} & e^{\beta}
        \end{pmatrix}.
    \]
    It is straightforward to work out that
    \begin{gather}
        Z_{\Lambda_n,\beta}^+\langle\sigma_0\rangle_{\Lambda_n,\beta}^+
        =
        \left(T^n
            \begin{pmatrix}
                1 & 0 \\ 0 & -1
            \end{pmatrix}
            T^n
            \right)_{1,1};
        \\
        Z_{\Lambda_n,\beta}^+ = \left(T^{2n}\right)_{1,1},
    \end{gather}
    see the exercise below.
    One may then conclude that the ratio of these two numbers tends to zero 
    with $n\to\infty$
    by simply diagonalising $T$.
\end{proof}

\begin{remark}
    Although the intuition is reminiscent
    of the theory of Markov chains, we stress that the matrix
    $T$ above is \emph{not} a stochastic matrix.
    This is why we need to consider the partition function
    (normalising constant) separately.
\end{remark}

\begin{exercise}
    Let $(f_k)_k$ denote a family of functions
    of the form $f_k:\{+1,-1\}\to\R$.
    For any $k$, define
    \[ M_k:=\begin{pmatrix}
        f_{k}(+1) & 0 \\
        0 & f_{k}(-1)
    \end{pmatrix}.\] 
    Prove that for any $n$,
    we have
    \[
        Z_{\Lambda_n,\beta}^+
        \langle\textstyle\prod_{k\in\Lambda_n} f_k(\sigma_{k})\rangle_{\Lambda_n,\beta}^+
        =
        \left(
            T
            M_{n-1}
            T
            M_{n-2}
            T
            \cdots
            T
            M_{-n+1}
            T
        \right)_{1,1}.
    \]
\end{exercise}

\begin{remark}
    Ising conjectured that the absence of magnetisation in the one-dimensional
    model would also hold in higher dimensions. This was later shown to be
    false: Peierls proved in 1936 that the two-dimensional model magnetises for
    sufficiently large $\beta$.
\end{remark}

\section{Early developments. 1936: Peierls' argument}
\label{sec:peierls}

\begin{theorem}[Peierls, 1936]
    \label{thm:peierls}
    The Ising model exhibits magnetisation in two dimensions.
\end{theorem}

We shall discuss a slight variation of Peierls' original setup,
so that we can fully focus the proof on the core idea.
Let $\T$ denote the triangular lattice graph,
comprised of vertices of the form
\[
    \T:=\left\{n+m e^{\pi i/3}:n,m\in\Z \right\}\subset\C,
\]
and such that each vertex is connected to the six
vertices at distance one.
Let $\Lambda_n\subset\T$ denote the set of vertices at a graph
distance at most $n-1$ from $0\in\T$.
We consider the Ising model on the infinite graph $\T$.
We shall prove the following version of Peierls' result.

\begin{theorem}[Peierls, 1936]
    \label{thm:peierls_triangles}
    Consider the Ising model on the two-dimensional
    triangular lattice graph $\T$.
    For sufficiently large $\beta$,
    we have
    \[
        \inf_{n}\langle\sigma_0\rangle_{\Lambda_n,\beta}^+
        >0.
    \]
\end{theorem}

Let $\H:=\T^*$ denote the hexagonal lattice
that is dual to the triangular lattice.
For a fixed configuration $\sigma\in\Omega$,
we let $\calI(\sigma)\subset E(\H)$ denote the set of
hexagonal lattice edges separating hexagons with different spins.
The set $\calI(\sigma)$ is called the
\emph{interface} between the spins valued $+1$
and those valued $-1$.
\todo{Add figure}
Notice that $\calI(\sigma)$ has a partition into
loops and bi-infinite paths.
If only finitely many spins of $\sigma$ are valued $-1$,
then there are no bi-infinite paths,
and all connected components of $\calI(\sigma)$
are loops.
This happens almost surely when sampling from $\langle\blank\rangle_{\Lambda_n,\beta}^+$.

The core of Peierls' argument is the following lemma.

\begin{lemma}[Exponential decay of loop lengths]
    \label{lem:exp_decay_ising_loops}
    Consider the Ising model on the two-dimensional triangular lattice $\T$
    at inverse temperature $\beta$.
    Suppose that $e^{-2\beta}<\frac12$.
    Then for any hexagonal lattice edge $e\in E(\H)$
    and for any minimal loop length $\ell\in\Z_{\geq 1}$,
    we get
    \[
        \P_{\Lambda_n,\beta}^+
        (\{\text{$\calI(\sigma)$ has a loop of length at least $\ell$ through $e$}\})
        \leq \frac{(2e^{-2\beta})^\ell}{1-2e^{-2\beta}},
    \]
    uniformly in $n$.
\end{lemma}

\begin{proof}
    Fix $\beta$, $e$, and $n$.
    Let $\calL$ denote a loop through $e$,
    and consider the event $\{\calL\subset\calI\}$.
    We claim that
    \[
        \P_{\Lambda_n,\beta}^+
        (\{\calL\subset\calI\})
        \leq e^{-2\beta|\calL|}.
    \]

    To prove the claim,
    we introduce the injective ``loop erasure map''
    \[
        \calE_\calL:\{\calL\subset\calI\}
        \to \Omega\setminus \{\calL\subset\calI\},
    \]
    which is defined such that it flips all the spins inside the loop $\calL$.
    As a consequence, $\calI(\calE_\calL(\sigma))=\calI(\sigma)\setminus\calL$.
    For any $\sigma\in\{\calL\subset\calI\}$, we have
    \[
        \P_{\Lambda_n,\beta}^+
        (\sigma)
        =
        e^{-2\beta|\calL|}
        \cdot
        \P_{\Lambda_n,\beta}^+
        (\calE_\calL(\sigma))
        .
    \]
    We can write down this identity because we know that the loop erasure map
    decreases the Hamiltonian by $2\beta|\calL|$.
    Since $\calE_\calL$ is injective, we get
    \[
        \P_{\Lambda_n,\beta}^+
        (\{\calL\subset\calI\})
        = e^{-2\beta|\calL|}\cdot \P_{\Lambda_n,\beta}^+(\operatorname{Image}(\calE_\calL))
        \leq e^{-2\beta|\calL|},
    \]
    which proves the claim.

    To prove the lemma, observe simply that the number of loops of length $k$
    through $e$ is bounded by $2^k$, so that 
    \begin{align}
        &\P_{\Lambda_n,\beta}^+
        (\{\text{$\calI(\sigma)$ has a loop of length at least $\ell$ through $e$}\})
        \\&\qquad=
        \sum\nolimits_{\text{$\calL$ is a loop of length at least $\ell$ through $e$}}
        \P_{\Lambda_n,\beta}^+
        (\{\calL\subset\calI\})
        \\&\qquad\leq 
        \sum\nolimits_{k\geq \ell} 2^k\cdot e^{-2\beta k}.
    \end{align}
    The final expression is a geometric series converging to the upper bound
    in the lemma.
\end{proof}

\begin{remark}
    In the previous proof,
    the interplay between entropy and energy is quite transparent.
    The entropy in the argument comes from the number of loops of length
    $\ell$, which we upper bounded by $2^\ell$.
    Such a loop contributes a total of $2\beta\ell$ to the Hamiltonian.
    When $2e^{-2\beta}<1$, the energy term dominates,
    forcing the loops to be small.
\end{remark}

\begin{proof}[Proof of Theorem~\ref{thm:peierls_triangles}]
    For a fixed configuration $\sigma\in\Omega$
    sampled from $\P_{\Lambda_n,\beta}^+$,
    we may express $\sigma_0$ as the parity of the number
    of loops in $\calI(\sigma)$ surrounding $0$.
    In particular, if no loop surrounds $0$,
    then $\sigma_0=+1$.
    Thus, for Theorem~\ref{thm:peierls_triangles},
    it suffices to prove that
    \begin{equation}
        \label{eq:peierls_target_equation}
        \P_{\Lambda_n,\beta}^+
        (\{\text{$\calI(\sigma)$ contains a loop surrounding $0$}\})
        <\frac12,
    \end{equation}
    for sufficiently large $\beta$,
    and uniformly in $n$.

    Suppose given some loop $\calL\subset E(\H)$ surrounding $0$.
    Then $\calL$ must intersect the half-line $\R_{\geq 0}\subset\C$.
    More precisely, $\calL$ must contain some edge $e$ whose midpoint
    lies precisely in the set of half-integers $-\frac12+\Z_{\geq 1}$.
    If the endpoint of $e$ is $k-\frac12$, then
    $|\calL|\geq k$, otherwise it cannot surround $0$.
    We are now ready to complete Peierls' argument,
    using exponential decay of the loop lengths (Lemma~\ref{lem:exp_decay_ising_loops}).
    
    Let us perform a union bound over the intersection point,
    in order to obtain
    \begin{align}
        &\P_{\Lambda_n,\beta}^+
        (\{\text{$\calI(\sigma)$ contains a loop surrounding $0$}\})
        \\
        &\qquad\leq
        \sum_{k=1}^\infty
        \P_{\Lambda_n,\beta}^+
        (\{\text{$\calI(\sigma)$ contains a loop surrounding $0$ and hitting $k-\tfrac12$}\})
        \\
        &\qquad\leq
        \sum_{k=1}^\infty
        \frac{(2e^{-2\beta})^k}{1-2e^{-2\beta}}
        =
        \frac{2e^{-2\beta}}{(1-2e^{-2\beta})^2}.
    \end{align}
    This upper bound is independent of $n$
    and tends to $0$ with $\beta\to\infty$,
    thus establishing Equation~\eqref{eq:peierls_target_equation}.
\end{proof}

\begin{exercise}
    \begin{enumerate}
        \item     Consider the Ising model on the two-dimensional square lattice graph $\mathbb{Z}^2$.
        In this case, the interface $\calI(\sigma)$ does not consist of loops, but of even subgraphs
        of the dual lattice.
        How can Peierls' argument be adapted to this case?
        \item  Now consider the $d$-dimensional square lattice for $d\geq 3$.
        What is the structure of the interface in this case?
        Can we adapt Peierls' to prove magnetisation for sufficiently large $\beta$?
    \end{enumerate}
\end{exercise}

\begin{remark}
    Peierls' is robust,
    in the sense that it can be adapted to many other models in statistical mechanics.
\end{remark}


\section{Boundary conditions and the Markov property}

We now introduce two related concepts:
\emph{boundary conditions} and the \emph{Markov property}.
Boundary conditions arise when taking an Ising model and conditioning
on the spins at certain vertices.
An essential property of the nearest-neighbour Ising model is that the spins
can only communicate via the graph that the model is defined on.
This property is expressed in terms of the Markov property.

\begin{definition}[The Ising model with boundary conditions]
    \label{def:ising_bc}
    Let $G=(V,E)$ denote a fixed locally finite graph.
    We consider the measurable space 
    $(\Omega,\calF)$ where $\Omega=\{\pm1\}^V$
    and where $\calF$ is the product-$\sigma$-algebra.

    Let $\Lambda$ denote a \emph{domain},
    that is, a finite subset of $V$.
    The Ising model in the domain $\Lambda$
    with inverse temperature $\beta \geq 0$ and boundary conditions
    $\zeta\in\{\pm1\}^{\Lambda^c}$
    is the probability measure $\P_{\Lambda,\beta}^{\operatorname{Ising},\zeta}$
    on $(\Omega,\calF)$ defined via:
    \[
        \P_{\Lambda,\beta}^{\operatorname{Ising},\zeta}(\sigma)
        =
        \frac1{Z_{\Lambda,\beta}^{\operatorname{Ising},\zeta}}
        \cdot
        \true{\sigma|_{\Lambda^c}=\zeta}\cdot e^{-H_{\Lambda,\beta}^{\operatorname{Ising}}(\sigma)}
        ,
    \]
    where $H_{\Lambda,\beta}^{\operatorname{Ising}}(\sigma)$ is the Hamiltonian given by
    \[
        H_{\Lambda,\beta}^{\operatorname{Ising}}(\sigma)
        =
        -\beta\sum_{uv\in E(\Lambda)}\sigma_u\sigma_v
        ,
    \]
    and where the partition function $Z_{\Lambda,\beta}^{\operatorname{Ising},\zeta}$ is given by
    \[
        Z_{\Lambda,\beta}^{\operatorname{Ising},\zeta}
        =
        \sum_{\sigma \in \Omega} \true{\sigma|_{\Lambda^c}=\zeta} \cdot e^{-H_{\Lambda,\beta}^{\operatorname{Ising}}(\sigma)}
        .
    \]
    The set $E(\Lambda)\subset E$ denotes 
    the set of edges with at least one endpoint in $\Lambda$.
    Indeed, adding a constant to the Hamiltonian does not affect the measure,
    and edges which do not intersect $\Lambda$ contribute with a constant.

    We write $+$ and $-$ for the boundary conditions
    $+1\in\Omega$ and $-1\in\Omega$ respectively.
\end{definition}



Recall the definition of the Ising model on a finite graph
(Definition~\ref{def:ising_finite})
and on general graphs with boundary conditions (Definition~\ref{def:ising_bc}).
The second definition includes the first, since we may simply 
choose our domain $\Lambda$ to be the full vertex set whenever the
graph $G=(V,E)$ is finite.
That is why we state our results for general graphs with boundary conditions
in this section.

One important property of the definition with boundary conditions is that
it in fact encodes \emph{conditional probability measures}.

\begin{lemma}[Boundary conditions as conditional measures]
    \label{lemma:boundary_conditions_conditional_measures}
    Let $G$ denote a locally finite graph and $\beta\in[0,\infty)$
    an inverse temperature.
    Let $\Lambda\subset\Delta$ denote two finite domains
    and fix $\xi\in\{\pm1\}^{\Delta^c}$
    and $\zeta\in\{\pm1\}^{\Delta\setminus\Lambda}$.
    Then
    \[
        \P_{\Delta,\beta}^\xi(\blank|\{\sigma|_{\Delta\setminus\Lambda}=\zeta\})
        =
        \P_{\Lambda,\beta}^{\xi\zeta}.
    \]
\end{lemma}

\begin{proof}
    For the two measures, we get
    \begin{align}
        \P_{\Delta,\beta}^\xi(\sigma|\{\sigma|_{\Delta\setminus\Lambda}=\zeta\})
        &\propto\true{\sigma|_{\Lambda^c}=\xi\zeta}\cdot e^{-H_{\Delta,\beta}(\sigma)};
        \\\P_{\Lambda,\beta}^{\xi\zeta}(\sigma)
        &\propto\true{\sigma|_{\Lambda^c}=\xi\zeta}\cdot e^{-H_{\Lambda,\beta}(\sigma)}.
    \end{align}
    But $H_{\Delta,\beta}-H_{\Lambda,\beta}$ is constant on the
    event $\{\sigma|_{\Lambda^c}=\xi\zeta\}$,
    which means that the two probability measures are the same.
\end{proof}

The Ising model is a \emph{nearest-neighbour model},
meaning that the interactions are associated with the edges of the graph.
A consequence of this is the so-called \emph{Markov property}.
There are several ways to state it.
We shall first state and prove the following lemma.
For any domain $\Lambda\subset V$,
we let $\partial\Lambda\subset V$ denote the set of vertices
at graph distance one from $\Lambda$.
This is called the \emph{boundary} of $\Lambda$.

\begin{lemma}[Markov property]
    \label{lemma:markov_property_general}
    Consider a locally finite graph $G$,
    an inverse temperature $\beta$,
    a domain $\Lambda$,
    and a boundary condition $\zeta$.
    Let $(\Lambda_i)_i$ denote the partition of $\Lambda$
    into connected components.
    Then in the measure
    $\P_{\Lambda,\beta}^\zeta$,
    the family $(\sigma|_{\Lambda_i})_i$
    is a family of independent random variables.
    Moreover, the distribution of $\sigma|_{\Lambda_i}$ only depends on
    $\zeta|_{\partial\Lambda_i}$.
\end{lemma}

\begin{proof}
    We have
    $\P_{\Lambda,\beta}^\zeta(\sigma)\propto \true{\sigma|_{\Lambda^c}=\zeta}\cdot e^{-H_{\Lambda,\beta}(\sigma)}$.
    The Hamiltonian may be written
    \[
        H_{\Lambda,\beta}(\sigma)
        =
        \sum_i H_{\Lambda_i,\beta}(\sigma).
    \]
    But each term $H_{\Lambda_i,\beta}(\sigma)$
    is a function of $\sigma|_{\Lambda_i}$ and $\zeta|_{\partial\Lambda_i}$.
    This means that $e^{-H_{\Lambda,\beta}(\sigma)}$ 
    may be written as a product of factors, where the factor corresponding
    to $\Lambda_i$ only depends on $\sigma|_{\Lambda_i}$ and $\zeta|_{\partial\Lambda_i}$.
    This implies the desired independence.
\end{proof}

The Markov property is often phrased in a slightly different fashion.

\begin{theorem}[Markov property]
    Consider a locally finite graph $G$,
    and inverse temperature $\beta$,
    and two domains $\Lambda\subset\Delta$.
    Let $\zeta\in\{\pm1\}^{\Delta^c}$ denote a boundary condition,
    and fix $\xi\in\{\pm1\}^{\partial\Lambda}$.
    If $\P_{\Delta,\beta}^\zeta(\{\sigma|_{\partial\Lambda=\xi}\})>0$,
    then in the conditional probability measure
    \[
        \P_{\Delta,\beta}^\zeta(\blank|\{\sigma|_{\partial\Lambda=\xi}\})
        =
        \P_{\Delta\cup\partial\Lambda,\beta}^{\zeta\xi}
        ,
    \]
    the random variables $\sigma|_{\Lambda}$
    and $\sigma|_{\Lambda^c}$ are independent.
    Moreover, the distribution of $\sigma|_{\Lambda}$ only depends on $\xi$.
\end{theorem}

\begin{proof}
The two measures in the display in this theorem are equal because of Lemma~\ref{lemma:boundary_conditions_conditional_measures}.
The theorem is then a mere corollary of the previous lemma (Lemma~\ref{lemma:markov_property_general}).
\end{proof}



\section{Correlation inequalities}
\label{sec:correlation}

Peierls' argument is simple and robust, but also quite ad-hoc in the sense
that it does not serve as a building block for further analysis.
We now want to take a more systematic approach to the Ising model.
At the centre of the modern study of the Ising model are \emph{correlation functions}
and \emph{correlation inequalities}.

Let $\langle\blank\rangle$ denote an Ising model (in a finite graph,
or in a finite domain with boundary conditions).
For any finite set $A\subset V$, we define
\[
    \sigma_{A}:=\prod_{u\in A}\sigma_u.
\]
Its expectation $\langle\sigma_A\rangle$ is called a \emph{correlation function}.

\begin{exercise}
    Consider an Ising model $\langle\blank\rangle_{G,\beta}$
    on a finite graph $G=(V,E)$.
    This is a probability measure on $\Omega=\{\pm1\}^V$.
    Notice that the sample space $\Omega$ has the structure of a finite Abelian group.
    How is the Fourier transform of $\langle\blank\rangle_{G,\beta}$ related
    to the family $(\langle\sigma_A\rangle_{G,\beta})_A$ of correlation functions?
\end{exercise}

Correlation functions are at the centre of the study of the Ising model.
Inequalities between correlation functions are called \emph{correlation inequalities}.
We state some examples in the finite graph setting.
These shall all be proved rigorously later.
\begin{itemize}
    \item The \emph{first Griffiths inequality}, which asserts that for any $A\subset V$,
        \[
            \langle\sigma_A\rangle\geq 0.
        \]
    \item The \emph{second Griffiths inequality}, which asserts that for any $A,B\subset V$,
    \[
        \langle\sigma_A\sigma_B\rangle
        \geq
        \langle\sigma_A\rangle\langle\sigma_B\rangle.
    \]
    \item The \emph{Fortuin--Kasteleyn--Ginibre (FKG) inequality}, which asserts that
    if $X,Y:\Omega\to\R$ are two non-decreasing functions on the partially ordered set $\Omega$,
    then
    \[
        \langle XY\rangle
        \geq
        \langle X\rangle\langle Y\rangle.
    \]    
\end{itemize}
Such inequalities may be used to prove interesting properties about the Ising model.

\begin{remark}[Infinite graphs with boundary conditions as finite graphs]
    Until now, we always made a distinction between the Ising model on a finite graph
    and the Ising model on the infinite lattice with boundary conditions.
    But are they really different?
    Let us discuss the case of $+$ boundary conditions.
    Let $G$ denote an infinite graph and $\Lambda$ a domain.
    Then we may consider the Ising model on the finite graph
    \[
        G':=(V',E');
        \qquad
        V':=\Lambda\cup \{\Lambda^c\};
        \qquad
        E':=E(\Lambda).
    \]
    This means that the vertices in $\Lambda^c$ are collapsed
    into a single vertex.
    It is then easy to check that
    the distribution of $\sigma|_{\Lambda}$ is the same in the following two probability measures:
    \[
        \P^+_{\Lambda,\beta}
        \qquad
        \text{and}
        \qquad
        \P_{G',\beta}(\blank|\{\sigma_{\Lambda^c}=+\}).
    \]
    This enables us to state all our inequalities in a unified way,
    namely on finite graphs.
\end{remark}

\begin{exercise}
    First consider the Ising model on a finite graph $G$ at inverse temperature
    $\beta\in[0,\infty)$.
    Prove that $\langle\sigma_A\rangle_{G,\beta}=0$
    whenever $|A|$ is odd.

    Now consider the Ising model on locally finite graph $G$ at inverse temperature
    $\beta\in[0,\infty)$ with $+$ boundary conditions outside the domain
    $\Lambda\subset V$, chosen such that $\Lambda\neq V$.
    Recall the construction in the previous remark.
    Prove that if $A\subset\Lambda$ contains an odd number of vertices,
    then
    \[
        \langle\sigma_A\rangle_{\Lambda,\beta}^+
        =
        \E_{G',\beta}^+[\sigma_A|\{\sigma_{\Lambda^c}=+\}]
        =
        \langle\sigma_{A\cup\{\Lambda^c\}}\rangle_{G',\beta}.
    \]
\end{exercise}

\section{The random-currents expansion}

Previous sections explained the existence of a phase transition in dimension
$d\geq 2$, by demonstrating that the qualitative behaviour of the model 
is different at low and high temperature.
This section introduces a new representation of the Ising model
which is adapted to studying the Ising model at and around the critical temperature.

\begin{definition}[Currents]
    Let $G=(V,E)$ denote a graph.
    A \emph{current} is a map $\n:E\to\Z_{\geq 0}$.
    We think of $(V,\n)$ as a multigraph,
    where for each edge $uv\in E$ we have $\n_{uv}$ multi-edges between $u$ and $v$.
    The set of \emph{sources} $\partial\n\subset V$ of a current
    $\n$ is defined as the set of vertices $u\in V$ with an odd degree in the multigraph.
    We let $\hat\n:=(\n\wedge 1)\in\{0,1\}^E$ denote the associated percolation,
    which simply contains the edges carrying at least one current.

    If $G$ is finite and $\beta\in[0,\infty)$, then the
    \emph{weight} of a current is defined as
    \[
        w(\n):=w_{G,\beta}(\n):=\prod_{xy\in E}
        \frac{\beta^{\n_{xy}}}{\n_{xy}!}.
    \]
    The \emph{random-currents measure} is the measure $\M_{G,\beta}$ on $(\Z_{\geq 0})^E$
    defined by
    \[
        \M[\n]:=\M_{G,\beta}[\n]:=w_{G,\beta}(\n).
    \]
\end{definition}

\begin{remark}
    Notice that $e^{-\beta|E|}\M_{G,\beta}$ is a probability measure
    in which $(\n_{xy})_{xy\in E}$ is a family of independent
    random variables with distribution $\operatorname{Poisson}(\beta)$.
\end{remark}

Currents can be used to encode correlation functions of the Ising model.

\begin{theorem}[Current representation of correlation functions]
    \label{thm:current_representation_of_correlation_functions}
    Consider the Ising model on a finite graph $G$ at inverse temperature $\beta$.
    Let $A\subset V$ be a subset of vertices.
    Then
    \[
        Z\langle\sigma_A\rangle
        =2^{|V|}\sum_{\n:\:\partial\n=A}
        w(\n)
        =
        2^{|V|}\M[\{\partial\n=A\}].
    \]
    In particular, the partition function is given by
    \[
        Z=2^{|V|}\sum_{\n:\:\partial\n=\emptyset}
        w(\n)
        =
        2^{|V|}\M[\{\partial\n=\emptyset\}].
    \]
\end{theorem}

\begin{proof}
    The numbers $\cosh\beta$ and $\sinh\beta$ are obtained from the expansion
    $\sum_{\n=0}^\infty \beta^\n/\n!$ of $e^\beta$ by keeping only
    the even and odd terms respectively.
    The result then simply follows from
    the high-temperature representation (Theorem~\ref{thm:High-temperature expansion for correlation functions})
    by
    expanding $\cosh\beta$ and $\sinh\beta$.
\end{proof}


\begin{remark}[Switching example]
    The above setup opens the door to a powerfull technique called \emph{switching}.
    Two elements are key to switching:
    Poisson randomness, and parity constraints
    (both of which are present in the currents representation of correlation functions).
    Let us quickly describe how switching may be applied to a simple example.
    Suppose that we record cars traversing a bridge on a road.
    Blue cars pass according to a Poisson point process with rate $1$ (per second).
    Let $X_B$ denote the number of blue cars that pass after recording $\beta$ seconds.
    Can we easily prove, without a calculation, that
    \[
        \P[\{\text{$X_B$ is even}\}]\geq \P[\{\text{$X_B$ is odd}\}]?
    \]

    Suppose that there are also yellow cars,
    which arrive according to an independent Poisson process with the same rate.
    Let $X_Y$ denote the number of yellow cars that passed.
    Suppose that, after waiting $\beta$ seconds, $X_B+X_Y=N>0$ cars passed.
    What is the \emph{conditional} probability that $X_B$ is even?

    Well, we must have $\P[\{\text{$X_B$ is even}\}|\{X_B+X_Y=N\}]=1/2$.
    Indeed, by the properties Poisson point processes,
    the $n$-th car (for $1\leq n\leq N$) is blue or yellow with equal probability, 
    independently of the other cars.
    Thus, we may condition on the colours of the first $N-1$ cars,
    then flip a fair coin to decide the colour of the last car
    and thus the parity of $X_B$.
    Equivalently, we observe that \emph{repainting} or \emph{switching}
    the colour of the last car leaves the conditional distribution invariant.

    But we cannot always do the switch.
    If $N=X_B+X_Y=0$
    then there is no car to repaint, and also
    $X_B=0$.
    Thus, we conclude that
    \[
        \P[\{\text{$X_B$ is even}\}]-\P[\{\text{$X_B$ is odd}\}]=\P[\{X_B+X_Y=0\}]\geq 0.
    \]
    
    Notice that we originally asked a question about blue cars,
    but introducing yellow cars allowed us to answer it.
    This is the essence of the switching lemma.
\end{remark}

\begin{lemma}[Explicit switching lemma]
    \label{lem:explicit_switching_lemma}
    Let $G$ denote a finite graph and $\beta\in[0,\infty)$.
    Consider the measurable pair $(\n,\m)\sim \M^2=\M_{G,\beta}^2$.
    Fix $\s\in(\Z_{\geq 0})^E$ and $\eta\subset\hat\s$.
    Then for any $A\subset V$,
    we have
    \[
        \M^2[\{\n+\m=\s\}\cap\{\partial\n=A\}]
        =
        \M^2[\{\n+\m=\s\}\cap\{\partial\n=A\Delta\partial\eta\}].
    \]
    Notice also that if $\n+\m=\s$, then $\partial\s=(\partial\n)\Delta(\partial\m)$.
\end{lemma}

Slight variations of the proof will be used later.

\begin{proof}
    Introduce the probability measure $\P:\propto\M^2[\{\n+\m=\s\}\cap(\blank)]$.
    Our objective is to prove that
    \[
        \P[\{\partial\n=A\}]
        =
        \P[\{\partial\n=A\Delta\partial\eta\}].
    \]

    In the probability measure $\P$, the family $((\n_{xy},\m_{xy}))_{xy\in E}$
    is independent over the edges $xy\in E$,
    and each pair $(\n_{xy},\m_{xy})$ follows the distrubution of two independent 
    distributions $\operatorname{Poisson}(\beta)$ conditioned to sum to $\s_{xy}$.
    By analogy with the example of blue and yellow cars,
    we may interpret $\P$ as follows:
    \begin{itemize}
        \item $\calM=\calM_\s$ is the fixed multigraph $\{(xy,k)\in E\times\Z_{\geq 0}:k<\s_{xy}\}$,
        \item $\calA$ is a uniformly random subset of $\calM$,
        \item $\calB$ is the complement of $\calA$ in $\calM$,
        \item $\n_{xy}=\n_{xy}(\calA)$ is the number of multi-edges in $\calA$ between $x$ and $y$,
        \item $\m_{xy}=\m_{xy}(\calB)$ is the number of multi-edges in $\calB$ between $x$ and $y$.
    \end{itemize}

    Define $\tilde\eta:=\eta\times\{0\}\subset\calM$.
    Since $\calA$ is a uniformly random subset of $\calM$ in the measure $\P$,
    the set $\calA\Delta\tilde\eta$ is also uniformly random in $\calM$.
    Therefore we have
    \[
        \P[\{\partial\n=A\}]=
        \P[\{\partial\n(\calA)=A\}]
        =
        \P[\{\partial\n(\calA\Delta\tilde\eta)=A\}]
        =
        \P[\{\partial\n=A\Delta\partial\eta\}].
    \]
    This is the desired equality.
\end{proof}

\begin{definition}[Percolation of currents]
    Let $G$ denote a graph and $\n$ a current.
    Write \[\{u\xleftrightarrow{\hat\n}v\}\]
    for the event there is an open path from $u$ to $v$
    ($u$ and $v$ may represent vertices or sets of vertices).
    For fixed $S\subset V$, we shall also write $\calE_S$ for the set
    \[
        \{O\subset E:\text{$|C\cap S|$ is even for any connected component $C\subset V$ of $(V,O)$}\}.
    \]
\end{definition}

\begin{exercise}
    Let $G$ denote a graph and 
    $x,y\in V$  distinct vertices. Prove the following.
    \begin{itemize}
        \item If $S=\{x,y\}$, then $\{\hat\n\in\calE_S\}=\{x\xleftrightarrow{\hat\n}y\}$.
        \item If $\omega\in\calE_S$, then we may find a subset $\eta\subset\omega$ with $\partial\eta=S$.
        \item If $G$ is a finite graph and $\partial\n=S$, then $\hat\n\in\calE_S$.
        \item For any $S\subset V$, the event $\{\hat\n\in\calE_S\}$ is an increasing event of the current $\n$.
    \end{itemize}
\end{exercise}

\begin{lemma}[Switching lemma]
    \label{lem:switching_lemma}
    Let $G$ denote a finite graph and $\beta\in[0,\infty)$.
    Consider the measurable pair $(\n,\m)\sim \M^2=\M_{G,\beta}^2$.
    Then for any $A,B,S\subset V$ and for any bounded function $F:(\Z_{\geq 0})^E\to\C$,
    we have
    \begin{align}
        &\M^2[F(\n+\m)\true{\widehat{\n+\m} \in \calE_S}\true{\partial\n=A}\true{\partial\m=B}]
        \\={}&
        \M^2[F(\n+\m)\true{\widehat{\n+\m} \in \calE_S}\true{\partial\n=A\Delta S}\true{\partial\m=B\Delta S}].
    \end{align}
\end{lemma}

\begin{proof}
    By linearity of integration, it suffices to consider the case that $F(\n+\m):=\true{\n+\m=\s}$
    for some fixed current $\s$ with $\hat\s\in\calE_S$.
    By the previous exercise, we may find some $\eta\subset\hat\s$
    such that $\partial\eta=S$.
    The desired equality
    \begin{align}
    &
    \M^2[\true{\n+\m =\s}\true{\partial\n=A}\true{\partial\m=B}]
    \\={}&
    \M^2[\true{\n+\m =\s}\true{\partial\n=A\Delta S}\true{\partial\m=B\Delta S}]
    \end{align}
    then follows by the explicit switching lemma.
\end{proof}

In practice, we do not care so much about the function $F$, and simply set it to $F\equiv 1$.

An important corollary of the switching lemma is the \emph{second Griffits inequality}.

\begin{lemma}[Second Griffiths inequality]
    Consider the Ising model on a finite graph $G$ at inverse temperature $\beta$.
    Then for any $A,B\subset V$, we have
    $\langle\sigma_{A}\sigma_B\rangle-\langle\sigma_A\rangle\langle\sigma_B\rangle\geq 0$.
\end{lemma}

The second Griffiths inequality is more subtle than the first,
as it bounds a \emph{difference} of correlation functions.
This is typical for the switching lemma.

\begin{proof}
    Claim that
    \begin{align}
        &\M^2[\{\partial\n=A\}\cap\{\partial\m=B\}]
        \\
        &\qquad=
        \M^2[\{\widehat{\n+\m} \in \calE_B\}\cap\{\partial\n=A\}\cap\{\partial\m=B\}]
        \\
        &\qquad\stackrel{\text{switch}}=
        \M^2[\{\widehat{\n+\m} \in \calE_B\}\cap\{\partial\n=A\Delta B\}\cap\{\partial\m=\emptyset\}]
        \\
        &\qquad\leq
        \M^2[\{\partial\n=A\Delta B\}\cap\{\partial\m=\emptyset\}].
    \end{align}
    The switch is just the switching lemma with $F\equiv 1$ and $S=B$.
    For the first equality, we simply observe that $\{\partial\m=B\}\subset \{\widehat{\n+\m} \in \calE_B\}$
    (see the exercise above).
    The inequality is inclusion of events.

    By the random currents expansion of correlation functions (Theorem~\ref{thm:current_representation_of_correlation_functions}),
    the left- and rightmost expressions are given by
    \[
        Z^2\langle\sigma_A\rangle\langle\sigma_B\rangle
        \leq
        Z^2\langle\sigma_A\sigma_B\rangle\langle\sigma_\emptyset\rangle.
    \]
    Since $\langle\sigma_\emptyset\rangle=1$, this is the desired inequality.
\end{proof}

\section{Double random currents}

The previous section explained how correlation functions are expressed
in terms of random currents.
We also proved a basic result, namely the existence of a demagnetised phase
of the Ising model on graphs of bounded degree.

All results discussed so far concern the behaviour of the Ising model
in the off-critical regime (very large values of $\beta$, very small values of $\beta$).
Our main interest is however in the \emph{critical regimes}:
the values for $\beta$ where the model undergoes a qualitative change,
such as values in the topological boundary of the set
\[
    \{\beta\in[0,\infty):\lim_{n\to\infty}\langle\sigma_u\rangle_{\Lambda_n,\beta}^+=0\}
\]
for a given infinite graph $G$ with a reference point $u$
(as per usual, $\Lambda_n$ refers to the graph metric ball around $u$).

We shall now introduce a new tool to study correlation functions
and random currents: the \emph{switching lemma}.
In recent years this tool has proved to be instrumental in the derivation
of rigorous results on the critical
behaviour of the Ising model, especially in graph dimensions $3$ and $4$.

\begin{lemma}[Switching lemma]
    Consider the Ising model on a finite graph $G$ at inverse temperature $\beta$.
    Let $A,B,S\subset V$.
    Then for any bounded function $F:(\Z_{\geq 0})^E\to\C$,
    the following identities hold true:
    \[
        \M^A\times \M^B[F(\n+\m)\true{\n+\m\in\calE_S}]
        =
        \M^{A\Delta S}\times \M^{B\Delta S}[F(\n+\m)\true{\n+\m\in\calE_S}],
    \]
    where $A\Delta S$ denotes the symmetric difference of $A$ and $S$.

    In terms of weights, this is equivalent to
    \begin{multline}
        \sum_{\substack{\n:\:\partial\n=A\\\m:\:\partial\m=B}}
        w_\beta(\n)w_\beta(\m)
        F(\n+\m)\true{\n+\m\in\calE_S}
        \\
        =
        \sum_{\substack{\n:\:\partial\n=A\Delta S\\\m:\:\partial\m=B\Delta S}}
        w_\beta(\n)w_\beta(\m)
        F(\n+\m)\true{\n+\m\in\calE_S}.
    \end{multline}
\end{lemma}

\begin{proof}
    By linearity of expectation,
    it suffices to consider the case that $F(\b):=\true{\b=\a}$
    for some fixed $\a\in\calE_S$.
    Our objective is then to derive the equality
    \[
        \M^A\times \M^B[\{\n+\m=\a\}]
        =
        \M^{A\Delta S}\times \M^{B\Delta S}[\{\n+\m=\a\}]
    \]
    or
    \begin{align}
        \M^2[\{\partial\n=A,\,\partial\m=B,\,\n+\m=\a\}]
        =
        \M^2[\{\partial\n=A\Delta S,\,\partial\m=B\Delta S,\,\n+\m=\a\}].
    \end{align}
    Define the probability measure
    \[
        \P:\propto\M^2[(\blank)\true{\n+\m=\a}].
    \]
    It suffices to prove that
    \begin{equation}
        \label{eq:switching_target_equation}
    \P[\{\partial\n=A,\,\partial\m=B\}]
    =
    \P[\{\partial\n=A\Delta S,\,\partial\m=B\Delta S\}].
    \end{equation}

    By going back to the definition of $\M$ in terms of $w_\beta$,
    it is straightforward to see that the pair $(\n,\m)$ has the following probability distribution under $\P$:
    \begin{itemize}
        \item The family $(\n_{uv})_{uv}$ is a family of independent random variables,
        \item The distribution of $\n_{uv}$ is $\operatorname{Binomial}(\a_{uv},1/2)$,
        \item We have $\n+\m=\a$ almost surely, which fixes the joint distribution of $(\n,\m)$.
    \end{itemize}
    In fact, we may interpret $\P$ in a different way.
    Define the \emph{multigraph}
    \[
        \calM_\a:=\{(uv,k)\in E\times\Z_{\geq 0}:\a_{uv}<k\}
    \]
    on the vertex set $V$.
    Then $\P$ is interpreted as follows:
    \begin{itemize}
        \item We let $\calK$ denote a uniformly random subset of $\calM_\a$,
        \item We let $\n_{uv}$ denote the number of multiedges in $\calK$ between $u$ and $v$,
        \item We let $\m_{uv}$ denote the number of multiedges in $\calM_\a\setminus\calK$ between $u$ and $v$.
    \end{itemize}
    Indeed, this definition of $\P$ is consistent with our previous one.

    Proving Equation~\eqref{eq:switching_target_equation} now comes down to
    proving that the number of submultigraphs $\calK\subset\calM_\a$ contributing
    to the events on the left and right, is the same.
    Let $E_S\subset E(\a)$ denote an arbitrary subset such that $\partial E_S=S$,
    and write $E_{S,0}:=E_S\times\{0\}\subset\calM_\a$.
    The existence of the set $E_S$ follows from the fact that $\a\in\calE_S$.
    The reader may now verify that the map
    \[
        \{\partial\n=A,\,\partial\m=B\}
        \to
        \{\partial\n=A\Delta S,\,\partial\m=B\Delta S\}
        ,\,
        \calK\mapsto \calK\Delta E_{S,0}
    \]
    is a bijection.
    This proves that the two sets have the same cardinality,
    and thus the same probability under the measure $\P$.
    We have now established Equation~\eqref{eq:switching_target_equation}
    and therefore the lemma.
\end{proof}

\begin{corollary}[Second Griffiths inequality]
    Consider the Ising model on a finite graph $G$ at inverse temperature $\beta$.
    Then for any $A,B\subset V$, we have
    $\langle\sigma_{A\Delta B}\rangle_{G,\beta}-\langle\sigma_A\rangle_{G,\beta}\langle\sigma_B\rangle_{G,\beta}\geq 0$.
\end{corollary}

\begin{proof}
    We have $1=\langle1\rangle=\langle\sigma_\emptyset\rangle$.
    By the previous section (for example Corollary~\ref{cor:current_representation_of_correlation_functions}),
    \[
    Z^2
    (\langle\sigma_{A\Delta B}\rangle\langle\sigma_\emptyset\rangle-\langle\sigma_A\rangle\langle\sigma_B\rangle)
    =
    2^{2|V|}
    (
    \M^{A\Delta B}\times\M^\emptyset[1]
    -
    \M^A\times\M^B[1]
    ).
    \]
    Claim that the quantity on the right is nonnegative.
    Notice that if $\partial\m= B$, then $\m\in\calE_S$,
    and therefore
    \begin{multline}
        \M^A\times\M^B[1]
        =
        \M^A\times\M^B[\true{\n+\m\in\calE_B}]
        \stackrel{\text{switch}}=
        M^{A\Delta B}\times\M^\emptyset[\true{\n+\m\in\calE_B}]
        \\
        \leq
        M^{A\Delta B}\times\M^\emptyset[1].
    \end{multline}
    This inequality implies the claim, and therefore the second Griffiths inequality.
\end{proof}

\begin{exercise}[Conditioning on equality increases the correlation functions]
    \label{exo:conditioning_equality}
    Consider the Ising model on a finite graph $G$
    at inverse temperature $\beta$, and fix some subset $A\subset V$.
    \begin{itemize}
        \item     Prove that for any two distinct vertices $u,v\in V$,
        we have
        \[
            \E_{G,\beta}[\sigma_A|\{\sigma_u=\sigma_v\}]
            \geq
            \E_{G,\beta}[\sigma_A]=\langle\sigma_A\rangle_{G,\beta}.
        \]
        \item Prove for any $X\subset Y\subset V$, we have
        \[
            \E_{G,\beta}[\sigma_A|\{\text{$\sigma$ is constant on $X$}\}]
            \leq
            \E_{G,\beta}[\sigma_A|\{\text{$\sigma$ is constant on $Y$}\}]
            .
        \]
    \end{itemize}
\end{exercise}

\begin{exercise}[The two-point function as a metric]
    Consider the Ising model on a finite graph $G$
    at inverse temperature $\beta>0$.
    Prove that $V\times V\to [0,\infty],\,
    (u,v)\mapsto-\log\langle\sigma_u\sigma_v\rangle_{G,\beta}$
    defines a metric on $V$.
\end{exercise}

\begin{definition}[Probability measures on currents]
    Consider the Ising model on a finite graph $G$
    at inverse temperature $\beta$.
    For any $A\subset V$,
    define the probability measure
    \[
        \P^A_{G,\beta}:=\frac{2^{|V|}}{Z_{G,\beta}\langle\sigma_A\rangle_{G,\beta}}\M^A_{G,\beta}.
    \]
    For any $A_1,\ldots,A_n$,
    write $\P^{A_1,\ldots,A_n}:=\P^{A_1}\times\cdots\times\P^{A_n}$.
\end{definition}

\begin{exercise}[Correlation functions in terms of sourceless currents]
    Consider the Ising model on a finite graph $G$
    at inverse temperature $\beta$.
    Prove that for any $A\subset V$,
    \[
        \langle\sigma_A\rangle^2
        =
        \P^{\emptyset,\emptyset}[\{\n+\m\in\calE_A\}].
    \]

    Observe that we can now express all correlation functions in terms of a single
    fixed probability measure on sourceless random currents.
\end{exercise}


\section{Monotonicity via the second Griffiths inequality}

\begin{theorem}[Monotonicity in the temperature]
    Let $G$ denote a finite graph and $A\subset V$ a finite set.
    Then the function $\beta\mapsto\langle\sigma_A\rangle_{G,\beta}$
    is non-decreasing.
\end{theorem}

\begin{proof}
    We want to prove that
    \[
        \frac{\partial}{\partial\beta}
        \langle\sigma_A\rangle_{G,\beta}
        =
        \frac{\partial}{\partial\beta}
        \left(
            \frac{
                \sum_\sigma\sigma_A\prod_{uv}e^{\beta\sigma_u\sigma_v}
            }{
                \sum_\sigma\prod_{uv}e^{\beta\sigma_u\sigma_v}
            }
        \right)
        \geq 0.
    \]
    Since we are differentiating a fraction,
    it suffices to show that the numerator grows at a faster rate
    than the denominator,
    that is,
    \[
        \frac{\frac{\partial}{\partial\beta}
            \sum_\sigma\sigma_A\prod_{uv}e^{\beta\sigma_u\sigma_v}
        }{
            Z\langle\sigma_A\rangle
        }
        \geq
        \frac{\frac{\partial}{\partial\beta}
            \sum_\sigma\prod_{uv}e^{\beta\sigma_u\sigma_v}
        }{
            Z
        }.
    \]
    We perform the differential and then multiply each side by $\langle\sigma_A\rangle$,
    to see that this inequality is equivalent to
    \[
       \sum_{xy}
        \frac{\sum_\sigma
            \sigma_x\sigma_y\sigma_A\prod_{uv}e^{\beta\sigma_u\sigma_v}
        }{Z}
        \geq
        \langle\sigma_A\rangle
        \sum_{xy}
        \frac{
            \sum_\sigma
            \sigma_x\sigma_y
            \prod_{uv}e^{\beta\sigma_u\sigma_v}
        }{Z}.
    \]
    Each fraction may now be reinterpreted as a correlation function,
    so that the previous inequality is equivalent to
    \[
        \sum_{xy}\langle\sigma_x\sigma_y\sigma_A\rangle
        \geq
        \langle\sigma_A\rangle
        \sum_{xy}\langle\sigma_x\sigma_y\rangle.
    \]
    But this is just the second Griffiths inequality.
\end{proof}

\begin{exercise}[Regularity properties of the correlation functions in $\beta$]
    Prove that the function $[0,\infty)\to\R,\,\beta\mapsto\langle\sigma_A\rangle_{G,\beta}$
    in the above context is an analytic function.
\end{exercise}

Next, we want to prove monotonicity in domains.
We first challenge the reader to prove the following exercise.


\begin{exercise}[Conditioning on equality increases the correlation functions]
    \label{exo:conditioning_equality}
    Consider the Ising model on a finite graph $G$
    at inverse temperature $\beta$, and fix some subset $A\subset V$.
    \begin{itemize}
        \item     Prove that for any two distinct vertices $u,v\in V$,
        we have
        \[
            \E_{G,\beta}[\sigma_A|\{\sigma_u=\sigma_v\}]
            \geq
            \E_{G,\beta}[\sigma_A]=\langle\sigma_A\rangle_{G,\beta}.
        \]
        \item Prove for any $X\subset Y\subset V$, we have
        \[
            \E_{G,\beta}[\sigma_A|\{\text{$\sigma$ is constant on $X$}\}]
            \leq
            \E_{G,\beta}[\sigma_A|\{\text{$\sigma$ is constant on $Y$}\}]
            .
        \]
    \end{itemize}
\end{exercise}

\begin{lemma}[Monotonicity in domains]
    \label{lemma:correlation_functions_monotone_both}
    Consider the Ising model on a locally finite graph
    $G=(V,E)$ at inverse temperature $\beta$.
    Consider two finite domains $\Lambda\subset\Lambda'\subset V$
    and a subset $A\subset \Lambda$.
    \begin{itemize}
        \item \textbf{Free boundary.}
        We have $\langle\sigma_A\rangle^\f_{\Lambda,\beta}\leq\langle\sigma_A\rangle^\f_{\Lambda',\beta}$.
        \item \textbf{Wired boundary.}
        We have $\langle\sigma_A\rangle^+_{\Lambda,\beta}\geq\langle\sigma_A\rangle^+_{\Lambda',\beta}$.
    \end{itemize}
\end{lemma}

\begin{proof}[Proof for $\langle\blank\rangle^\f$]
    We first prove the following claim:
    if $G'$ and $G''$ are finite graphs on the same vertex set,
    and such that $E(G'')=E(G')\cup\{xy\}$,
    then
    \[
        \langle\sigma_A\rangle_{G',\beta}
        \leq
        \langle\sigma_A\rangle_{G'',\beta}
    \]
    for any $A\subset V(G')$.
    To prove the claim, we simply expand 
    \[
        \langle\sigma_A\rangle_{G'',\beta}
        =
        \frac{
            \langle   e^{\beta\sigma_x\sigma_y}\sigma_A\rangle_{G'}
        }{
            \langle e^{\beta\sigma_x\sigma_y}\rangle_{G'}
        }.
    \]
    Thus, we want to show that
    \[
            \langle e^{\beta\sigma_x\sigma_y} \sigma_A \rangle_{G'}
            \geq 
            \langle e^{\beta\sigma_x\sigma_y}\rangle_{G'}
            \langle\sigma_A\rangle_{G'}.
    \]
    This follows from the second Griffiths inequality.
    We have now proved the claim.

    Recall the definition of the finite graph $\Lambda^\f$.
    Let $\tilde\Lambda^\f:=((\Lambda')^\f,E(\Lambda^\f))$;
    this is just the graph $\Lambda^\f$
    supplemented with some isolated vertices $\Lambda'\setminus\Lambda$.
    The law of $\sigma$ in $\langle\blank\rangle_{\tilde\Lambda^\f}$
    is just given by $\langle\blank\rangle_{\Lambda^\f}$,
    with independent fair coin flips for the isolated vertices in $\Lambda'\setminus\Lambda$.
    Thus, it suffices to prove that
    \[
        \langle\sigma_A\rangle^\f_{\Lambda}
        =
        \langle\sigma_A\rangle_{\tilde\Lambda^\f}
        \leq
        \langle\sigma_A\rangle_{(\Lambda')^\f}
        =
        \langle\sigma_A\rangle^\f_{\Lambda'}
        .
    \]
    This follows from the claim.
\end{proof}


\begin{proof}[Proof for $\langle\blank\rangle^+$]
    Without loss of generality,
    $\Lambda'\setminus\Lambda=\{u\}$ for some
    vertex $u\in V$.
    We make all calculations in the graph $(\Lambda')^\frakg$ with the ghost vertex:
    we get
    \[
        \langle\sigma_A\rangle_{\Lambda'}^+
        =
        \E_{(\Lambda')^\frakg}[\sigma_A|\{\sigma_\frakg=+\}];
        \qquad
        \langle\sigma_A\rangle_{\Lambda}^+
        =
        \E_{(\Lambda')^\frakg}[\sigma_A|\{\sigma_\frakg=+\}\cap\{\sigma_u=\sigma_\frakg\}].
    \]

    Assume first that $|A|$ is even for now.
    Then
    \begin{equation}
        \langle\sigma_A\rangle_{\Lambda}^+
        =
        \E_{(\Lambda')^\frakg}[\sigma_A|\{\sigma_u=\sigma_\frakg\}]
        \geq
        \E_{(\Lambda')^\frakg}[\sigma_A]
        =
        \langle\sigma_A\rangle_{\Lambda'}^+,
    \end{equation}
    due to Exercise~\ref{exo:conditioning_equality}.

    If $|A|$ is odd, we just need to replace the set $A$
    by $A':=A\cup\{\frakg\}$.
    More precisely,
    \begin{equation}
        \langle\sigma_A\rangle_{\Lambda}^+
        =
        \E_{(\Lambda')^\frakg}[\sigma_{A'}|\{\sigma_u=\sigma_\frakg\}]
        \geq
        \E_{(\Lambda')^\frakg}[\sigma_{A'}]
        =
        \langle\sigma_A\rangle_{\Lambda'}^+,
    \end{equation}
    where the inequality uses the same exercise.

    Those are the desired inequalities.
\end{proof}

\begin{definition}[Infinite-volume limit]
    Let $G=(V,E)$ denote a locally finite graph.
    Write
    \[
        \lim_{\Lambda\uparrow V}f(\Lambda)
        \qquad\text{for}\qquad
        \lim_{n\to\infty}f(\Lambda_n),
    \]
where $(\Lambda_n)_n$ is any increasing sequence of finite domains 
with $\cup_n\Lambda_n=V$.
This notation makes sense only when the limit is independent
of the precise choice of the sequence $(\Lambda_n)_n$,
and is called the \emph{thermodynamical limit} or \emph{infinite-volume limit}.

Let $(\Omega,\calF)$ denote the measurable space $\Omega:=\{\pm1\}^V$
endowed with the product $\sigma$-algebra.
For a domain $\Lambda$, we write $\calF_\Lambda$
for the $\sigma$-algebra generated by spins in $\Lambda$.
An observable $X:\Omega\to\C$ is called \emph{local} if it is measurable
with respect to $\calF_\Lambda$ for some domain $\Lambda$.

Let $\calP(\Omega,\calF)$ denote the set of all probability measures
on this measurable space.
We endow this set with the \emph{local convergence topology},
which is defined as the topology making the map
\[
    \calP(\Omega,\calF)\to\C,\,\mu\mapsto\mu[X]
\]
continuous for any local observable $X$.
\end{definition}

\begin{remark}
    This topology is sometimes known under different names in the literature
    (such as the \emph{weak topology}).
    I like the name \emph{local convergence topology} because it captures the essence quite literally:
    if the statistics of the measures within a fixed domain $\Lambda$ converge,
    then we have local convergence.
\end{remark}

\begin{exercise}
    Prove that $\calP(\Omega,\calF)$ is a compact space in this topology.
\end{exercise}

\begin{theorem}[Existence of the thermodynamical limit]
    Consider the Ising model on a locally finite graph $G$
    at inverse temperature $\beta$.
    Then there exists unique probability measures
    $\langle\blank\rangle_{G,\beta}^\f,\langle\blank\rangle^+_{G,\beta}\in\calP(\Omega,\calF)$
    such that
    \[
        \lim_{\Lambda\uparrow V}\langle X\rangle_{\Lambda,\beta}^*=\langle X\rangle_{G,\beta}^*
    \]
    for $*\in\{\f,+\}$ and
    for any local observable $X:\Omega\to\R$.
    In other words,
    \[
        \lim_{\Lambda\uparrow V}\langle \blank\rangle_{\Lambda,\beta}^*
        =:
        \langle \blank\rangle_{G,\beta}^*.
    \]
    The measures $\langle\blank\rangle_{G,\beta}^*$ are called
    the \emph{thermodynamical limits} or \emph{infinite-volume limits}.
\end{theorem}

\begin{proof}
    Any local observable may be written as a finite linear conbination
    of observables of the form $\sigma_A$ where $A$ is a finite subset of
    $V$.
    The theorem then follows by compactness and Lemma~\ref{lemma:correlation_functions_monotone}.
\end{proof}

\begin{exercise}[Continuity properties in $\beta$]
    Consider the Ising model on a locally finite graph $G=(V,E)$.
    Fix $A\subset V$ finite.
    \begin{itemize}
        \item The function $\beta\mapsto \langle\sigma_A\rangle_{G,\beta}^*$ is non-decreasing for $*\in\{\f,+\}$.
        \item The function $\beta\mapsto \langle\sigma_A\rangle_{G,\beta}^\f$ is left continuous.
        \item The function $\beta\mapsto \langle\sigma_A\rangle_{G,\beta}^+$ is right continuous.
    \end{itemize}
    \emph{Hint.} Argue that $\beta\mapsto \langle\sigma_A\rangle_{G,\beta}^\f$
    is a limit of a non-decreasing sequence of non-decreasing functions.
\end{exercise}

\begin{definition}[Magnetisation and critical temperature]
    Let $G$ be a vertex-transitive locally finite graph and $u$ some distinguished reference vertex.
    The non-decreasing right-continuous function
    \[
        m=m_G:[0,\infty)\to\R,\,\beta\mapsto\langle\sigma_u\rangle_{G,\beta}^+
    \]
    is called the \emph{magnetisation}.

    The \emph{critical (inverse) temperature} is defined via
    \[
        \beta_c:=\beta_c(G):=\inf\{\beta\in[0,\infty):m(\beta)>0\}.
    \]
\end{definition}

We have already proved that $\beta_c\in(0,\infty)$ for $G=\Z^d$
in dimension $d\geq 2$,
and that $\beta_c=\infty$ for $G=\Z$.

\section{The thermodynamical limit. Wired boundary}
\label{sec:infinite_volume}

Consider the Ising model on the infinite graph $\Z^d$ for $d\geq 2$.
Let $u=0\in\Z^d$ and let $\Lambda_n$ denote the graph metric ball around $u$.
We have already derived the following results.
\begin{itemize}
    \item For large $\beta$,
    the Ising model exhibits magnetisation
    in the sense that
    \[
        \inf_{n}\langle\sigma_0\rangle_{\Lambda_n,\beta}^+
        >0.
    \]
    This was proved via the Peierls argument, see Theorem~\ref{thm:peierls}
    and Exercise~\ref{exo:peierls_general}.
    \item For small $\beta$,
    the Ising model \emph{does not} exhibit magnetisation:
    \[
        \lim_{n\to\infty}\langle\sigma_0\rangle_{\Lambda_n,\beta}^+
        =0.
    \]
    This was proved via a Peierls argument for random currents,
    see Exercise~\ref{exercise:currents_peierls}.
\end{itemize}
At the time moment of stating the Peierls argument (Section~\ref{sec:peierls}),
we knew almost nothing about the Ising model.
Our understanding is now advancing.
We already used the first Griffiths inequality to show that $\langle\sigma_0\rangle_{\Lambda_n,\beta}^+\geq 0$
(Corollary~\ref{cor:griffiths_1}
and Exercise~\ref{exo:correlation_functions_with_odd_sets}).
Our first objective is now to prove the following result.
To state it, we write
\[
    \lim_{\Lambda\uparrow V}f(\Lambda)
    \qquad\text{for}\qquad
    \lim_{n\to\infty}f(\Lambda_n),
\]
where $(\Lambda_n)_n$ is any increasing sequence of domains 
with $\cup_n\Lambda_n=V$.
This notation makes sense only when the limit is independent
of the precise choice of the sequence $(\Lambda_n)_n$,
and is called the \emph{thermodynamical limit} or \emph{infinite-volume limit}.

\begin{lemma}[Correlation functions are monotone in the domain (wired boundary)]
    \label{lemma:correlation_functions_monotone}
    Consider the Ising model on a locally finite graph
    $G$ at inverse temperature $\beta$.
    Let $A\subset V$ denote any finite subset.
    Then the function
    \[
        \Lambda\mapsto
        \langle\sigma_A\rangle_{\Lambda,\beta}^+
    \]
    is a nonincreasing function of the domain $\Lambda$.

    In particular, we have well-definedness of the thermodynamical limit
    \[
        \lim_{\Lambda\uparrow V}
        \langle\sigma_A\rangle_{\Lambda,\beta}^+.
    \]
\end{lemma}

\begin{proof}
    Consider two domains $\Lambda\subset\bar\Lambda$.
    We want to show that
    \[
        \langle\sigma_A\rangle_{\Lambda,\beta}^+
        \geq
        \langle\sigma_A\rangle_{\bar\Lambda,\beta}^+.
    \]
    Without loss of generality,
    $A\subset\bar\Lambda$ and
    $\bar\Lambda\setminus\Lambda=\{u\}$ for some
    vertex $u\in V$.

    Let $G'=(\bar\Lambda\cup\{\bar\Lambda^c\},E(\bar\Lambda))$ denote the graph obtained from $\bar\Lambda$
    as in Remark~\ref{remark:infinite_graphs_as_finite_graphs}
    and Exercise~\ref{exercise:infinite_graphs_as_finite_graphs}.
    We refer to the Ising model on $G'$ when subscripts are submitted from now on.
    Assume that $|A|$ is even for now.
    Then
    \begin{equation}
        \langle\sigma_A\rangle_{\bar\Lambda,\beta}^+
        =
        \E[\sigma_A];\qquad
        \langle\sigma_A\rangle_{\Lambda,\beta}^+
        =
        \E[\sigma_A|\{\sigma_u=\sigma_{\bar\Lambda^c}\}].
    \end{equation}
    It suffices to show that the conditioning increases the expectation.
    This is just Exercise~\ref{exo:conditioning_equality}.
    
    % Define the event $Q_\pm:=\{\sigma_u\sigma_{\bar\Lambda^c}=\pm 1\}$.
    % It suffices to prove that
    % \[
    %     \E[\sigma_A|Q_+]
    %     \geq
    %     \E[\sigma_A|Q_-].
    % \]
    % Notice that
    % \begin{multline}
    %     \P[Q_+]\E[\sigma_A|Q_+]
    %     -
    %     \P[Q_-]\E[\sigma_A|Q_-]
    %     =
    %     \E[\sigma_A\sigma_{\{u,\bar\Lambda^c\}}]
    %     \\
    %     \stackrel{\text{second Griffiths}}\geq
    %     \E[\sigma_A]\E[\sigma_{\{u,\bar\Lambda^c\}}]
    %     =(\P[Q_+]
    %     -
    %     \P[Q_-])\E[\sigma_A].
    % \end{multline}
    % This is the desired inequality.

    If $|A|$ is odd then we just need to replace the set $A$
    by $A':=A\cup\{\bar\Lambda^c\}$.
    More precisely,
    we have
    \begin{equation}
        \langle\sigma_A\rangle_{\bar\Lambda,\beta}^+
        =
        \E[\sigma_{A'}];\qquad
        \langle\sigma_A\rangle_{\Lambda,\beta}^+
        =
        \E[\sigma_{A'}|\{\sigma_u=\sigma_{\bar\Lambda^c}\}].
    \end{equation}
    One may then simply apply Exercise~\ref{exo:conditioning_equality} as for the even case.
\end{proof}

Perhaps we were wondering if $\langle\sigma_0\rangle_{\Lambda_n,\beta}^+$
was decreasing in $n$ in the statement of the Peierls argument
(Theorem~\ref{thm:peierls_triangles}),
but the result we proved just now is much stronger:
we proved that the thermodynamical limit of any ``local Fourier coefficient''
is well-defined.
Rather than taking a thermodynamical of observables,
we would however like to make sense of the thermodynamical limit
of the family of measures $\langle\blank\rangle_{\Lambda,\beta}^+$.
The previous lemma enables us to do this;
we only need to set up the definitions to make formal sense of our limit.

\begin{definition}[The compact ``local convergence'' topology]
    Let $G$ denote a locally finite graph.
    Recall that $(\Omega,\calF)$ is the measurable space $\Omega:=\{\pm1\}^V$
    endowed with the product $\sigma$-algebra.
    For a domain $\Lambda$, we write $\calF_\Lambda$
    for the $\sigma$-algebra generated by spins in $\Lambda$.
    An observable $X:\Omega\to\C$ is called \emph{local} if it measurable
    with respect to $\calF_\Lambda$ for some domain $\Lambda$.

    Let $\calP(\Omega,\calF)$ denote the set of all probability measures
    on this measurable space.
    We endow this set with the \emph{local convergence topology},
    which is defined as the topology making the map
    \[
        \calP(\Omega,\calF)\to\C,\,\mu\mapsto\mu[X]
    \]
    continuous for any local observable $X$.
\end{definition}

\begin{remark}
    This topology is sometimes known under different names in the literature
    (such as the \emph{weak topology}).
    I like the name \emph{local convergence topology} because it captures the essence quite literally:
    if the statistics of the measures within a fixed domain $\Lambda$ converge,
    then we have local convergence.
\end{remark}

\begin{remark}
    The local convergence topology turns $\calP(\Omega,\calF)$ into a compact space.
    For a fixed domain $\Lambda$, the set of probability measures
    on $(\Omega,\calF_\Lambda)$ is a compact simplex in some finite-dimensional
    real vector space.
    Sequences of probability measures of this type have converging subsequences by standard
    arguments.
    Convergence for arbitrary $\Lambda$ may be obtained by a standard diagonal
    argument.
\end{remark}

\begin{theorem}[Existence of the thermodynamical limit (wired boundary)]
    Consider the Ising model on a locally finite graph $G$
    at inverse temperature $\beta$.
    Then there exists a unique probability measure
    $\langle\blank\rangle_{G,\beta}^+\in\calP(\Omega,\calF)$
    such that
    \[
        \lim_{\Lambda\uparrow V}\langle X\rangle_{\Lambda,\beta}^+=\langle X\rangle_{G,\beta}^+
    \]
    for any local observable $X:\Omega\to\R$.
    In other words,
    \[
        \lim_{\Lambda\uparrow V}\langle \blank\rangle_{\Lambda,\beta}^+
        =
        \langle \blank\rangle_{G,\beta}^+.
    \]
    The measure $\langle\blank\rangle_{G,\beta}^+$ is called
    the \emph{thermodynamical limit} or \emph{infinite-volume limit}
    with $+$ boundary conditions.
\end{theorem}

\begin{proof}
    Any local observable may be written as a finite linear conbination
    of observables of the form $\sigma_A$ where $A$ is a finite subset of
    $V$.
    The theorem then follows by compactness and Lemma~\ref{lemma:correlation_functions_monotone}.
\end{proof}

\begin{exercise}[Right-continuity in $\beta$ for a wired boundary]
    Consider the Ising model on a locally finite graph $G$.
    Prove all of the following statements.
    \begin{itemize}
        \item For any finite set $A\subset V$,
        the function $[0,\infty)\to\R,\,\beta\mapsto\langle\sigma_A\rangle_{G,\beta}^+$
        is non-decreasing and right-continuous.
        \item The function $[0,\infty)\to\calP(\Omega,\calF),\,\beta\mapsto\langle\blank\rangle_{G,\beta}^+$
        is a right-continuous function.
        \item The points of discontinuity form a countable subset of $[0,\infty)$.
    \end{itemize}
\end{exercise}

\begin{definition}[Magnetisation and critical temperature]
    Let $G$ denote a locally finite graph and $u$ some distinguished reference vertex.
    The function
    \[
        m=m_G:[0,\infty)\to\R,\,\beta\mapsto\langle\sigma_u\rangle_{G,\beta}^+
    \]
    is called the \emph{magnetisation}.

    The \emph{critical (inverse) temperature} is defined via
    \[
        \beta_c:=\inf\{\beta\in[0,\infty):m(\beta)>0\}.
    \]
\end{definition}

\begin{remark}
    The exercise above proves that $m$ is a non-decreasing right-continuous function.
    We have already seen that:
    \begin{itemize}
        \item If $G$ has max-degree $d$,
        then $m(\beta)=0$ for $\beta<1/d$ (Exercise~\ref{exercise:currents_peierls}),
        \item If $G$ is the graph $\Z^d$ for $d\geq 2$,
        then $\lim_{\beta\to\infty}m(\beta)>0$ (Exercise~\ref{exo:peierls_general}).
    \end{itemize}
    This implies in particular that on the square lattice graph $\Z^d$ in dimension
    $d\geq 2$,
    the critical inverse temperature $\beta_c$ is a strictly positive real number.
    A key objective of our field is to understand the behaviour of the Ising model
    at $\beta=\beta_c$ and at $\beta\approx\beta_c$.
    As a very first question we can ask:
    is the function $m$ continuous?
    We will see the answer in Theorem~\ref{}\todo{Add ref to theorem}.
\end{remark}



\section{The thermodynamical limit: Wired boundary, demagnetisation}
\label{sec:vanishing_magnetisation}

Our next goal is to prove the following theorem.

\begin{theorem}[$+$ and $-$ boundary conditions coincide
    when the magnetisation vanishes] 
    \label{thm:vanishing_magnetisation}
    Let $G$ denote a connected locally finite graph,
    endowed with some reference vertex $u$.
    Then 
    \[
        \langle\blank\rangle_{G,\beta}^+=\langle\blank\rangle_{G,\beta}^-
        \qquad
        \iff 
        \qquad
        m_G(\beta)=0.
    \]
\end{theorem}

The theorem can be proved using the following bound.

\begin{exercise}[Pairing bound, difficult]
    Consider the Ising model on a finite graph $G$ at inverse temperature $\beta$.
    Let $A\subset V$ denote any finite subset,
    and fix $u\in A$.
    Use the switching lemma to prove that
    \[
        \langle\sigma_A\rangle \leq
        \sum_{v\in A\setminus\{u\}}
        \langle\sigma_u\sigma_v\rangle
        \langle\sigma_{A\setminus\{u,v\}}\rangle.
    \]
    Hint: argue that
    \[
        \true{\n\in\calE_A}
        \leq
        \sum_{v\in A\setminus\{u\}}
        \true{\n\in\calE_{\{u,v\}}}
        \true{\n\in\calE_A}.
    \]

    Conclude that
    \[
        \langle\sigma_A\rangle \leq
        \sum_{\pi}
        \prod_{\{u,v\}\in \pi}
        \langle\sigma_u\sigma_v\rangle
    \]
    where $\pi$ ranges over the \emph{pairings} of $A$,
    that is, the set of partitions of $A$ in which each member has two elements.
\end{exercise}


\begin{proof}[Proof of Theorem~\ref{thm:vanishing_magnetisation}]
    Notice that $\langle\blank\rangle_{G,\beta}^+$ and $\langle\blank\rangle_{G,\beta}^-$ are related
    by a global spin flip (the pushforward map corresponding to $\sigma\mapsto-\sigma$).
    Therefore all of the following are equivalent:
    \begin{itemize}
        \item $\langle\blank\rangle_{G,\beta}^+=\langle\blank\rangle_{G,\beta}^-$,
        \item $\langle\blank\rangle_{G,\beta}^+$ is invariant under the map $\sigma\mapsto-\sigma$,
        \item $\langle\sigma_A\rangle_{G,\beta}^+=0$ whenever $A\subset V$ has odd cardinal.
    \end{itemize}

    The implication ``$\implies$'' is now obvious, and we focus on  ``$\impliedby$''.
    Suppose that $m(\beta)=0$,
    that is, $\langle\sigma_u\rangle^+=0$ where $u$ is the reference vertex.
    Fix $A\subset V$ with $|A|$ odd.
    It suffices to prove that $\langle\sigma_A\rangle^+=0$.
    We shall in fact give \emph{two} proofs of this fact.
    In both proofs, we shall fix a sequence $(\Lambda_n)_n$ of increasing subsets of $V$ with $\cup_n\Lambda_n=V$.
    \begin{itemize}
        \item \emph{Proof~1, using the pairing bound.}
        For fixed $n$, we get
        \begin{align}
            \langle\sigma_A\rangle_{\Lambda_n}^+
            =\langle\sigma_{A\cup\{\Lambda_n^c\}}\rangle_{G_n'}
            &\leq
            \sum_{v\in A}\langle\sigma_{\{v,\Lambda_n^c\}}\rangle_{G_n'}
            \langle\sigma_{A\setminus\{v\}}\rangle_{G_n'}
            =\sum_{v\in A}\langle\sigma_v\rangle_{\Lambda_n}^+
            \langle\sigma_{A\setminus\{v\}}\rangle_{\Lambda_n}^+
            \\&\leq 
            \sum_{v\in A}\langle\sigma_v\rangle_{\Lambda_n}^+
            \to_{n\to\infty}0.
        \end{align}
        The first inequality is the pairing bound,
        the second the generic bound $\langle\sigma_{A\setminus\{v\}}\rangle_{\Lambda_n}^+\in[0,1]$,
        and the convergence follows from Exercise~\ref{exercise:Definition of critical beta does not depend on ref point}.

        \item \emph{Proof~2, using directly the high-temperature expansion.}
        We only consider $\beta>0$, otherwise the spins are independent fair coin flips,
        and the result is automatic.
        Recall Exercise~\ref{exercise:Log-Lipschitz property of correlation functions}.
        For fixed $n$, we get
        \[
            (\tanh\beta)^{d_{\operatorname{Transport}}(A,\{u\})}\cdot\langle\sigma_A\rangle_{\Lambda_n}^+
            \leq 
            \langle\sigma_u\rangle_{\Lambda_n}^+.
        \]
        On the left, the transport distance is calculated in the graph $G_n'$
        (the dependence on $n$ is implicit).
        As $n\to\infty$, this transport distance stabilises at the finite transport distance
        in the infinite graph $G$.
        Since the right hand side tends to zero with $n$,
        we know that the left hand side also tends to zero.
        Since the prefactor remains uniformly positive,
        we must have $\langle\sigma_A\rangle^+=0$.
    \end{itemize}    
\end{proof}





% \section{The FKG inequality. Proof}

Cees Fortuin, Pieter Kasteleyn, and Jean Ginibre discovered a general
way to prove that increasing observables are positive correlated.
This inequality already known in the context of percolation
theory (independent randomness) as the \emph{Harris inequality},
after Theodore Harris.
In these lecture notes, we shall state and prove the FKG inequality in a simplified
context, which will be sufficient for our purposes.
The interested reader may consult the original paper, which is an accessible classic in statistical mechanics.

\begin{definition}[FKG inequality]
    Let $(\Omega,\preceq)$ denote a partially ordered set
    and $\mu$ a probability measure on $\Omega$.
    We say that $\mu$ satisfies the \emph{FKG inequality}
    if
    \begin{equation}
        \label{eq:FKG}
        \mu[fg]\geq \mu[f]\mu[g]
    \end{equation}
    for any bounded $\preceq$-nondecreasing functions
    $f,g:\Omega\to\R$.
\end{definition}

For simplicity, we often call $\preceq$-nondecreasing functions \emph{increasing}
and $\preceq$-nonincreasing functions \emph{decreasing}.
Notice that $f$ is increasing if and only if $-f$ is decreasing.
The FKG inequality may therefore be formulated in terms of decreasing functions,
or in terms of a mixture of increasing and decreasing functions.

\begin{remark}[The FKG inequality and increasing events]
    We stated the FKG inequality in terms of observables.
    Examples of observables are functions of the form $\ind{A}$.
    In that case, the FKG inequality states that, if $\mu(A)>0$,
    then
    \[
        \mu(B|A)\geq \mu(B)
    \]
    whenever $\ind{A}$ and $\ind{B}$ are increasing functions.
    Such events are called \emph{increasing events}.
\end{remark}

\begin{exercise}[Iterating the FKG inequality]
    Let $\mu$ denote a probability measure satisfying the FKG inequality and
     $(A_i)_i$ a finite family of increasing events.
    Prove that
    \[
        \mu(\cap_i A_i)\geq \prod_i\mu(A_i).
    \]
    
    If $(f_i)_i$ is a finite family of increasing functions,
    does it hold true that $\mu[\prod_i f_i]\geq \prod_i \mu[f_i]$?
    Why (not)?
\end{exercise}

We have now defined the FKG inequality,
but to derive it, we require the notion of a distributive lattice.

\begin{definition}[Distributive lattices]
    A \emph{distributive lattice} is a tuple $(\Omega,\preceq,\vee,\wedge)$
    where $(\Omega,\preceq)$ is a partially ordered set
    and where $\vee,\wedge:\Omega\times\Omega\to\Omega$
    are binary operators satisfying the following properties
    for any $x,y,z\in\Omega$:
    \begin{enumerate}
        \item $x\vee y$ equals the least upper bound of $x$ and $y$ with respect to $\preceq$,
        \item $x\wedge y$ equals the greatest lower bound of $x$ and $y$ with respect to $\preceq$,
        \item The following two \emph{distribution equations}:
        \begin{itemize}
            \item $x\wedge (y\vee z)=(x\wedge y)\vee(x\wedge z)$,
            \item $x\vee (y\wedge z)=(x\vee y)\wedge(x\vee z)$.
        \end{itemize}
    \end{enumerate}
    A distributive lattice is called \emph{finite} or \emph{countable}
    whenever $\Omega$ has these respective properties.
    It is called a \emph{binary lattice} if it is isomorphic
    to $\{0,1\}^I$ for some index set $I$.
\end{definition}


\begin{definition}[FKG lattice condition]
    Let $X:\Omega\to[0,\infty)$ denote a
    function defined on some distributive lattice $(\Omega,\preceq,\vee,\wedge)$.
    We say that $X$ satisfies the \emph{FKG lattice condition}
    if 
    \begin{equation}
        X(\omega\vee\eta)\cdot X(\omega\wedge\eta)
        \geq
        X(\omega)\cdot X(\eta)
        \qquad
        \forall\omega,\eta\in\Omega.
    \end{equation}
\end{definition}

\begin{theorem}[FKG, 1971]
    \label{thm:original_FKG}
    Let $(\Omega,\preceq,\vee,\wedge)$ denote a finite binary lattice,
    and let $X:\Omega\to[0,\infty)$ denote a strictly positive function
    satisfying the FKG lattice condition.
    Then the probability measure $\mu$ defined by its expectation functional
    \[
        \mu[f]:=\frac1Z\sum_{\omega\in\Omega}X(\omega)f(\omega);
        \qquad Z:=\sum_{\omega\in\Omega}X(\omega)
    \]
    satisfies the FKG inequality on $(\Omega,\preceq)$.
\end{theorem}

\begin{proof}
    Without loss of generality, $\Omega=\{0,1\}^n$ for some $n\in\Z_{\geq 0}$.
    We induct on $n$.
    The case $n=0$ is trivial.
    The case $n=1$ is elementary and left as an exercise for the interested reader.
    Notice that the measures
    \[
        \mu_\pm:=\mu[\blank|\Omega_\pm];
        \qquad
        \Omega_-:=\{\omega_n=0\};
        \qquad
        \Omega_+:=\{\omega_n=1\}
    \]
    satisfy the FKG inequality due to the induction hypothesis.

    \begin{claim*}
    For any increasing function $f$  on $(\Omega,\preceq)$ we have \(
            \mu_-[f]\leq\mu_+[f]
        \).
    \end{claim*}

    We shall first see how the claim implies the theorem,
    then prove the claim.
    Let $f,g:\Omega\to\R$ denote increasing functions.
    We then simply assert that
    \[
        \mu[fg] = \mu[\mu[fg|\omega_n]]
        \geq
        \mu[\mu[f|\omega_n]\mu[g|\omega_n]]
        \geq
        \mu[\mu[f|\omega_n]]\mu[\mu[g|\omega_n]]
        =
        \mu[f]\mu[g].
    \]
    The two equalities are just the tower property.
    The first inequality is the FKG inequality applied
    to the measures $\mu_\pm$.
    For the second inequality, notice that $\mu[f|\omega_n]$
    and $\mu[g|\omega_n]$ are increasing functions
    of the bit $\omega_n$,
    so that we may simply apply the FKG inequality
    coming from the $n=1$ case already discussed above.

    We now prove the claim.
    Remark that we have not yet used the FKG lattice condition;
    this will be crucial in the proof of the claim.
    Write $\omega\mapsto \omega^+$ for the obvious bijection
    from $\Omega_-$ to $\Omega_+$ (which flips the last bit).
    Then
    \[
        \mu_+[ f]
        =
        \frac{\sum_{\omega\in\Omega_-}X(\omega^+)f(\omega^+)}{\sum_{\omega\in\Omega_-}X(\omega^+)}.
    \]
    Writing $X(\omega^+)=X(\omega)X'(\omega)$
    where
    $X'(\omega):=\frac{X(\omega^+)}{X(\omega)}$,
    we get 
    \[
        \mu_+[ f]
        =
        \frac{\mu_-[ f(\omega^+) X']}{\mu_-[ X']}
        \geq
        \frac{\mu_-[ f X']}{\mu_-[ X']}
        .
    \]
    For the inequality in this display we just used that $f(\omega^+)\geq f(\omega)$.
    To conclude that the right hand side equals at least $\mu_-[ f]$
    we apply the FKG inequality to $\mu_-[\blank]$,
    observing that $f$ is increasing by assumption and that $X'$ is increasing
    due to the FKG lattice condition.
    This establishes the claim, and thus the theorem.
\end{proof}

\begin{remark}[No FKG after conditioning]
    Suppose that $\mu$ is a probability measure satisfying the FKG inequality.
    Then we have $\mu(B|A)\geq\mu(B)$ whenever $A$ and $B$ are increasing events
    with $\mu(A)>0$,
    but we \emph{do not} know if the conditional probability measure $\mu(\blank|A)$
    satisfies the FKG inequality.
\end{remark}

\begin{exercise}[No FKG after conditioning]
    Consider the following example of two independent coin flips:
    $\Omega=\{\pm1\}^2$ and $X\equiv 1$,
    so that $\mu$ is the uniform distribution on $\Omega$.
    Since $X$ satisfies the FKG lattice condition,
    the measure $\mu$ satisfies the FKG inequality.
    Let $A$ denote the increasing event that \emph{at least}
    one of the coins is valued $+1$.
    What is the correlation of the two coins in the conditional measure
    $\mu(\blank|A)$?
    Argue that this conditional measure does not satisfy the FKG inequality.
\end{exercise}

\begin{remark}[Preservation of FKG after conditioning on binary sublattices]
    Consider the setting of Theorem~\ref{thm:original_FKG}.
    That theorem does imply that the FKG inequality is preserved
    under conditioning on an event $A\subset\Omega$
    which is itself a binary lattice.
\end{remark}

% \section{The FKG inequality. Application to the Ising spins}

\begin{lemma}
    Consider the Ising model on a finite graph $G$ at inverse
    temperature $\beta\geq 0$.
    Then the map
    \[
        \sigma\mapsto e^{-H_{G,\beta}(\sigma)}
    \]
    satisfies the FKG lattice condition.
    In particular, the Ising measure
    $\langle\blank\rangle_{G,\beta}$ satisfies the FKG inequality.
\end{lemma}

\begin{proof}
    Let $\sigma,\eta\in\Omega$ denote two spin configurations.
    It suffices to show that
    \[
        H(\sigma\vee\eta)+H(\sigma\wedge\eta)
        \leq
        H(\sigma)+H(\eta).
    \]
    This is immediate from the definition of the Hamiltonian:
    it is a sum of terms of the form
    \[
        -\beta\sigma_u\sigma_v,
    \]
    while these terms satisfy the obvious inequality
    \[
        (\sigma_u\vee \eta_u)(\sigma_v\vee \eta_v)
        +
        (\sigma_u\wedge \eta_u)(\sigma_v\wedge \eta_v)
        \geq
        \sigma_u\sigma_v+\eta_u\eta_v.
    \]
    This finishes the proof.
\end{proof}

We can now already prove the first Griffiths inequality for the case of
two vertices.

\begin{corollary}
    Let $G$ be a finite graph and let $\beta\geq 0$.
    Then the associated Ising model satisfies $\langle\sigma_u\sigma_v\rangle_{G,\beta}\geq 0$.
\end{corollary}

\begin{proof}
    Note that $\sigma_u$ and $\sigma_v$ are increasing functions
    of the spin configuration $\sigma$,
    while they have zero expectation due to flip-symmetry.
    The result then follows from the FKG inequality.
\end{proof}


\bibliographystyle{amsalpha}
\bibliography{}
\end{document}
