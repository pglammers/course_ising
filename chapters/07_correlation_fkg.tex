\section{The FKG inequality. Proof}

Cees Fortuin, Pieter Kasteleyn, and Jean Ginibre discovered a general
way to prove that increasing observables are positive correlated.
This inequality already known in the context of percolation
theory (independent randomness) as the \emph{Harris inequality},
after Theodore Harris.
In these lecture notes, we shall state and prove the FKG inequality in a simplified
context, which will be sufficient for our purposes.
The interested reader may consult the original paper, which is an accessible classic in statistical mechanics.

\begin{definition}[FKG inequality]
    Let $(\Omega,\preceq)$ denote a partially ordered set
    and $\mu$ a probability measure on $\Omega$.
    We say that $\mu$ satisfies the \emph{FKG inequality}
    if
    \begin{equation}
        \label{eq:FKG}
        \mu[fg]\geq \mu[f]\mu[g]
    \end{equation}
    for any bounded $\preceq$-nondecreasing functions
    $f,g:\Omega\to\R$.
\end{definition}

For simplicity, we often call $\preceq$-nondecreasing functions \emph{increasing}
and $\preceq$-nonincreasing functions \emph{decreasing}.
Notice that $f$ is increasing if and only if $-f$ is decreasing.
The FKG inequality may therefore be formulated in terms of decreasing functions,
or in terms of a mixture of increasing and decreasing functions.

\begin{remark}[The FKG inequality and increasing events]
    We stated the FKG inequality in terms of observables.
    Examples of observables are functions of the form $\ind{A}$.
    In that case, the FKG inequality states that, if $\mu(A)>0$,
    then
    \[
        \mu(B|A)\geq \mu(B)
    \]
    whenever $\ind{A}$ and $\ind{B}$ are increasing functions.
    Such events are called \emph{increasing events}.
\end{remark}

\begin{exercise}[Iterating the FKG inequality]
    Let $\mu$ denote a probability measure satisfying the FKG inequality and
     $(A_i)_i$ a finite family of increasing events.
    Prove that
    \[
        \mu(\cap_i A_i)\geq \prod_i\mu(A_i).
    \]
    
    If $(f_i)_i$ is a finite family of increasing functions,
    does it hold true that $\mu[\prod_i f_i]\geq \prod_i \mu[f_i]$?
    Why (not)?
\end{exercise}

We have now defined the FKG inequality,
but to derive it, we require the notion of a distributive lattice.

\begin{definition}[Distributive lattices]
    A \emph{distributive lattice} is a tuple $(\Omega,\preceq,\vee,\wedge)$
    where $(\Omega,\preceq)$ is a partially ordered set
    and where $\vee,\wedge:\Omega\times\Omega\to\Omega$
    are binary operators satisfying the following properties
    for any $x,y,z\in\Omega$:
    \begin{enumerate}
        \item $x\vee y$ equals the least upper bound of $x$ and $y$ with respect to $\preceq$,
        \item $x\wedge y$ equals the greatest lower bound of $x$ and $y$ with respect to $\preceq$,
        \item The following two \emph{distribution equations}:
        \begin{itemize}
            \item $x\wedge (y\vee z)=(x\wedge y)\vee(x\wedge z)$,
            \item $x\vee (y\wedge z)=(x\vee y)\wedge(x\vee z)$.
        \end{itemize}
    \end{enumerate}
    A distributive lattice is called \emph{finite} or \emph{countable}
    whenever $\Omega$ has these respective properties.
    It is called a \emph{binary lattice} if it is isomorphic
    to $\{0,1\}^I$ for some index set $I$.
\end{definition}


\begin{definition}[FKG lattice condition]
    Let $X:\Omega\to[0,\infty)$ denote a
    function defined on some distributive lattice $(\Omega,\preceq,\vee,\wedge)$.
    We say that $X$ satisfies the \emph{FKG lattice condition}
    if 
    \begin{equation}
        X(\omega\vee\eta)\cdot X(\omega\wedge\eta)
        \geq
        X(\omega)\cdot X(\eta)
        \qquad
        \forall\omega,\eta\in\Omega.
    \end{equation}
\end{definition}

\begin{theorem}[FKG, 1971]
    \label{thm:original_FKG}
    Let $(\Omega,\preceq,\vee,\wedge)$ denote a finite binary lattice,
    and let $X:\Omega\to[0,\infty)$ denote a strictly positive function
    satisfying the FKG lattice condition.
    Then the probability measure $\mu$ defined by its expectation functional
    \[
        \mu[f]:=\frac1Z\sum_{\omega\in\Omega}X(\omega)f(\omega);
        \qquad Z:=\sum_{\omega\in\Omega}X(\omega)
    \]
    satisfies the FKG inequality on $(\Omega,\preceq)$.
\end{theorem}

\begin{proof}
    Without loss of generality, $\Omega=\{0,1\}^n$ for some $n\in\Z_{\geq 0}$.
    We induct on $n$.
    The case $n=0$ is trivial.
    The case $n=1$ is elementary and left as an exercise for the interested reader.
    Notice that the measures
    \[
        \mu_\pm:=\mu[\blank|\Omega_\pm];
        \qquad
        \Omega_-:=\{\omega_n=0\};
        \qquad
        \Omega_+:=\{\omega_n=1\}
    \]
    satisfy the FKG inequality due to the induction hypothesis.

    \begin{claim*}
    For any increasing function $f$  on $(\Omega,\preceq)$ we have \(
            \mu_-[f]\leq\mu_+[f]
        \).
    \end{claim*}

    We shall first see how the claim implies the theorem,
    then prove the claim.
    Let $f,g:\Omega\to\R$ denote increasing functions.
    We then simply assert that
    \[
        \mu[fg] = \mu[\mu[fg|\omega_n]]
        \geq
        \mu[\mu[f|\omega_n]\mu[g|\omega_n]]
        \geq
        \mu[\mu[f|\omega_n]]\mu[\mu[g|\omega_n]]
        =
        \mu[f]\mu[g].
    \]
    The two equalities are just the tower property.
    The first inequality is the FKG inequality applied
    to the measures $\mu_\pm$.
    For the second inequality, notice that $\mu[f|\omega_n]$
    and $\mu[g|\omega_n]$ are increasing functions
    of the bit $\omega_n$,
    so that we may simply apply the FKG inequality
    coming from the $n=1$ case already discussed above.

    We now prove the claim.
    Remark that we have not yet used the FKG lattice condition;
    this will be crucial in the proof of the claim.
    Write $\omega\mapsto \omega^+$ for the obvious bijection
    from $\Omega_-$ to $\Omega_+$ (which flips the last bit).
    Then
    \[
        \mu_+[ f]
        =
        \frac{\sum_{\omega\in\Omega_-}X(\omega^+)f(\omega^+)}{\sum_{\omega\in\Omega_-}X(\omega^+)}.
    \]
    Writing $X(\omega^+)=X(\omega)X'(\omega)$
    where
    $X'(\omega):=\frac{X(\omega^+)}{X(\omega)}$,
    we get 
    \[
        \mu_+[ f]
        =
        \frac{\mu_-[ f(\omega^+) X']}{\mu_-[ X']}
        \geq
        \frac{\mu_-[ f X']}{\mu_-[ X']}
        .
    \]
    For the inequality in this display we just used that $f(\omega^+)\geq f(\omega)$.
    To conclude that the right hand side equals at least $\mu_-[ f]$
    we apply the FKG inequality to $\mu_-[\blank]$,
    observing that $f$ is increasing by assumption and that $X'$ is increasing
    due to the FKG lattice condition.
    This establishes the claim, and thus the theorem.
\end{proof}

\begin{remark}[No FKG after conditioning]
    Suppose that $\mu$ is a probability measure satisfying the FKG inequality.
    Then we have $\mu(B|A)\geq\mu(B)$ whenever $A$ and $B$ are increasing events
    with $\mu(A)>0$,
    but we \emph{do not} know if the conditional probability measure $\mu(\blank|A)$
    satisfies the FKG inequality.
\end{remark}

\begin{exercise}[No FKG after conditioning]
    Consider the following example of two independent coin flips:
    $\Omega=\{\pm1\}^2$ and $X\equiv 1$,
    so that $\mu$ is the uniform distribution on $\Omega$.
    Since $X$ satisfies the FKG lattice condition,
    the measure $\mu$ satisfies the FKG inequality.
    Let $A$ denote the increasing event that \emph{at least}
    one of the coins is valued $+1$.
    What is the correlation of the two coins in the conditional measure
    $\mu(\blank|A)$?
    Argue that this conditional measure does not satisfy the FKG inequality.
\end{exercise}

\begin{remark}[Preservation of FKG after conditioning on binary sublattices]
    Consider the setting of Theorem~\ref{thm:original_FKG}.
    That theorem does imply that the FKG inequality is preserved
    under conditioning on an event $A\subset\Omega$
    which is itself a binary lattice.
\end{remark}
