\section{Monotonicity via the second Griffiths inequality}

\begin{theorem}[Monotonicity in the temperature]
    Let $G$ denote a finite graph and $A\subset V$ a finite set.
    Then the function $\beta\mapsto\langle\sigma_A\rangle_{G,\beta}$
    is non-decreasing.
\end{theorem}

\begin{proof}
    We want to prove that
    \[
        \frac{\partial}{\partial\beta}
        \langle\sigma_A\rangle_{G,\beta}
        =
        \frac{\partial}{\partial\beta}
        \left(
            \frac{
                \sum_\sigma\sigma_A\prod_{uv}e^{\beta\sigma_u\sigma_v}
            }{
                \sum_\sigma\prod_{uv}e^{\beta\sigma_u\sigma_v}
            }
        \right)
        \geq 0.
    \]
    Since we are differentiating a fraction,
    it suffices to show that the numerator grows at a faster rate
    than the denominator,
    that is,
    \[
        \frac{\frac{\partial}{\partial\beta}
            \sum_\sigma\sigma_A\prod_{uv}e^{\beta\sigma_u\sigma_v}
        }{
            Z\langle\sigma_A\rangle
        }
        \geq
        \frac{\frac{\partial}{\partial\beta}
            \sum_\sigma\prod_{uv}e^{\beta\sigma_u\sigma_v}
        }{
            Z
        }.
    \]
    We perform the differential and then multiply each side by $\langle\sigma_A\rangle$,
    to see that this inequality is equivalent to
    \[
       \sum_{xy}
        \frac{\sum_\sigma
            \sigma_x\sigma_y\sigma_A\prod_{uv}e^{\beta\sigma_u\sigma_v}
        }{Z}
        \geq
        \langle\sigma_A\rangle
        \sum_{xy}
        \frac{
            \sum_\sigma
            \sigma_x\sigma_y
            \prod_{uv}e^{\beta\sigma_u\sigma_v}
        }{Z}.
    \]
    Each fraction may now be reinterpreted as a correlation function,
    so that the previous inequality is equivalent to
    \[
        \sum_{xy}\langle\sigma_x\sigma_y\sigma_A\rangle
        \geq
        \langle\sigma_A\rangle
        \sum_{xy}\langle\sigma_x\sigma_y\rangle.
    \]
    But this is just the second Griffiths inequality.
\end{proof}

\begin{exercise}[Regularity properties of the correlation functions in $\beta$]
    Prove that the function $[0,\infty)\to\R,\,\beta\mapsto\langle\sigma_A\rangle_{G,\beta}$
    in the above context is an analytic function.
\end{exercise}

Next, we want to prove monotonicity in domains.
We first challenge the reader to prove the following exercise.


\begin{exercise}[Conditioning on equality increases the correlation functions]
    \label{exo:conditioning_equality}
    Consider the Ising model on a finite graph $G$
    at inverse temperature $\beta$, and fix some subset $A\subset V$.
    \begin{itemize}
        \item     Prove that for any two distinct vertices $u,v\in V$,
        we have
        \[
            \E_{G,\beta}[\sigma_A|\{\sigma_u=\sigma_v\}]
            \geq
            \E_{G,\beta}[\sigma_A]=\langle\sigma_A\rangle_{G,\beta}.
        \]
        \item Prove for any $X\subset Y\subset V$, we have
        \[
            \E_{G,\beta}[\sigma_A|\{\text{$\sigma$ is constant on $X$}\}]
            \leq
            \E_{G,\beta}[\sigma_A|\{\text{$\sigma$ is constant on $Y$}\}]
            .
        \]
    \end{itemize}
\end{exercise}

\begin{lemma}[Monotonicity in domains]
    \label{lemma:correlation_functions_monotone_both}
    Consider the Ising model on a locally finite graph
    $G=(V,E)$ at inverse temperature $\beta$.
    Consider two finite domains $\Lambda\subset\Lambda'\subset V$
    and a subset $A\subset \Lambda$.
    \begin{itemize}
        \item \textbf{Free boundary.}
        We have $\langle\sigma_A\rangle^\f_{\Lambda,\beta}\leq\langle\sigma_A\rangle^\f_{\Lambda',\beta}$.
        \item \textbf{Wired boundary.}
        We have $\langle\sigma_A\rangle^+_{\Lambda,\beta}\geq\langle\sigma_A\rangle^+_{\Lambda',\beta}$.
    \end{itemize}
\end{lemma}

\begin{proof}[Proof for $\langle\blank\rangle^\f$]
    We first prove the following claim:
    if $G'$ and $G''$ are finite graphs on the same vertex set,
    and such that $E(G'')=E(G')\cup\{xy\}$,
    then
    \[
        \langle\sigma_A\rangle_{G',\beta}
        \leq
        \langle\sigma_A\rangle_{G'',\beta}
    \]
    for any $A\subset V(G')$.
    To prove the claim, we simply expand 
    \[
        \langle\sigma_A\rangle_{G'',\beta}
        =
        \frac{
            \langle   e^{\beta\sigma_x\sigma_y}\sigma_A\rangle_{G'}
        }{
            \langle e^{\beta\sigma_x\sigma_y}\rangle_{G'}
        }.
    \]
    Thus, we want to show that
    \[
            \langle e^{\beta\sigma_x\sigma_y} \sigma_A \rangle_{G'}
            \geq 
            \langle e^{\beta\sigma_x\sigma_y}\rangle_{G'}
            \langle\sigma_A\rangle_{G'}.
    \]
    This follows from the second Griffiths inequality.
    We have now proved the claim.

    Recall the definition of the finite graph $\Lambda^\f$.
    Let $\tilde\Lambda^\f:=((\Lambda')^\f,E(\Lambda^\f))$;
    this is just the graph $\Lambda^\f$
    supplemented with some isolated vertices $\Lambda'\setminus\Lambda$.
    The law of $\sigma$ in $\langle\blank\rangle_{\tilde\Lambda^\f}$
    is just given by $\langle\blank\rangle_{\Lambda^\f}$,
    with independent fair coin flips for the isolated vertices in $\Lambda'\setminus\Lambda$.
    Thus, it suffices to prove that
    \[
        \langle\sigma_A\rangle^\f_{\Lambda}
        =
        \langle\sigma_A\rangle_{\tilde\Lambda^\f}
        \leq
        \langle\sigma_A\rangle_{(\Lambda')^\f}
        =
        \langle\sigma_A\rangle^\f_{\Lambda'}
        .
    \]
    This follows from the claim.
\end{proof}


\begin{proof}[Proof for $\langle\blank\rangle^+$]
    Without loss of generality,
    $\Lambda'\setminus\Lambda=\{u\}$ for some
    vertex $u\in V$.
    We make all calculations in the graph $(\Lambda')^\frakg$ with the ghost vertex:
    we get
    \[
        \langle\sigma_A\rangle_{\Lambda'}^+
        =
        \E_{(\Lambda')^\frakg}[\sigma_A|\{\sigma_\frakg=+\}];
        \qquad
        \langle\sigma_A\rangle_{\Lambda}^+
        =
        \E_{(\Lambda')^\frakg}[\sigma_A|\{\sigma_\frakg=+\}\cap\{\sigma_u=\sigma_\frakg\}].
    \]

    Assume first that $|A|$ is even for now.
    Then
    \begin{equation}
        \langle\sigma_A\rangle_{\Lambda}^+
        =
        \E_{(\Lambda')^\frakg}[\sigma_A|\{\sigma_u=\sigma_\frakg\}]
        \geq
        \E_{(\Lambda')^\frakg}[\sigma_A]
        =
        \langle\sigma_A\rangle_{\Lambda'}^+,
    \end{equation}
    due to Exercise~\ref{exo:conditioning_equality}.

    If $|A|$ is odd, we just need to replace the set $A$
    by $A':=A\cup\{\frakg\}$.
    More precisely,
    \begin{equation}
        \langle\sigma_A\rangle_{\Lambda}^+
        =
        \E_{(\Lambda')^\frakg}[\sigma_{A'}|\{\sigma_u=\sigma_\frakg\}]
        \geq
        \E_{(\Lambda')^\frakg}[\sigma_{A'}]
        =
        \langle\sigma_A\rangle_{\Lambda'}^+,
    \end{equation}
    where the inequality uses the same exercise.

    Those are the desired inequalities.
\end{proof}

