\section{Boundary conditions and the Markov property}

We now introduce two related concepts:
\emph{boundary conditions} and the \emph{Markov property}.
Boundary conditions arise when taking an Ising model and conditioning
on the spins at certain vertices.
An essential property of the nearest-neighbour Ising model is that the spins
can only communicate via the graph that the model is defined on.
This property is expressed in terms of the Markov property.

\begin{definition}[The Ising model with boundary conditions]
    \label{def:ising_bc}
    Let $G=(V,E)$ denote a fixed locally finite graph.
    We consider the measurable space 
    $(\Omega,\calF)$ where $\Omega=\{\pm1\}^V$
    and where $\calF$ is the product-$\sigma$-algebra.

    Let $\Lambda$ denote a \emph{domain},
    that is, a finite subset of $V$.
    The Ising model in the domain $\Lambda$
    with inverse temperature $\beta \geq 0$ and boundary conditions
    $\zeta\in\{\pm1\}^{\Lambda^c}$
    is the probability measure $\P_{\Lambda,\beta}^{\operatorname{Ising},\zeta}$
    on $(\Omega,\calF)$ defined via:
    \[
        \P_{\Lambda,\beta}^{\operatorname{Ising},\zeta}(\sigma)
        =
        \frac1{Z_{\Lambda,\beta}^{\operatorname{Ising},\zeta}}
        \cdot
        \true{\sigma|_{\Lambda^c}=\zeta}\cdot e^{-H_{\Lambda,\beta}^{\operatorname{Ising}}(\sigma)}
        ,
    \]
    where $H_{\Lambda,\beta}^{\operatorname{Ising}}(\sigma)$ is the Hamiltonian given by
    \[
        H_{\Lambda,\beta}^{\operatorname{Ising}}(\sigma)
        =
        -\beta\sum_{uv\in E(\Lambda)}\sigma_u\sigma_v
        ,
    \]
    and where the partition function $Z_{\Lambda,\beta}^{\operatorname{Ising},\zeta}$ is given by
    \[
        Z_{\Lambda,\beta}^{\operatorname{Ising},\zeta}
        =
        \sum_{\sigma \in \Omega} \true{\sigma|_{\Lambda^c}=\zeta} \cdot e^{-H_{\Lambda,\beta}^{\operatorname{Ising}}(\sigma)}
        .
    \]
    The set $E(\Lambda)\subset E$ denotes 
    the set of edges with at least one endpoint in $\Lambda$.
    Indeed, adding a constant to the Hamiltonian does not affect the measure,
    and edges which do not intersect $\Lambda$ contribute with a constant.

    We write $+$ and $-$ for the boundary conditions
    $+1\in\Omega$ and $-1\in\Omega$ respectively.
\end{definition}



Recall the definition of the Ising model on a finite graph
(Definition~\ref{def:ising_finite})
and on general graphs with boundary conditions (Definition~\ref{def:ising_bc}).
The second definition includes the first, since we may simply 
choose our domain $\Lambda$ to be the full vertex set whenever the
graph $G=(V,E)$ is finite.
That is why we state our results for general graphs with boundary conditions
in this section.

One important property of the definition with boundary conditions is that
it in fact encodes \emph{conditional probability measures}.

\begin{lemma}[Boundary conditions as conditional measures]
    \label{lemma:boundary_conditions_conditional_measures}
    Let $G$ denote a locally finite graph and $\beta\in[0,\infty)$
    an inverse temperature.
    Let $\Lambda\subset\Delta$ denote two finite domains
    and fix $\xi\in\{\pm1\}^{\Delta^c}$
    and $\zeta\in\{\pm1\}^{\Delta\setminus\Lambda}$.
    Then
    \[
        \P_{\Delta,\beta}^\xi(\blank|\{\sigma|_{\Delta\setminus\Lambda}=\zeta\})
        =
        \P_{\Lambda,\beta}^{\xi\zeta}.
    \]
\end{lemma}

\begin{proof}
    For the two measures, we get
    \begin{align}
        \P_{\Delta,\beta}^\xi(\sigma|\{\sigma|_{\Delta\setminus\Lambda}=\zeta\})
        &\propto\true{\sigma|_{\Lambda^c}=\xi\zeta}\cdot e^{-H_{\Delta,\beta}(\sigma)};
        \\\P_{\Lambda,\beta}^{\xi\zeta}(\sigma)
        &\propto\true{\sigma|_{\Lambda^c}=\xi\zeta}\cdot e^{-H_{\Lambda,\beta}(\sigma)}.
    \end{align}
    But $H_{\Delta,\beta}-H_{\Lambda,\beta}$ is constant on the
    event $\{\sigma|_{\Lambda^c}=\xi\zeta\}$,
    which means that the two probability measures are the same.
\end{proof}

The Ising model is a \emph{nearest-neighbour model},
meaning that the interactions are associated with the edges of the graph.
A consequence of this is the so-called \emph{Markov property}.
There are several ways to state it.
We shall first state and prove the following lemma.
For any domain $\Lambda\subset V$,
we let $\partial\Lambda\subset V$ denote the set of vertices
at graph distance one from $\Lambda$.
This is called the \emph{boundary} of $\Lambda$.

\begin{lemma}[Markov property]
    \label{lemma:markov_property_general}
    Consider a locally finite graph $G$,
    an inverse temperature $\beta$,
    a domain $\Lambda$,
    and a boundary condition $\zeta$.
    Let $(\Lambda_i)_i$ denote the partition of $\Lambda$
    into connected components.
    Then in the measure
    $\P_{\Lambda,\beta}^\zeta$,
    the family $(\sigma|_{\Lambda_i})_i$
    is a family of independent random variables.
    Moreover, the distribution of $\sigma|_{\Lambda_i}$ only depends on
    $\zeta|_{\partial\Lambda_i}$.
\end{lemma}

\begin{proof}
    We have
    $\P_{\Lambda,\beta}^\zeta(\sigma)\propto \true{\sigma|_{\Lambda^c}=\zeta}\cdot e^{-H_{\Lambda,\beta}(\sigma)}$.
    The Hamiltonian may be written
    \[
        H_{\Lambda,\beta}(\sigma)
        =
        \sum_i H_{\Lambda_i,\beta}(\sigma).
    \]
    But each term $H_{\Lambda_i,\beta}(\sigma)$
    is a function of $\sigma|_{\Lambda_i}$ and $\zeta|_{\partial\Lambda_i}$.
    This means that $e^{-H_{\Lambda,\beta}(\sigma)}$ 
    may be written as a product of factors, where the factor corresponding
    to $\Lambda_i$ only depends on $\sigma|_{\Lambda_i}$ and $\zeta|_{\partial\Lambda_i}$.
    This implies the desired independence.
\end{proof}

The Markov property is often phrased in a slightly different fashion.

\begin{theorem}[Markov property]
    Consider a locally finite graph $G$,
    and inverse temperature $\beta$,
    and two domains $\Lambda\subset\Delta$.
    Let $\zeta\in\{\pm1\}^{\Delta^c}$ denote a boundary condition,
    and fix $\xi\in\{\pm1\}^{\partial\Lambda}$.
    If $\P_{\Delta,\beta}^\zeta(\{\sigma|_{\partial\Lambda=\xi}\})>0$,
    then in the conditional probability measure
    \[
        \P_{\Delta,\beta}^\zeta(\blank|\{\sigma|_{\partial\Lambda=\xi}\})
        =
        \P_{\Delta\cup\partial\Lambda,\beta}^{\zeta\xi}
        ,
    \]
    the random variables $\sigma|_{\Lambda}$
    and $\sigma|_{\Lambda^c}$ are independent.
    Moreover, the distribution of $\sigma|_{\Lambda}$ only depends on $\xi$.
\end{theorem}

\begin{proof}
The two measures in the display in this theorem are equal because of Lemma~\ref{lemma:boundary_conditions_conditional_measures}.
The theorem is then a mere corollary of the previous lemma (Lemma~\ref{lemma:markov_property_general}).
\end{proof}


