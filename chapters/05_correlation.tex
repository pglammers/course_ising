\section{Correlation inequalities}
\label{sec:correlation}

Peierls' argument is simple and robust, but also quite ad-hoc in the sense
that it does not serve as a building block for further analysis.
We now want to take a slightly more systematic approach to the Ising model.
At the centre of the study of the Ising model are \emph{correlation functions}
and \emph{correlation inequalities}.

Let $\langle\blank\rangle$ denote an Ising model (in a finite graph,
or in a finite domain with boundary conditions).
For any finite set $A\subset V$, we define
\[
    \sigma_{A}:=\prod_{u\in A}\sigma_u.
\]
Its expectation $\langle\sigma_A\rangle$ is called a \emph{correlation function}.

\begin{exercise}
    Consider an Ising model $\langle\blank\rangle_{G,\beta}$
    on a finite graph $G=(V,E)$.
    This is a probability measure on $\Omega=\{\pm1\}^V$.
    Notice that the sample space $\Omega$ has the structure of a finite Abelian group.
    How is the Fourier transform of $\langle\blank\rangle_{G,\beta}$ related
    to the family $(\langle\sigma_A\rangle_{G,\beta})_A$ of correlation functions?
\end{exercise}

Correlation functions are at the centre of the study of the Ising model.
Inequalities between correlation functions are called \emph{correlation inequalities}.
We state some examples in the finite graph setting:
\begin{itemize}
    \item The \emph{first Griffiths inequality}, which asserts that for any $A\subset V$,
        \[
            \langle\sigma_A\rangle\geq 0.
        \]
    \item The \emph{second Griffiths inequality}, which asserts that for any $A,B\subset V$,
    \[
        \langle\sigma_A\sigma_B\rangle
        \geq
        \langle\sigma_A\rangle\langle\sigma_B\rangle.
    \]
    \item The \emph{Fortuin--Kasteleyn--Ginibre (FKG) inequality}, which asserts that
    if $X,Y:\Omega\to\R$ are two non-decreasing functions on the partially ordered set $\Omega$,
    then
    \[
        \langle XY\rangle
        \geq
        \langle X\rangle\langle Y\rangle.
    \]    
\end{itemize}
Such inequalities may be used to prove interesting properties about the Ising model.

\begin{remark}
    \todo{write remark about boundary conditions}
\end{remark}

\begin{exercise}

\end{exercise}


