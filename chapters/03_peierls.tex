\section{Peierls' argument}
\label{sec:peierls}

Peierls disproved Ising's conjecture for the absence of phase transition
in dimension $d\geq 2$.

\begin{theorem}[Peierls, 1936]
    \label{thm:peierls}
    Consider the finite domains $\Lambda_n:=\{-n,\dots,n\}^2$ of the square lattice graph $\Z^2$.
    Then for sufficiently large $\beta\in[0,\infty)$, we have
    \[
        \inf_n\langle\sigma_{(0,0)}\rangle_{\Lambda_n,\beta}^+>0.
    \]
\end{theorem}

\begin{proof}
    Our objective is to prove that $\P_{\Lambda_n,\beta}^+[\{\sigma_{(0,0)}=-\}]\leq\frac13$
    for all $n$.
    Fix $n$.

    Consider the set $\Omega'\subset\Omega$ of spin configurations on $\bar\Lambda_n$
    which assign $+$ to $\partial\Lambda_n$.
    The two-dimensional square lattice graph $G=\Z^2$ is planar, and
    therefore we may consider its planar dual $G^*$.
    For any spin configuration $\sigma\in\Omega'$,
    we let $\calI(\sigma)\subset E(G^*)$ denote its \emph{interface},
    that is, the set of dual edges separating two spins with a \emph{distinct} value.
    Notice that:
    \begin{itemize}
        \item The map $\sigma\mapsto \calI(\sigma)$ is injective,
        \item If $\sigma_{(0,0)}=-$, then $\calI(\sigma)$ contains at least one self-avoiding loop around $(0,0)$.
    \end{itemize}
    In particular, inclusion of events yields
    \begin{equation}
        \label{eq:peierlsgeneral}
        \P_{\Lambda_n,\beta}^+[\{\sigma_{(0,0)}=-\}]
        \leq
        \P_{\Lambda_n,\beta}^+[\{\text{$\calI(\sigma)$ contains a loop $\gamma$ around $(0,0)$}\}].
    \end{equation}

    We would now like to make a competition between entropy and energy appear,
    as for the Curie-Weiss model.
    The entropy comes from the choice of the loop $\gamma$;
    the energy comes into play when upper bounding the probability that a particular loop
    belongs to $\calI(\sigma)$.
    For large $\beta$, energy wins over entropy, yielding the desired bound.
    Let us start with the energy bound.

    \begin{claim*}
        For any fixed loop $\gamma$, we may bound $\P_{\Lambda_n,\beta}^+[\{\gamma\subset\calI(\sigma)\}]\leq e^{-2\beta|\gamma|}$.
    \end{claim*}

    \begin{proof}[Proof of the claim]
    We would like to define a \emph{loop erasure map} $\calE:\{\gamma\subset\calI(\sigma)\}\to\Omega'$,
    which has the property that it removes the loop $\gamma$ from the interface,
    that is,
    \[
        \calI(\calE(\sigma))=\calE(\sigma)\setminus\gamma.
    \]
    It is easy to realise such a map: we simply define $\calE$ such that it
    flips the sign of every vertex of $\Lambda_n$ which is surrounded by $\gamma$.
    Since $\calI$ is injective, the map $\calE$ is also injective,
    and we have
    \[
        \P_{\Lambda_n,\beta}^+[\{\gamma\subset\calI(\sigma)\}]
        =
        \frac
        {\sum_{\sigma\in\operatorname{Domain}(\calE)}e^{-H(\sigma)}}
        {\sum_{\sigma\in\Omega'}e^{-H(\sigma)}}
        \leq
        \frac
        {\sum_{\sigma\in\operatorname{Domain}(\calE)}e^{-H(\sigma)}}
        {\sum_{\sigma\in\operatorname{Image}(\calE)}e^{-H(\sigma)}}
        =
        e^{-2\beta|\gamma|}.
    \]
    The last equality is easy,
    since for any $\sigma\in\{\gamma\subset\calI(\sigma)\}$,
    we have
    \[
        H(\calE(\sigma))=H(\sigma)-2\beta|\gamma|,
    \]
    since $\calE$ removes precisely $|\gamma|$ disagreement edges from the interface.
    This proves the claim.
    \renewcommand{\qedsymbol}{}
    \end{proof}

    We use the energy bound to prove another interesting intermediate result.

    \begin{claim*}[Exponential decay of the loop length]
        For any dual edge $e$, we have
        \[
        \P_{\Lambda_n,\beta}^+[\{\text{$\calI(\sigma)$ contains a loop of length at least $\ell$ through $e$}\}]
        \leq (3e^{-2\beta})^\ell \frac{1}{1-3e^{-2\beta}}
        \]
        whenever $3e^{-2\beta}<1$.
    \end{claim*}

    \begin{proof}[Proof of the claim]
        Let $\calL_k$ denote the set of self-avoiding loops through $e$
        of length $k$.
        Notice that $|\calL_k|\leq 3^k$.
        A union bound yields
        \begin{multline}
            \P_{\Lambda_n,\beta}^+[\{\text{$\calI(\sigma)$ contains a loop of length at least $\ell$ through $e$}\}]
            \\
            \leq
            \sum_{k=\ell}^\infty \sum_{\gamma\in\calL_k} 
            \P_{\Lambda_n,\beta}^+[\{\gamma\subset\calI(\sigma)\}]
            \leq
            \sum_{k=\ell}^\infty |\calL_k|\cdot e^{-2\beta k}
            \leq 
            \sum_{k=\ell}^\infty 3^k\cdot e^{-2\beta k}
            \\
            =
            (3e^{-2\beta})^\ell \frac{1}{1-3e^{-2\beta}}.
        \end{multline}
        This is the desired bound.
    \renewcommand{\qedsymbol}{}
    \end{proof}

    Return to Equation~\eqref{eq:peierlsgeneral}.
    If $\calI(\sigma)$ contains a loop around $(0,0)$, then this loop 
    must hit $(k-\frac12,0)$ for some $k\in\Z_{\geq 1}$,
    and this loop must have at least $k$ steps.
    Thus, another union bound yields
    \[
        \P_{\Lambda_n,\beta}^+[\{\sigma_{(0,0)}=-\}]
        \leq
        \sum_{k=1}^\infty 
        (3e^{-2\beta})^{k} \frac{1}{1-3e^{-2\beta}}
        =(3e^{-2\beta})\frac{1}{(1-3e^{-2\beta})^2}.
    \]
    The right hand side is smaller than $\frac13$ when $\beta$ is sufficiently large,
    independently of $n$.
\end{proof}

\begin{remark*}
    Peierls' is robust,
    in the sense that it can be adapted to many other models in statistical mechanics.
\end{remark*}

\begin{exercise}[The Peierls argument in higher dimensions]
    \label{exo:peierls_general}
    Now consider the square lattice graph $\Z^d$ in dimension $d\geq 3$.
    What is the structure of the interface in this case?
    Can we adapt Peierls' to prove magnetisation for sufficiently large $\beta$?
\end{exercise}
