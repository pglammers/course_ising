\section{Early developments. 1936: Peierls' argument}
\label{sec:peierls}

\begin{theorem}[Peierls, 1936]
    \label{thm:peierls}
    The Ising model exhibits phase transition in two dimensions.
\end{theorem}

We shall prove a slight variation of Peierls' original setup,
so that we can fully focus the proof on the core idea.
Let $\T$ denote the triangular lattice graph,
comprised of vertices of the form
\[
    \T:=\left\{n+m e^{\pi i/3}:n,m\in\Z \right\}\subset\C,
\]
and such that each vertex is connected to the six
vertices at distance one.
Let $\Lambda_n\subset\T$ denote the set of vertices at a graph
distance at most $n-1$ from $0\in\T$.
We consider the Ising model on the infinite graph $\T$.
We shall prove the following version of Peierls' result.

\begin{theorem}[Peierls, 1936]
    \label{thm:peierls_triangles}
    Consider the Ising model on the two-dimensional
    triangular lattice graph $\T$.
    For sufficiently large $\beta$,
    we have
    \[
        \inf_{n}\langle\sigma_0\rangle_{\Lambda_n,\beta}^+
        >0.
    \]
\end{theorem}

Let $\H:=\T^*$ denote the hexagonal lattice
that is dual to the triangular lattice.
For a fixed configuration $\sigma\in\Omega$,
we let $\calI(\sigma)\subset E(\H)$ denote the set of
hexagonal lattice edges separating hexagons with different spins.
The set $\calI(\sigma)$ is called the
\emph{interface} between the spins valued $+1$
and those valued $-1$.
\todo{Add figure}
Notice that $\calI(\sigma)$ has a partition into
loops and bi-infinite paths.
If only finitely many spins of $\sigma$ are valued $-1$,
then there are no bi-infinite paths,
and all connected components of $\calI(\sigma)$
are loops.
This happens almost surely when sampling from $\langle\blank\rangle_{\Lambda_n,\beta}^+$.

The core of Peierls' argument is the following lemma.

\begin{lemma}[Exponential decay of loop lengths]
    \label{lem:exp_decay_ising_loops}
    Consider the Ising model on the two-dimensional triangular lattice $\T$
    at inverse temperature $\beta$.
    Suppose that $e^{-2\beta}<\frac12$.
    Then for any hexagonal lattice edge $e\in E(\H)$
    and for any minimal loop length $\ell\in\Z_{\geq 1}$,
    we get
    \[
        \P_{\Lambda_n,\beta}^+
        (\{\text{$\calI(\sigma)$ has a loop of length at least $\ell$ through $e$}\})
        \leq \frac{(2e^{-2\beta})^\ell}{1-2e^{-2\beta}},
    \]
    uniformly in $n$.
\end{lemma}

\begin{proof}
    Fix $\beta$, $e$, and $n$.
    Let $\calL$ denote a loop through $e$,
    and consider the event $\{\calL\subset\calI\}$.
    We claim that
    \[
        \P_{\Lambda_n,\beta}^+
        (\{\calL\subset\calI\})
        \leq e^{-2\beta|\calL|}.
    \]

    To prove the claim,
    we introduce the injective ``loop erasure map''
    \[
        \calE_\calL:\{\calL\subset\calI\}
        \to \Omega\setminus \{\calL\subset\calI\},
    \]
    which is defined such that it flips all the spins inside the loop $\calL$.
    As a consequence, $\calI(\calE_\calL(\sigma))=\calI(\sigma)\setminus\calL$.
    For any $\sigma\in\{\calL\subset\calI\}$, we have
    \[
        \P_{\Lambda_n,\beta}^+
        (\sigma)
        =
        e^{-2\beta|\calL|}
        \cdot
        \P_{\Lambda_n,\beta}^+
        (\calE_\calL(\sigma))
        .
    \]
    We can write down this identity because we know that the loop erasure map
    decreases the Hamiltonian by $2\beta|\calL|$.
    Since $\calE_\calL$ is injective, we get
    \[
        \P_{\Lambda_n,\beta}^+
        (\{\calL\subset\calI\})
        = e^{-2\beta|\calL|}\cdot \P_{\Lambda_n,\beta}^+(\operatorname{Image}(\calE_\calL))
        \leq e^{-2\beta|\calL|},
    \]
    which proves the claim.

    To prove the lemma, observe simply that the number of loops of length $k$
    through $e$ is bounded by $2^k$, so that 
    \begin{align}
        &\P_{\Lambda_n,\beta}^+
        (\{\text{$\calI(\sigma)$ has a loop of length at least $\ell$ through $e$}\})
        \\&\qquad=
        \sum\nolimits_{\text{$\calL$ is a loop of length at least $\ell$ through $e$}}
        \P_{\Lambda_n,\beta}^+
        (\{\calL\subset\calI\})
        \\&\qquad\leq 
        \sum\nolimits_{k\geq \ell} 2^k\cdot e^{-2\beta k}.
    \end{align}
    The final expression is a geometric series converging to the upper bound
    in the lemma.
\end{proof}

\begin{remark}
    In the previous proof,
    the interplay between entropy and energy is quite transparent.
    The entropy in the argument comes from the number of loops of length
    $\ell$, which we upper bounded by $2^\ell$.
    Such a loop contributes a total of $2\beta\ell$ to the Hamiltonian.
    When $2e^{-2\beta}<1$, the energy term dominates,
    forcing the loops to be small.
\end{remark}

\begin{proof}[Proof of Theorem~\ref{thm:peierls_triangles}]
    For a fixed configuration $\sigma\in\Omega$
    sampled from $\P_{\Lambda_n,\beta}^+$,
    we may express $\sigma_0$ as the parity of the number
    of loops in $\calI(\sigma)$ surrounding $0$.
    In particular, if no loop surrounds $0$,
    then $\sigma_0=+1$.
    Thus, for Theorem~\ref{thm:peierls_triangles},
    it suffices to prove that
    \begin{equation}
        \label{eq:peierls_target_equation}
        \P_{\Lambda_n,\beta}^+
        (\{\text{$\calI(\sigma)$ contains a loop surrounding $0$}\})
        <\frac12,
    \end{equation}
    for sufficiently large $\beta$,
    and uniformly in $n$.

    Suppose given some loop $\calL\subset E(\H)$ surrounding $0$.
    Then $\calL$ must intersect the half-line $\R_{\geq 0}\subset\C$.
    More precisely, $\calL$ must contain some edge $e$ whose midpoint
    lies precisely in the set of half-integers $-\frac12+\Z_{\geq 1}$.
    If the endpoint of $e$ is $k-\frac12$, then
    $|\calL|\geq k$, otherwise it cannot surround $0$.
    We are now ready to complete Peierls' argument,
    using exponential decay of the loop lengths (Lemma~\ref{lem:exp_decay_ising_loops}).
    
    Let us perform a union bound over the intersection point,
    in order to obtain
    \begin{align}
        &\P_{\Lambda_n,\beta}^+
        (\{\text{$\calI(\sigma)$ contains a loop surrounding $0$}\})
        \\
        &\qquad\leq
        \sum_{k=1}^\infty
        \P_{\Lambda_n,\beta}^+
        (\{\text{$\calI(\sigma)$ contains a loop surrounding $0$ and hitting $k-\tfrac12$}\})
        \\
        &\qquad\leq
        \sum_{k=1}^\infty
        \frac{(2e^{-2\beta})^k}{1-2e^{-2\beta}}
        =
        \frac{2e^{-2\beta}}{(1-2e^{-2\beta})^2}.
    \end{align}
    This upper bound is independent of $n$
    and tends to $0$ with $\beta\to\infty$,
    thus establishing Equation~\eqref{eq:peierls_target_equation}.
\end{proof}

\begin{exercise}
    \begin{enumerate}
        \item     Consider the Ising model on the two-dimensional square lattice graph $\mathbb{Z}^2$.
        In this case, the interface $\calI(\sigma)$ does not consist of loops, but of even subgraphs
        of the dual lattice.
        How can Peierls' argument be adapted to this case?
        \item  Now consider the $d$-dimensional square lattice for $d\geq 3$.
        What is the structure of the interface in this case?
        Can we adapt Peierls' to prove magnetisation for sufficiently large $\beta$?
    \end{enumerate}
\end{exercise}

\begin{remark}
    Peierls' is robust,
    in the sense that it can be adapted to many other models in statistical mechanics.
\end{remark}
