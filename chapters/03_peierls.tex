\section{Early developments. 1936: Peierls' argument}
\label{sec:peierls}

\begin{theorem}[Peierls, 1936]
    \label{thm:peierls}
    The Ising model exhibits a phase transition in two dimensions.
\end{theorem}

We shall prove a slight variation of Peierls' original setup,
so that we can fully focus the proof on the core idea.
Let $\T$ denote the triangular lattice graph,
comprised of vertices of the form
\[
    \T:=\left\{n+m e^{\pi i/3}:n,m\in\Z \right\}\subset\C,
\]
and such that each vertex is connected to the six
vertices at distance one.
Let $\Lambda_n\subset\T$ denote the set of vertices at a graph
distance at most $n-1$ from $0\in\T$.
We consider the Ising model on the infinite graph $\T$.
We shall prove the following version of Peierls' result.

\begin{theorem}[Peierls, 1936]
    Consider the Ising model on the two-dimensional
    triangular lattice graph $\T$.
    For sufficiently large $\beta$,
    we have
    \[
        \inf_{n}\langle\sigma_0\rangle_{\Lambda_n,\beta}^+
        >0.
    \]
\end{theorem}

\begin{proof}
    Let $\H:=\T^*$ denote the hexagonal lattice
    that is dual to the triangular lattice.
    For a fixed configuration $\sigma\in\Omega$,
    we let $D(\sigma)\subset E(\H)$ denote the set of
    hexagonal lattice edges such that the spin on either side is
    different.
    \todo{Add figure}
    The set $D(\sigma)$ is called the
    \emph{interface} between the spins valued $+1$
    and those valued $-1$.
    Notice that $D(\sigma)$ has a partition into
    loops and bi-infinite paths.
    If only finitely many spins of $\sigma$ are $-1$,
    then there are no bi-infinite paths,
    and all connected components of $D(\sigma)$
    are loops.
    This happens almost surely when sampling from $\P_{\Lambda_n,\beta}^+$.

    For the above theorem, it suffices to prove that
    for sufficiently large $\beta$, we have
    \begin{equation}
        \label{eq:peierls_target}
        \inf_{n}\P_{\Lambda_n,\beta}^+
        (\{\text{some loop surrounds $0$}\})
        <\frac12.
    \end{equation}
    Notice that any loop surrounding zero,
    must necessarily intersect the half-axis
    $\R_{\geq 0}\subset\C$ at a half-integer.
    Claim that for sufficiently large $\beta$,
    and for any $k\in\Z_{\geq 1}$
    we have 
    \begin{equation}
        \label{eq:peierls_target_2}
        \inf_{n}\P_{\Lambda_n,\beta}^+
        (\{\text{there exists a loop which surrounds $0$ and intersects $k-\tfrac12$}\})
        \leq 4^{-k}.
    \end{equation}
    Before proving the claim, we observe that the claim
    implies Equation~\eqref{eq:peierls_target}
    via a union bound since $1/4+1/16+1/64+\cdots<1/2$.

    We now focus on the claim.
    Fix $k$ (the lower bound for $\beta$ that we find shall be independant
    of it).
    To prove the claim,
    we define a ``loop erasure map'' $F:\Omega\to\Omega$
    which (informally) erases the loop through $k-\frac12$
    around $0$
    (if there is one).
    Formally, it is defined as the unique map with the following properties.
    \begin{itemize}
        \item If $D(\sigma)$ does not contain a loop which hits $k-\frac12$
        and surrounds $0$,
        then $F$ does not modify $\sigma$,
        that is, $F(\sigma)=\sigma$.
        \item If $D(\sigma)$ contains a loop intersecting $k-\frac12$
        and surrounding $0$,
        then $F$ erases that loop.
        In this case, we formally define $F(\sigma)$ as follows.
        Let $\calC\subset D(\sigma)$ denote the loop of interest.
        Then $F(\sigma)$ is defined such that
        \[
            F(\sigma)_z \cdot \sigma_z = \begin{cases}
                +1 &\text{if $z$ is outside $\calC$,}\\
                -1 &\text{if $z$ is inside $\calC$.}
            \end{cases}
        \]
        The reader should verify that this implies indeed
        that $D(F(\sigma))$ equals $D(\sigma)$
        but with loop $\calC$ removed.
    \end{itemize}

    The loop erasure map is a projection,
    in the sense that $F\circ F=F$.
    The event in the claim (Equation~\eqref{eq:peierls_target_2}) is clearly
    related to $F$, since
    \[
        \operatorname{Image}(F)
        =
        \Omega\setminus
        \{\text{there exists a loop which surrounds $0$ and intersects $k-\tfrac12$}\}.
    \]
    For Equation~\eqref{eq:peierls_target_2},
    it suffices to prove that for any $\sigma\in\Omega$,
    we have
    \begin{equation}
        \label{eq:peierls_target_3}
        \P_{\Lambda_n,\beta}^+
        (F^{-1}(\sigma)\setminus\{\sigma\})
        \leq 4^{-k}
        \cdot
        \P_{\Lambda_n,\beta}^+
        (\sigma).
    \end{equation}
    Fix $\sigma$.
    What does the preimage $F^{-1}(\sigma)$ look like?
    It contains all configurations $\sigma^\calC$ obtained from $\sigma$
    by \emph{adding} one loop $\calC$ through $k-\frac12$ around $0$.
    We notice that this changes the Hamiltonian by exactly $2\beta|\calC|$,
    where $|\calC|$ is the number of edges of the loop.
    Thus, we get
    \[
        \P_{\Lambda_n,\beta}^+
        (F^{-1}(\sigma)\setminus\{\sigma\})
        =
        \sum_{\calC\subset E^*(\Lambda_n)\setminus D(\sigma)}
        \P_{\Lambda_n,\beta}^+
        (\sigma^\calC)
        =
        \P_{\Lambda_n,\beta}^+
        (\sigma)
        \sum_{\calC\subset E^*(\Lambda_n)\setminus D(\sigma)}
        e^{-2\beta|\calC|}.
    \]
    It suffices to prove that the rightmost sum is bounded by $4^{-k}$.
    Observe the following:
    \begin{itemize}
        \item If a loop hits $k-\frac12$ and surrounds $0$, then its length must exceed $2+4k\geq k$,
        \item Since at each turn the loop can turn left or right, there are at most $2^{\ell-1}$ loops of length $\ell$ going through $k-\frac12$.
    \end{itemize}
    Thus, we get
    \[
        % \P_{\Lambda_n,\beta}^+
        % (F^{-1}(\sigma)\setminus\{\sigma\})
        % =
        % \P_{\Lambda_n,\beta}^+
        % (\sigma)
        \sum_{\calC\subset E^*(\Lambda_n)\setminus D(\sigma)}
        e^{-2\beta|\calC|}
        \leq
        % \P_{\Lambda_n,\beta}^+
        % (\sigma)
        \sum_{\ell=k}^\infty
        (2e^{-2\beta})^\ell
        =\frac{(2e^{-2\beta})^k}{1-2e^{-2\beta}}.
    \]
    If $\beta$ is large enough,
    then the rightmost expression is bounded by $4^{-k}$,
    which proves Equation~\eqref{eq:peierls_target_3}
    and therefore Equations~\eqref{eq:peierls_target_2}
    and~\eqref{eq:peierls_target}
    and therefore the theorem.
\end{proof}

\begin{remark}
    In the Peierls argument,
    the interplay between entropy and energy is extremely important.
    The entropy in the argument comes from the number of loops of length
    $\ell$, which we upper bounded by $2^\ell$.
    Such a loop contributes a total of $2\beta\ell$ to the Hamiltonian.
    When $e^{-2\beta}<2$, the energy term dominates,
    forcing loops to be small.
\end{remark}

\begin{exercise}
    \begin{enumerate}
        \item     Consider the Ising model on the two-dimensional square lattice $\mathbb{Z}^2$.
        In this case, the interface $D(\sigma)$ does not consist of loops, but of even subgraphs
        of the dual lattice.
        How can Peierls' argument be adapted to this case?
        \item  Now consider the $d$-dimensional square lattice for $d\geq 3$.
        What is the structure of the interface in this case?
        Can we adapt Peierls' to prove magnetisation for sufficiently large $\beta$?
    \end{enumerate}
\end{exercise}






