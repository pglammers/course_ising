\section{The high-temperature expansion and switching}

We already used the high-temperature expansion to prove one correlation inequality:
the first Griffiths inequality, which asserts that $\langle\sigma_A\rangle\geq 0$.
This was an immediate consequence of the fact that the high-temperature expansion is a sum of positive terms.

There are many other interested inequalities.
Many of those are obtained via the \emph{switching lemma}.
The switching lemma is traditionally stated for the random-current expansion
(which is a refinement of the high-temperature expansion introduced in the next section),
but we shall first state it in the context of the high-temperature expansion
because the setup is a little bit simpler.
We can already use it to prove two interesting inequalities:
\begin{itemize}
    \item The \emph{pairing bound}, which relates multi-point and two-point correlation functions,
    \item The \emph{Simon--Lieb inequality}, which yields a finite-size criterion for exponential decay.
\end{itemize}

We first prove the following switching lemma.
We state it in its most general form,
namely for overlapping graphs $G$ and $G'$.
In practice, we often care about the special case that $G=G'$, or the slightly more general
case that $G\subset G'$.

\begin{lemma}[Switching lemma for the high-temperature expansion]
    Let $G=(V,E)$ and $G'=(V',E')$ denote two finite graphs and fix $\beta\in[0,\infty)$.
    For $A\subset V$ and $A'\subset V'$,
    write $S_{A,A'}:=\{\partial\omega=A,\,\partial\omega'=A'\}\subset \{0,1\}^E\times\{0,1\}^{E'}$.

    Fix $\eta\subset E\cup E'$
    and $Q\subset \eta\cap E\cap E'$.
    Then for any $A\subset V$ and $A'\subset V'$,
    we get
    \begin{equation}
        \bfM_{G,\beta}\times\bfM_{G',\beta}
        [\{\omega\Delta\omega'=\eta\}\cap S_{A,A'}]
        =
        \bfM_{G,\beta}\times\bfM_{G',\beta}
        [\{\omega\Delta\omega'=\eta\}\cap S_{A\Delta\partial Q,A'\Delta\partial Q}].
    \end{equation}
\end{lemma}

\begin{proof}
    First, assume simply that $G=G'$.
    Write $\Xi_Q^2$ for the map
    \[
        \Xi_Q^2:
        (\{0,1\}^E)^2
        \to
        (\{0,1\}^E)^2
        ,\,(\omega,\omega')\mapsto (\omega\Delta Q,\omega'\Delta Q).
    \]
    We make two important observations:
    \begin{itemize}
        \item $\Xi_Q^2$ restricts to a involution on $\{\omega\Delta\omega'=\eta\}$,
        \item On $\{\omega\Delta\omega'=\eta\}$, the map $\Xi_Q^2$ does not modify the total number $|\omega|+|\omega'|$
        of edges.
    \end{itemize} 
    Since the weight of each configuration $(\omega,\omega')$ is a function of $|\omega|+|\omega'|$,
    the two observations imply that
    the measure \[\bfM_{G,\beta}\times\bfM_{G',\beta}
        [\{\omega\Delta\omega'=\eta\}\cap (\blank)]\] is preserved by the involution
    $\Xi_Q^2$.
    The result follows since $\Xi_Q^2$ is is also a bijection from $S_{A,A'}$
    to $S_{A\Delta\partial Q,A'\Delta\partial Q}$.

    If $G\neq G'$ then we simply view $\Xi_Q^2$
    as an involution on $\{0,1\}^E\times\{0,1\}^{E'}$,
    and the rest of the proof works in the same way.
\end{proof}

To apply this switching lemma,
it is useful to have some simple terminology for graphs and percolations.

\begin{definition}[Percolation events]
    Let $G=(V,E)$ denote a graph and $\omega\subset E$ a percolation configuration.
    Write \[\{u\xleftrightarrow{\omega}v\}\]
    for the event there is an open path from $u$ to $v$
    ($u$ and $v$ may represent vertices or sets of vertices).
    For fixed $A\subset V$, we shall also write $\calE_A$ for the set
    \[
        \{\omega\subset E:\text{$|C\cap A|$ is even for any connected component $C\subset V$ of $(V,\omega)$}\}.
    \]
\end{definition}

\begin{exercise}
    Let $G$ denote a graph and 
    $x,y\in V$  distinct vertices. Prove that:
    \begin{itemize}
        \item If $A=\{x,y\}$, then $\{\omega\in\calE_A\}=\{x\xleftrightarrow{\omega}y\}$,
        \item If $\omega\in\calE_A$, then we may find a finite subset $\eta\subset\omega$ with $\partial\eta=A$,
        \item If $G$ is a finite graph and $\partial\omega=A$, then $\omega\in\calE_A$,
        \item For any $A\subset V$, the event $\{\omega\in\calE_A\}$ is an increasing event of the percolation $\omega$.
    \end{itemize}
\end{exercise}

We can now prove some interesting bounds.
The first bound is the pairing bound.
This bound suggests that multi-correlation functions may be viewed
as interacting paths which pair up the sources in our source set.

\begin{theorem}[Pairing bound]
    Let $G=(V,E)$ denote a finite graph and $\beta\in[0,\infty)$.
    For any $x\in A\subset V$ we have
    \begin{equation}
        \langle\sigma_A\rangle
        \leq
        \sum_{y\in A\setminus\{x\}}
        \langle\sigma_x\sigma_y\rangle
        \langle \sigma_{A\setminus\{x,y\}}\rangle.
    \end{equation}
    In particular, iterating yields
    \[
        \langle\sigma_A\rangle
        \leq
        \sum_{\pi}
        \prod_{xy\in\pi}
        \langle\sigma_x\sigma_y\rangle,
    \]
    where $\pi$ runs over all \emph{pairings} of $A$,
    that is, over all partitions of $A$ into pairs.
\end{theorem}

\begin{proof}
    By the high-temperature expansion,
    we get
    \[
        (2^{-|V|}Z)^2 \langle\sigma_A\rangle
        =
        \bfM^2[\{\partial\omega = A,\,\partial\omega'=\emptyset\}].
    \]
    But on this event we have $\partial(\omega\Delta\omega')=A$,
    which means that $\omega\Delta\omega'$ contains a path
    from $x$ to at least one other vertex in $A$ (see the exercise).
    In other words,
    \[
        \true{\partial\omega = A,\,\partial\omega'=\emptyset}
        \leq
        \sum\nolimits_{y\in A\setminus\{x\},\,\eta\in\{0,1\}^E,\,\{x\xleftrightarrow{\eta}y\}}
        \true{\omega\Delta\omega'=\eta,\,\partial\omega = A,\,\partial\omega'=\emptyset}.
    \]
    We now claim that
    \begin{align}
        &\bfM^2[\{\partial\omega = A,\,\partial\omega'=\emptyset\}]
        \\&\leq 
            \sum\nolimits_{y\in A\setminus\{x\},\,\eta\in\{0,1\}^E,\,\{x\xleftrightarrow{\eta}y\}}
        \bfM^2[
        \{\omega\Delta\omega'=\eta,\,\partial\omega = A,\,\partial\omega'=\emptyset\}
        ]
        \\&=
            \sum\nolimits_{y\in A\setminus\{x\},\,\eta\in\{0,1\}^E,\,\{x\xleftrightarrow{\eta}y\}}
        \bfM^2[
        \{\omega\Delta\omega'=\eta,\,\partial\omega = A\setminus\{x,y\},\,\partial\omega'=\{x,y\}\}
        ].
    \end{align}
    The inequality is the previous inequality,
    the equality is the switching lemma for the high-temperature expansion
    applied to each term $(y,\eta)$,
    where $Q$ is simply some path in $\eta$ from $x$ to $y$.

    The final expression in the claim is equal to
    \[
        \sum\nolimits_{y\in A\setminus\{x\}}
        \bfM^2[
        \{\partial\omega = A\setminus\{x,y\},\,\partial\omega'=\{x,y\}\}
        ]
        =
        (2^{-|V|}Z)^2 \langle\sigma_{A\setminus\{x,y\}}\rangle\langle\sigma_x\sigma_y\rangle,
    \]
    which finishes the proof.
\end{proof}

Next, we focus on Simon's inequality.
We already proved that there is exponential decay at high temperature.
Simon's inequality says that \emph{if} there is exponential decay of correlations,
then it can be detected within a finite volume.

Recall that if $G=(V,E)$ is a graph and $\Lambda\subset V$
a subset, then $\partial\Lambda$ denotes the set of vertices in $V\setminus\Lambda$
which are adjacent to $\Lambda$.
Write $\partial_\circ\Lambda$ for the \emph{interior boundary},
that is, the set of vertices in $\Lambda$ adjacent to $V\setminus\Lambda$.
Write $\partial_e\Lambda$ for the \emph{edge boundary},
that is, the set of edges connecting $\Lambda$ and $V\setminus\Lambda$.

\begin{theorem}[Simon's inequality]
    Let $G=(V,E)$ denote a finite graph,
    $\Lambda\subset V$ some domain,
    and let $\beta\in[0,\infty)$.
    Fix $x\in \Lambda$ and $y\in V\setminus\Lambda$.
    \begin{itemize}
        \item We have
        \[
            \langle\sigma_x\sigma_y\rangle_{G,\beta}
            \leq
            \sum_{z\in\partial_\circ\Lambda}
            \langle\sigma_x\sigma_z\rangle_{\Lambda,\beta}^\f
            \langle\sigma_y\sigma_z\rangle_{G,\beta}.
        \]
        \item We have
        \[
            \langle\sigma_x\sigma_y\rangle_{G,\beta}
            \leq
            (\tanh\beta)
            \sum_{zz'\in\partial_e\Lambda}
            \langle\sigma_x\sigma_z\rangle_{\Lambda,\beta}^\f
            \langle\sigma_y\sigma_{z'}\rangle_{G,\beta}.
        \]
    \end{itemize}
    In fact, the converse inequalities are also true if we 
    multiply the right hand sides by $1/|\partial_\circ\Lambda|$
    and $1/|\partial_e\Lambda|$ respectively.
\end{theorem}

\begin{proof}
    Focus on the first inequality.
    Let $\bfM':=\bfM_{G,\beta}\times\bfM_{\Lambda^\f,\beta}$.
    Expanding the left hand side yields
    \[
        \frac{Z_GZ_{\Lambda^\f}}{2^{|V|}2^{|\Lambda|}}
        \langle\sigma_x\sigma_y\rangle_{G,\beta}
        =
        \bfM'[\{\partial\omega=\{x,y\},\,\partial\omega'=\emptyset\}].
    \]
    But on the event on the right,
    we have $\partial(\omega\Delta\omega')=\{x,y\}$,
    which means that $\omega\Delta\omega'$ contains a path
    which remains in $\Lambda$
    and which connects $x$ to some vertex in $\partial_\circ\Lambda$.
    In other words,
    \begin{multline}
        \label{eq:converse}
        \true{\partial\omega=\{x,y\},\,\partial\omega'=\emptyset}
        \\
        \leq
        \sum\nolimits_{z\in\partial_\circ\Lambda,\,\eta\in\{0,1\}^E,\,x\xleftrightarrow{\eta\cap E(\Lambda^\f)}z}
        \true{\omega\Delta\omega'=\eta,\,\partial\omega=\{x,y\},\,\partial\omega'=\emptyset}.
    \end{multline}
    Using the switching lemma like for the pairing bound,
    we obtain
    \begin{align}
        &\bfM'[\{\partial\omega=\{x,y\},\,\partial\omega'=\emptyset\}]
        \\&
        \leq 
        \sum\nolimits_{z\in\partial_\circ\Lambda,\,\eta\in\{0,1\}^E,\,x\xleftrightarrow{\eta\cap E(\Lambda^\f)}z}
        \bfM'[
        \{\omega\Delta\omega'=\eta,\,\partial\omega=\{x,y\},\,\partial\omega'=\emptyset\}]
        \\&
        =
        \sum\nolimits_{z\in\partial_\circ\Lambda,\,\eta\in\{0,1\}^E,\,x\xleftrightarrow{\eta\cap E(\Lambda^\f)}z}
        \bfM'[
        \{\omega\Delta\omega'=\eta,\,\partial\omega=\{y,z\},\,\partial\omega'=\{x,z\}\}]
        \\&
        =
        \sum\nolimits_{z\in\partial_\circ\Lambda}
        \bfM'[
        \{\partial\omega=\{y,z\},\,\partial\omega'=\{x,z\}\}]
        \\&=
        \frac{Z_GZ_{\Lambda^\f}}{2^{|V|}2^{|\Lambda|}}
        \sum\nolimits_{z\in\partial_\circ\Lambda}
        \langle\sigma_y\sigma_z\rangle_{G,\beta}
        \langle\sigma_x\sigma_z\rangle_{\Lambda^\f,\beta}.
    \end{align}
    We use the switching lemma for the first equality;
    we choose $Q$ to be a path from $x$ to $z$ in $\eta\cap E(\Lambda^\f)$
    to get an equality for each term.
    This proves the first inequality in the statement of the theorem.

    The converse inequality is true simply because we may reverse the inequality in
    Equation~\eqref{eq:converse} if we multiply the right hand side by $1/|\partial_\circ\Lambda|$.

    For the second inequality we only give a proof outline.
    It is obtained in a similar fashion,
    noticing that if $\omega\Delta\omega'$ connects
    $x$ and $y$,
    then there must be some edge $zz'\in\partial_e\Lambda$
    such that $\omega\Delta\omega'$ contains a self-avoiding path
    from $x$ to $y$, which passes through $zz'$ 
    and which does not leave $\Lambda$ before using this edge.
    The switching lemma may then be applied in a similar fashion.
    By switching the edge $zz'$, which only appears in the bigger
    graph,
    we make the extra factor $\tanh\beta$ appear.
\end{proof}

\begin{corollary}[Finite size criterion]
    Consider the Ising model on $\Z^d$ at inverse temperature $\beta$.
    For any finite $\Lambda\subset\Z^d$ containing $0\in\Z^d$,
    we define
    \[
        \phi_\beta(\Lambda):=
        (\tanh\beta)
        \sum_{zz'\in\partial_e\Lambda}
            \langle\sigma_0\sigma_z\rangle_{\Lambda,\beta}^\f.
    \]
    If $\phi_\beta(\Lambda)<1$ for some $\Lambda$, then
    there exists a constant $c=c_{d,\beta}>0$
    such that
    \[
        \langle\sigma_x\rangle_{\Delta,\beta}^+
        \leq \tfrac1ce^{-c \operatorname{Distance}(x,\Delta^c)}
    \]
    for any $x\in\Z^d$ and any domain $\Delta\subset\Z^d$.
\end{corollary}

\begin{proof}
    Fix $\Lambda$ with $\phi_\beta(\Lambda)<1$,
    and fix $\Delta$.
    For any $x\in\Z^d$, define
    \[
        a(x):=\left\lfloor
        \frac{
            \operatorname{Distance}(x,\Delta^c)
        }{
            \operatorname{Diameter}(\Lambda)+1
        }
    \right\rfloor.
    \]
    It suffices to prove that for any $x\in\Z^d$ and $n\in\Z_{\geq 0}$, we have
    \[
        a(x)\geq n
        \qquad\implies\qquad
        \langle\sigma_x\rangle_{\Delta,\beta}^+ \leq \phi_\beta(\Lambda)^n.
    \]
    The induction basis $n=0$ is obvious; we limit this proof to the induction step.

    Using Simon's inequality, we get
    \[
        \langle\sigma_x\rangle_{\Delta,\beta}^+
        =
        \langle\sigma_x\sigma_\frakg\rangle_{\Delta^\frakg,\beta}
        \leq 
        (\tanh\beta)
            \sum_{zz'\in\partial_e(\Lambda+x)}
            \langle\sigma_x\sigma_z\rangle_{\Lambda+x,\beta}^\f
            \langle\sigma_\frakg\sigma_{z'}\rangle_{\Delta^\frakg,\beta}.
    \]
    By induction, we may bound $\langle\sigma_\frakg\sigma_{z'}\rangle_{\Delta^\frakg,\beta}=\langle\sigma_{z'}\rangle_{\Delta,\beta}^+\leq \phi_\beta(\Lambda)^{a(x)-1}$,
    so that the previous line is upper bounded by
    \[
        \langle\sigma_x\rangle_{\Delta,\beta}^+
        \leq
        \left(\textstyle
        (\tanh\beta)
            \sum_{zz'\in\partial_e(\Lambda+x)}
            \langle\sigma_x\sigma_z\rangle_{\Lambda+x,\beta}^\f
            \right)
            \phi_\beta(\Lambda)^{a(x)-1}
            =
            \phi_\beta(\Lambda)^{a(x)}.
    \]
    This is the desired bound.
\end{proof}

\begin{remark}
    At this point, we have proved that
    \[
        \inf_\Lambda \phi_\beta(\Lambda) < 1
        \qquad
        \implies
        \qquad
        \text{exponential decay of correlations},
    \]
    but not the converse implication.

    It is easy to see that the converse implication is also true.
    Indeed, the converse version of the Simon inequality yields
    \[
        \langle \sigma_0\rangle_{\Lambda,\beta}^+
        \geq \frac{\tanh\beta}{|\partial_e\Lambda|} 
        \sum_{zz'\in\partial_e\Lambda}
            \langle\sigma_0\sigma_z\rangle_{\Lambda,\beta}^\f
            =\frac{\phi_\beta(S)}{|\partial_e\Lambda|}.
    \]
    If the left hand side decays exponentially fast in the distance from $0$ to the boundary,
    then one may clearly choose $\Lambda$ so large that 
    $\langle \sigma_0\rangle_{\Lambda,\beta}^+\leq 1/|\partial_e\Lambda|$, in which case $\phi_\beta(\Lambda)<1$.
\end{remark}
