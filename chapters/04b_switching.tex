\section{The high-temperature expansion and switching}

We already used the high-temperature expansion to prove one correlation inequality:
the first Griffiths inequality, which asserts that $\langle\sigma_A\rangle\geq 0$.
This was an immediate consequence of the fact that the high-temperature expansion is a sum of positive terms.

There are many other interested inequalities.
Many of those are obtained via the \emph{switching lemma}.
The switching lemma is traditionally stated for the random-current expansion
(which is a refinement of the high-temperature expansion introduced in the next section),
but we shall first state it in the context of the high-temperature expansion
because the setup is a little bit simpler.
We can already use it to prove three interesting inequalities:
\begin{itemize}
    \item The \emph{pairing bound}, which relates multi-point and two-point correlation functions,
    \item The \emph{Simon--Lieb inequality}, which yields a finite-size criterion for exponential decay.
\end{itemize}

We first prove the following switching lemma.

\begin{lemma}[Switching lemma for the high-temperature expansion]
    Let $G=(V,E)$ and $G'=(V',E')$ denote two finite graphs and fix $\beta\in[0,\infty)$.
    For $A\subset V$ and $A'\subset V'$,
    write $S_{A,A'}:=\{\partial\omega=A,\,\partial\omega'=A'\}\subset \{0,1\}^E\times\{0,1\}^{E'}$.

    Fix $\eta\subset E\cup E'$
    and $Q\subset \eta\cap E\cap E'$.
    Then for any $A\subset V$ and $A'\subset V'$,
    we get
    \begin{equation}
        \bfM_{G,\beta}\times\bfM_{G',\beta}
        [\{\omega\Delta\omega'=\eta\}\cap S_{A,A'}]
        =
        \bfM_{G,\beta}\times\bfM_{G',\beta}
        [\{\omega\Delta\omega'=\eta\}\cap S_{A\Delta\partial Q,A'\Delta\partial Q}].
    \end{equation}
\end{lemma}

\begin{proof}
    Assume simply that $G=G'$.
    This case is already sufficient for proving the pairing bound below.
    The proof of the general case is similar,
    we briefly discuss it at the end.

    Write $\Xi_Q^2$ for the map
    \[
        \Xi_Q^2:
        (\{0,1\}^E)^2
        \to
        (\{0,1\}^E)^2
        ,\,(\omega,\omega')\mapsto (\omega\Delta Q,\omega'\Delta Q).
    \]
    We make two important observations:
    \begin{itemize}
        \item $\Xi_Q^2$ restricts to a involution on $\{\omega\Delta\omega'=\eta\}$,
        \item On $\{\omega\Delta\omega'=\eta\}$, the map $\Xi_Q^2$ does not modify the total number $|\omega|+|\omega'|$
        of edges.
    \end{itemize} 
    Since the weight of each configuration $(\omega,\omega')$ is a function of $|\omega|+|\omega'|$,
    the two observations imply that
    the measure \[\bfM_{G,\beta}\times\bfM_{G',\beta}
        [\{\omega\Delta\omega'=\eta\}\cap (\blank)]\] is preserved by the involution
    $\Xi_Q^2$.
    The result follows since $\Xi_Q^2$ is is also a bijection from $S_{A,A'}$
    to $S_{A\Delta\partial Q,A'\Delta\partial Q}$.

    If $G\neq G'$ then we simply view $\Xi_Q^2$
    as an involution on $\{0,1\}^E\times\{0,1\}^{E'}$.
\end{proof}

To apply this switching lemma,
we need some simply combinatorial tools.

\begin{definition}[Percolation events]
    Let $G=(V,E)$ denote a graph and $\omega\subset E$ a percolation configuration.
    Write \[\{u\xleftrightarrow{\omega}v\}\]
    for the event there is an open path from $u$ to $v$
    ($u$ and $v$ may represent vertices or sets of vertices).
    For fixed $A\subset V$, we shall also write $\calE_A$ for the set
    \[
        \{\omega\subset E:\text{$|C\cap A|$ is even for any connected component $C\subset V$ of $(V,\omega)$}\}.
    \]
\end{definition}

\begin{exercise}
    Let $G$ denote a graph and 
    $x,y\in V$  distinct vertices. Prove that:
    \begin{itemize}
        \item If $A=\{x,y\}$, then $\{\omega\in\calE_A\}=\{x\xleftrightarrow{\omega}y\}$,
        \item If $\omega\in\calE_A$, then we may find a subset $\eta\subset\omega$ with $\partial\eta=A$,
        \item If $G$ is a finite graph and $\partial\omega=A$, then $\omega\in\calE_A$,
        \item For any $A\subset V$, the event $\{\omega\in\calE_A\}$ is an increasing event of the percolation $\omega$.
    \end{itemize}
\end{exercise}

We can now prove some interesting bounds.

\begin{theorem}[Pairing bound]
    Let $G=(V,E)$ denote a finite graph and $\beta\in[0,\infty)$.
    For any $x\in A\subset V$ we have
    \begin{equation}
        \langle\sigma_A\rangle
        \leq
        \sum_{y\in A\setminus\{x\}}
        \langle\sigma_x\sigma_y\rangle
        \langle \sigma_{A\setminus\{x,y\}}\rangle.
    \end{equation}
    In particular, iterating yields
    \[
        \langle\sigma_A\rangle
        \leq
        \sum_{\pi}
        \prod_{xy\in\pi}
        \langle\sigma_x\sigma_y\rangle,
    \]
    where $\pi$ runs over all \emph{pairings} of $A$,
    that is, over all partitions of $A$ into pairs.
\end{theorem}

\begin{proof}
    By the high-temperature expansion,
    we get
    \[
        (2^{-|V|}Z)^2 \langle\sigma_A\rangle
        =
        \bfM^2[\{\partial\omega = A,\,\partial\omega'=\emptyset\}].
    \]
    But on this event we have $\partial(\omega\Delta\omega')=A$,
    which means that $\omega\Delta\omega'$ contains a path
    from $x$ to at least one other vertex in $A$ (see the exercise).
    In other words,
    \[
        \true{\partial\omega = A,\,\partial\omega'=\emptyset}
        \leq
        \sum\nolimits_{y\in A\setminus\{x\},\,\eta\in\{0,1\}^E,\,\{x\xleftrightarrow{\eta}y\}}
        \true{\omega\Delta\omega'=\eta,\,\partial\omega = A,\,\partial\omega'=\emptyset}.
    \]
    We now claim that
    \begin{align}
        &\bfM^2[\{\partial\omega = A,\,\partial\omega'=\emptyset\}]
        \\&\leq 
            \sum\nolimits_{y\in A\setminus\{x\},\,\eta\in\{0,1\}^E,\,\{x\xleftrightarrow{\eta}y\}}
        \bfM^2[
        \{\omega\Delta\omega'=\eta,\,\partial\omega = A,\,\partial\omega'=\emptyset\}
        ]
        \\&=
            \sum\nolimits_{y\in A\setminus\{x\},\,\eta\in\{0,1\}^E,\,\{x\xleftrightarrow{\eta}y\}}
        \bfM^2[
        \{\omega\Delta\omega'=\eta,\,\partial\omega = A\setminus\{x,y\},\,\partial\omega'=\{x,y\}\}
        ].
    \end{align}
    The inequality is the previous inequality,
    the equality is the switching lemma for the high-temperature expansion
    applied to each term $(y,\eta)$,
    where $Q$ is simply some path in $\eta$ from $x$ to $y$.

    The final sum in the claim is upper bounded by
    \[
        \sum\nolimits_{y\in A\setminus\{x\}}
        \bfM^2[
        \{\partial\omega = A\setminus\{x,y\},\,\partial\omega'=\{x,y\}\}
        ]
        =
        (2^{-|V|}Z)^2 \langle\sigma_{A\setminus\{x,y\}}\rangle\langle\sigma_x\sigma_y\rangle,
    \]
    which finishes the proof.
\end{proof}

Recall that if $G=(V,E)$ is a graph and $\Lambda\subset V$
a subset, then $\partial\Lambda$ denotes the set of vertices in $V\setminus\Lambda$
which are adjacent to $\Lambda$.
Write $\partial_\circ\Lambda$ for the \emph{interior boundary},
that is, the set of vertices in $\Lambda$ adjacent to $V\setminus\Lambda$.
Write $\partial_e\Lambda$ for the \emph{edge boundary},
that is, the set of edges connecting $\Lambda$ and $V\setminus\Lambda$.

\begin{theorem}[Simon's inequality]
    Let $G=(V,E)$ denote a finite graph,
    $\Lambda\subset V$ some domain,
    and let $\beta\in[0,\infty)$.
    Fix $x\in \Lambda$ and $y\in V\setminus\Lambda$.
    \begin{itemize}
        \item We have
        \[
            \langle\sigma_x\sigma_y\rangle_{G,\beta}
            \leq
            \sum_{z\in\partial_\circ\Lambda}
            \langle\sigma_x\sigma_z\rangle_{\Lambda,\beta}^\f
            \langle\sigma_y\sigma_z\rangle_{G,\beta}.
        \]
        \item We have
        \[
            \langle\sigma_x\sigma_y\rangle_{G,\beta}
            \leq
            (\tanh\beta)
            \sum_{zz'\in\partial_e\Lambda}
            \langle\sigma_x\sigma_z\rangle_{\Lambda,\beta}^\f
            \langle\sigma_y\sigma_{z'}\rangle_{G,\beta}.
        \]
    \end{itemize}
\end{theorem}

\begin{proof}
    Focus on the first inequality.
    Expanding the left hand side yields
    \[
        \frac{Z_GZ_{\Lambda^\f}}{2^{|V|}2^{|\Lambda|}}
        \langle\sigma_x\sigma_y\rangle_{G,\beta}
        =
        \bfM_{G,\beta}\times\bfM_{\Lambda^\f,\beta}[\{\partial\omega=\{x,y\},\,\partial\omega'=\emptyset\}].
    \]
    But on the event on the right,
    we have $\partial(\omega\Delta\omega')=\{x,y\}$,
    which means that $\omega\Delta\omega'$ contains a path
    from $x$ to at least one vertex in $\partial_\circ\Lambda$
    that does not leave $\Lambda$.
    In other words,
    \begin{multline}
        \true{\partial\omega=\{x,y\},\,\partial\omega'=\emptyset}
        \\
        \leq
        \sum\nolimits_{z\in\partial_\circ\Lambda,\,\eta\in\{0,1\}^E,\,x\xleftrightarrow{\eta\cap E(\Lambda^\f)}z}
        \true{\omega\Delta\omega'=\eta,\,\partial\omega=\{x,y\},\,\partial\omega'=\emptyset}.
    \end{multline}
    Using the switching lemma like for the pairing bound,
    we obtain
    \begin{align}
        &\bfM_{G,\beta}\times\bfM_{\Lambda^\f,\beta}[\{\partial\omega=\{x,y\},\,\partial\omega'=\emptyset\}]
        \\&
        \leq 
        \sum\nolimits_{z\in\partial_\circ\Lambda,\,\eta\in\{0,1\}^E,\,x\xleftrightarrow{\eta\cap E(\Lambda^\f)}z}
        \bfM_{G,\beta}\times\bfM_{\Lambda^\f,\beta}[
        \{\omega\Delta\omega'=\eta,\,\partial\omega=\{x,y\},\,\partial\omega'=\emptyset\}]
        \\&
        =
        \sum\nolimits_{z\in\partial_\circ\Lambda,\,\eta\in\{0,1\}^E,\,x\xleftrightarrow{\eta\cap E(\Lambda^\f)}z}
        \bfM_{G,\beta}\times\bfM_{\Lambda^\f,\beta}[
        \{\omega\Delta\omega'=\eta,\,\partial\omega=\{y,z\},\,\partial\omega'=\{x,y\}\}]
        \\&
        \leq
        \sum\nolimits_{z\in\partial_\circ\Lambda}
        \bfM_{G,\beta}\times\bfM_{\Lambda^\f,\beta}[
        \{\partial\omega=\{y,z\},\,\partial\omega'=\{x,y\}\}]
        \\&=
        \frac{Z_GZ_{\Lambda^\f}}{2^{|V|}2^{|\Lambda|}}
        \sum\nolimits_{z\in\partial_\circ\Lambda}
        \langle\sigma_y\sigma_z\rangle_{G,\beta}
        \langle\sigma_x\sigma_y\rangle_{\Lambda^\f,\beta}.
    \end{align}
    We use the switching lemma for the second inequality;
    we choose $Q$ to be a path from $x$ to $y$ in $\eta\cap E(\Lambda^\f)$
    to get an equality for each term.

    For the second inequality we only give a proof outline.
    It is obtained in a similar fashion,
    noticing that if $\omega\Delta\omega'$ connects
    $x$ and $y$,
    then there must be some edge $zz'\in\partial_e\Lambda$
    such that $\omega\Delta\omega'$ contains a self-avoiding path
    from $x$ to $y$, which passes through $zz'$ 
    and which does not leave $\Lambda$ before using this edge.
    The switching lemma may then be applied in a similar fashion.
    By switching the edge $zz'$, which only appears in the bigger
    graph,
    we make the extra factor $\tanh\beta$ appear.
\end{proof}
