\section{Monotonicity in the temperature}

\begin{theorem}[Monotonicity in the temperature]
    Let $G$ denote a finite graph and $A\subset V$ a finite set.
    Then the function $\beta\mapsto\langle\sigma_A\rangle_{G,\beta}$
    is non-decreasing.
\end{theorem}

\begin{proof}
    We want to prove that
    \[
        \frac{\partial}{\partial\beta}
        \langle\sigma_A\rangle_{G,\beta}
        =
        \frac{\partial}{\partial\beta}
        \left(
            \frac{
                \sum_\sigma\sigma_A\prod_{uv}e^{\beta\sigma_u\sigma_v}
            }{
                \sum_\sigma\prod_{uv}e^{\beta\sigma_u\sigma_v}
            }
        \right)
        \geq 0.
    \]
    Since we are differentiating a fraction,
    it suffices to show that the numerator grows at a faster rate
    than the denominator,
    that is,
    \[
        \frac{\frac{\partial}{\partial\beta}
            \sum_\sigma\sigma_A\prod_{uv}e^{\beta\sigma_u\sigma_v}
        }{
            Z\langle\sigma_A\rangle
        }
        \geq
        \frac{\frac{\partial}{\partial\beta}
            \sum_\sigma\prod_{uv}e^{\beta\sigma_u\sigma_v}
        }{
            Z
        }.
    \]
    By multiplying either side by $\langle\sigma_A\rangle$
    and differentiating each side,
    we see that this inequality is equivalent to
    \[
       \sum_{xy}
        \frac{\sum_\sigma
            \sigma_x\sigma_y\sigma_A\prod_{uv}e^{\beta\sigma_u\sigma_v}
        }{Z}
        \geq
        \langle\sigma_A\rangle
        \sum_{xy}
        \frac{
            \sum_\sigma
            \sigma_x\sigma_y
            \prod_{uv}e^{\beta\sigma_u\sigma_v}
        }{Z}.
    \]
    Each fraction may now be reinterpreted as a correlation function,
    so that the previous inequality is equivalent to
    \[
        \sum_{xy}\langle\sigma_x\sigma_y\sigma_A\rangle
        \geq
        \langle\sigma_A\rangle
        \sum_{xy}\langle\sigma_x\sigma_y\rangle.
    \]
    But this is just the second Griffiths inequality.
\end{proof}

\begin{exercise}
    What are some other properties of the function $\beta\mapsto\langle\sigma_A\rangle_{G,\beta}$
    in the context of the theorem above?
\end{exercise}
