\section{Early developments. 1907: The Curie--Weiss model}
\label{sec:definitions_examples}

The Ising model is actually a generalisation of the Curie--Weiss model.
Weiss introduced this mathematical model in 1907 in order to find a theoretical
explanation for the spontaneous magnetisation of ferromagnetic materials
that had previously been studied by Pierre Curie.
The Curie--Weiss model is a special case of the Ising model
where the graph is a complete graph.
This model can be solved with elementary methods
and exhibits a phase transition.
Moreover, it exhibits the interplay between entropy and energy in an
extremely transparent fashion,
which is why we quickly review it here.

\begin{definition}[Curie-Weiss model]
    The Curie-Weiss model is the Ising model on the complete graph \( G=K_n \) on \( n \) vertices.
    We typically fix a parameter $\alpha\in[0,\infty)$ and then set the inverse temperature $\beta$ to $\beta=\alpha/n$.
    We shall write
    \[\P^{\operatorname{CW}}_{n,\alpha}:=\P^{\operatorname{Ising}}_{K_n,\alpha/n}\]
    for the associated Boltzmann distribution.
    The Curie-Weiss model is often used to study mean-field behavior in statistical mechanics, as it captures the essential features of phase transitions and spontaneous magnetization in a simplified setting.
\end{definition}

Let $n_+=n_+(\sigma)$ denote the number of vertices with spin $+1$ in a configuration $\sigma\in\Omega$.
This is a random variable.
Let us try to calculate the probability of the event $\{n_+=k\}$,
without worrying about the partition function (the normalising constant).
One may easily check that
the Hamiltonian satisfies
\begin{align}
    H(\sigma)=2\beta n_+(n-n_+) + \text{const}(n).
\end{align}
The distribution of $n_+$ can then be calculated as follows:
\begin{align}
    \label{eq:CurieWeissDistribution}
    \P_{K_n,\beta}^{\operatorname{Ising}}(\{n_+ = k\}) &\propto \binom{n}{k} e^{-2\beta  k(n-k)}
    \propto \frac1{k! (n-k)!} e^{-2\beta k (n-k)}.
\end{align}
Using Stirling's formula for the factorials, we find that
\begin{gather}
    \log\P_{K_n,\beta}^{\operatorname{Ising}}(\{n_+ = k\}) 
    \stackrel{\text{Stirling}}{\approx}
    -n f_{(n\beta)}(k/n) + \text{const}(n);
    \\
    f_{(\alpha)}:[0,1]\to\R,\,
    x\mapsto x \log x + (1-x)\log (1-x) + 2 \alpha x (1-x).
\end{gather}
If we fix $\alpha$ and send $n$ to infinity, then 
we discover a large deviations principle for the random variable $n_+/n$
with rate function $f_{(\alpha)}$ and speed $n$.
In particular, the random variable $n_+/n$ is concentrated
around the minimisers of the function $f_{(\alpha)}$.

\begin{exercise}
    \begin{enumerate}
        \item Formally verify that all of the above calculations are correct.
        \item Show that for small $\alpha$, the function \( f_{(\alpha)} \) has a single minimum, which means that the random variable \( n_+/n \) is concentrated around the value \( 1/2 \).
        \item Show that for large enough $\alpha$, the function \( f_{(\alpha)} \) has two minima at \( (1 \pm m)/2 \) for some $m>0$, which means that the random variable \( n_+/n \) is concentrated around these minima.
        The value of $m$ is called the \emph{magnetisation}.
        \item Calculate the critical value for $\alpha$. At this value, the second derivative of \( f_{(\alpha)} \) vanishes at \( x=1/2 \). What does this mean for the distribution of \( n_+/n \)?
        More precisely, what is the order of magnitude of $\Var|\frac{n_+}{n}-\frac12|$ as $n\to\infty$?
    \end{enumerate}
\end{exercise}

\begin{remark}
    Reconsider Equation~\eqref{eq:CurieWeissDistribution}.
    In this equation, the competition between the two factors is extremely transparent.
    \begin{itemize}
        \item     First, there is a combinatorial term or \emph{entropy}, which favours values $k$ for the random variable
        $n_+$ such that the cardinality of the set $\{n_+=k\}$ is large.
        This means that values $k\approx n/2$ are preferred.
        \item     Second, there is the \emph{energy} term, which favours values such that the Hamiltonian 
        is minimised. This favours configurations where as many spins as possible align.    
    \end{itemize}
    The interaction parameter $\beta$ allows us to put more emphasis
    on the entropy term or on the energy term.
    In the $n\to\infty$ limit, there is a precise value for $n\beta$
    where the behaviour of the random system undergoes a qualitative change:
    a rudimentary example of a  \emph{phase transition}.
\end{remark}

While the competition between entropy and energy is transparent in the
case of the complete graph (the Curie--Weiss model), the situation is much more
complicated when considering other graphs.
We are particularly interested in graphs modelling Euclidean space,
by taking large finite subgraphs of the square lattice graph $\Z^d$
in dimension $d\geq 1$.
The geometry of the Ising model (absent in the case of Curie--Weiss)
is an extremely rich and beautiful object,
and most of this course is dedicated to understanding it.
