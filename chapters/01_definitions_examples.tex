\section{Definitions of the Ising model and analysis of the Curie--Weiss model}
\label{sec:definitions_examples}

\begin{definition}[Ising model on a finite graph]
    The Ising model on a finite graph \( G = (V, E) \) with inverse temperature \( \beta \in [0,\infty) \) is defined as follows.
    Let $\Omega:=\{\pm1\}^V$ denote the set of spin configurations on the vertices of the graph;
    a typical element of $\Omega$ is denoted by $\sigma=(\sigma_i)_{i\in V}$.
    Elements $\sigma\in\Omega$ are called \emph{spins}.
    The \emph{energy} or \emph{Hamiltonian} of a spin configuration $\sigma$ is given by
    \[
    H(\sigma) := H_{G,\beta}(\sigma) := -\beta \sum_{\{i,j\} \in E} \sigma_i \sigma_j.
    \]
    The probability of a configuration \(\sigma\in\Omega\) is given by the Boltzmann distribution
    \[
    \P(\sigma) := \P_{G,\beta}(\sigma) := \frac{1}{Z_{G,\beta}} e^{-H_{G,\beta}(\sigma)},
    \]
    where \(Z_{G,\beta}\) is normalization constant or \emph{partition function} defined by
    \[
    Z_{G,\beta} := \sum_{\sigma\in\Omega} e^{-H_{G,\beta}(\sigma)}.
    \]
\end{definition}

\begin{definition}[Curie-Weiss model]
    The Curie-Weiss model is a special case of the Ising model where the graph \( G \) is a complete graph on \( N \) vertices. In this model, every pair of vertices is connected by an edge, and the Hamiltonian simplifies to
    \[
    H(\sigma) = -\beta \sum_{1 \leq i < j \leq N} \sigma_i \sigma_j.
    \]
    Write $\P_{N,\beta}$ for the corresponding probability measure.
    The Curie-Weiss model is often used to study mean-field behavior in statistical mechanics, as it captures the essential features of phase transitions and spontaneous magnetization in a simplified setting.
\end{definition}

Let $N_+$ denote the number of vertices with spin $+1$ in a configuration $\sigma\in\Omega$.
This is a random variable.
Let us try to calculate the probability of the event $\{N_+=K\}$,
without worrying about the normalization constant $Z_{G,\beta}$.
It is straightforward to verify that for $\sigma\in\{N_+=K\}$,
the Hamiltonian satisfies
\begin{align}
    H(\sigma)=2\beta K(N-K) + \text{const}(N).
\end{align}
The distribution of $N_+$ can then be calculated as follows:
\begin{align}
    \label{eq:CurieWeissDistribution}
    \P_{N,\beta}(N_+ = K) &\propto \binom{N}{K} e^{-2\beta  K(N-K)}
    \propto \frac1{K! (N-K)!} e^{-2\beta K (N-K)}.
\end{align}
Using Stirling's formula for the factorials, we find that
\begin{gather}
    \log\P_{N,\beta}(N_+ = K) 
    \stackrel{\text{Stirling}}{\approx}
    -N f_{(N\beta)}(K/N) + \text{const}(N);
    \\
    f_{(\alpha)}:[0,1]\to\R,\,
    x\mapsto x \log x + (1-x)\log (1-x) + 2 \alpha x (1-x).
\end{gather}
For large $N$, the random variable $N_+/N$ is concentrated
around the minimisers of the function $f_{(N\beta)}$.
It is thus natural to fix some $\alpha\geq 0$ and set $\beta:=\alpha/N$
while sending $N$ to infinity.
The phase transition of the Curie-Weiss model then simply reduces to studying
the minimisers of the function $f_{(\alpha)}$ and the behaviour of this function
around these minimisers.

\begin{exercise}
    \begin{enumerate}
        \item Formally verify that all the calculations are correct.
        \item Show that for small $\alpha$, the function \( f_{(\alpha)} \) has a single minimum, which means that the value of \( K/N \) is concentrated around the value \( 1/2 \).
        \item Show that for large enough $\alpha$, the function \( f_{(\alpha)} \) has two minimizers at \( (1 \pm m)/2 \) for some $m>0$, which means that the value of \( K/N \) is concentrated around these minimizers.
        \item Calculate the critical value for $\alpha$. At this value, the second derivative of \( f_{(\alpha)} \) vanishes at \( x=1/2 \). What does this mean for the distribution of \( N_+/N \)?
    \end{enumerate}
\end{exercise}

\begin{remark}
    Reconsider Equation~\eqref{eq:CurieWeissDistribution}.
    In this equation, the competition between the two factors is extremely transparent.
    \begin{itemize}
        \item     First, there is an comabinatorial term or \emph{entropy}, which favours values $K$ for the random variable
        $N_+$ such that the cardinality of the set $\{N_+=K\}$ is large.
        This means that values $K\approx N/2$ are preferred.
        \item     Second, there is the \emph{energy} term, which favours values such that the Hamiltonian 
        is minimised. This favours configurations where as many spins as possible align.    
    \end{itemize}
    The interaction parameter $\beta$ allows us to put more emphasis
    on the entropy term or on the energy term.
    In the $N\to\infty$ limit, there is a precise value for $N\beta$
    where the behaviour of the random system undergoes a qualitative change:
    a rudimentary example of a  \emph{phase transition}.
\end{remark}

While the competition between entropy and energy is transparent in the
case of the complete graph (the Curie--Weiss model), the situation is much more
complicated when considering other graphs.
We are particularly interested in graphs modelling Euclidean space,
by taking large finite subgraphs of the square lattice graph $\Z^d$
in dimension $d\geq 1$.
The geometry of the Ising model (absent in the case of Curie--Weiss)
is an extremely rich and beautiful object,
and most of this course is dedicated to understanding it.
