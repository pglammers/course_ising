\section{The Curie--Weiss model}
\label{sec:definitions_examples}

At the end of the 19th century, Curie published his experimental results on 
\emph{ferromagnetism}:
the
magnetic properties of metals.
He made three striking observations.
\begin{itemize}
    \item The magnetic strength of a metal varies with the temperature.
    Increasing the temperature decreases the magnetic strength.
    \item Each metal has a certain temperature, specific to that metal,
    at which the magnetic properties disappear entirely.
    We call this temperature the \emph{Curie temperature}.
    \item Around the Curie temperature, the magnetic strength drops continuously
    to zero. In other words, the magnetic strength does not ``jump'' to zero.
\end{itemize}
The first observation singles out the temperature as the driving parameter of the system.
This is good news for us, since the temperature may be regarded informally as the amount of 
``randomness'' or ``entropy'' in the system, justifying a probabilistic analysis of the situation.
The second observation implies that there is a \emph{phase transition}:
there is a special temperature (in this case the Curie temperature) at which
the system's behaviour undergoes a qualitative change.
The third observation entails an important property of this phase transition.

The first mathematical explanation for Curie's experimental results came from
Weiss.
He proposed the following mathematical axioms for studying the magnetic properties of metals.
\begin{itemize}
    \item The metal consists of $n$ atoms.
    \item Each atom acts like a small magnet in itself.
    It is in one of two states, denoted $\pm$.
    \item The total strength of the metal is obtained by summing the states of all atoms.
    \item Each atom interacts with all other atoms.
    The atoms prefer to \emph{align}, that is, to be in the same state.
    The temperature regulates the strength of the interaction.
\end{itemize}
Physically, it makes sense that the temperature regulates the interaction strength.
When atoms move slowly, they will stabilise, oriented in alignment with the magnetic field imposed by the other atoms.
When atoms move fast, they will not bother with the states of the other atoms, and simply align themselves randomly.
It is thus natural to think of the interaction strength as the \emph{inverse temperature}.

\begin{definition}[Curie-Weiss model]
    The Curie-Weiss model is the probability measure $\P^{\operatorname{CW}}_{n,\beta}$
    on $\sigma\in\Omega:=\{+,-\}^n$
    defined via
    \[
        \P(\sigma):=\P^{\operatorname{CW}}_{n,\beta}(\sigma)
        \propto
        e^{-H^{\operatorname{CW}}_{n,\beta}(\sigma)}
        ;
        \qquad
        H(\sigma):=H^{\operatorname{CW}}_{n,\beta}(\sigma):=
        -\frac\beta{n} \sum_{i<j}\sigma_i\sigma_j,
    \]
    where $n\in\Z_{\geq 1}$ and $\beta\in[0,\infty)$.
    The parameter $\beta$ is called the \emph{interaction strength} or \emph{inverse temperature}.
    The function $H$ is called the \emph{Hamiltonian} and captures the \emph{energy} in the system.
    The probability measure $\P$ is also called the \emph{Boltzmann distribution}.
\end{definition}

Let $n_+=n_+(\sigma)$ denote the number of vertices with spin $+$ in a configuration $\sigma\in\Omega$.
This is a random variable.
Let us try to calculate the probability of the event $\{n_+=k\}$,
without worrying about the partition function (the normalising constant).
One may easily check that
the Hamiltonian satisfies
\begin{align}
    H(\sigma)=2\frac\beta{n} n_+(n-n_+) + \text{const}(n).
\end{align}
The distribution of $n_+$ can then be calculated as follows:
\begin{align}
    \label{eq:CurieWeissDistribution}
    \P(\{n_+ = k\}) &\propto \binom{n}{k} e^{-2\frac\beta{n}  k(n-k)}
    \propto \frac1{k! (n-k)!} e^{-2\frac\beta{n} k (n-k)}.
\end{align}
Using Stirling's approximation for the factorials, we find that
\begin{gather}
    \log\P(\{n_+ = k\})
    \stackrel{\text{Stirling}}{\approx}
    -n f_{\beta}(k/n) + \text{const}(n);
    \\
    f_{\beta}:[0,1]\to\R,\,
    x\mapsto x \log x + (1-x)\log (1-x) + 2 \beta x (1-x).
\end{gather}
If we fix $\beta$ and send $n$ to infinity, then 
we discover a large deviations principle for the random variable $n_+/n$
with rate function $f_{\beta}$ and speed $n$.
In particular, the random variable $n_+/n$ concentrates
around the minimisers of the function $f_{\beta}$.

\begin{exercise}[The rate function of the Curie--Weiss model]
    \begin{enumerate}
        \item Show that for small $\beta$, the function \( f_{\beta} \) has a single minimum at $x=1/2$, which means that the random variable \( n_+/n \) concentrates around the value \( 1/2 \).
        \item Show that for large $\beta$, the function \( f_{\beta} \) has two minima at \( (1 \pm m)/2 \) for some $m>0$, which means that the random variable \( n_+/n \) is concentrated around these minima.
        The value of $m$ is called the \emph{magnetisation}.
        \item Calculate the critical value for $\beta$. At this value, the second derivative of \( f_{\beta} \) vanishes at \( x=1/2 \). What does this mean for the distribution of \( n_+/n \)?
        Estimate the order of magnitude of $\Var\frac{n_+}{n}$ as $n\to\infty$ for this value of $\beta$.
    \end{enumerate}
\end{exercise}

\begin{remark}[Entropy versus energy in the Curie--Weiss model]
    Reconsider Equation~\eqref{eq:CurieWeissDistribution}.
    In this equation, the competition between the two factors is extremely transparent.
    \begin{itemize}
        \item     First, there is a combinatorial term or \emph{entropy}, which favours values $k$ for the random variable
        $n_+$ such that the cardinality of the set $\{n_+=k\}$ is large.
        This means that values $k\approx n/2$ are preferred.
        \item     Second, there is the \emph{energy} term, which favours values such that the energy 
        is minimised. This favours configurations where as many spins as possible align.    
    \end{itemize}
    The interaction parameter $\beta$ allows us to put more emphasis
    on the entropy term or on the energy term.
    In the $n\to\infty$ limit, there is a precise value for $\beta$
    where the behaviour of the random system undergoes a qualitative change:
    a rudimentary example of a \emph{phase transition}.
\end{remark}
