\section{The random-currents expansion}

Next, we introduce random currents.
Random currents are a refinement of the high-temperature expansion.
Let us make precise what we mean, before turning to the details.
A significant drawback of Lemma~\ref{lemma:switching_high_t}
is the fact that we can switch subsets of the \emph{symmetric difference}
$\omega\Delta\omega'$, but not of the \emph{union} $\omega\cup\omega'$.
Random currents carry slightly more information than high-temperature expansion,
which enables this ``upgrade''.
Switching over the union of the two percolations is important for several
new correlation inequalities, and eventually the proof of continuity of the phase transition.

\begin{definition}[Currents]
    Let $G=(V,E)$ denote a graph.
    A \emph{current} is a map $\n:E\to\Z_{\geq 0}$.
    We think of $(V,\n)$ as a multigraph,
    where for each edge $uv\in E$ we have $\n_{uv}$ multi-edges between $u$ and $v$.
    The set of \emph{sources} $\partial\n\subset V$ of a current
    $\n$ is defined as the set of vertices $u\in V$ with an odd degree in the multigraph.
    We let $\hat\n:=(\n\wedge 1)\in\{0,1\}^E$ denote the associated percolation,
    which simply contains the edges carrying at least one current.

    If $G$ is finite and $\beta\in[0,\infty)$, then the
    \emph{weight} of a current is defined as
    \[
        w(\n):=w_{G,\beta}(\n):=\prod_{xy\in E}
        \frac{\beta^{\n_{xy}}}{\n_{xy}!}.
    \]
    The \emph{random-currents measure} is the measure $\M_{G,\beta}$ on $(\Z_{\geq 0})^E$
    defined by
    \[
        \M[\n]:=\M_{G,\beta}[\n]:=w_{G,\beta}(\n).
    \]
\end{definition}

\begin{remark}
    Notice that $e^{-\beta|E|}\M_{G,\beta}$ is a probability measure
    in which $(\n_{xy})_{xy\in E}$ is a family of independent
    random variables with distribution $\operatorname{Poisson}(\beta)$.
\end{remark}

The random-currents measure is a richer object than the high-temperature expansion.
To see this, consider a single fixed edge $xy\in E$.
Then we have the following correspondence between the high-temperature weights
and the random-currents weights:
\[
    \cosh\beta = \sum_{\n\in 2\Z_{\geq 0}} \frac{\beta^\n}{\n!}
    ;
    \qquad
    \sinh\beta = \sum_{\n\in 2\Z_{\geq 0}+1} \frac{\beta^\n}{\n!}.
\]
We may thus interpret the relation between $e^{-\beta|E|}\bfM$ and $e^{-\beta|E|}\M$ as follows:
\begin{itemize}
    \item $\M$ is a nonnormalised family of independent $\operatorname{Poisson}(\beta)$-variables $(\n_{xy})_{xy\in E}$,
    \item $\bfM$ is obtained from $\M$ by writing $\omega_{xy}$ for the \emph{parity} of $\n_{xy}$,
    \item In particular, $\partial\n\sim\M$ and $\partial\omega\sim\bfM$ have the same distribution,
    \item Moreover, the percolation $\hat\n$ may be viewed as
    \[
        \hat\n = \omega \cup\{xy\in E: \n_{xy}\in\{2,4,8,\ldots\}\}.
    \]
    In particular, the following distributions are the same:
    \begin{equation}
        \label{eq:equivalence}
        \text{$\hat\n$ in the measure $\M$}
        \qquad
        \text{and}
        \qquad
        \text{$\omega\cup\eta$ in the measure $\bfM\times\P_{p}$,}
    \end{equation}
    where $\eta\sim\P_p$ is an independent bond percolation on $G$
    with parameter
    \[
        p
        =\frac{\cosh\beta-1}{\cosh\beta}
        =\frac{
            \sum_{\n\in 2\Z_{\geq 1}} \frac{\beta^\n}{\n!}
        }{
            \sum_{\n\in 2\Z_{\geq 0}} \frac{\beta^\n}{\n!}
        }
        .
    \]
\end{itemize}

The last observation arises from the simple fact that edges carring an even current
still have a probability $p$ of carrying a strictly positive even current.

Theorem~\ref{thm:High-temperature expansion for correlation functions}
translates to random currents as follows.

\begin{theorem}[Current representation of correlation functions]
    \label{thm:current_representation_of_correlation_functions}
    Consider the Ising model on a finite graph $G$ at inverse temperature $\beta$.
    Let $A\subset V$ be a subset of vertices.
    Then
    \[
        Z\langle\sigma_A\rangle
        =2^{|V|}\sum_{\n:\:\partial\n=A}
        w(\n)
        =
        2^{|V|}\M[\{\partial\n=A\}].
    \]
    In particular, the partition function is given by
    \[
        Z=2^{|V|}\sum_{\n:\:\partial\n=\emptyset}
        w(\n)
        =
        2^{|V|}\M[\{\partial\n=\emptyset\}].
    \]
\end{theorem}

We will now explain how random currents allow us to ``upgrade'' from switching
over subsets of $\omega\Delta\omega'$ to subsets of $\hat\n\cup\hat\m$.

\begin{exercise}[Poisson switching]
    Suppose that we record cars traversing a bridge on a road.
    Blue cars pass according to a Poisson point process with rate $1$ (per second).
    Let $X_B$ denote the number of blue cars that pass after recording $\beta$ seconds.
    Can we easily prove, without a calculation, that
    \[
        \P[\{\text{$X_B$ is even}\}]\geq \P[\{\text{$X_B$ is odd}\}]?
    \]

    Suppose that there are also yellow cars,
    which arrive according to an independent Poisson process with the same rate.
    Let $X_Y$ denote the number of yellow cars that passed.

    By considering the colour of the \emph{last} car that passed the bridge, prove that:
    \begin{enumerate}
        \item $\P[\{\text{$X_B$ is even}\}|\{X_B+X_Y= N \}]=\frac12$ whenever $N>0$,
        \item $\P[\{\text{$X_B$ is even}\}|\{X_B+X_Y= N \}]=0$ whenever $N=0$,
        \item $\P[\{\text{$X_B$ is even}\}]-\P[\{\text{$X_B$ is odd}\}]=\P[\{\text{$X_B+X_Y=0$}\}]\geq 0$.
    \end{enumerate}
\end{exercise}

\begin{lemma}[Switching lemma]
    \label{lem:explicit_switching_lemma}
    Let $G$ and $G'$ denote two finite graphs and fix $\beta\in[0,\infty)$.
    If $A\subset V$, $\s\in(\Z_{\geq 0})^{E\cup E'}$, and $Q\subset \hat\s\cap E\cap E'$,
    then 
    \begin{equation}
        \M_{G,\beta}\times\M_{G',\beta}
        [\{\n+\m=\s\}\cap \{\partial\n=A\}]
        =
        \M_{G,\beta}\times\M_{G',\beta}
        [\{\n+\m=\s\}\cap \{\partial\n= A\Delta\partial Q\}].
    \end{equation}
    Notice also that if $\n+\m=\s$, then $\partial\s=(\partial\n)\Delta(\partial\m)$.
\end{lemma}

\begin{proof}
    By induction, we may simply suppose that $|Q|=1$, say $Q=\{xy\}$.
    Introduce the probability measure
    \[
        \P:\propto  \M_{G,\beta}\times\M_{G',\beta}
        [\{\n+\m=\s,\,\n|_{Q^c}=\a,\,\m|_{Q^c}=\b\}\cap(\blank)];
    \]
    for the claimed identity it suffices to derive, for fixed $\a$ and $\b$, the stronger equality
    \[
        \P[\{\partial\n=A\}]=\P[\{\partial\n=A\Delta\partial Q\}].
    \]
    But the only randomness that is left in the measure $\P$,
    are the values of $\n_{xy}$ and $\m_{xy}$,
    which are independent Poisson random variables conditioned to sum to $\s_{xy}>0$.
    By the previous exercise, the parity of $\n_{xy}$ has the distribution of a fair coin flip,
    which proves the previous identity.
\end{proof}


% \begin{lemma}[Switching lemma]
%     \label{lem:switching_lemma}
%     Let $G$ denote a finite graph and $\beta\in[0,\infty)$.
%     Consider the measurable pair $(\n,\m)\sim \M^2=\M_{G,\beta}^2$.
%     Then for any $A,B,S\subset V$ and for any bounded function $F:(\Z_{\geq 0})^E\to\C$,
%     we have
%     \begin{align}
%         &\M^2[F(\n+\m)\true{\widehat{\n+\m} \in \calE_S}\true{\partial\n=A}\true{\partial\m=B}]
%         \\={}&
%         \M^2[F(\n+\m)\true{\widehat{\n+\m} \in \calE_S}\true{\partial\n=A\Delta S}\true{\partial\m=B\Delta S}].
%     \end{align}
% \end{lemma}

% \begin{proof}
%     By linearity of integration, it suffices to consider the case that $F(\n+\m):=\true{\n+\m=\s}$
%     for some fixed current $\s$ with $\hat\s\in\calE_S$.
%     By the previous exercise, we may find some $\eta\subset\hat\s$
%     such that $\partial\eta=S$.
%     The desired equality
%     \begin{align}
%     &
%     \M^2[\true{\n+\m =\s}\true{\partial\n=A}\true{\partial\m=B}]
%     \\={}&
%     \M^2[\true{\n+\m =\s}\true{\partial\n=A\Delta S}\true{\partial\m=B\Delta S}]
%     \end{align}
%     then follows by the explicit switching lemma.
% \end{proof}

% In practice, we do not care so much about the function $F$, and simply set it to $F\equiv 1$.

An important corollary of the switching lemma is the \emph{second Griffiths inequality}.

\begin{lemma}[Second Griffiths inequality]
    Consider the Ising model on a finite graph $G$ at inverse temperature $\beta$.
    Then for any $A,B\subset V$, we have
    $\langle\sigma_{A}\sigma_B\rangle-\langle\sigma_A\rangle\langle\sigma_B\rangle\geq 0$.
\end{lemma}

The second Griffiths inequality is more subtle than the first,
as it bounds a \emph{difference} of correlation functions.
This is typical for the switching lemma.

\begin{proof}
    Claim that
    \begin{align}
        &\M^2[\{\partial\n=A\}\cap\{\partial\m=B\}]
        \\
        &\qquad=
        \M^2[\{\widehat{\n+\m} \in \calE_B\}\cap\{\partial\n=A\}\cap\{\partial\m=B\}]
        \\
        &\qquad\stackrel{\text{switch}}=
        \M^2[\{\widehat{\n+\m} \in \calE_B\}\cap\{\partial\n=A\Delta B\}\cap\{\partial\m=\emptyset\}]
        \\
        &\qquad\leq
        \M^2[\{\partial\n=A\Delta B\}\cap\{\partial\m=\emptyset\}].
    \end{align}
    For the first equality, we simply observe that $\{\partial\m=B\}\subset \{\widehat{\n+\m} \in \calE_B\}$
    (see Exercice~\ref{exo:basic_perco}).
    The switch is the switching lemma, where we switch over a subset of $\widehat{\n+\m}$ 
    having $B$ as source set.
    The inequality is inclusion of events.

    By the random currents expansion of correlation functions (Theorem~\ref{thm:current_representation_of_correlation_functions}),
    the left- and rightmost expressions are given by
    \[
        Z^2\langle\sigma_A\rangle\langle\sigma_B\rangle/4^{|V|}
        \leq
        Z^2\langle\sigma_A\sigma_B\rangle\langle\sigma_\emptyset\rangle/4^{|V|}.
    \]
    Since $\langle\sigma_\emptyset\rangle=1$, this is the desired inequality.
\end{proof}

\begin{exercise}[The two-point function as a metric]
    Consider the Ising model on a finite graph $G$
    at inverse temperature $\beta>0$.
    Prove that $V\times V\to [0,\infty],\,
    (u,v)\mapsto-\log\langle\sigma_u\sigma_v\rangle_{G,\beta}$
    defines a metric on $V$.
    Use directly the switching lemma, and not the second Griffiths inequality.
    What does the percolation event $\calE_S$ look like in this case?
\end{exercise}

% \begin{definition}[Probability measures on currents]
%     Consider the Ising model on a finite graph $G$
%     at inverse temperature $\beta$.
%     For any $A\subset V$,
%     define the probability measure $\P^A_{G,\beta}$ by
%     \[
%         \P^A:=\P^A_{G,\beta}:=\frac{2^{|V|}}{Z_{G,\beta}\langle\sigma_A\rangle_{G,\beta}}\M_{G,\beta}[\true{\partial\n=A}(\blank)].
%     \]
%     For any $A_1,\ldots,A_n$,
%     write $\P^{A_1,\ldots,A_n}:=\P^{A_1}\times\cdots\times\P^{A_n}$.
% \end{definition}

\begin{exercise}[Correlation functions in terms of sourceless currents]
    Consider the Ising model on a finite graph $G$
    at inverse temperature $\beta$.
    Prove that for any $A\subset V$,
    \[
        Z^2\langle\sigma_A\rangle^2/4^{|V|}
        =
        \M^2[\{\partial\n=\partial\m=\emptyset\}\cap\{\widehat{\n+\m}\in\calE_A\}].
    \]

    Observe that we can now express all correlation functions in terms of a single
    fixed probability measure $4^{|V|}\M^2[\{\partial\n=\partial\m=\emptyset\}\cap(\blank)]/Z^2$ on sourceless random currents.
\end{exercise}
