\section{The random-currents expansion}

Previous sections explained the existence of a phase transition in dimension
$d\geq 2$, by demonstrating that the qualitative behaviour of the model 
is different at low and high temperature.
This section introduces a new representation of the Ising model
which is adapted to studying the Ising model at and around the critical temperature.

\begin{definition}[Currents]
    Let $G=(V,E)$ denote a graph.
    A \emph{current} is a map $\n:E\to\Z_{\geq 0}$.
    We think of $(V,\n)$ as a multigraph,
    where for each edge $uv\in E$ we have $\n_{uv}$ multi-edges between $u$ and $v$.
    The set of \emph{sources} $\partial\n\subset V$ of a current
    $\n$ is defined as the set of vertices $u\in V$ with an odd degree in the multigraph.
    We let $\hat\n:=(\n\wedge 1)\in\{0,1\}^E$ denote the associated percolation,
    which simply contains the edges carrying at least one current.

    If $G$ is finite and $\beta\in[0,\infty)$, then the
    \emph{weight} of a current is defined as
    \[
        w(\n):=w_{G,\beta}(\n):=\prod_{xy\in E}
        \frac{\beta^{\n_{xy}}}{\n_{xy}!}.
    \]
    The \emph{random-currents measure} is the measure $\M_{G,\beta}$ on $(\Z_{\geq 0})^E$
    defined by
    \[
        \M[\n]:=\M_{G,\beta}[\n]:=w_{G,\beta}(\n).
    \]
\end{definition}

\begin{remark}
    Notice that $e^{-\beta|E|}\M_{G,\beta}$ is a probability measure
    in which $(\n_{xy})_{xy\in E}$ is a family of independent
    random variables with distribution $\operatorname{Poisson}(\beta)$.
\end{remark}

Currents can be used to encode correlation functions of the Ising model.

\begin{theorem}[Current representation of correlation functions]
    \label{thm:current_representation_of_correlation_functions}
    Consider the Ising model on a finite graph $G$ at inverse temperature $\beta$.
    Let $A\subset V$ be a subset of vertices.
    Then
    \[
        Z\langle\sigma_A\rangle
        =2^{|V|}\sum_{\n:\:\partial\n=A}
        w(\n)
        =
        2^{|V|}\M[\{\partial\n=A\}].
    \]
    In particular, the partition function is given by
    \[
        Z=2^{|V|}\sum_{\n:\:\partial\n=\emptyset}
        w(\n)
        =
        2^{|V|}\M[\{\partial\n=\emptyset\}].
    \]
\end{theorem}

\begin{proof}
    The numbers $\cosh\beta$ and $\sinh\beta$ are obtained from the expansion
    $\sum_{\n=0}^\infty \beta^\n/\n!$ of $e^\beta$ by keeping only
    the even and odd terms respectively.
    The result then simply follows from
    the high-temperature representation (Theorem~\ref{thm:High-temperature expansion for correlation functions})
    by
    expanding $\cosh\beta$ and $\sinh\beta$.
\end{proof}


\begin{remark}[Switching example]
    The above setup opens the door to a powerfull technique called \emph{switching}.
    Two elements are key to switching:
    Poisson randomness, and parity constraints
    (both of which are present in the currents representation of correlation functions).
    Let us quickly describe how switching may be applied to a simple example.
    Suppose that we record cars traversing a bridge on a road.
    Blue cars pass according to a Poisson point process with rate $1$ (per second).
    Let $X_B$ denote the number of blue cars that pass after recording $\beta$ seconds.
    Can we easily prove, without a calculation, that
    \[
        \P[\{\text{$X_B$ is even}\}]\geq \P[\{\text{$X_B$ is odd}\}]?
    \]

    Suppose that there are also yellow cars,
    which arrive according to an independent Poisson process with the same rate.
    Let $X_Y$ denote the number of yellow cars that passed.
    Suppose that, after waiting $\beta$ seconds, $X_B+X_Y=N>0$ cars passed.
    What is the \emph{conditional} probability that $X_B$ is even?

    Well, we must have $\P[\{\text{$X_B$ is even}\}|\{X_B+X_Y=N\}]=1/2$.
    Indeed, by the properties Poisson point processes,
    the $n$-th car (for $1\leq n\leq N$) is blue or yellow with equal probability, 
    independently of the other cars.
    Thus, we may condition on the colours of the first $N-1$ cars,
    then flip a fair coin to decide the colour of the last car
    and thus the parity of $X_B$.
    Equivalently, we observe that \emph{repainting} or \emph{switching}
    the colour of the last car leaves the conditional distribution invariant.

    But we cannot always do the switch.
    If $N=X_B+X_Y=0$
    then there is no car to repaint, and also
    $X_B=0$.
    Thus, we conclude that
    \[
        \P[\{\text{$X_B$ is even}\}]-\P[\{\text{$X_B$ is odd}\}]=\P[\{X_B+X_Y=0\}]\geq 0.
    \]
    
    Notice that we originally asked a question about blue cars,
    but introducing yellow cars allowed us to answer it.
    This is the essence of the switching lemma.
\end{remark}

\begin{lemma}[Explicit switching lemma]
    \label{lem:explicit_switching_lemma}
    Let $G$ denote a finite graph and $\beta\in[0,\infty)$.
    Consider the measurable pair $(\n,\m)\sim \M^2=\M_{G,\beta}^2$.
    Fix $\s\in(\Z_{\geq 0})^E$ and $\eta\subset\hat\s$.
    Then for any $A\subset V$,
    we have
    \[
        \M^2[\{\n+\m=\s\}\cap\{\partial\n=A\}]
        =
        \M^2[\{\n+\m=\s\}\cap\{\partial\n=A\Delta\partial\eta\}].
    \]
    Notice also that if $\n+\m=\s$, then $\partial\s=(\partial\n)\Delta(\partial\m)$.
\end{lemma}

Slight variations of the proof will be used later.

\begin{proof}
    Introduce the probability measure $\P:\propto\M^2[\{\n+\m=\s\}\cap(\blank)]$.
    Our objective is to prove that
    \[
        \P[\{\partial\n=A\}]
        =
        \P[\{\partial\n=A\Delta\partial\eta\}].
    \]

    In the probability measure $\P$, the family $((\n_{xy},\m_{xy}))_{xy\in E}$
    is independent over the edges $xy\in E$,
    and each pair $(\n_{xy},\m_{xy})$ follows the distrubution of two independent 
    distributions $\operatorname{Poisson}(\beta)$ conditioned to sum to $\s_{xy}$.
    By analogy with the example of blue and yellow cars,
    we may interpret $\P$ as follows:
    \begin{itemize}
        \item $\calM=\calM_\s$ is the fixed multigraph $\{(xy,k)\in E\times\Z_{\geq 0}:k<\s_{xy}\}$,
        \item $\calA$ is a uniformly random subset of $\calM$,
        \item $\calB$ is the complement of $\calA$ in $\calM$,
        \item $\n_{xy}=\n_{xy}(\calA)$ is the number of multi-edges in $\calA$ between $x$ and $y$,
        \item $\m_{xy}=\m_{xy}(\calB)$ is the number of multi-edges in $\calB$ between $x$ and $y$.
    \end{itemize}

    Define $\tilde\eta:=\eta\times\{0\}\subset\calM$.
    Since $\calA$ is a uniformly random subset of $\calM$ in the measure $\P$,
    the set $\calA\Delta\tilde\eta$ is also uniformly random in $\calM$.
    Therefore we have
    \[
        \P[\{\partial\n=A\}]=
        \P[\{\partial\n(\calA)=A\}]
        =
        \P[\{\partial\n(\calA\Delta\tilde\eta)=A\}]
        =
        \P[\{\partial\n=A\Delta\partial\eta\}].
    \]
    This is the desired equality.
\end{proof}

\begin{definition}[Percolation of currents]
    Let $G$ denote a graph and $\n$ a current.
    Write \[\{u\xleftrightarrow{\hat\n}v\}\]
    for the event there is an open path from $u$ to $v$
    ($u$ and $v$ may represent vertices or sets of vertices).
    For fixed $S\subset V$, we shall also write $\calE_S$ for the set
    \[
        \{O\subset E:\text{$|C\cap S|$ is even for any connected component $C\subset V$ of $(V,O)$}\}.
    \]
\end{definition}

\begin{exercise}
    Let $G$ denote a graph and 
    $x,y\in V$  distinct vertices. Prove the following.
    \begin{itemize}
        \item If $S=\{x,y\}$, then $\{\hat\n\in\calE_S\}=\{x\xleftrightarrow{\hat\n}y\}$.
        \item If $\omega\in\calE_S$, then we may find a subset $\eta\subset\omega$ with $\partial\eta=S$.
        \item If $G$ is a finite graph and $\partial\n=S$, then $\hat\n\in\calE_S$.
        \item For any $S\subset V$, the event $\{\hat\n\in\calE_S\}$ is an increasing event of the current $\n$.
    \end{itemize}
\end{exercise}

\begin{lemma}[Switching lemma]
    \label{lem:switching_lemma}
    Let $G$ denote a finite graph and $\beta\in[0,\infty)$.
    Consider the measurable pair $(\n,\m)\sim \M^2=\M_{G,\beta}^2$.
    Then for any $A,B,S\subset V$ and for any bounded function $F:(\Z_{\geq 0})^E\to\C$,
    we have
    \begin{align}
        &\M^2[F(\n+\m)\true{\widehat{\n+\m} \in \calE_S}\true{\partial\n=A}\true{\partial\m=B}]
        \\={}&
        \M^2[F(\n+\m)\true{\widehat{\n+\m} \in \calE_S}\true{\partial\n=A\Delta S}\true{\partial\m=B\Delta S}].
    \end{align}
\end{lemma}

\begin{proof}
    By linearity of integration, it suffices to consider the case that $F(\n+\m):=\true{\n+\m=\s}$
    for some fixed current $\s$ with $\hat\s\in\calE_S$.
    By the previous exercise, we may find some $\eta\subset\hat\s$
    such that $\partial\eta=S$.
    The desired equality
    \begin{align}
    &
    \M^2[\true{\n+\m =\s}\true{\partial\n=A}\true{\partial\m=B}]
    \\={}&
    \M^2[\true{\n+\m =\s}\true{\partial\n=A\Delta S}\true{\partial\m=B\Delta S}]
    \end{align}
    then follows by the explicit switching lemma.
\end{proof}

In practice, we do not care so much about the function $F$, and simply set it to $F\equiv 1$.

An important corollary of the switching lemma is the \emph{second Griffits inequality}.

\begin{lemma}[Second Griffiths inequality]
    Consider the Ising model on a finite graph $G$ at inverse temperature $\beta$.
    Then for any $A,B\subset V$, we have
    $\langle\sigma_{A}\sigma_B\rangle-\langle\sigma_A\rangle\langle\sigma_B\rangle\geq 0$.
\end{lemma}

The second Griffiths inequality is more subtle than the first,
as it bounds a \emph{difference} of correlation functions.
This is typical for the switching lemma.

\begin{proof}
    Claim that
    \begin{align}
        &\M^2[\{\partial\n=A\}\cap\{\partial\m=B\}]
        \\
        &\qquad=
        \M^2[\{\widehat{\n+\m} \in \calE_B\}\cap\{\partial\n=A\}\cap\{\partial\m=B\}]
        \\
        &\qquad\stackrel{\text{switch}}=
        \M^2[\{\widehat{\n+\m} \in \calE_B\}\cap\{\partial\n=A\Delta B\}\cap\{\partial\m=\emptyset\}]
        \\
        &\qquad\leq
        \M^2[\{\partial\n=A\Delta B\}\cap\{\partial\m=\emptyset\}].
    \end{align}
    The switch is just the switching lemma with $F\equiv 1$ and $S=B$.
    For the first equality, we simply observe that $\{\partial\m=B\}\subset \{\widehat{\n+\m} \in \calE_B\}$
    (see the exercise above).
    The inequality is inclusion of events.

    By the random currents expansion of correlation functions (Theorem~\ref{thm:current_representation_of_correlation_functions}),
    the left- and rightmost expressions are given by
    \[
        Z^2\langle\sigma_A\rangle\langle\sigma_B\rangle
        \leq
        Z^2\langle\sigma_A\sigma_B\rangle\langle\sigma_\emptyset\rangle.
    \]
    Since $\langle\sigma_\emptyset\rangle=1$, this is the desired inequality.
\end{proof}

\begin{exercise}[The two-point function as a metric]
    Consider the Ising model on a finite graph $G$
    at inverse temperature $\beta>0$.
    Prove that $V\times V\to [0,\infty],\,
    (u,v)\mapsto-\log\langle\sigma_u\sigma_v\rangle_{G,\beta}$
    defines a metric on $V$.
    Use directly the switching lemma, and not the second Griffiths inequality.
    What does the percolation event $\calE_S$ look like in this case?
\end{exercise}

\begin{definition}[Probability measures on currents]
    Consider the Ising model on a finite graph $G$
    at inverse temperature $\beta$.
    For any $A\subset V$,
    define the probability measure $\P^A_{G,\beta}$ by
    \[
        \P^A:=\P^A_{G,\beta}:=\frac{2^{|V|}}{Z_{G,\beta}\langle\sigma_A\rangle_{G,\beta}}\M_{G,\beta}[\true{\partial\n=A}(\blank)].
    \]
    For any $A_1,\ldots,A_n$,
    write $\P^{A_1,\ldots,A_n}:=\P^{A_1}\times\cdots\times\P^{A_n}$.
\end{definition}

\begin{exercise}[Correlation functions in terms of sourceless currents]
    Consider the Ising model on a finite graph $G$
    at inverse temperature $\beta$.
    Prove that for any $A\subset V$,
    \[
        \langle\sigma_A\rangle^2
        =
        \P^{\emptyset,\emptyset}[\{\widehat{\n+\m}\in\calE_A\}].
    \]

    Observe that we can now express all correlation functions in terms of a single
    fixed probability measure on sourceless random currents.
\end{exercise}
