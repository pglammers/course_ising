\section{Early developments. 1924: Ising's analysis}
\label{sec:ising_1d}

Wilhelm Lenz challenged his doctoral student Ernst Ising
to solve the model of interest on the one-dimensional line graph $\Z$.
Readers acquainted with percolation theory
will suspect that such a simple model is unlikely to exhibit
a phase transition.
This suspicion is correct,
but we stress that the percolation model had not yet
been described at the time that Ising undertook his doctoral research.

We defined the Ising model in the previous section,
but only on finite graphs.
Let us extend this definition to infinite graphs.

\begin{definition}[The Ising model with boundary conditions]
    Let $G=(V,E)$ denote a fixed locally finite graph.
    We consider the measurable space 
    $(\Omega,\calF)$ where $\Omega=\{\pm1\}^V$
    and where $\calF$ is the product-$\sigma$-algebra.

    Let $\Lambda$ denote a \emph{domain},
    that is, a finite subset of $V$.
    The Ising model in the domain $\Lambda$
    with inverse temperature $\beta \geq 0$ and boundary conditions
    $\zeta\in\{\pm1\}^{V\setminus\Lambda}$
    is the probability measure $\P_{\Lambda,\beta}^{\operatorname{Ising},\zeta}$
    on $(\Omega,\calF)$ defined via:
    \[
        \P_{\Lambda,\beta}^{\operatorname{Ising},\zeta}(\sigma)
        =
        \frac1{Z_{\Lambda,\beta}^{\operatorname{Ising},\zeta}}
        \cdot
        \true{\sigma|_{V\setminus\Lambda}=\zeta}\cdot e^{-H_{\Lambda,\beta}^{\operatorname{Ising}}(\sigma)}
        ,
    \]
    where $H_{\Lambda,\beta}^{\operatorname{Ising}}(\sigma)$ is the Hamiltonian given by
    \[
        H_{\Lambda,\beta}^{\operatorname{Ising}}(\sigma)
        =
        -\beta\sum_{uv\in E(\Lambda)}\sigma_u\sigma_v
        ,
    \]
    and where the partition function $Z_{\Lambda,\beta}^{\operatorname{Ising},\zeta}$ is given by
    \[
        Z_{\Lambda,\beta}^{\operatorname{Ising},\zeta}
        =
        \sum_{\sigma \in \Omega} \true{\sigma|_{V\setminus\Lambda}=\zeta} \cdot e^{-H_{\Lambda,\beta}^{\operatorname{Ising}}(\sigma)}
        .
    \]
    The set $E(\Lambda)\subset E$ denotes 
    the set of edges with at least one endpoint in $\Lambda$.
    Indeed, adding a constant to the Hamiltonian does not affect the measure,
    and edges which do not intersect $\Lambda$ contribute with a constant.

    We write $+$ and $-$ for the boundary conditions
    $+1\in\Omega$ and $-1\in\Omega$ respectively.
\end{definition}

\begin{theorem}[Ising, 1924]
    The one-dimensional Ising model is demagnetised at all temperatures.
    This means the following.
    Let $G=(V,E)$ denote the one-dimensional lattice $\Z$,
    and define $\Lambda_n:=\{-n+1,\ldots,n-1\}$.
    Then, for any $\beta\geq 0$, we have
    \[
        \lim_{n\to\infty}\langle\sigma_0\rangle_{\Lambda_n,\beta}^+
        =
        0
        .
    \]
\end{theorem}

\begin{proof}
    Write $T$ for the matrix
    \[
        T:=
        \begin{pmatrix}
            e^{\beta} & e^{-\beta} \\
            e^{-\beta} & e^{\beta}
        \end{pmatrix}.
    \]
    It is straightforward to work out that
    \begin{gather}
        Z_{\Lambda_n,\beta}^+\langle\sigma_0\rangle_{\Lambda_n,\beta}^+
        =
        \left(T^n
            \begin{pmatrix}
                1 & 0 \\ 0 & -1
            \end{pmatrix}
            T^n
            \right)_{1,1};
        \\
        Z_{\Lambda_n,\beta}^+ = \left(T^{2n}\right)_{1,1},
    \end{gather}
    see the exercise below.
    One may then conclude that the ratio of these two numbers tends to zero 
    with $n\to\infty$
    by simply diagonalising $T$.
\end{proof}

\begin{remark}
    Although the intuition is reminiscent
    of the theory of Markov chains, we stress that the matrix
    $T$ above is \emph{not} a stochastic matrix.
    This is why we need to consider the partition function
    (normalising constant) separately.
\end{remark}

\begin{exercise}
    Let $(f_k)_k$ denote a family of functions
    of the form $f_k:\{+1,-1\}\to\R$.
    For any $k$, define
    \[ M_k:=\begin{pmatrix}
        f_{k}(+1) & 0 \\
        0 & f_{k}(-1)
    \end{pmatrix}.\] 
    Prove that for any $n$,
    we have
    \[
        Z_{\Lambda_n,\beta}^+
        \langle\textstyle\prod_{k\in\Lambda_n} f_k(\sigma_{k})\rangle_{\Lambda_n,\beta}^+
        =
        \left(
            T
            M_{n-1}
            T
            M_{n-2}
            T
            \cdots
            T
            M_{-n+1}
            T
        \right)_{1,1}.
    \]
\end{exercise}

\begin{remark}
    Ising conjectured that the absence of magnetisation in the one-dimensional
    model would also hold in higher dimensions. This was later shown to be
    false: Peierls proved in 1936 that the two-dimensional model magnetises for
    sufficiently large $\beta$.
\end{remark}
