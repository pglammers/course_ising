\section{Mixing properties of the infinite-volume limit}

In this section, we consider the Ising model on the infinite square lattice $\Z^d$.

\begin{definition}
    Consider a measure $\langle\blank\rangle\in\calP(\Omega,\calF)$.
    \begin{itemize}
        \item For any $u\in\Z^d$, we define the \emph{shift operator} $\tau_u:\Omega\to\Omega$ by
        \[
            (\tau_u\sigma)_x = \sigma_{x-u}.
        \]
        An event $A$ is \emph{shift-invariant} if $\tau_uA:=\{\tau_u\sigma:\sigma\in A\}$ for any $u\in\Z^d$.
        \item The measure is called \emph{shift-invariant} if
        \[
            \langle X\circ\tau_u \rangle = \langle X\rangle
        \]
        for any vertex $u\in\Z^d$ and
        for any bounded local observable $X$.
        \item The measure is called \emph{mixing} if
        \[
            \lim_{\|u\|_2\to\infty}
            \left(\langle X(Y\circ\tau_u)\rangle-\langle X\rangle\langle Y\circ\tau_u\rangle\right)
            =0
        \]
        for any bounded local observables $X$ and $Y$.
        \item The measure is called \emph{ergodic}
        if it is shift-invariant
        and
        \[\langle\ind{A}\rangle\in\{0,1\}\]
        for any shift-invariant event $A\in\calF$.
    \end{itemize}
\end{definition}

\begin{lemma}[Mixing implies ergodicity]
    If a shift-invariant measure is mixing, then it is ergodic.
\end{lemma}

\begin{proof}
    Let $\langle\blank\rangle$ denote a shift-invariant measure that is mixing, but not ergodic.
    We aim to derive a contradiction.
    Let $A$ denote a shift-invariant event with $p:=\langle\ind{A}\rangle\in(0,1)$.
    We shall derive a contradiction by constructing two events which
    are both extremely correlated with $A$ (using ergodicity),
    while also being almost independent (using mixing).

    Fix $\epsilon>0$.
    By the martingale convergence theorem,
    there exists a finite domain $\Lambda\subset\Z^d$
    and an $\calF_\Lambda$-measurable event
    $A_\Lambda$
    such that
    $\langle \ind{A\Delta A_\Lambda}\rangle<\epsilon$.
    Write $p':=\langle \ind{A_\Lambda}\rangle$;
    notice that $|p'-p|<\epsilon$.
    Define $A_{\Lambda+u}:=\tau_uA_\Lambda\in\calF_{\Lambda+u}$.
    We claim that there exists some $u\in\Z^d$ such that:
    \[
        \langle \ind{A_\Lambda\Delta A_{\Lambda+u}}\rangle
        <2\epsilon
        \qquad
        \text{and}
        \qquad
        \langle \ind{A_\Lambda\Delta A_{\Lambda+u}}\rangle
        \geq 2p'(1-p')-\epsilon.
    \]
    This yields the desired contradiction when $\epsilon$ is small enough.

    The inequality on the left is easy to obtain:
    for any $u\in\Z^d$,
    the event $A_{\Lambda+u}$
    also satisfies
    $\langle \ind{A\Delta A_{\Lambda+u}}\rangle=\langle \ind{A\Delta A_\Lambda}\rangle<\epsilon$.
    By the triangular inequality, we have
    \(
        \langle \ind{A_\Lambda\Delta A_{\Lambda+u}}\rangle
        <2\epsilon.
    \)

    For the inequality on the right, we use mixing:
    we get
    \[
        \langle \ind{A_\Lambda\Delta A_{\Lambda+u}}\rangle
        =
        \langle \ind{A_\Lambda}+\ind{A_{\Lambda+u}}
        -
        2 \ind{A_\Lambda}\ind{A_{\Lambda+u}}\rangle
        \to_{\|u\|_2\to\infty}
        p'+p'-2p'p'
        =2p'(1-p').
    \]
    We then simply choose $u$ such that $\|u\|_2$ is sufficiently large.
\end{proof}

\begin{theorem}
    Consider the Ising model on $\Z^d$ at inverse temperature $\beta$.
    \begin{itemize}
        \item \textbf{Wired boundary.}
        The measure $\langle\blank\rangle_{\Z^d,\beta}^+$ is mixing.
        \item \textbf{Free boundary.}
        The measure $\langle\blank\rangle_{\Z^d,\beta}^\f$ is mixing if
        $\lim_{\|y\|_2\to\infty}\langle\sigma_x\sigma_y\rangle_{\Z^d,\beta}^\f=0$.
    \end{itemize}
\end{theorem}

\begin{proof}[Proof for $\langle\blank\rangle^+$]
    It suffices to prove that for any finite sets $A,B\subset\Z^d$,
    we have
    \[
        \lim_{\|u\|_2\to\infty}
        \langle\sigma_A\sigma_{B+u}\rangle_{\Z^d,\beta}^+
        =
        \langle\sigma_A\rangle_{\Z^d,\beta}^+
        \langle\sigma_B\rangle_{\Z^d,\beta}^+.
    \]
    By the second Griffiths inequality,
    we get
    \[
        \langle\sigma_A\sigma_{B+u}\rangle_{\Z^d,\beta}^+
        \geq
        \langle\sigma_A\rangle_{\Z^d,\beta}^+
        \langle\sigma_B\rangle_{\Z^d,\beta}^+.
    \]
    It suffices to prove the other inequality.
    Consider an extremely large finite domain $\Lambda\supset A\cup B$.
    Define $\Lambda_u:=\Lambda\cup(\Lambda+u)$.
    If $\|u\|_2$ is sufficiently large,
    then $\Lambda$ is not connected to $\Lambda_u$
    in the set $\Lambda_u$, and so we get
    \[
        \limsup_{\|u\|_2\to\infty}
        \langle\sigma_A\sigma_{B+u}\rangle_{\Z^d,\beta}^+
        \leq
        \limsup_{\|u\|_2\to\infty}
        \langle\sigma_A\sigma_{B+u}\rangle_{\Lambda_u,\beta}^+
        =
        \langle\sigma_A\rangle_{\Lambda,\beta}^+
        \langle\sigma_B\rangle_{\Lambda,\beta}^+
        .
    \]
    Sending $\Lambda\uparrow\Z^d$ yields the desired inequality.
\end{proof}

\begin{proof}[Proof for $\langle\blank\rangle^\f$]
    As for $+$ boundary conditions,
    it suffices to fix two finite sets $A,B\subset\Z^d$,
    and prove that
    \[
        \limsup_{\|u\|_2\to\infty}
        \langle\sigma_A\sigma_{B+u}\rangle_{\Z^d,\beta}^\f
        \leq
        \langle\sigma_A\rangle_{\Z^d,\beta}^\f
        \langle\sigma_B\rangle_{\Z^d,\beta}^\f
        .
    \]
    Fix $u$, and
    consider an extremely large finite domain $\Lambda\supset A\cup (B+u)$.
    By the switching lemma, it is straightforward to deduce that
    \[
        \langle\sigma_A\sigma_{B+u}\rangle_{\Lambda,\beta}^\f
        \leq
        \langle\sigma_A\rangle_{\Lambda,\beta}^\f\langle\sigma_{B+u}\rangle_{\Lambda,\beta}^\f
        +\sum_{x\in A,\,y\in B}\langle\sigma_x\sigma_{y+u}\rangle_{\Lambda,\beta}^\f.
    \]
    By sending $\Lambda\uparrow\Z^d$, we get
    \[
        \langle\sigma_A\sigma_{B+u}\rangle_{\Z^d,\beta}^\f
        \leq
        \langle\sigma_A\rangle_{\Z^d,\beta}^\f\langle\sigma_B\rangle_{\Z^d,\beta}^\f
        +\sum_{x\in A,\,y\in B}\langle\sigma_x\sigma_{y+u}\rangle_{\Z^d,\beta}^\f.
    \]
    The error term vanishes by our additional assumption in the statement of the theorem.
\end{proof}
