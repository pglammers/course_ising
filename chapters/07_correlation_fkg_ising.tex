\section{The FKG inequality. Applications to the Ising spins}

\begin{lemma}
    Consider the Ising model on a finite graph $G$ at inverse
    temperature $\beta\geq 0$.
    Then the map
    \[
        \sigma\mapsto e^{-H_{G,\beta}(\sigma)}
    \]
    satisfies the FKG lattice condition.
    In particular, the Ising measure
    $\langle\blank\rangle_{G,\beta}$ satisfies the FKG inequality.
\end{lemma}

\begin{proof}
    Let $\sigma,\eta\in\Omega$ denote two spin configurations.
    It suffices to show that
    \[
        H(\sigma\vee\eta)+H(\sigma\wedge\eta)
        \leq
        H(\sigma)+H(\eta).
    \]
    This is immediate from the definition of the Hamiltonian:
    it is a sum of terms of the form
    \[
        -\beta\sigma_u\sigma_v,
    \]
    while these terms satisfy the obvious inequality
    \[
        (\sigma_u\vee \eta_u)(\sigma_v\vee \eta_v)
        +
        (\sigma_u\wedge \eta_u)(\sigma_v\wedge \eta_v)
        \geq
        \sigma_u\sigma_v+\eta_u\eta_v.
    \]
    This finishes the proof.
\end{proof}

We can now already prove the first Griffiths inequality for the case of
two vertices.

\begin{corollary}
    Let $G$ be a finite graph and let $\beta\geq 0$.
    Then the associated Ising model satisfies $\langle\sigma_u\sigma_v\rangle_{G,\beta}\geq 0$.
\end{corollary}

\begin{proof}
    Note that $\sigma_u$ and $\sigma_v$ are increasing functions
    of the spin configuration $\sigma$,
    while they have zero expectation due to flip-symmetry.
    The result then follows from the FKG inequality.
\end{proof}
