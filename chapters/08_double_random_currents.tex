\section{Double random currents}

The previous section explained how correlation functions can be expressed
in terms of correlation functions.
We also proved a basic result, namely the existence of a demagnetised phase
of the Ising model on graphs of bounded degree.

All results discussed so far, concern the behaviour of the Ising model
in the off-critical regime (very large values of $\beta$, very small values of $\beta$).
Our main interest is however in the \emph{critical point}:
the values for $\beta$ where the model undergoes a qualitative change,
such as values in the topological boundary of the set
\[
    \{\beta\in[0,\infty):\lim_{n\to\infty}\langle\sigma_u\rangle_{\Lambda_n,\beta}^+=0\}
\]
for a given infinite graph $G$ with a reference point $u$
(as per usual, $\Lambda_n$ refers to the graph metric ball around $u$).

We shall now introduce a new tool to study correlation functions
and random currents: the \emph{switching lemma}.
In recent years this tool has proved to be instrumental in the derivation
of rigorous results on the critical
behaviour of the Ising model, especially in dimensions $3$ and $4$.
The switching lemma may be stated in terms of a combinatorial language,
or in terms of an analytical language.
We shall start with the combinatorial perspective,
and perhaps develop the analytical perspective later.\todo{Decide what to do}

% \begin{lemma}[Switching lemma]
%     Let $G$ denote a finite graph.
%     Let $A,B,S\subset V$.
%     Let $\calE_S$ denote the set of currents 
% \end{lemma}
