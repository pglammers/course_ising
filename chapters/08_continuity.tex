\section{Continuity of the magnetisation in dimension $d\geq 3$}

The objective of this section is to prove the following deep theorem.

\begin{theorem}[Continuity in dimension $d\geq 3$]
    \label{thm:continuity}
    Consider the Ising model on the square lattice graph $G=\Z^d$
    in dimension $d\in\Z_{\geq 3}$.
    Then $m(\beta_c)=0$,
    that is, the magnetisation is continuous at $\beta=\beta_c$.
    Moreover, for $\beta\in[0,\beta_c]$,
    we have $\langle\blank\rangle^\f_{\Z^d,\beta}=\langle\blank\rangle^+_{\Z^d,\beta}$.
\end{theorem}

The proof presented here works only in dimension $d\geq 3$,
because we use an essential input called the \emph{infrared bound}.
The infrared bound is a classical tool in the analysis of spin systems.
Unfortunately, its proof is beyond the scope of these lecture notes.

\begin{theorem}[Infrared bound]
    Consider the Ising model on the square lattice graph $\Z^d$ for fixed $d\in\Z_{\geq 1}$.
    Then there exists a constant $C\in\R_{\geq 0}$ such that
    \[
        \langle\sigma_x\sigma_y\rangle_{\Z^d,\beta}^\f
        \leq C\frac1{\|y-x\|_2^{d-2}}
    \]
    for any $\beta\in[0,\beta_c]$.
    In particular, if $d\geq 3$,
    then
    \begin{equation}
        \label{eq:infrared_simple_decay}
        \lim_{\|y-x\|_2\to\infty}\langle\sigma_x\sigma_y\rangle_{\Z^d,\beta}^\f=0.
    \end{equation}    
\end{theorem}

Thus, we aim to prove that the infrared bound implies continuity (Theorem~\ref{thm:continuity}).
In fact, once we proved that $m(\beta_c)=0$,
it is quite easy to deduce the last part of Theorem~\ref{thm:continuity}.
We focus on proving that $m(\beta_c)=0$ for now.
Globally, the proof consists of the following two lemmas.

\begin{lemma}[Continuity, Step~1]
    Consider the Ising model on $\Z^d$ for $d\in\Z_{\geq 1}$ at $\beta\in[0,\infty)$.
    Then
    \[
        m(\beta)^2=\inf_{x,y}\langle\sigma_x\sigma_y\rangle_{\Z^d,\beta}^+.
    \]
\end{lemma}

\begin{lemma}[Continuity, Step~2]
    Consider the Ising model on $\Z^d$ for $d\in\Z_{\geq 3}$ at $\beta\in[0,\beta_c]$.
    Then
    \[
        \langle\sigma_x\sigma_y\rangle_{\Z^d,\beta}^+
        =
        \langle\sigma_x\sigma_y\rangle_{\Z^d,\beta}^\f.
    \]
    for any $x,y\in\Z^d$.
    More generally, for any subset $A\subset\Z^d$ of even cardinal, we have 
    \[
        \langle\sigma_A\rangle_{\Z^d,\beta}^+
        =
        \langle\sigma_A\rangle_{\Z^d,\beta}^\f.
    \]
\end{lemma}

\begin{proof}[Proof that the two steps imply Theorem~\ref{thm:continuity}]
    Assume the two lemmas.
    The infrared bound then tells us that at $\beta_c$ the two-point function tends
    to zero with the distance (for both free and wired boundary conditions, due to Step~2).
    Step~1 then tells us that the magnetisation vanishes.
    All odd correlation functions then vanish for $\langle\blank\rangle^+$ by Theorem~\ref{thm:vanishing_magnetisation}.
    The even correlation functions match those of $\langle\blank\rangle^\f$ by the last part of Step~2.
\end{proof}

Step~2 is the hard step; we start with a proof of Step~1.

\begin{proof}[Proof of Continuity, Step~1]
    Fix $x,y\in\Z^d$.
    For any finite domain $\Lambda\ni x,y$,
    we have
    \[
        \langle\sigma_x\rangle_{\Lambda,\beta}^+
        \langle\sigma_y\rangle_{\Lambda,\beta}^+
        =
        \langle\sigma_x\sigma_\frakg\rangle_{\Lambda^\frakg,\beta}
        \langle\sigma_y\sigma_\frakg\rangle_{\Lambda^\frakg,\beta}
        \leq
        \langle\sigma_x\sigma_y\rangle_{\Lambda,\beta}^+
    \]
    by the second Griffiths inequality.
    Sending $\Lambda\uparrow\Z^d$ yields
    \[
        m(\beta)^2\leq \langle\sigma_x\sigma_y\rangle_{\Z^d,\beta}^+.
    \]
    It suffices to prove the other bound.

    Fix $x=0\in\Z^d$, and let $\Lambda\ni x$ denote a large finite domain.
    For any $y\in\Z^d$, let $\Lambda_y:=\Lambda\cup(\Lambda+y)$.
    Then
    \[
        \limsup_{\|y\|_2\to\infty}\langle\sigma_x\sigma_y\rangle_{\Z^d,\beta}^+
        \leq
        \limsup_{\|y\|_2\to\infty}\langle\sigma_x\sigma_y\rangle_{\Lambda_y,\beta}^+
        =
        (\langle\sigma_x\rangle_{\Lambda,\beta}^+)^2
        \to_{\Lambda\uparrow\Z^d}m(\beta)^2.
    \]
    The equality holds true because for $\|y\|_2$ sufficiently large,
    $\Lambda$ and $\Lambda+y$ are no longer adjancent,
    and therefore the restrictions $\sigma|_\Lambda$
    and $\sigma|_{\Lambda+y}$ behave like independent Ising models.
\end{proof}

We now turn to the proof of Step~2. Fix $A\subset\Z^d$
with $|A|$ even.
We want to prove that
\begin{equation}
    \label{eq:continuity_target_limit}
    \lim_{\Lambda\uparrow\Z^d}
    \left(
    \langle\sigma_A\rangle_{\Lambda^\frakg,\beta}
    -\langle\sigma_A\rangle_{\Lambda^\f,\beta}
    \right)
    =
    0.
\end{equation}
In order to state a useful upper bound,
let us first introduce the probability measure
\[
    \P_{G}^A:=\frac{2^{|V|}}{Z_G \langle\sigma_A\rangle_G} \M_G[\{\partial\n=A\}\cap(\blank)].
\]

\begin{lemma}[Continuity, Step~2a]
    Fix $\Lambda\subset\Z^d$ finite, fix $\beta\in[0,\infty)$,
    and fix $A\subset\Lambda$ of even cardinal.
    Consider the random pair $(\n,\m)\sim \P_{\Lambda^\frakg,\beta}^A\times\P_{\Lambda^\f,\beta}^\emptyset$
    Then
    \begin{equation}
        \frac{
            \langle\sigma_A\rangle_{\Lambda^\frakg,\beta}
            -\langle\sigma_A\rangle_{\Lambda^\f,\beta}
        }{
            \langle\sigma_A\rangle_{\Lambda^\frakg,\beta}
        }
        =
        \P_{\Lambda^\frakg}^A\times\P_{\Lambda^\f}^\emptyset
        [\{(\widehat{\n+\m}\cap E(\Lambda^\f))\not\in\calE_A\}].
    \end{equation}
    The event on the right means that in order to pair up the vertices in $A$
    with edges in $\widehat{\n+\m}$,
    it is necessary to include a path through the ghost.
\end{lemma}

\begin{proof}
    We have
    \begin{multline}
        Z_{\Lambda^\frakg}Z_{\Lambda^\f}\langle\sigma_A\rangle_{\Lambda^\f,\beta}=
        2^{2|\Lambda|+1}\M_{\Lambda^\frakg}\times\M_{\Lambda^\f}[\{\partial\n=\emptyset,\,\partial\m= A\}]
        \\
        =
        2^{2|\Lambda|+1}\M_{\Lambda^\frakg}\times\M_{\Lambda^\f}[\{\partial\n=A,\,\partial\m= \emptyset\}\cap\{\widehat{\n+\m}\cap E(\Lambda^\f)\in\calE_A\}],
    \end{multline}
    where we use switching for the last equality.
    In other words,
    \[
        \langle\sigma_A\rangle_{\Lambda^\f,\beta}=
        \langle\sigma_A\rangle_{\Lambda^\frakg,\beta}\cdot
        \P_{\Lambda^\frakg}^A\times\P_{\Lambda^\f}^\emptyset
        [\{(\widehat{\n+\m}\cap E(\Lambda^\f))\in\calE_A\}].
    \]
    This implies the desired identity.
\end{proof}

In order to prove Step~2 (or equivalently, Equation~\eqref{eq:continuity_target_limit}),
we now use the previous lemma to make a number of reductions.
\begin{itemize}
    \item By the previous lemma,
    proving Equation~\eqref{eq:continuity_target_limit} is equivalent to proving that 
    \[
        \lim_{\Lambda\uparrow\Z^d}
        \P_{\Lambda^\frakg}^A\times\P_{\Lambda^\f}^\emptyset
            [\{(\widehat{\n+\m}\cap E(\Lambda^\f))\not\in\calE_A\}]=0.
    \]
    \item Let $B_n=[-n,n]^d\cap \Z^d$ denote an extremely large box
    containing $A$.
    If the ghost is required to pair up the vertices in $A$,
    then $B_n$ must be connected to the ghost.
    Therefore it suffices to prove that
    \begin{equation}
        \label{eq:targetinter}
        \lim_{\Lambda\uparrow\Z^d}
        \P_{\Lambda^\frakg}^A\times\P_{\Lambda^\f}^\emptyset
            [\{B_n \xleftrightarrow{\widehat{\n+\m}}\frakg\}]=0.
    \end{equation}
    \item The percolation measure $\P_{\Lambda^\frakg}^A$ may be viewed
    as a normalised version of \[\bfM_{\Lambda^\frakg}[\{\partial\omega=A\}\cap(\blank)]\times \P_p,\] see Equation~\eqref{eq:equivalence}.
    Fix $Q\subset E(B_n)$ such that $\partial Q=A$.
    Equation~\eqref{eq:XiQ} relates the measures $\bfM_{\Lambda^\frakg}^A$
    and $\bfM_{\Lambda^\frakg}^\emptyset$:
    the map $\Xi_Q$ acts as a pushforward map
    between the measures, up to a Radon--Nikodym derivative which is uniformly
    lower- and upper bounded.
    Since $\Xi_Q$ only modifies the edges in $E(B_n)$, it leaves the event \[\{B_n \xleftrightarrow{\widehat{\n+\m}}\frakg\}\] invariant.
    This implies that Equation~\eqref{eq:targetinter} is equivalent to 
    \begin{equation}
        \lim_{\Lambda\uparrow\Z^d}
        \P_{\Lambda^\frakg}^\emptyset\times\P_{\Lambda^\f}^\emptyset
            [\{B_n \xleftrightarrow{\widehat{\n+\m}}\frakg\}]=0.
    \end{equation}
    \item 
    By inclusion of events, it suffices to show that
    \begin{equation}
        \lim_{N\to\infty}\lim_{\Lambda\uparrow\Z^d}
        \P_{\Lambda^\frakg}^\emptyset\times\P_{\Lambda^\f}^\emptyset
            [\{B_n \xleftrightarrow{\widehat{\n+\m}}\partial B_N\}]=0.
    \end{equation}
    In fact, the event $\{B_n \xleftrightarrow{\widehat{\n+\m}}\partial B_N\}$
    is a \emph{local} event.
    We shall show below that the measures $\P_{\Lambda^\frakg}^\emptyset\times\P_{\Lambda^\f}^\emptyset$
    converge in the local convergence topology as $\Lambda\uparrow\Z^d$.
    Thus, if we write $\P^{\emptyset,\emptyset}$ for its limit, it then suffices
    to show that
    \[
        \lim_{N\to\infty}\P^{\emptyset,\emptyset}
            [\{B_n \xleftrightarrow{\widehat{\n+\m}}\partial B_N\}]=0.
    \]
    Since $n$ is arbitrary, this is equivalent to showing
    that $\widehat{\n+\m}$ does not percolate $\P^{\emptyset,\emptyset}$-almost surely.
\end{itemize}
 
Thus, as explained above, it suffices to prove the following two steps.

\begin{lemma}[Continuity, Step~2b]\label{lemma:continuity_step2b}
    Consider the Ising model on $\Z^d$ with $d\in\Z_{\geq 1}$
    and $\beta\in[0,\infty)$.
    Then the limit $\P^{\emptyset,\emptyset}:=
        \lim_{\Lambda\uparrow\Z^d}
        \P_{\Lambda^\frakg}^\emptyset\times\P_{\Lambda^\f}^\emptyset
        $ converges in the local convergence topology.
\end{lemma}

\begin{lemma}[Continuity, Step~2c]\label{lemma:continuity_step2c}
    Consider the Ising model on $\Z^d$ with $d\in\Z_{\geq 3}$
    and $\beta\in[0,\beta_c]$.
    Then $\widehat{\n+\m}$ does not percolate in the limit $\P^{\emptyset,\emptyset}$ constructed
    in Step~2b.
\end{lemma}

Step~2b is straightforward, and follows
from the convergence of the infinite-volume measures $\langle\blank\rangle^+_{\Z^d}$
and $\langle\blank\rangle^\f_{\Z^d}$.
For Step~2c we combine a simple version of the Burton--Keane argument
(not using ergodicity)
with the infrared bound.

\begin{exercise}[Proof of Continuity, Step~2b]
    We want to prove convergence
    of $\P_{\Lambda^\frakg}^\emptyset$ and $\P_{\Lambda^\f}^\emptyset$;
    the proof of convergence is the same for the two families,
    and we focus on the second.
    \begin{itemize}
        \item Argue that the sprinkling relation implies that it suffices to prove that the probability measures
        $\bfP_{\Lambda^\f}^\emptyset:\propto\bfM_{\Lambda^\f}[\{\partial\n=\emptyset\}\cap(\blank)]$
        converges in the local convergence topology.
        \item Prove that for a fixed edge $xy$,
        we have 
        \[
            \bfP_{\Lambda^\f}^\emptyset[\{\omega_{xy}=a\}]\propto 
            \langle
            (\sigma_x\sigma_y\tanh\beta)^a e^{-\beta\sigma_x\sigma_y}
            \rangle_{\Lambda^\f,\beta}
        \]
        as $a$ ranges over $\{0,1\}$.
        Conclude that this value converges as $\Lambda\uparrow\Z^d$.
        \item Let $Q\subset\E(\Z^d)$ finite.
        Find a local observable $X_\zeta$ for each $\zeta\in\{0,1\}^Q$ such that
        \[
            \bfP_{\Lambda^\f}[\{\omega|_Q=\zeta\}]\propto \langle X_Q\rangle_{\Lambda^\f}.
        \]
        Argue that the probabilities on the left converge as $\Lambda\uparrow\Z^d$.
    \end{itemize}
\end{exercise}


\begin{exercise}[Burton--Keane argument without ergodicity]
    The measure $\P^{\emptyset,\emptyset}$ is insertion tolerant (due to sprinkling)
    and shift-invariant.
    Let $N_\infty$ denote the number of infinite clusters of $\hat\n\cup\hat\m$.
    Prove that $\P^{\emptyset,\emptyset}[\{N_\infty>2\}]=0$
    by appealing to trifurcation boxes.
    Why can we not rule out that $\P^{\emptyset,\emptyset}[\{N_\infty=2\}]=0$
    at this stage?
\end{exercise}



\begin{proof}[Proof of Continuity, Step~2c]
    We want to prove that $p:=\P^{\emptyset,\emptyset}[\{0\xleftrightarrow{\widehat{\n+\m}}\infty\}]=0$.

    Define the random set $C_\infty:=\{x:x\xleftrightarrow{\widehat{\n+\m}}\infty\}$.
    Recall that $B_m:=[-m,m]^d\cap \Z^d$.
    Shift-invariance implies that
    \[
        \E^{\emptyset,\emptyset}[|C_\infty\cap B_m|^2]
        \geq
        \E^{\emptyset,\emptyset}[|C_\infty\cap B_m|]^2
        =
        p^2|B_m|^2.
    \]
    Since $N_\infty\leq 2$ almost surely,
    we have
    \[
        \E^{\emptyset,\emptyset}[|\{(x,y)\in B_m\times B_m:x\xleftrightarrow{\widehat{\n+\m}}y\}|]
        \geq
        \frac12 \E^{\emptyset,\emptyset}[|C_\infty\cap B_m|^2]
        \geq
        \frac12 p^2 |B_m|^2.
    \]

    By the switching lemma,
    \[
        \langle\sigma_x\sigma_y\rangle_{\Z^d,\beta}^\f
        \geq
        \langle\sigma_x\sigma_y\rangle_{\Z^d,\beta}^+
        \langle\sigma_x\sigma_y\rangle_{\Z^d,\beta}^\f
        \geq
        \P^{\emptyset,\emptyset}[\{x\xleftrightarrow{\widehat{\n+\m}}y\}].
    \]
    Thus, the inequality above yields
    \[
        \sum_{x,y\in B_m}\langle\sigma_x\sigma_y\rangle_{\Z^d,\beta}^\f
        \geq \frac12 p^2 |B_m|^2.
    \]
    By 
    the infrared bound (Equation~\eqref{eq:infrared_simple_decay}),
    the left side is of order $o(|B_m|^2)$
    as $m\to\infty$.
\end{proof}
