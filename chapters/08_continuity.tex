\section{Continuity of the magnetisation in dimension $d\geq 3$}

The objective of this section is to prove the following deep theorems.

\begin{theorem}[Continuity in dimension $d\geq 3$]
    \label{thm:continuity}
    Consider the Ising model on the square lattice graph $G=\Z^d$
    in dimension $d\in\Z_{\geq 3}$.
    Then $m(\beta_c)=0$,
    that is, the magnetisation is continuous at $\beta=\beta_c$.
    Moreover, for $\beta\in[0,\beta_c]$,
    we have $\langle\blank\rangle^\f_{\Z^d,\beta}=\langle\blank\rangle^+_{\Z^d,\beta}$.
\end{theorem}

The proof presented here works only in dimension $d\geq 3$,
because we use an essential input called the \emph{infrared bound}.
The infrared bound is a classical tool in the analysis of spin systems.
Unfortunately, its proof is beyond the scope of these lecture notes.

\begin{theorem}[Infrared bound]
    Consider the Ising model on the square lattice graph $\Z^d$ for fixed $d\in\Z_{\geq 1}$.
    Then there exists a constant $C\in\R_{\geq 0}$ such that
    \[
        \langle\sigma_x\sigma_y\rangle_{\Z^d,\beta}^\f
        \leq C\frac1{\|y-x\|_2^{d-2}}
    \]
    for any $\beta\in[0,\beta_c]$.
    In particular, if $d\geq 3$,
    then
    \begin{equation}
        \lim_{\|y-x\|_2\to\infty}\langle\sigma_x\sigma_y\rangle_{\Z^d,\beta}^\f=0.
    \end{equation}    
\end{theorem}

Thus, we aim to prove that the infrared bound implies continuity (Theorem~\ref{thm:continuity}).
In fact, once we proved that $m(\beta_c)=0$,
it is quite easy to deduce the last part of Theorem~\ref{thm:continuity}.
We focus on proving that $m(\beta_c)=0$ for now.
Globally, the proof consists of the following two lemmas.

\begin{lemma}[Continuity, Step~1]
    Consider the Ising model on $\Z^d$ for $d\in\Z_{\geq 1}$ at $\beta\in[0,\infty)$.
    Then
    \[
        m(\beta)^2=\inf_{x,y}\langle\sigma_x\sigma_y\rangle_{\Z^d,\beta}^+.
    \]
\end{lemma}

\begin{lemma}[Continuity, Step~2]
    Consider the Ising model on $\Z^d$ for $d\in\Z_{\geq 3}$ at $\beta\in[0,\beta_c]$.
    Then
    \[
        \langle\sigma_x\sigma_y\rangle_{\Z^d,\beta}^+
        =
        \langle\sigma_x\sigma_y\rangle_{\Z^d,\beta}^\f.
    \]
    for any $x,y\in\Z^d$.
    More generally, for any subset $A\subset\Z^d$ of even cardinal, we have 
    \[
        \langle\sigma_A\rangle_{\Z^d,\beta}^+
        =
        \langle\sigma_A\rangle_{\Z^d,\beta}^\f.
    \]
\end{lemma}

Suppose that we have proved these two lemmas.
The infrared bound then tells us that at $\beta_c$ the two point function tends
to zero with the distance (for both free and wired boundary conditions, due to Step~2).
Step~1 then tells us that the magnetisation vanishes.
Step~2 is the hard step; we start with a proof of Step~1.

\begin{proof}[Proof of Continuity, Step~1]
    Fix $x,y\in\Z^d$.
    For any finite domain $\Lambda\ni x,y$,
    we have
    \[
        \langle\sigma_x\rangle_{\Lambda,\beta}^+
        \langle\sigma_y\rangle_{\Lambda,\beta}^+
        =
        \langle\sigma_x\sigma_\frakg\rangle_{\Lambda^\frakg,\beta}
        \langle\sigma_y\sigma_\frakg\rangle_{\Lambda^\frakg,\beta}
        \leq
        \langle\sigma_x\sigma_y\rangle_{\Lambda,\beta}^+
    \]
    by the second Griffiths inequality.
    Sending $\Lambda\uparrow\Z^d$ yields
    \[
        m(\beta)^2\leq \langle\sigma_x\sigma_y\rangle_{\Z^d,\beta}^+.
    \]
    It suffices to prove the other bound.

    Fix $x=0\in\Z^d$, and let $\Lambda\ni x$ denote a large finite domain.
    For any $y\in\Z^d$, let $\Lambda_y:=\Lambda\cup(\Lambda+y)$.
    Then
    \[
        \limsup_{\|y\|_2\to\infty}\langle\sigma_x\sigma_y\rangle_{\Z^d,\beta}^+
        \leq
        \limsup_{\|y\|_2\to\infty}\langle\sigma_x\sigma_y\rangle_{\Lambda_y,\beta}^+
        =
        (\langle\sigma_x\rangle_{\Lambda,\beta}^+)^2
        \to_{\Lambda\uparrow\Z^d}m(\beta)^2.
    \]
    The equality holds true because for $\|y\|_2$ sufficiently large,
    $\Lambda$ and $\Lambda+y$ are no longer adjancent,
    and therefore the restrictions $\sigma|_\Lambda$
    and $\sigma|_{\Lambda+y}$ behave like independent Ising models.
\end{proof}



\begin{proof}[Proof of continuity in dimension $d\geq 3$]
    We have already seen that $m(\beta_c)=0$
    (and also $m(\beta)=0$ for $\beta\leq\beta_c$).
    It suffices to prove that $\langle\blank\rangle^+_{\Z^d,\beta}=\langle\blank\rangle^\f_{\Z^d,\beta}$.
    We shall prove that
    \[
        \langle\sigma_A\rangle^+_{\Z^d,\beta}=\langle\sigma_A\rangle^\f_{\Z^d,\beta}
    \]
    for any finite $A\subset\Z^d$.
    If $|A|$ is even then this also follows from Step~2.
    If $|A|$ is odd then we must simply show that $\langle\sigma_A\rangle^+_{\Z^d,\beta}=0$.
    For $|A|=1$ this is just the statement that $m(\beta)=0$.
    For $|A|>1$ we can deduce this from the pairing bound (Lemma~\ref{???}).
\end{proof}
