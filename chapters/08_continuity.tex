\section{Continuity of the magnetisation in dimension $d\geq 3$}

The objective of this section is to prove the following deep theorems.

\begin{theorem}[Continuity in dimension $d\geq 3$]
    \label{thm:continuity}
    Consider the Ising model on the square lattice graph $G=\Z^d$
    in dimension $d\in\Z_{\geq 3}$.
    Then $m(\beta_c)=0$,
    that is, the magnetisation is continuous at $\beta=\beta_c$.
    Moreover, for $\beta\in[0,\beta_c]$,
    we have $\langle\blank\rangle^\f_{\Z^d,\beta}=\langle\blank\rangle^+_{\Z^d,\beta}$.
\end{theorem}

The proof presented here works only in dimension $d\geq 3$,
because we use an essential input called the \emph{infrared bound}.
The infrared bound is a classical tool in the analysis of spin systems.
Unfortunately, its proof is beyond the scope of these lecture notes.

\begin{theorem}[Infrared bound]
    Consider the Ising model on the square lattice graph $\Z^d$ for fixed $d\in\Z_{\geq 1}$.
    Then there exists a constant $C\in\R_{\geq 0}$ such that
    \[
        \langle\sigma_x\sigma_y\rangle_{\Z^d,\beta}^\f
        \leq C\frac1{\|y-x\|_2^{d-2}}
    \]
    for any $\beta\in[0,\beta_c]$.
    In particular, if $d\geq 3$,
    then
    \begin{equation}
        \lim_{\|y-x\|_2\to\infty}\langle\sigma_x\sigma_y\rangle_{\Z^d,\beta}^\f=0.
    \end{equation}    
\end{theorem}

Thus, we aim to prove that the infrared bound implies continuity (Theorem~\ref{thm:continuity}).
In fact, once we proved that $m(\beta_c)=0$,
it is quite easy to deduce the last part of Theorem~\ref{thm:continuity}.
We focus on proving that $m(\beta_c)=0$ for now.
Globally, the proof consists of the following two lemmas.

\begin{lemma}[Continuity, Step~1]
    Consider the Ising model on $\Z^d$ for $d\in\Z_{\geq 1}$ at $\beta\in[0,\infty)$.
    Then
    \[
        m(\beta)^2=\inf_{x,y}\langle\sigma_x\sigma_y\rangle_{\Z^d,\beta}^+.
    \]
\end{lemma}

\begin{lemma}[Continuity, Step~2]
    Consider the Ising model on $\Z^d$ for $d\in\Z_{\geq 3}$ at $\beta\in[0,\beta_c]$.
    Then
    \[
        \langle\sigma_x\sigma_y\rangle_{\Z^d,\beta}^+
        =
        \langle\sigma_x\sigma_y\rangle_{\Z^d,\beta}^\f.
    \]
    for any $x,y\in\Z^d$.
    More generally, for any subset $A\subset\Z^d$ of even cardinal, we have 
    \[
        \langle\sigma_A\rangle_{\Z^d,\beta}^+
        =
        \langle\sigma_A\rangle_{\Z^d,\beta}^\f.
    \]
\end{lemma}

\begin{proof}[Proof that the two steps imply that $m(\beta_c)=0$]
    Suppose that we have proved these two lemmas.
    The infrared bound then tells us that at $\beta_c$ the two point function tends
    to zero with the distance (for both free and wired boundary conditions, due to Step~2).
    Step~1 then tells us that the magnetisation vanishes.    
\end{proof}

Step~2 is the hard step; we start with a proof of Step~1.

\begin{proof}[Proof of Continuity, Step~1]
    Fix $x,y\in\Z^d$.
    For any finite domain $\Lambda\ni x,y$,
    we have
    \[
        \langle\sigma_x\rangle_{\Lambda,\beta}^+
        \langle\sigma_y\rangle_{\Lambda,\beta}^+
        =
        \langle\sigma_x\sigma_\frakg\rangle_{\Lambda^\frakg,\beta}
        \langle\sigma_y\sigma_\frakg\rangle_{\Lambda^\frakg,\beta}
        \leq
        \langle\sigma_x\sigma_y\rangle_{\Lambda,\beta}^+
    \]
    by the second Griffiths inequality.
    Sending $\Lambda\uparrow\Z^d$ yields
    \[
        m(\beta)^2\leq \langle\sigma_x\sigma_y\rangle_{\Z^d,\beta}^+.
    \]
    It suffices to prove the other bound.

    Fix $x=0\in\Z^d$, and let $\Lambda\ni x$ denote a large finite domain.
    For any $y\in\Z^d$, let $\Lambda_y:=\Lambda\cup(\Lambda+y)$.
    Then
    \[
        \limsup_{\|y\|_2\to\infty}\langle\sigma_x\sigma_y\rangle_{\Z^d,\beta}^+
        \leq
        \limsup_{\|y\|_2\to\infty}\langle\sigma_x\sigma_y\rangle_{\Lambda_y,\beta}^+
        =
        (\langle\sigma_x\rangle_{\Lambda,\beta}^+)^2
        \to_{\Lambda\uparrow\Z^d}m(\beta)^2.
    \]
    The equality holds true because for $\|y\|_2$ sufficiently large,
    $\Lambda$ and $\Lambda+y$ are no longer adjancent,
    and therefore the restrictions $\sigma|_\Lambda$
    and $\sigma|_{\Lambda+y}$ behave like independent Ising models.
\end{proof}

We now turn to the proof of Step~2. Fix $A\subset\Z^d$
with $|A|$ even.
We want to prove that
\begin{equation}
    \label{eq:continuity_target_limit}
    \lim_{\Lambda\uparrow\infty}
    \left(
    \langle\sigma_A\rangle_{\Lambda^\frakg,\beta}
    -\langle\sigma_A\rangle_{\Lambda^\f,\beta}
    \right)
    =
    0.
\end{equation}
The switching lemma is not yet adapted to this setup,
since (until now) we only compared correlation functions on the same graph.
To state our new switching lemma, we introduce some new notations:
for any finite graph $G=(V,E)$ and any source set $A\subset V$,
we define the probability measure on random currents
\[
    \P_{G}^A:=\frac1{Z_G \langle\sigma_A\rangle_G} \M_G[\{\partial\n=A\}\cap(\blank)].
\]
If $G':=(V',E')$ is another finite graph,
then we view a current $\n\in(\Z_{\geq 0})^E$
also as a current on $E\cup E'$, by setting $\n_{xy}:=0$ for $xy\in E'\setminus E$,
and vice versa.

\begin{lemma}[Continuity, Step~2a]
    Fix $\Lambda\subset\Z^d$ finite, fix $\beta\in[0,\infty)$,
    and fix $A\subset\Lambda$ of even cardinal.
    Consider the random pair $(\n,\m)\sim \P_{\Lambda^\frakg,\beta}^A\times\P_{\Lambda^\f,\beta}^\emptyset$
    Then
    \begin{equation}
        \frac{
            \langle\sigma_A\rangle_{\Lambda^\frakg,\beta}
            -\langle\sigma_A\rangle_{\Lambda^\f,\beta}
        }{
            \langle\sigma_A\rangle_{\Lambda^\frakg,\beta}
        }
        =
        \P_{\Lambda^\frakg}^A\times\P_{\Lambda^\f}^\emptyset
        [\{(\widehat{\n+\m}\cap E(\Lambda^\f))\not\in\calE_A\}].
    \end{equation}
    The event on the right means that in order to pair up the vertices in $A$
    with edges in $\widehat{\n+\m}$,
    one must necessarily use the edges incident to the ghost.
\end{lemma}

\begin{proof}
    
\end{proof}

By the previous lemma,
it suffices to prove that
\[
    \lim_{\Lambda\uparrow\Z^d}
    \P_{\Lambda^\frakg}^A\times\P_{\Lambda^\f}^\emptyset
        [\{(\widehat{\n+\m}\cap E(\Lambda^\f))\not\in\calE_A\}]=0.
\]
But if the ghost is required to pair up the vertices in $A$,
then at least one of the vertices in $A$ must be connected to the ghost.
Therefore it suffices to prove that
\[
    \lim_{\Lambda\uparrow\Z^d}
    \P_{\Lambda^\frakg}^A\times\P_{\Lambda^\f}^\emptyset
        [\{A \xleftrightarrow{\widehat{\n+\m}}\frakg\}]=0.
\]

This is roughly proved as follows.
Imagine that we can somehow exchange the limit with the measure:
we first take a limit in the measures (in the local convergence topology),
then we evaluate the event.
Then the appropriate ``limit event'' should of course be the event that
$A$ intersects an infinite component of the percolation $\widehat{\n+\m}$.

Formally, we would like to say that
\begin{align}
    \lim_{\Lambda\uparrow\Z^d}
    \P_{\Lambda^\frakg}^A\times\P_{\Lambda^\f}^\emptyset
    [\{A \xleftrightarrow{\widehat{\n+\m}}\frakg\}]
    &
    \leq
    \lim_{\Delta\uparrow\Z^d}
    \left(
    \lim_{\Lambda\uparrow\Z^d}
    \P_{\Lambda^\frakg}^A\times\P_{\Lambda^\f}^\emptyset
    [\{A \xleftrightarrow{\widehat{\n+\m}}\partial\Delta\}]
    \right) 
    \\&
    =
    \lim_{\Delta\uparrow\Z^d}
    \left(
    \lim_{\Lambda\uparrow\Z^d}
    \P_{\Lambda^\frakg}^A\times\P_{\Lambda^\f}^\emptyset
    \right)
    [\{A \xleftrightarrow{\widehat{\n+\m}}\partial\Delta\}]
    \\&
    =
    \left(
    \lim_{\Lambda\uparrow\Z^d}
    \P_{\Lambda^\frakg}^A\times\P_{\Lambda^\f}^\emptyset
    \right)
    [\{A \xleftrightarrow{\widehat{\n+\m}}\infty\}]
    \\&\leq
    \left(
    \lim_{\Lambda\uparrow\Z^d}
    \P_{\Lambda^\frakg}^A\times\P_{\Lambda^\f}^\emptyset
    \right)
    [\{\text{$\widehat{\n+\m}$ percolates}\}]=0.
\end{align}
To justify these relations, we must verify two statements:
\begin{itemize}
    \item The limit $\P^{A,\emptyset}:=\left(
        \lim_{\Lambda\uparrow\Z^d}
        \P_{\Lambda^\frakg}^A\times\P_{\Lambda^\f}^\emptyset\right)
        $ converges in the local convergence topology,
    \item The currents $\widehat{\n+\m}$ do not percolate in this limit.
\end{itemize}

Both statements are nontrivial and require a proof.
The first statement is straightforward, and follows
from the convergence of the infinite-volume measures $\langle\blank\rangle^+_{\Z^d}$
and $\langle\blank\rangle^\f_{\Z^d}$.
For the second statement, we combine a simple version of the Burton--Keane argument
with the infrared bound.

\todo{Explain why we may work with $\P^{\emptyset,\emptyset}$}

\begin{lemma}[Continuity, Step~2b]\label{lemma:continuity_step2b}
    Consider the Ising model on $\Z^d$ with $d\in\Z_{\geq 1}$
    and $\beta\in[0,\infty)$.
    Then the limit $\P^{\emptyset,\emptyset}:=
        \lim_{\Lambda\uparrow\Z^d}
        \P_{\Lambda^\frakg}^\emptyset\times\P_{\Lambda^\f}^\emptyset
        $ converges in the local convergence topology.
\end{lemma}

\begin{lemma}[Continuity, Step~2c]\label{lemma:continuity_step2c}
    Consider the Ising model on $\Z^d$ with $d\in\Z_{\geq 3}$
    and $\beta\in[0,\beta_c]$.
    Then $\widehat{\n+\m}$ does not percolate in the limit $\P^{\emptyset,\emptyset}$ constructed
    in Step~2b.
\end{lemma}

\begin{proof}[Proof of Continuity, Step~2b]
    \todo{Write the proof}
\end{proof}

\begin{proof}[Proof of Continuity, Step~2c]
    Let $N_\infty$ denote the random number of infinite connected components of $\widehat{\n+\m}$.
    By a Burton--Keane argument using shift-invariance and insertion tolerance,
    we have $\P^{\emptyset,\emptyset}[\{N_\infty=\infty\}]=0$.
    Fix $N\in\Z_{\geq 1}$; our objective is to prove that
    \[
        \P^{\emptyset,\emptyset}[\{N_\infty= N\}]=:p=0.
    \]

    Suppose that $p>0$ in order to derive a contradiction.
    The idea is to use shift-invariance and the infra-red bound.
    Let
    \[
        \delta:=\P^{\emptyset,\emptyset}[\{0\xleftrightarrow{\widehat{\n+\m}}\infty\}|\{N_\infty= N\}]
    \]
    denote the (conditional) density of the infinite clusters.
    Let $\Lambda_n:=\{-n,\dots,n\}^d\subset\Z^d$.
    Then
    \[
        \P^{\emptyset,\emptyset}\left[\left\{|\{x\in\Lambda_n:x\leftrightarrow\infty\}|>\tfrac{\delta}2|\Lambda_n|\right\}\middle|\{N_\infty= N\}\right]
        \geq
        \frac\delta2.
    \]
    But if $|\{x\in\Lambda_n:x\leftrightarrow\infty\}|>\tfrac{\delta}2|\Lambda_n|$
    and there are $N$ infinite clusters,
    then $\Lambda_n$ contains at least $\tfrac{\delta}{2N}|\Lambda_n|$
    vertices which are all connected to one another.
    In particular,
    \[
        \E^{\emptyset,\emptyset}\left[|\{(x,y)\in\Lambda_n\times\Lambda_n:x\leftrightarrow y\}|\right]
        \geq
        p
        \frac{\delta^2}{4N^2}|\Lambda_n|^2.
    \]

    But clearly
    \[
        \langle\sigma_x\sigma_y\rangle_{\Z^d,\beta}^\f
        \geq
        \langle\sigma_x\sigma_y\rangle_{\Z^d,\beta}^+
        \langle\sigma_x\sigma_y\rangle_{\Z^d,\beta}^\f
        \geq
        \P^{\emptyset,\emptyset}[\{x\xleftrightarrow{\widehat{\n+\m}}y\}],
    \]
    and therefore the previous bound implies that
    \[
        \liminf_{n\to\infty}\frac1{|\Lambda_n|^2}\sum_{x,y\in\Lambda_n}
        \langle\sigma_x\sigma_y\rangle_{\Z^d,\beta}^\f
        \geq p
        \frac{\delta^2}{4N^2}>0.
    \]
    This contradicts the infrared bound, as desired.
\end{proof}


\begin{proof}[Proof that $\langle\blank\rangle_{\Z^d,\beta}^+=\langle\blank\rangle_{\Z^d,\beta}^\f$ for $\beta\leq\beta_c$]
    We shall prove that
    \[
        \langle\sigma_A\rangle^+_{\Z^d,\beta}=\langle\sigma_A\rangle^\f_{\Z^d,\beta}
    \]
    for any finite $A\subset\Z^d$.
    If $|A|$ is even then this also follows from Step~2.
    If $|A|$ is odd then we must simply show that $\langle\sigma_A\rangle^+_{\Z^d,\beta}=0$.
    For $|A|=1$ this is just the statement that $m(\beta)=0$.
    For $|A|>1$ we can deduce this from the pairing bound (Lemma~\ref{???}).
\end{proof}
