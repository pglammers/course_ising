\section{The thermodynamical limit: Wired boundary, demagnetisation}
\label{sec:vanishing_magnetisation}

Our next goal is to prove the following theorem.

\begin{theorem}[$+$ and $-$ boundary conditions coincide
    when the magnetisation vanishes] 
    \label{thm:vanishing_magnetisation}
    Let $G$ denote a connected locally finite graph,
    endowed with some reference vertex $u$.
    Then 
    \[
        \langle\blank\rangle_{G,\beta}^+=\langle\blank\rangle_{G,\beta}^-
        \qquad
        \iff 
        \qquad
        m_G(\beta)=0.
    \]
\end{theorem}

We prove this lemma in several steps.

\begin{lemma}
    For any $\beta\in\R_{>0}$ and $N\in\Z_{\geq 0}$,
    there exists a constant $C=C_{\beta,N}<\infty$ with the following property.
    Consider the Ising model at inverse temperature $\beta$
    on some finite graph $G$.
    Then for any $u,v,g\in V$,
    we have
    \[
        \langle \sigma_v\sigma_g\rangle \leq C \langle \sigma_u\sigma_g\rangle
    \]
    whenever $d_G(u,v)\leq N$.
\end{lemma}

\begin{proof}
    \todo{Add proof}
\end{proof}

\begin{lemma}
    Let $G$ denote a connected locally finite graph,
    endowed with some reference vertex $u$.
    Then 
    \[
        m_G(\beta) := \langle \sigma_u\rangle_{G,\beta}^+=0
        \qquad
        \iff
        \qquad
        \forall v\in V,\,
        \langle \sigma_v\rangle_{G,\beta}^+=0
        .
    \]
\end{lemma}

\begin{proof}
    Fix $\beta$ and $v\in V$.
    It suffices to find a constant $C<\infty$ such that
    \[
        \langle \sigma_v\rangle_{\Lambda,\beta}^+
        \leq C\langle \sigma_u\rangle_{\Lambda,\beta}^+
    \]
    for any domain $\Lambda$.
    But then we may simply apply the ghost trick and apply the previous lemma.
\end{proof}

\begin{exercise}[difficult]
    Consider the Ising model on a finite graph $G$ at inverse temperature $\beta$.
    Let $A\subset V$ denote any finite subset,
    and fix $u\in A$.
    Use the switching lemma to prove that
    \[
        \langle\sigma_A\rangle \leq
        \sum_{v\in A\setminus\{u\}}
        \langle\sigma_u\sigma_v\rangle
        \langle\sigma_{A\setminus\{u,v\}}\rangle.
    \]
    Hint: argue that
    \[
        \true{\n\in\calE_A}
        \leq
        \sum_{v\in A\setminus\{u\}}
        \true{\n\in\calE_{\{u,v\}}}
        \true{\n\in\calE_A}.
    \]
\end{exercise}


\begin{proof}[Proof of Theorem~\ref{thm:vanishing_magnetisation}]
    
\end{proof}


