\section{The thermodynamical limit: Wired boundary, demagnetisation}
\label{sec:vanishing_magnetisation}

Our next goal is to prove the following theorem.

\begin{theorem}[$+$ and $-$ boundary conditions coincide
    when the magnetisation vanishes] 
    \label{thm:vanishing_magnetisation}
    Let $G$ denote a connected locally finite graph,
    endowed with some reference vertex $u$.
    Then 
    \[
        \langle\blank\rangle_{G,\beta}^+=\langle\blank\rangle_{G,\beta}^-
        \qquad
        \iff 
        \qquad
        m_G(\beta)=0.
    \]
\end{theorem}

The theorem can be proved using the following bound.

\begin{exercise}[Pairing bound, difficult]
    Consider the Ising model on a finite graph $G$ at inverse temperature $\beta$.
    Let $A\subset V$ denote any finite subset,
    and fix $u\in A$.
    Use the switching lemma to prove that
    \[
        \langle\sigma_A\rangle \leq
        \sum_{v\in A\setminus\{u\}}
        \langle\sigma_u\sigma_v\rangle
        \langle\sigma_{A\setminus\{u,v\}}\rangle.
    \]
    Hint: argue that
    \[
        \true{\n\in\calE_A}
        \leq
        \sum_{v\in A\setminus\{u\}}
        \true{\n\in\calE_{\{u,v\}}}
        \true{\n\in\calE_A}.
    \]

    Conclude that
    \[
        \langle\sigma_A\rangle \leq
        \sum_{\pi}
        \prod_{\{u,v\}\in \pi}
        \langle\sigma_u\sigma_v\rangle
    \]
    where $\pi$ ranges over the \emph{pairings} of $A$,
    that is, the set of partitions of $A$ in which each member has two elements.
\end{exercise}


\begin{proof}[Proof of Theorem~\ref{thm:vanishing_magnetisation}]
    Notice that $\langle\blank\rangle_{G,\beta}^+$ and $\langle\blank\rangle_{G,\beta}^-$ are related
    by a global spin flip (the pushforward map corresponding to $\sigma\mapsto-\sigma$).
    Therefore all of the following are equivalent:
    \begin{itemize}
        \item $\langle\blank\rangle_{G,\beta}^+=\langle\blank\rangle_{G,\beta}^-$,
        \item $\langle\blank\rangle_{G,\beta}^+$ is invariant under the map $\sigma\mapsto-\sigma$,
        \item $\langle\sigma_A\rangle_{G,\beta}^+=0$ whenever $A\subset V$ has odd cardinal.
    \end{itemize}

    The implication ``$\implies$'' is now obvious, and we focus on  ``$\impliedby$''.
    Suppose that $m(\beta)=0$,
    that is, $\langle\sigma_u\rangle^+=0$ where $u$ is the reference vertex.
    Fix $A\subset V$ with $|A|$ odd.
    It suffices to prove that $\langle\sigma_A\rangle^+=0$.
    We shall in fact give \emph{two} proofs of this fact.
    In both proofs, we shall fix a sequence $(\Lambda_n)_n$ of increasing subsets of $V$ with $\cup_n\Lambda_n=V$.
    \begin{itemize}
        \item \emph{Proof~1, using the pairing bound.}
        For fixed $n$, we get
        \begin{align}
            \langle\sigma_A\rangle_{\Lambda_n}^+
            =\langle\sigma_{A\cup\{\Lambda_n^c\}}\rangle_{G_n'}
            &\leq
            \sum_{v\in A}\langle\sigma_{\{v,\Lambda_n^c\}}\rangle_{G_n'}
            \langle\sigma_{A\setminus\{v\}}\rangle_{G_n'}
            =\sum_{v\in A}\langle\sigma_v\rangle_{\Lambda_n}^+
            \langle\sigma_{A\setminus\{v\}}\rangle_{\Lambda_n}^+
            \\&\leq 
            \sum_{v\in A}\langle\sigma_v\rangle_{\Lambda_n}^+
            \to_{n\to\infty}0.
        \end{align}
        The first inequality is the pairing bound,
        the second the generic bound $\langle\sigma_{A\setminus\{v\}}\rangle_{\Lambda_n}^+\in[0,1]$,
        and the convergence follows from Exercise~\ref{exercise:Definition of critical beta does not depend on ref point}.

        \item \emph{Proof~2, using directly the high-temperature expansion.}
        We only consider $\beta>0$, otherwise the spins are independent fair coin flips,
        and the result is automatic.
        Recall Exercise~\ref{exercise:Log-Lipschitz property of correlation functions}.
        For fixed $n$, we get
        \[
            (\tanh\beta)^{d_{\operatorname{Transport}}(A,\{u\})}\cdot\langle\sigma_A\rangle_{\Lambda_n}^+
            \leq 
            \langle\sigma_u\rangle_{\Lambda_n}^+.
        \]
        On the left, the transport distance is calculated in the graph $G_n'$
        (the dependence on $n$ is implicit).
        As $n\to\infty$, this transport distance stabilises at the finite transport distance
        in the infinite graph $G$.
        Since the right hand side tends to zero with $n$,
        we know that the left hand side also tends to zero.
        Since the prefactor remains uniformly positive,
        we must have $\langle\sigma_A\rangle^+=0$.
        \qedhere
    \end{itemize}    
\end{proof}


