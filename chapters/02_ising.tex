\section{Ising's model and basic notions}
\label{sec:ising_1d}

The Curie--Weiss model succeeds at explaining Curie's observations qualitatively,
in terms of a competition between \emph{energy} and \emph{entropy}.
However, it does not take into account the \emph{geometry} of the atoms in the metal.
It would perhaps be more realistic to place the atoms on a Euclidean grid,
and let the interactions strength between two atoms depend on their distance.
In the simplest case, we could simply let each atom interact only with the atoms
closest to it. This is called the \emph{nearest-neighbour interaction}.
We mainly focus on this setup in these lecture notes.

Wilhelm Lenz challenged his doctoral student Ernst Ising
to solve this nearest-neighbour model for magnetism on the one-dimensional line graph $\Z$.
Lenz was not entirely precise when posing this question,
and it was Ising who first formulated a definition for the model under consideration.
The model is therefore called the \emph{Ising model} in his honour.
Ising solved the one-dimension Ising model, see Exercise~\ref{exercise:ising_1d} below.

\begin{definition}[Ising model]
    \label{def:ising_finite}
    The Ising model on a finite graph \( G = (V, E) \) with \emph{inverse temperature} \( \beta \in [0,\infty) \) is defined as follows.
    Let $\Omega:=\{\pm1\}^V$ denote the set of spin configurations on the vertices of the graph;
    a typical element of $\Omega$ is denoted by $\sigma=(\sigma_u)_{u\in V}$.
    Elements $\sigma\in\Omega$ are called \emph{spin configurations};
    elements $\sigma_u$ are called \emph{spins}.
    The \emph{energy} or \emph{Hamiltonian} of a spin configuration $\sigma$ is given by
    \[
        H_{G,\beta}^{\operatorname{Ising}}(\sigma) := -\beta \sum_{uv \in E} \sigma_u \sigma_v.
    \]
    We write $\mu_{G,\beta}^{\operatorname{Ising}}$ for the associated \emph{Boltzmann distribution} or \emph{Gibbs measure}:
    \[
        \mu_{G,\beta}^{\operatorname{Ising}}(\sigma) := \frac{1}{Z_{G,\beta}^{\operatorname{Ising}}} e^{-H^{\operatorname{Ising}}_{G,\beta}(\sigma)},
    \]
    where \(Z_{G,\beta}^{\operatorname{Ising}}\) is normalisation constant or \emph{partition function} defined by
    \[
        Z_{G,\beta}^{\operatorname{Ising}}:= \sum_{\sigma\in\Omega} e^{-H^{\operatorname{Ising}}_{G,\beta}(\sigma)}.
    \]
    We shall write $\langle\blank\rangle_{G,\beta}^{\operatorname{Ising}}$ for the expectation functional associated to this probability measure.
\end{definition}

\begin{remark}
    For brevity, we omit the superscript $\operatorname{Ising}$ in notations.
    We also occasionaly omit the reference to the graph $G$ or the inverse temperature $\beta$ when they are clear from the context.
\end{remark}

\begin{remark}
    We often prefer the notation $\langle X\rangle$ over $\mu[X]$ when taking the expectation of some random variable $X$,
    expect when considering \emph{conditional} expectations.
\end{remark}

\begin{remark}[Flip-symmetry]
    The Ising model is \emph{flip-symmetric} in the sense that the distribution of the spins is invariant under the transformation $\sigma\mapsto-\sigma$.
    This is because the Hamiltonian is invariant under this transformation.
\end{remark}

\begin{figure}
    \includegraphics{figures/1D_Ising.pdf}
    \caption{Ising proved that in the 1D Ising model with $+$ boundary conditions,
    the magnetisation of the middle site decays exponentially fast with the system size.}
\end{figure}

\begin{exercise}[Ising's 1924 result on the absence of a phase transition on $\Z$]
    \label{exercise:ising_1d}
    Consider the Ising model on the nearest-neighbour graph on the set $V_n:=\{0,1,\ldots,n\}$ for $n\in\Z_{\geq 1}$
    at inverse temperature $\beta\in[0,\infty)$.
    \begin{enumerate}[label=(\roman*)]
        \item Calculate $\langle\sigma_x\rangle$ for any $x\in V_n$.
        \item Calculate $\langle\sigma_x\sigma_y\rangle$ for any $x,y\in V_n$. Does this value depend on $n$?
        \item Use flip-symmetry to argue that $\langle\sigma_x\sigma_y\rangle=\mu[\sigma_y|\{\sigma_x=+1\}]$.
        \item \emph{Harder:} Calculate $\mu[\sigma_k|\{\sigma_0=\sigma_{2k}=+1\}]$, where $k\in\Z_{\geq 0}$ and $n=2k$.
    \end{enumerate}
    \emph{Hint.}
    Let $(X_{k})_{k\geq 0}$
    denote independent $\pm1$-valued coin flips in some probability measure $\P$,
    with $X_0$ a fair coin flip and $\P[\{X_k=+1\}]=e^\beta/2\cosh\beta$ for $k\geq 1$.
    Show that the distribution of $\sigma$ in the Ising model on $V_n$
    is the same as that of the random variable $(\sigma'_k)_{0\leq k\leq n}$ in $\P$,
    where $\sigma'_k:= X_0 X_1\cdots X_k$.
\end{exercise}

In the last question, we calculated the magnetisation of the middle site in a system of width $n$,
with fixed boundary conditions $+1$ imposed on both sides.
It turns out that it decays exponentially fast in $n$ for any fixed $\beta$.
Since the qualitative behaviour (exponential decay) is the same for all $\beta$,
Ising concluded that the model does not undergoe a phase transition in one dimension.
He conjectured that the same would be true in higher dimension.
Over the past one hundred years, our mathematical understanding of the Ising model has advanced considerably,
with the development of several techniques that are very different in nature.
We now know that Ising's conjecture is false in dimension two and higher,
and that the behaviour of the model changes significantly with the dimension
and with the precise value of the inverse temperature $\beta$.

% \begin{definition}[Correlation functions]
%     Consider $\Omega:=\{\pm\}^V$.
%     Then for any finite subset $A\subset V$,
%     we define $\sigma_A:\Omega\to\{\pm\},\,\sigma\mapsto\prod_{x\in A}\sigma_x$.
%     Its expectation $\langle\sigma_A\rangle$ in any probability measure $\langle\blank\rangle$
%     on $\Omega$ is called a \emph{correlation function}.
%     If $|A|=n$ then $\langle\sigma_A\rangle$ is also called an \emph{$n$-point correlation function}.
% \end{definition}


% \begin{exercise}[Flip-symmetry]
%     \label{exercise:flip-symmetry}
%     Consider the Ising model on a finite graph $G=(V,E)$.
%     \begin{itemize}
%         \item Prove that if $A\subset V$ contains an odd number of vertices,
%         then $\langle\sigma_A\rangle=0$.
%         \item Prove that if $A\subset V$ contains an odd number of vertices
%         and $x\in V$,
%         then \[\E[\sigma_A|\{\sigma_x=+\}]=\langle \sigma_{A}\sigma_x\rangle.\]
%         \item Prove that if $A\subset V$ contains an even number of vertices
%         and $x\in V$,
%         then \[\E[\sigma_A|\{\sigma_x=+\}]=\langle \sigma_{A}\rangle.\]
%     \end{itemize}
% \end{exercise}

% In practice, we are interested in the Ising model on finite portions of the square
% lattice $\Z^d$ endowed with nearest-neighbour connectivity.
% We now provide the definitions for this setup.

% \begin{definition}[Free boundary conditions]
%     Let $G=(V,E)$ denote a locally finite graph and $\Lambda\subset V$ a finite set.
%     Write $\Lambda^\f$ for the subgraph of $G$ induced by $\Lambda$.
%     Write $\langle\blank\rangle_{\Lambda,\beta}^\f:=\langle\blank\rangle_{\Lambda^\f,\beta}$ for the \emph{free-boundary
%     Ising model} in $\Lambda$ at inverse temperature $\beta\in[0,\infty)$.    
% \end{definition}

% \begin{definition}[Fixed boundary conditions]
%     Let $G=(V,E)$ denote a locally finite graph and $\Lambda\subset V$ a finite set.
%     Let $\partial\Lambda\subset V\setminus\Lambda$ denote the set of vertices adjacent to $\Lambda$.
%     Write $\bar\Lambda$ for the graph defined by
%     \[
%         V(\bar\Lambda):=\Lambda\cup\partial\Lambda;\qquad
%         E(\bar\Lambda):=\{\{x,y\}\in E:\{x,y\}\cap\Lambda\neq\emptyset\}.
%     \]
%     For any $\zeta\in\{\pm\}^{\partial\Lambda}$,
%     we shall write $\langle\blank\rangle_{\Lambda,\beta}^\zeta$
%     for the measure
%     \[
%         \langle\blank\rangle_{\Lambda,\beta}^\zeta:=\E_{\bar\Lambda,\beta}[\blank|\{\sigma|_{\partial\Lambda}=\zeta\}].
%     \]
%     This is called the \emph{fixed-boundary Ising model} with boundary conditions $\zeta$.
%     The boundary condition $\zeta\equiv\pm$ is of particular interest,
%     and it is denoted $\langle\blank\rangle_{\Lambda,\beta}^\pm$.
% \end{definition}

\section{Markov property}

We saw that the 1D Ising model is a Markov chain:
conditional on any fixed spin, the spins ``in the past''
are independent of the spins ``in the future''.
This property generalises to arbitrary graphs
and is called the \emph{Markov property} of the Ising model.
In tandem with the Markov property, it is useful to introduce the notion of \emph{boundary conditions}.

\begin{definition}[Boundary conditions, cf.~Figure~\ref{fig:triangular}]
    Let $G=(V,E)$ denote a locally finite graph and $\Lambda\subset V$ a finite set.
    Let $\partial\Lambda\subset V\setminus\Lambda$ denote the set of vertices adjacent to $\Lambda$.
    Write $\bar\Lambda$ for the graph defined by
    \[
        V(\bar\Lambda):=\Lambda\cup\partial\Lambda;\qquad
        E(\bar\Lambda):=\{xy\in E:xy\cap\Lambda\neq\emptyset\}.
    \]
    For any $\zeta\in\{\pm1\}^{\partial\Lambda}$,
    we shall write $\langle\blank\rangle_{\Lambda,\beta}^\zeta$
    for the measure
    \[
        \langle\blank\rangle_{\Lambda,\beta}^\zeta:=\mu_{\bar\Lambda,\beta}[\blank|\{\sigma|_{\partial\Lambda}=\zeta\}].
    \]
    This is called the \emph{fixed boundary Ising model} with boundary conditions $\zeta$.
    The boundary condition $\zeta\equiv+1$ is of particular interest,
    and it is denoted $\langle\blank\rangle_{\Lambda,\beta}^+$.
    We write $\langle\blank\rangle_{\Lambda,\beta}^-$ for the boundary condition $\zeta\equiv-1$.
\end{definition}


\begin{theorem}[Markov property of the Ising model]
    Let $G$ denote a finite simple graph,
    and consider the Ising model $\mu_{G,\beta}$ at inverse temperature $\beta\in[0,\infty)$.
    Let $\Lambda\subset V$ be a subset of the vertices of $G$,
    and let $\zeta\in\{\pm1\}^{V\setminus\Lambda}$ be a fixed spin configuration outside of $\Lambda$.
    Let $(\Lambda_i)_i$ denote the partition of $\Lambda$ into connected components.
    Then the conditional measure $\mu_{G,\beta}[\blank|\{\sigma|_{\Lambda^\complement}=\zeta\}]$ satisfies the following properties:
    \begin{enumerate}
        \item The family of random variables $(\sigma|_{\bar\Lambda_i})_i$ is independent,
        \item The law of $\sigma|_{\bar\Lambda_i}$ is $\langle\blank\rangle_{\Lambda_i,\beta}^{\zeta|_{\partial\Lambda_i}}$.
    \end{enumerate}
\end{theorem}

\begin{proof}
    Let $C$ be the constant defined by
    \[
        C:=-\beta\sum_{xy\in E,\,xy\subset\Lambda^\complement}\zeta_x\zeta_y.
    \]
    Then we can decompose the Hamiltonian for any $\sigma$ with $\sigma|_{\Lambda^\complement}=\zeta$ as
    \[
        H_{G,\beta}(\sigma)=C+\sum_i H_{\bar\Lambda_i,\beta}(\sigma),
    \]
    where each $H_{\bar\Lambda_i,\beta}$ is measurable in terms of $\zeta|_{\partial\Lambda_i}$ and $\sigma|_{\bar\Lambda_i}$.
    This decomposition implies that
    \[
        \mu_{G,\beta}[\blank|\{\sigma|_{\Lambda^\complement}=\zeta\}]
        =\bigotimes_i \langle\blank\rangle_{\Lambda_i,\beta}^{\zeta|_{\partial\Lambda_i}}.
    \]
    This implies the desired properties.
\end{proof}

\begin{remark}[Conditioning on multiple events]
    Recall that when conditioning on multiple events,
    the order of the conditioning does not matter
    (the only thing that matters is that the intersection of the events has positive probability).
    For example, if we take $\langle\blank\rangle_{\Lambda,\beta}^+$,
    and condition on the state of more spins within $\Lambda$,
    then this yields a new fixed boundary condition in a smaller domain.
\end{remark}

\begin{exercise}
    Fix $\beta\in[0,\infty)$
    and let $G=(V,E)$ denote a locally finite simple graph.
    Let $\Lambda\subset\Delta\subset V$ denote two finite subsets.
    Work out that $\langle\blank\rangle_{\Lambda,\beta}^+$ is the same as
    $\mu_{\Delta,\beta}[\blank|\{\sigma|_{\Lambda^\complement}\equiv+1\}]$,
    by going back to first principles (the definition of the Ising measure).
\end{exercise}

