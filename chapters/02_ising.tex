\section{Ising's model and basic notions}
\label{sec:ising_1d}

While the competition between entropy and energy is transparent in the Curie--Weiss model,
the model does not encode any kind of geometry.
Indeed, all atoms interact equally with all other atoms.
It would perhaps be more realistic to place the atoms on a Euclidean grid,
and let the interactions strength between two atoms depend on their distance.
In the simplest case, we could simply let each atom interact only with the atoms
closest to it. This is called the \emph{nearest-neighbour interaction}.
We mainly focus on this setup in these lecture notes.

Wilhelm Lenz challenged his doctoral student Ernst Ising
to solve this nearest-neighbour model for magnetism on the one-dimensional line graph $\Z$.
Lenz was not entirely precise when posing this question,
and it was Ising who first formulated a definition for the model under consideration.
The model is therefore called the \emph{Ising model} in his honour.
We shall later address Ising's analysis in an exercise (Exercise~\ref{}\todo{Add exercise}).

\begin{definition}[Ising model]
    \label{def:ising_finite}
    The Ising model on a finite graph \( G = (V, E) \) with \emph{inverse temperature} \( \beta \in [0,\infty) \) is defined as follows.
    Let $\Omega:=\{\pm1\}^V$ denote the set of spin configurations on the vertices of the graph;
    a typical element of $\Omega$ is denoted by $\sigma=(\sigma_u)_{u\in V}$.
    Elements $\sigma\in\Omega$ are called \emph{spin configurations};
    elements $\sigma_u$ are called \emph{spins}.
    The \emph{energy} or \emph{Hamiltonian} of a spin configuration $\sigma$ is given by
    \[
        H_{G,\beta}^{\operatorname{Ising}}(\sigma) := -\beta \sum_{uv \in E} \sigma_u \sigma_v.
    \]
    We write $\P_{G,\beta}^{\operatorname{Ising}}$ for the associated \emph{Boltzmann distribution} or \emph{Gibbs measure}:
    \[
        \P_{G,\beta}^{\operatorname{Ising}}(\sigma) := \frac{1}{Z_{G,\beta}^{\operatorname{Ising}}} e^{-H^{\operatorname{Ising}}_{G,\beta}(\sigma)},
    \]
    where \(Z_{G,\beta}^{\operatorname{Ising}}\) is normalisation constant or \emph{partition function} defined by
    \[
        Z_{G,\beta}^{\operatorname{Ising}}:= \sum_{\sigma\in\Omega} e^{-H^{\operatorname{Ising}}_{G,\beta}(\sigma)}.
    \]
    We shall write $\langle\blank\rangle_{G,\beta}^{\operatorname{Ising}}$ for the expectation functional associated to this probability measure.
\end{definition}

\begin{remark*}
    We shall often suppress subscripts and superscripts when they are clear from the context.
\end{remark*}

\begin{remark*}
    The mathematical community has widely adopted the terminology coming from the physics
    literature.
    We often prefer the symbol $\langle\blank\rangle$ over $\E[\blank]$ when taking expecations,
    expect when considering \emph{conditional} expectations.
\end{remark*}

\begin{exercise}[The edge graph]
    Consider the Ising model on the complete graph on the two vertices $V:=\{x,y\}$
    at inverse temperature $\beta\in[0,\infty)$.
    \begin{itemize}
        \item Calculate $\langle\sigma_x\rangle_\beta$.
        \item Calculate $\langle\sigma_x\sigma_y\rangle_\beta$.
    \end{itemize}
\end{exercise}

\begin{definition}[Correlation functions]
    Consider $\Omega:=\{\pm\}^V$.
    Then for any finite subset $A\subset V$,
    we define $\sigma_A:\Omega\to\{\pm\},\,\sigma\mapsto\prod_{x\in A}\sigma_x$.
    Its expectation $\langle\sigma_A\rangle$ in any probability measure $\langle\blank\rangle$
    on $\Omega$ is called a \emph{correlation function}.
    If $|A|=n$ then $\langle\sigma_A\rangle$ is also called an \emph{$n$-point correlation function}.
\end{definition}

\begin{remark*}[Flip-symmetry]
    The Ising model is \emph{flip-symmetric} in the sense that the distribution of the spins is invariant under the transformation $\sigma\mapsto-\sigma$.
    This is because the Hamiltonian is invariant under this transformation.
\end{remark*}

\begin{exercise}[Flip-symmetry]
    \label{exercise:flip-symmetry}
    Consider the Ising model on a finite graph $G=(V,E)$.
    \begin{itemize}
        \item Prove that if $A\subset V$ contains an odd number of vertices,
        then $\langle\sigma_A\rangle=0$.
        \item Prove that if $A\subset V$ contains an odd number of vertices
        and $x\in V$,
        then \[\E[\sigma_A|\{\sigma_x=+\}]=\langle \sigma_{A}\sigma_x\rangle.\]
        \item Prove that if $A\subset V$ contains an even number of vertices
        and $x\in V$,
        then \[\E[\sigma_A|\{\sigma_x=+\}]=\langle \sigma_{A}\rangle.\]
    \end{itemize}
\end{exercise}

In practice, we are interested in the Ising model on finite portions of the square
lattice $\Z^d$ endowed with nearest-neighbour connectivity.
We now provide the definitions for this setup.

\begin{definition}[Free boundary conditions]
    Let $G=(V,E)$ denote a locally finite graph and $\Lambda\subset V$ a finite set.
    Write $\Lambda^\f$ for the subgraph of $G$ induced by $\Lambda$.
    Write $\langle\blank\rangle_{\Lambda,\beta}^\f:=\langle\blank\rangle_{\Lambda^\f,\beta}$ for the \emph{free-boundary
    Ising model} in $\Lambda$ at inverse temperature $\beta\in[0,\infty)$.    
\end{definition}

\begin{definition}[Fixed boundary conditions]
    Let $G=(V,E)$ denote a locally finite graph and $\Lambda\subset V$ a finite set.
    Let $\partial\Lambda\subset V\setminus\Lambda$ denote the set of vertices adjacent to $\Lambda$.
    Write $\bar\Lambda$ for the graph defined by
    \[
        V(\bar\Lambda):=\Lambda\cup\partial\Lambda;\qquad
        E(\bar\Lambda):=\{\{x,y\}\in E:\{x,y\}\cap\Lambda\neq\emptyset\}.
    \]
    For any $\zeta\in\{\pm\}^{\partial\Lambda}$,
    we shall write $\langle\blank\rangle_{\Lambda,\beta}^\zeta$
    for the measure
    \[
        \langle\blank\rangle_{\Lambda,\beta}^\zeta:=\E_{\bar\Lambda,\beta}[\blank|\{\sigma|_{\partial\Lambda}=\zeta\}].
    \]
    This is called the \emph{fixed-boundary Ising model} with boundary conditions $\zeta$.
    The boundary condition $\zeta\equiv\pm$ is of particular interest,
    and it is denoted $\langle\blank\rangle_{\Lambda,\beta}^\pm$.
\end{definition}

\begin{exercise}[Markov property]
    Consider the Ising model on some finite graph $G=(V,E)$ at inverse temperature $\beta$.
    Fix some $\Lambda\subset V$ and let $(\Lambda_i)_i$ denote the partition of $\Lambda$ into connected components.
    Let $\zeta\in\{\pm\}^{\Lambda^c}$,
    and consider the conditional probability measure
    $\P[\blank|\{\sigma|_{\Lambda^c}=\zeta\}]$.
    \begin{itemize}
        \item Prove that $(\sigma|_{\Lambda_i})_i$ is a family of independent random variables in this measure.
        \item Prove that the law of $\sigma|_{\Lambda_i}$ is $\langle\blank\rangle_{\Lambda_i}^{\zeta|_{\partial\Lambda_i}}$.
    \end{itemize}
    \emph{Hint.}
    Decompose the Hamiltonian according to $H(\sigma)=C+\sum_i H_i(\sigma)$,
    where each $H_i$ is measurable in terms of $\zeta|_{\partial\Lambda_i}$ and $\sigma|_{\Lambda_i}$.
\end{exercise}

Ising proved that in one dimension, the Ising model exhibits exponential decay of correlations at all temperatures.
In other words, there is no phase transition.
We now state his result, without a proof.
While the proof is quite straightforward even with elementary methods,
its proof becomes entirely trivial after the introduction of more recent methods.

\begin{theorem}[Ising, 1924]
    \label{thm:Ising}
    Consider the finite domains $\Lambda_n:=\{-n,\dots,n\}$ of the graph $\Z$.
    Then for any $\beta\in[0,\infty)$,
    there exists a constant $c=c_\beta>0$ such that
    \[
        \langle\sigma_0\rangle_{\Lambda_n,\beta}^+
        \leq \frac1ce^{-c_\beta\cdot n}.
    \]
\end{theorem}

Unfortunately, Ising wrongly conjectured that the same would be true in higher dimension.
Disappointed with this prediction, he left academia.
