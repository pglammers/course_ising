\section{Infinite-volume limits}
\label{sec:vanishing_magnetisation}

\begin{definition}[Infinite-volume limit]
    Let $G=(V,E)$ denote a locally finite graph.
    Write
    \[
        \lim_{\Lambda\uparrow V}f(\Lambda)
        \qquad\text{for}\qquad
        \lim_{n\to\infty}f(\Lambda_n),
    \]
where $(\Lambda_n)_n$ is any increasing sequence of finite domains 
with $\cup_n\Lambda_n=V$.
This notation makes sense only when the limit is independent
of the precise choice of the sequence $(\Lambda_n)_n$,
and is called the \emph{thermodynamical limit} or \emph{infinite-volume limit}.

Let $(\Omega,\calF)$ denote the measurable space $\Omega:=\{\pm1\}^V$
endowed with the product $\sigma$-algebra.
For a domain $\Lambda$, we write $\calF_\Lambda$
for the $\sigma$-algebra generated by spins in $\Lambda$.
An observable $X:\Omega\to\C$ is called \emph{local} if it is measurable
with respect to $\calF_\Lambda$ for some domain $\Lambda$.

Let $\calP(\Omega,\calF)$ denote the set of all probability measures
on this measurable space.
We endow this set with the \emph{local convergence topology},
which is defined as the topology making the map
\[
    \calP(\Omega,\calF)\to\C,\,\mu\mapsto\mu[X]
\]
continuous for any local observable $X$.
\end{definition}

\begin{remark*}
    This topology is sometimes known under different names in the literature,
    such as the \emph{weak topology}.
    The name \emph{local convergence topology} is quite explicit:
    if the statistics of the measures within a fixed domain $\Lambda$ converge,
    then we have local convergence.
\end{remark*}

\begin{exercise}
    Prove that $\calP(\Omega,\calF)$ is a compact space in this topology.
\end{exercise}

\begin{theorem}[Existence of the thermodynamical limit]
    Consider the Ising model on a locally finite graph $G$
    at inverse temperature $\beta$.
    Then there exists unique probability measures
    $\langle\blank\rangle_{G,\beta}^\f,\langle\blank\rangle^+_{G,\beta}\in\calP(\Omega,\calF)$
    such that
    \[
        \lim_{\Lambda\uparrow V}\langle X\rangle_{\Lambda,\beta}^*=\langle X\rangle_{G,\beta}^*
    \]
    for $*\in\{\f,+\}$ and
    for any local observable $X:\Omega\to\R$.
    In other words,
    \[
        \lim_{\Lambda\uparrow V}\langle \blank\rangle_{\Lambda,\beta}^*
        =:
        \langle \blank\rangle_{G,\beta}^*.
    \]
    The measures $\langle\blank\rangle_{G,\beta}^*$ are called
    the \emph{thermodynamical limits} or \emph{infinite-volume limits}.
\end{theorem}

\begin{proof}
    Any local observable may be written as a finite linear combination
    of observables of the form $\sigma_A$ where $A$ is a finite subset of
    $V$.
    The theorem then follows by compactness and Lemma~\ref{lemma:correlation_functions_monotone_both}.
\end{proof}

\begin{definition}[Shift operator]
    Let $G=\Z^d$.
    Consider a measure $\langle\blank\rangle\in\calP(\Omega,\calF)$.
    \begin{itemize}
        \item For any $u\in\Z^d$, we define the \emph{shift operator} $\tau_u:\Omega\to\Omega$ by
        \[
            (\tau_u\sigma)_x = \sigma_{x-u}.
        \]
        An event $A$ is \emph{shift-invariant} if $\tau_uA:=\{\tau_u\sigma:\sigma\in A\}$ for any $u\in\Z^d$.
        \item The measure is called \emph{shift-invariant} if
        \[
            \langle X\circ\tau_u \rangle = \langle X\rangle
        \]
        for any vertex $u\in\Z^d$ and
        for any bounded local observable $X$.
    \end{itemize}
\end{definition}

\begin{theorem}[Shift-invariance]
    Let $G=\Z^d$.
    The measures $\langle\blank\rangle^\f_{\Z^d,\beta}$ and $\langle\blank\rangle^+_{\Z^d,\beta}$
    are shift-invariant.
\end{theorem}

\begin{proof}
    The desired symmetry simply follows from the symmetry in the definitions.
\end{proof}


\begin{exercise}[Continuity properties in $\beta$]
    Consider the Ising model on a locally finite graph $G=(V,E)$.
    Fix $A\subset V$ finite.
    \begin{itemize}
        \item The function $\beta\mapsto \langle\sigma_A\rangle_{G,\beta}^*$ is non-decreasing for $*\in\{\f,+\}$.
        \item The function $\beta\mapsto \langle\sigma_A\rangle_{G,\beta}^\f$ is left continuous.
        \item The function $\beta\mapsto \langle\sigma_A\rangle_{G,\beta}^+$ is right continuous.
    \end{itemize}
    \emph{Hint.} Argue that $\beta\mapsto \langle\sigma_A\rangle_{G,\beta}^\f$
    is a limit of a non-decreasing sequence of non-decreasing functions.
\end{exercise}

\begin{definition}[Magnetisation and critical temperature]
    Let $G$ be a vertex-transitive locally finite graph.
    The non-decreasing right-continuous function
    \[
        m=m_G:[0,\infty)\to\R,\,\beta\mapsto\langle\sigma_u\rangle_{G,\beta}^+
    \]
    is called the \emph{magnetisation} ($u$ is an arbitrary reference vertex).

    The \emph{critical (inverse) temperature} is defined via
    \[
        \beta_c:=\beta_c(G):=\inf\{\beta\in[0,\infty):m(\beta)>0\}.
    \]
\end{definition}

We have already proved that $\beta_c\in(0,\infty)$ for $G=\Z^d$
in dimension $d\geq 2$,
and that $\beta_c=\infty$ for $G=\Z$.

It is easy to derive the following result when the $m(\beta)=0$.

\begin{theorem}[$+$ and $-$ boundary conditions coincide
    when the magnetisation vanishes] 
    \label{thm:vanishing_magnetisation}
    Let $G$ denote a connected locally finite graph,
    endowed with some reference vertex $u$.
    Then 
    \[
        \langle\blank\rangle_{G,\beta}^+=\langle\blank\rangle_{G,\beta}^-
        \qquad
        \iff 
        \qquad
        m_G(\beta)=0.
    \]
\end{theorem}



\begin{proof}
    Notice that $\langle\blank\rangle_{G,\beta}^+$ and $\langle\blank\rangle_{G,\beta}^-$ are related
    by a global spin flip (the pushforward map corresponding to $\sigma\mapsto-\sigma$).
    Therefore all of the following are equivalent:
    \begin{itemize}
        \item $\langle\blank\rangle_{G,\beta}^+=\langle\blank\rangle_{G,\beta}^-$,
        \item $\langle\blank\rangle_{G,\beta}^+$ is invariant under the map $\sigma\mapsto-\sigma$,
        \item $\langle\sigma_A\rangle_{G,\beta}^+=0$ whenever $A\subset V$ has odd cardinal.
    \end{itemize}

    The implication ``$\implies$'' is now obvious, and we focus on  ``$\impliedby$''.
    Suppose that $m(\beta)=0$,
    that is, $\langle\sigma_x\rangle^+=0$ for any $x\in V$.
    Fix $A\subset V$ with $|A|$ odd.
    It suffices to prove that $\langle\sigma_A\rangle^+=0$.
    But we simply observe that
    \[
        \langle\sigma_A\rangle^+
        =
        \lim_{\Lambda\uparrow V}
        \langle\sigma_A\sigma_\frakg\rangle_{\Lambda^\frakg}
        \leq
        \lim_{\Lambda\uparrow V}
        \sum_{x\in A}
        \langle\sigma_{A\setminus\{x\}}\rangle_{\Lambda^\frakg}
        \langle\sigma_x\sigma_\frakg\rangle_{\Lambda^\frakg}
        \leq
        \sum_{x\in A}
        \langle\sigma_x\rangle_{\Lambda}^+= |A|\cdot m(\beta)=0.
    \]
    The first inequality is the pairing bound (Theorem~\ref{thm:pairing}).
\end{proof}

\todo{Add: Messager--Miracle-Solé}
