\section{Introduction}
\label{sec:intro}

The Ising model is the archetypal model for the study of phase transitions in
mathematical physics.
It was first introduced by Wilhelm Lenz in 1920 and later solved by Ernst Ising
in 1924 in the one-dimensional case.
The model consists of a lattice of spins,
each of which can be in one of two states,
up or down.
Informally,
one may think of these spins as the magnetic moments of atoms in a ferromagnetic material.
The behaviour of this probabilistic model depends strongly on a few different parameters:
\begin{itemize}
    \item The dimension $d$ of the lattice $\Z^d$ on which the spins are placed,
    \item The interaction strength $\beta$ between neighbouring spins,
    \item The way that boundary conditions are imposed,
    \item The strength of external magnetic field $h$.
\end{itemize}
In fact, we shall start by defining the Ising model on arbitrary finite graphs.
We shall now give a definition of the Ising model, although
we keep boundary conditions and external magnetic fields for later.

\begin{definition}[Ising model on a finite graph]
    \label{def:ising_finite}
    The Ising model on a finite graph \( G = (V, E) \) with \emph{inverse temperature} \( \beta \in [0,\infty) \) is defined as follows.
    Let $\Omega:=\{\pm1\}^V$ denote the set of spin configurations on the vertices of the graph;
    a typical element of $\Omega$ is denoted by $\sigma=(\sigma_u)_{u\in V}$.
    Elements $\sigma\in\Omega$ are called \emph{spin configurations};
    elements $\sigma_u$ are called \emph{spins}.
    The \emph{energy} or \emph{Hamiltonian} of a spin configuration $\sigma$ is given by
    \[
        H_{G,\beta}^{\operatorname{Ising}}(\sigma) := -\beta \sum_{uv \in E} \sigma_u \sigma_v.
    \]
    We write $\P_{G,\beta}^{\operatorname{Ising}}$ for the associated \emph{Boltzmann distribution} or \emph{Gibbs measure}:
    \[
        \P_{G,\beta}^{\operatorname{Ising}}(\sigma) := \frac{1}{Z_{G,\beta}^{\operatorname{Ising}}} e^{-H^{\operatorname{Ising}}_{G,\beta}(\sigma)},
    \]
    where \(Z_{G,\beta}^{\operatorname{Ising}}\) is normalisation constant or \emph{partition function} defined by
    \[
        Z_{G,\beta}^{\operatorname{Ising}}:= \sum_{\sigma\in\Omega} e^{-H^{\operatorname{Ising}}_{G,\beta}(\sigma)}.
    \]
    We shall write $\langle\blank\rangle_{G,\beta}^{\operatorname{Ising}}$ for the expectation functional associated to this probability measure.
\end{definition}

\begin{remark}[Flip-symmetry]
    The Ising model is \emph{flip-symmetric} in the sense that the distribution of the spins is invariant under the transformation $\sigma\mapsto-\sigma$.
    This is because the Hamiltonian is invariant under this transformation.
\end{remark}

\begin{remark}
    We shall often suppress subscripts and superscripts when they are clear from the context.
\end{remark}

\begin{remark}
    Adding a constant to the Hamiltonian does not change the distribution of the Ising model,
    even though it affects the partition function.
\end{remark}

\begin{remark}
    The mathematical community has widely adopted the terminology coming from the physics
    literature.
\end{remark}
