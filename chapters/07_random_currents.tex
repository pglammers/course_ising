\section{Random currents}

\begin{definition}[Currents]
    Let $G=(V,E)$ denote a graph.
    A \emph{current} is a map $\n:E\to\Z_{\geq 0}$.
    We think of $(V,\n)$ as a multigraph,
    where for each edge $uv\in E$ we have $\n_{uv}$ multi-edges between $u$ and $v$.
    The set of \emph{sources} $\partial\n\subset V$ of a current
    $\n$ is defined as the set of vertices $u\in V$ with an odd degree in the multigraph.
    If $G$ is finite and $\beta\in[0,\infty)$, then the
    \emph{weight} of a current is defined as
    \[
        w_\beta(\n):=\prod_{xy\in E}
        \frac{\beta^{\n_{xy}}}{\n_{xy}!}.
    \]
\end{definition}

Currents can be used to encode correlation functions of the Ising model.

\begin{theorem}[Current representation of correlation functions]
    \label{thm:current_representation_of_correlation_functions}
    Consider the Ising model on a finite graph $G$ at inverse temperature $\beta$.
    Let $A\subset V$ be a subset of vertices.
    Then
    \[
        Z_{G,\beta}\langle\sigma_A\rangle_{G,\beta}
        =2^{|V|}\sum_{\n:\:\partial\n=A}
        w_\beta(\n).
    \]
    In particular, the partition function is given by
    \[
        Z_{G,\beta}=2^{|V|}\sum_{\n:\:\partial\n=\emptyset}
        w_\beta(\n).
    \]
\end{theorem}

\begin{proof}
    We claim that
    \begin{align}
        \label{eq:curents:1}
        Z_{G,\beta}\langle\sigma_A\rangle_{G,\beta}
        &=
        \sum_{\sigma\in\{\pm1\}^V}
        \sigma_A
        \prod_{uv\in E}
        e^{\beta\sigma_u\sigma_v}
        \\
        \label{eq:curents:2}
        &=
        \sum_{\sigma\in\{\pm1\}^V}
        \sigma_A
        \prod_{uv\in E}
        \sum_{\n=0}^\infty
        \frac{(\beta\sigma_u\sigma_v)^\n}{\n!}
        \\
        \label{eq:curents:3}
        &=
        \sum_{\n:E\to\Z_{\geq 0}}
        \sum_{\sigma\in\{\pm1\}^V}
        \sigma_A
        \prod_{uv\in E}
        \frac{(\beta\sigma_u\sigma_v)^{\n_{uv}}}{\n_{uv}!}
        \\
        \label{eq:curents:4}
        &=
        2^{|V|}\sum_{\n:\:\partial\n=A}
        \prod_{uv\in E}
        \frac{\beta^{\n_{uv}}}{\n_{uv}!}
        \\
        \label{eq:curents:5}
        &=
        2^{|V|}
        \sum_{\n:\:\partial\n=A}
        w_\beta(\n).
    \end{align}
    Although this may come as a surprise, its justification is straightforward.
    \begin{itemize}
        \item Equation~\eqref{eq:curents:1} is derived from the definition
        of the Ising model by pulling out the sum in the Hamiltonian in
        the exponential as a product.
        \item Equation~\eqref{eq:curents:2} follows by expanding each exponential.
        \item Equation~\eqref{eq:curents:3} is Fubini's theorem:
        we first swap the product and the sum (so that we have to perform one sum for each
        factor in the product), then we interchange the two sums.
        Absolute convergene comes from the factorial terms in the denominator.
        \item Equation~\eqref{eq:curents:4} follows simply by noting that the sum over $\sigma$ is zero unless
        all the factors of the form $\sigma_u$ cancel.
        \item Equation~\eqref{eq:curents:5} is the definition of the weight above.
    \end{itemize}
    This finishes the proof.
\end{proof}

\begin{corollary}[First Griffiths inequality]
    \label{cor:griffiths_1}
    Consider the Ising model on a finite graph $G$ at inverse temperature $\beta$.
    Then for any $A\subset V$, we have $\langle\sigma_A\rangle_{G,\beta}\geq 0$.
\end{corollary}

\begin{proof}
    The current expansion shows that the correlation function is a sum of
    non-negative terms.
\end{proof}

\begin{exercise}[Correlation functions with odd sets]
    \label{exo:correlation_functions_with_odd_sets}
    \begin{itemize}
        \item     Consider the setting of the previous theorem.
        Show that if $|A|$ is odd, then $\langle\sigma_A\rangle_{G,\beta}=0$.
        \item Now consider an arbitrary graph $G$ with some domain $\Lambda$.
        Show that if $|A|$ is odd, then $\langle\sigma_A\rangle_{G,\beta}^+\geq 0$.
        For this part, one needs Remark~\ref{remark:infinite_graphs_as_finite_graphs}
        and Exercise~\ref{exercise:infinite_graphs_as_finite_graphs}.
    \end{itemize}
\end{exercise}

\begin{definition}[Percolation of currents]
    Let $G$ denote a graph and $\n$ a current.
    We associate $\n$ with the set $E(\n):=\{uv\in E:\n_{uv}>0\}\subset E$.
    Edges in $E(\n)$ are called \emph{open},
    the other edges \emph{closed}.
    We shall write $\{u\xleftrightarrow{\n}v\}$
    for the event there is an open path from $u$ to $v$
    ($u$ and $v$ may represent vertices or sets of vertices).

    For fixed $S\subset V$, we shall also write $\calE_S$ for the set
    \[
        \{O\subset E:\text{$|C\cap S|$ is even for any connected component $C\subset V$ of $(V,O)$}\}.
    \]
    Notice that if $\partial\n=S$,
    then $E(\n)\in\calE_S$.

    We shall often simply write $\n$ for $E(\n)$.
\end{definition}

\begin{exercise}[Currents and the Peierls argument]
    \label{exercise:currents_peierls}
    For the first two parts, consider
    the Ising model on a finite graph $G$ at inverse temperature $\beta$.
    \begin{enumerate}
        \item Let $\calL\subset E$ denote a circuit through $G$
        (a closed path which traverses no edge more than once).
        Prove that
        \[
            \frac{
                \sum_{\n:\:\partial\n=\emptyset,\,\calL\subset E(\n)} w_\beta(\n)
            }{
                \sum_{\n:\:\partial\n=\emptyset} w_\beta(\n)
            }
            \leq
            \beta^{|\calL|}.
        \]
        \item Let $\calP\subset E$ denote a path from $u$ to $v$ through $G$
        which traverses no edge more than once.
        Prove that
        \[
            \frac{
                \sum_{\n:\:\partial\n=\{u,v\},\,\calP\subset E(\n)} w_\beta(\n)
            }{
                \sum_{\n:\:\partial\n=\emptyset} w_\beta(\n)
            }
            \leq
            \beta^{|\calP|}.
        \]
        \item Let $G$ denote an infinite graph of maximum degree $d\in\Z_{\geq 0}$.
        Let $u$ denote a fixed vertex, and let $\Lambda_n$ denote the set of vertices
        at distance at most $n-1$ from $u$.
        Prove that for any $\beta<1/d$ and $n\in\Z_{\geq 0}$, we have
        \[
            \langle\sigma_u\rangle_{\Lambda_n,\beta}^+
            \leq
            \frac{(d\beta)^n}{1-d\beta}.
        \]
        For this part, one needs Remark~\ref{remark:infinite_graphs_as_finite_graphs}
        and Exercise~\ref{exercise:infinite_graphs_as_finite_graphs}.
    \end{enumerate}
    Indeed, this is the high-temperature counterpart to the Peierls argument
    (Theorem~\ref{thm:peierls_triangles}).
    We did not only prove that for small $\beta$, the magnetisation vanishes
    (tends to zero),
    but also that it tends to zero exponentially fast in the distance to the boundary.
\end{exercise}

In fact, the sum
\(\sum_{\n:\:\partial\n=A}
        w_\beta(\n)\) can be calculated in a different way.
        We can first sum over the \emph{parity} of each edge 
        (which already determines $\partial\n$),
        then over all random currents consistent with that
        parity.
        Once the parity is fixed, the remaining sum decomposes
        as an independent product over the edges.
        More precisely, we get
        \begin{align}
            \sum_{\n:\:\partial\n=A}
        w_\beta(\n)
       & =
        \sum_{\omega\subset E:\:\partial\ind{\omega}=A}
        \left(\sum\nolimits_{\n:\:\n-\ind{\omega}\in 2\Z^E}
        w_\beta(\n)\right)
        \\
        &=
        \sum_{\omega\subset E:\:\partial\ind{\omega}=A}
        (\sinh\beta)^{|\omega|}(\cosh\beta)^{|E\setminus\omega|}.
        \end{align}
The factors $\cosh\beta$ and $\sinh\beta$ appear when performing the sum
$\sum_{k\geq 0} \beta^k/k!$ but restricting to even and odd values
for $k$ respectively.
We have now proved the \emph{high-temperature expansion}
(see below for a formal statement).
The high-temperature expansion is less versatile than the currents representation
of the Ising model, but it is still somewhat useful.

\begin{theorem}[High-temperature expansion]
        \label{thm:High-temperature expansion}
        Consider the Ising model on a finite graph $G$ at inverse temperature $\beta$.
        Let $A\subset V$ be a subset of vertices.
        Then
        \[
            Z_{G,\beta}\langle\sigma_A\rangle_{G,\beta}
            =2^{|V|}\sum_{\omega\subset E:\:\partial\ind{\omega}=A}
            (\sinh\beta)^{|\omega|}(\cosh\beta)^{|E\setminus\omega|}.
        \]
        In particular, the partition function is given by
        \[
            Z_{G,\beta}=2^{|V|}\sum_{\omega\subset E:\:\partial\ind{\omega}=\emptyset}
            (\sinh\beta)^{|\omega|}(\cosh\beta)^{|E\setminus\omega|}.
        \]
\end{theorem}

\begin{exercise}[Log-Lipschitz property of correlation functions]
    \label{exercise:Log-Lipschitz property of correlation functions}
    Let $G$ denote any finite graph.
    We define the following metric on subsets of $V$:
    for any $A,B\subset V$, we set
    \[
        d_{\operatorname{Transport}}(A,B)
        :=
        \min\{|\omega|:\text{$\omega\subset E$ and $A\Delta B=\partial\ind{\omega}$}\}.
    \]
    Consider the Ising model on $G$ at inverse temperature $\beta\in\R_{\geq 0}$.
    Use the high-temperature expansion to prove that
    \[
        (\tanh\beta)^{d_{\operatorname{Transport}}(A,B)}\langle\sigma_A\rangle
        \leq
        \langle\sigma_B\rangle. 
    \]
    This implies that for any $u\in V$,
    the function $v\mapsto-\log \langle\sigma_u\sigma_v\rangle$ is $(-\log\tanh\beta)$-Lipschitz.
\end{exercise}


\begin{definition}[Replacing sums by measures]
    Sums are measures.
    Even though there is no added mathematical value in replacing sums by measures,
    doing so will slightly change our perspective,
    and also shorten notations.

    For an Ising model on some finite graph $G$ with inverse temperature
    $\beta$, we shall write
    \(
        \M_{G,\beta}
    \)
    for the measure on random currents $\n\in(\Z_{\geq 0})^E$
    such that
    \[
        \M_{G,\beta}(\n)=w_\beta(\n).
    \]
    For $A\subset V$, we shall also write $\M_{G,\beta}^A$
    for the measure
    \[
        \M_{G,\beta}[\blank]:=\M_{G,\beta}[(\blank)\ind{\{\partial\n=A\}}].
    \]
\end{definition}

\begin{corollary}
        The measure $\M_{G,\beta}$ is not a probability measure,
        but $e^{-\beta|E|}\M_{G,\beta}$ is a probability measure
        in which $(\n_{uv})_{uv\in E}$ is a family of independent
        random variables with distribution $\operatorname{Poisson}(\beta)$.
\end{corollary}

\begin{corollary}
    \label{cor:current_representation_of_correlation_functions}
    Theorem~\ref{thm:current_representation_of_correlation_functions}
    says that 
    \[
        Z_{G,\beta}\langle\sigma_A\rangle_{G,\beta}
        =2^{|V|}\M_{G,\beta}[\{\partial\n=A\}]
        =2^{|V|}\M_{G,\beta}^A[1].
    \]
\end{corollary}
