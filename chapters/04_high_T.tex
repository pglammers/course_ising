\section{The high-temperature expansion}

The previous section proved the Peierls argument.
An essential ingredient was to view the Ising model in two dimensions
through the \emph{interfaces} of the spins.
Such a transformation of the model may be viewed as a rudimentary version of
an \emph{expansion}.
The interface perspective is sometimes called the \emph{low-temperature expansion}
because it works well in the low-temperature regime (when $\beta$ is large).
There are several useful expansions for the Ising model;
each one of them is adapted to a different setting.
In this section we discuss another expansion: the \emph{high-temperature expansion}.
As the name suggests, this expansion is well-adapted to situations where $\beta$
is small, even though we can also use it to prove useful results in other regimes.
Appendix~\ref{}\todo{Add Appendix and reference} contains an overview of the expansions
discussed in these notes, and may serve as a reference.

For a streamlined presentation, we will henceforth present all our expansions
for the Ising model on \emph{finite graphs without boundary conditions}.
This obviously includes the free boundary conditions.
It is straightforward to see that fixed boundary conditions
also fit into this framework, see Definition~\ref{definition:ghost} and Lemma~\ref{lemma:ghost} below.

Consider the Ising model on a finite graph $G$.
We are typically interested in the correlation functions,
defined via
\[
    \langle\sigma_A\rangle
    =
    \frac{\sum_{\sigma\in\Omega}\sigma_A e^{-H(\sigma)}}{\sum_{\sigma\in\Omega}e^{-H(\sigma)}}
    =
    \frac{\sum_{\sigma\in\Omega}\sigma_A \prod_{xy\in E} e^{\beta\sigma_x\sigma_y} }{\sum_{\sigma\in\Omega}\prod_{xy\in E} e^{\beta\sigma_x\sigma_y} }.
\]
An \emph{expansion} of the Ising model involves rewriting the sum
\[
    \sum_{\sigma\in\Omega}\sigma_A \prod_{xy\in E} e^{\beta\sigma_x\sigma_y}.
\]
A typical expansion comes down to rewriting the exponential, for example:
\begin{itemize}
    \item We may write $e^{\beta\sigma_x\sigma_y}=\cosh \beta + \sigma_x\sigma_y \sinh \beta$,
    \item We may write $e^{\beta\sigma_x\sigma_y}=\sum_{k=0}^\infty (\beta\sigma_x\sigma_y)^k/k!$,
    \item We may write $e^{\beta\sigma_x\sigma_y}=e^{-2\beta} + 2\cdot\true{\sigma_x=\sigma_y}\sinh\beta$.
\end{itemize}
Every expansion comes with its own advantages and disadvantages.
The high-temperature expansion derives from the first identity.

\begin{definition}[High-temperature expansion]
    Consider the Ising model on a finite graph $G=(V,E)$ at inverse temperature $\beta$.
    We consider \emph{percolation configurations} $\omega\in\{0,1\}^E$;
    each $\omega$ is also regarded a (random) set of edges.
    We write $\partial\omega\subset V$ for the set of vertices having
    \emph{odd} degree in the graph $(V,\omega)$.

    The \emph{high-temperature expansion} is the measure $\bfM_{G,\beta}$
    on $\omega\in\{0,1\}^E$ defined by
    \[
        \bfM[\omega]:=\bfM_{G,\beta}[\omega]:=(\cosh\beta)^{|E\setminus\omega|}(\sinh\beta)^{|\omega|}.
    \]
\end{definition}

\begin{theorem}[High-temperature expansion for correlation functions]
    \label{thm:High-temperature expansion for correlation functions}
    Consider the Ising model on a finite graph $G=(V,E)$.
    Then for any $A\subset V$, we have
    \[
        Z\langle\sigma_A\rangle=2^{|V|}\bfM[\{\partial\omega=A\}].
    \]
    In particular, $Z=2^{|V|}\bfM[\{\partial\omega=\emptyset\}]$.
\end{theorem}

\begin{proof}
    We claim that
    \begin{align}
        \label{eq:hight:1}
        Z\langle\sigma_A\rangle
        &=
        \sum_{\sigma\in\Omega}
        \sigma_A
        \prod_{xy\in E}
        e^{\beta\sigma_x\sigma_y}
        \\
        \label{eq:hight:2}
        &=
        \sum_{\sigma\in\Omega}
        \sigma_A
        \prod_{xy\in E}
        (\cosh\beta+\sigma_x\sigma_y\sinh\beta)
        \\
        \label{eq:hight:3}
        &=
        \sum_{\sigma\in\Omega}
        \sigma_A
        \sum_{\omega\in\{0,1\}^E}
        \prod_{xy\in E}
        (\cosh\beta)^{1-\omega_{xy}}(\sigma_x\sigma_y\sinh\beta)^{\omega_{xy}}
        \\
        \label{eq:hight:4}
        &=
        \sum_{\omega\in\{0,1\}^E}
        (\cosh\beta)^{|E\setminus\omega|}(\sinh\beta)^{|\omega|}
        \sum_{\sigma\in\Omega}
        \sigma_A
        \sigma_{\partial\omega}
        \\
        \label{eq:hight:5}
        &=
        \sum_{\omega\in\{0,1\}^E}
        (\cosh\beta)^{|E\setminus\omega|}(\sinh\beta)^{|\omega|}
        2^{|V|}\true{A=\partial\omega}
        \\
        &=
        2^{|V|}\bfM[\{\partial\omega=A\}].
    \end{align}
    Equations~\eqref{eq:hight:1} and~\eqref{eq:hight:2}
    come down to definitions
    and the identity for $e^{\beta\sigma_x\sigma_y}$.
    Swapping the sum and the product yields Equation~\eqref{eq:hight:3}.
    Equation~\eqref{eq:hight:4} is a rearrangement of the terms,
    noting that $\prod_{xy}(\sigma_x\sigma_y)^{\omega_{xy}}=\sigma_{\partial\omega}$.
    Equation~\eqref{eq:hight:5} is obtained by resolving the sum over $\sigma$.
    The final equation is the definition of $\bfM$.
\end{proof}

We can use this theorem to state our first important \emph{correlation inequality}.

\begin{theorem}[First Griffiths inequality]
    Consider the Ising model on a finite graph $G=(V,E)$.
    Then for any $A\subset V$, we have
    \(
        \langle\sigma_A\rangle\geq 0
    \).
\end{theorem}

\begin{proof}
    The previous theorem yields a nonnegative number for $Z\langle\sigma_A\rangle$.
\end{proof}

\begin{theorem}[Exponential decay at high temperature]
    \label{thm:Exponential decay at high temperature}
    Consider the Ising model with $+$ boundary conditions 
    on the graph $\Z^d$ for $d\in\Z_{\geq 1}$.
    Then for any $\beta\in[0,\infty)$
    such that $(2d-1)\tanh\beta < 1$,
    there exists a constant $c=c_{d,\beta}>0$
    such that
    \[
        \langle\sigma_x\rangle_{\Lambda,\beta}^+
        \leq \tfrac1ce^{-c \operatorname{Distance}(x,\Lambda^c)}
    \]
    for any $x\in\Z^d$ and any domain $\Lambda\subset\Z^d$.
\end{theorem}

We would like to use the high-temperature expansion,
but for this we must first write $\langle\blank\rangle^+_\Lambda$
as an Ising model on a finite graph without boundary condition.

\begin{definition}[Ghost vertex]
    \label{definition:ghost}
    Let $G=(V,E)$ denote a locally finite graph,
    and $\Lambda\subset V$ a finite domain.
    We already defined the graphs $\Lambda^\f$ and $\bar\Lambda$.
    Now define the graph $\Lambda^\frakg$ as follows:
    it is obtained from the graph $\bar\Lambda$
    by replacing all vertices in $\partial\Lambda$ by a single distinguished
    vertex $\frakg$, called the \emph{ghost vertex}.
    Its vertex set is given by
    \(V(\Lambda^\frakg):=\Lambda\cup\{\frakg\}\),
    and there is a natural bijection from $E(\bar\Lambda)$
    to $E(\Lambda^\frakg)$.

    Notice that $\Lambda^\frakg$ is a multigraph when some $x\in\Lambda$
    is connected to multiple vertices in $\partial\Lambda$ in the graph $\bar\Lambda$,
    but this does not really affect our setup.
\end{definition}

It is easy to see that the following lemma holds true.

\begin{lemma}
    \label{lemma:ghost}
    Let $G=(V,E)$ denote a locally finite graph,
    and $\Lambda\subset V$ a finite domain.
    Then the distribution of $\sigma|_{\Lambda}$ is the same in the following
    two measures:
    \[
        \langle\blank\rangle_{\Lambda}^+
        \qquad\text{and}\qquad
        \E_{\Lambda^\frakg}[\blank|\sigma_\frakg=+].
    \]
    Correlation functions can thus be expressed in terms of correlation functions
    on finite graphs via Exercise~\ref{exercise:flip-symmetry}.
\end{lemma}

\begin{proof}[Proof overview of Theorem~\ref{thm:Exponential decay at high temperature}]
    We have
    \[
        \langle\sigma_x\rangle_{\Lambda}^+
        =
        \langle\sigma_x\sigma_\frakg\rangle_{\Lambda^\frakg}
        =
        \frac{
        \bfM_{\Lambda^\frakg}[\{\partial\omega=\{x,\frakg\}\}]
        }{
        \bfM_{\Lambda^\frakg}[\{\partial\omega=\emptyset\}]
        }.
    \]
    If $\partial\omega=\{x,\frakg\}$, then $\omega$ contains a self-avoiding
    walk $\gamma$ from $x$ to $\frakg$.
    A union bound yields
    \[
        \langle\sigma_x\rangle_{\Lambda}^+
        \leq
        \sum_{\gamma}\frac{\bfM_{\Lambda^\frakg}[\{\partial\omega=\{x,\frakg\}\}\cap\{\gamma\subset\omega\}]}{\bfM_{\Lambda^\frakg}[\{\partial\omega=\emptyset\}]}.
    \]
    The proof is now completed after performing the two steps of the Peierls argument:
    \begin{itemize}
        \item One bounds each term by $(\tanh\beta)^{|\gamma|}$,
        \item One bounds the number of walks $\gamma$ of length $n$ from $x$ by $2d(2d-1)^{|\gamma|-1}$.
    \end{itemize} 
    \qedhere
\end{proof}

\begin{exercise}
    Fill in the details of the previous proof overview.
\end{exercise}

Let us summarise what we have proved so far.
\begin{itemize}
    \item Theorem~\ref{thm:Exponential decay at high temperature}
    implies that there is exponential decay of correlations when $\beta$ is sufficiently small.
    \item In dimension $d=1$, Theorem~\ref{thm:Exponential decay at high temperature} also implies Ising's result (Theorem~\ref{thm:Ising}),
    since there was no requirement on $\beta$ when $d=1$.
    \item In dimension $d\geq 2$, we proved that there is \emph{magnetisation} via the
    Peierls argument (Theorem~\ref{thm:peierls}).
    Thus, in dimension $d\geq 2$, there must be a phase transition,
    and we aim to investigate further.
\end{itemize}

The high-temperature expansion is typically used to find upper bounds on correlation
functions.
However, it is also possible to use it to find lower bounds.
Let $G=(V,E)$ denote a finite graph.
For any fixed set $Q\subset E$ of edges,
we define the \emph{XOR map}
\[
    \Xi_Q:\{0,1\}^E\to\{0,1\}^E,\,\omega\mapsto\omega\Delta Q,
\]
where $\Delta$ denotes the symmetric difference of two sets.
This map is an involution.
Moreover, for any $A\subset V$, it restricts to a bijection
\begin{equation}
    \label{eq:XiQ}
\Xi_Q:
\{\partial\omega=A\}\to\{\partial\omega=A\Delta\partial Q\}.
\end{equation}
The measure $\bfM$ is not invariant under the involution $\Xi_Q$,
but it is easy to see how the map affects the measure.
More precisely, for any $\eta\in\{0,1\}^E$, we have
\begin{equation}
    \label{eq:reweighting}
    \bfM[\{\omega = \Xi_Q(\eta)\}]
    =
    (\tanh\beta)^{|Q\setminus\eta|-|Q\cap\eta|}
    \cdot
    \bfM[\{\omega = \eta\}].
\end{equation}
The prefactor is upper bounded by $(\tanh\beta)^{-|Q|}$.
Thus, writing $\Xi_Q^*$ for the pushforward map,
we obtain
\[
    \Xi_Q^*\bfM\leq (\tanh\beta)^{-|Q|}
    \cdot
    \bfM.
\]
For example, using the bijection in Equation~\eqref{eq:XiQ},
we obtain
\begin{equation}
    \label{eq:M-Xi-Q-high-temperature}
    \bfM[\{\partial\omega=A\}]
    \leq
    (\tanh\beta)^{-|Q|}
    \cdot
    \bfM[\{\partial\omega=A\Delta\partial Q\}].
\end{equation}
We have now proved the following result.

\begin{lemma}
    \label{lemma:high-temperature}
    Consider the Ising model on a finite graph $G=(V,E)$.
    Then for any $A\subset V$ and any $Q\subset E$, we have
    \[
        \langle\sigma_A\sigma_{\partial Q}\rangle
        \geq
        (\tanh\beta)^{|Q|}\cdot\langle\sigma_A\rangle.
    \]
    In particular,
    \[
        \langle\sigma_x\sigma_y\rangle
        \geq
        (\tanh\beta)^{\operatorname{Distance}(x,y)}.
    \]
\end{lemma}

\begin{proof}
    The first inequality is Equation~\eqref{eq:M-Xi-Q-high-temperature}.
    For the second inequality, simply set $A=\emptyset$
    and let $Q$ denote a shortest path from $x$ to $y$.
\end{proof}

This lemma complements Theorem~\ref{thm:Exponential decay at high temperature}
at high temperature, as the lemma asserts that the correlation functions cannot
decay \emph{faster} than exponentially at any finite temperature
(that is, when $\beta>0$).

