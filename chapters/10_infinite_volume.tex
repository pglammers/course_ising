\section{The thermodynamical limit. Wired boundary, existence}
\label{sec:infinite_volume}

Consider the Ising model on the infinite graph $\Z^d$ for $d\geq 2$.
Let $u=0\in\Z^d$ and let $\Lambda_n$ denote the graph metric ball around $u$.
We have already derived the following results.
\begin{itemize}
    \item For large $\beta$,
    the Ising model exhibits magnetisation
    in the sense that
    \[
        \inf_{n}\langle\sigma_0\rangle_{\Lambda_n,\beta}^+
        >0.
    \]
    This was proved via the Peierls argument, see Theorem~\ref{thm:peierls}
    and Exercise~\ref{exo:peierls_general}.
    \item For small $\beta$,
    the Ising model \emph{does not} exhibit magnetisation:
    \[
        \lim_{n\to\infty}\langle\sigma_0\rangle_{\Lambda_n,\beta}^+
        =0.
    \]
    This was proved via a Peierls argument for random currents,
    see Exercise~\ref{exercise:currents_peierls}.
\end{itemize}
At the time moment of stating the Peierls argument (Section~\ref{sec:peierls}),
we knew almost nothing about the Ising model.
Our understanding is now advancing.
We already used the first Griffiths inequality to show that $\langle\sigma_0\rangle_{\Lambda_n,\beta}^+\geq 0$
(Corollary~\ref{cor:griffiths_1}
and Exercise~\ref{exo:correlation_functions_with_odd_sets}).
Our first objective is now to prove the following result.
To state it, we write
\[
    \lim_{\Lambda\uparrow V}f(\Lambda)
    \qquad\text{for}\qquad
    \lim_{n\to\infty}f(\Lambda_n),
\]
where $(\Lambda_n)_n$ is any increasing sequence of domains 
with $\cup_n\Lambda_n=V$.
This notation makes sense only when the limit is independent
of the precise choice of the sequence $(\Lambda_n)_n$,
and is called the \emph{thermodynamical limit} or \emph{infinite-volume limit}.

\begin{lemma}[Correlation functions are monotone in the domain (wired boundary)]
    \label{lemma:correlation_functions_monotone}
    Consider the Ising model on a locally finite graph
    $G$ at inverse temperature $\beta$.
    Let $A\subset V$ denote any finite subset.
    Then the function
    \[
        \Lambda\mapsto
        \langle\sigma_A\rangle_{\Lambda,\beta}^+
    \]
    is a nonincreasing function of the domain $\Lambda$.

    In particular, we have well-definedness of the thermodynamical limit
    \[
        \lim_{\Lambda\uparrow V}
        \langle\sigma_A\rangle_{\Lambda,\beta}^+.
    \]
\end{lemma}

\begin{proof}
    Consider two domains $\Lambda\subset\bar\Lambda$.
    We want to show that
    \[
        \langle\sigma_A\rangle_{\Lambda,\beta}^+
        \geq
        \langle\sigma_A\rangle_{\bar\Lambda,\beta}^+.
    \]
    Without loss of generality,
    $A\subset\bar\Lambda$ and
    $\bar\Lambda\setminus\Lambda=\{u\}$ for some
    vertex $u\in V$.

    Let $G'=(\bar\Lambda\cup\{\bar\Lambda^c\},E(\bar\Lambda))$ denote the graph obtained from $\bar\Lambda$
    as in Remark~\ref{remark:infinite_graphs_as_finite_graphs}
    and Exercise~\ref{exercise:infinite_graphs_as_finite_graphs}.
    We refer to the Ising model on $G'$ when subscripts are submitted from now on.
    Assume that $|A|$ is even for now.
    Then
    \begin{equation}
        \langle\sigma_A\rangle_{\bar\Lambda,\beta}^+
        =
        \E[\sigma_A];\qquad
        \langle\sigma_A\rangle_{\Lambda,\beta}^+
        =
        \E[\sigma_A|\{\sigma_u=\sigma_{\bar\Lambda^c}\}].
    \end{equation}
    It suffices to show that the conditioning increases the expectation.
    This is just Exercise~\ref{exo:conditioning_equality}.
    
    % Define the event $Q_\pm:=\{\sigma_u\sigma_{\bar\Lambda^c}=\pm 1\}$.
    % It suffices to prove that
    % \[
    %     \E[\sigma_A|Q_+]
    %     \geq
    %     \E[\sigma_A|Q_-].
    % \]
    % Notice that
    % \begin{multline}
    %     \P[Q_+]\E[\sigma_A|Q_+]
    %     -
    %     \P[Q_-]\E[\sigma_A|Q_-]
    %     =
    %     \E[\sigma_A\sigma_{\{u,\bar\Lambda^c\}}]
    %     \\
    %     \stackrel{\text{second Griffiths}}\geq
    %     \E[\sigma_A]\E[\sigma_{\{u,\bar\Lambda^c\}}]
    %     =(\P[Q_+]
    %     -
    %     \P[Q_-])\E[\sigma_A].
    % \end{multline}
    % This is the desired inequality.

    If $|A|$ is odd then we just need to replace the set $A$
    by $A':=A\cup\{\bar\Lambda^c\}$.
    More precisely,
    we have
    \begin{equation}
        \langle\sigma_A\rangle_{\bar\Lambda,\beta}^+
        =
        \E[\sigma_{A'}];\qquad
        \langle\sigma_A\rangle_{\Lambda,\beta}^+
        =
        \E[\sigma_{A'}|\{\sigma_u=\sigma_{\bar\Lambda^c}\}].
    \end{equation}
    One may then simply apply Exercise~\ref{exo:conditioning_equality} as for the even case.
\end{proof}

Perhaps we were wondering if $\langle\sigma_0\rangle_{\Lambda_n,\beta}^+$
was decreasing in $n$ in the statement of the Peierls argument
(Theorem~\ref{thm:peierls_triangles}),
but the result we proved just now is much stronger:
we proved that the thermodynamical limit of any ``local Fourier coefficient''
is well-defined.
Rather than taking a thermodynamical limit of observables,
we would however like to make sense of the thermodynamical limit
of the family of measures $\langle\blank\rangle_{\Lambda,\beta}^+$.
The previous lemma enables us to do this;
we only need to set up the definitions to make formal sense of our limit.

\begin{definition}[The compact ``local convergence'' topology]
    Let $G$ denote a locally finite graph.
    Recall that $(\Omega,\calF)$ is the measurable space $\Omega:=\{\pm1\}^V$
    endowed with the product $\sigma$-algebra.
    For a domain $\Lambda$, we write $\calF_\Lambda$
    for the $\sigma$-algebra generated by spins in $\Lambda$.
    An observable $X:\Omega\to\C$ is called \emph{local} if it is measurable
    with respect to $\calF_\Lambda$ for some domain $\Lambda$.

    Let $\calP(\Omega,\calF)$ denote the set of all probability measures
    on this measurable space.
    We endow this set with the \emph{local convergence topology},
    which is defined as the topology making the map
    \[
        \calP(\Omega,\calF)\to\C,\,\mu\mapsto\mu[X]
    \]
    continuous for any local observable $X$.
\end{definition}

\begin{remark}
    This topology is sometimes known under different names in the literature
    (such as the \emph{weak topology}).
    I like the name \emph{local convergence topology} because it captures the essence quite literally:
    if the statistics of the measures within a fixed domain $\Lambda$ converge,
    then we have local convergence.
\end{remark}

\begin{exercise}
    Prove that $\calP(\Omega,\calF)$ is a compact space in this topology.
\end{exercise}

\begin{theorem}[Existence of the thermodynamical limit (wired boundary)]
    Consider the Ising model on a locally finite graph $G$
    at inverse temperature $\beta$.
    Then there exists a unique probability measure
    $\langle\blank\rangle_{G,\beta}^+\in\calP(\Omega,\calF)$
    such that
    \[
        \lim_{\Lambda\uparrow V}\langle X\rangle_{\Lambda,\beta}^+=\langle X\rangle_{G,\beta}^+
    \]
    for any local observable $X:\Omega\to\R$.
    In other words,
    \[
        \lim_{\Lambda\uparrow V}\langle \blank\rangle_{\Lambda,\beta}^+
        =:
        \langle \blank\rangle_{G,\beta}^+.
    \]
    The measure $\langle\blank\rangle_{G,\beta}^+$ is called
    the \emph{thermodynamical limit} or \emph{infinite-volume limit}
    with $+$ boundary conditions.
\end{theorem}

\begin{proof}
    Any local observable may be written as a finite linear conbination
    of observables of the form $\sigma_A$ where $A$ is a finite subset of
    $V$.
    The theorem then follows by compactness and Lemma~\ref{lemma:correlation_functions_monotone}.
\end{proof}

\begin{exercise}[Continuity properties in $\beta$ (wired boundary)]
    Consider the Ising model on a locally finite graph $G$.
    Prove all of the following statements.
    \begin{itemize}
        \item For any finite sets $A,\Lambda\subset V$,
        the function $[0,\infty)\to\R,\,\beta\mapsto\langle\sigma_A\rangle_{\Lambda,\beta}^+$
        is non-decreasing and continuous.
        \item For any finite set $A\subset V$,
        the function $[0,\infty)\to\R,\,\beta\mapsto\langle\sigma_A\rangle_{G,\beta}^+$
        is non-decreasing and right-continuous.
        \item The function $[0,\infty)\to\calP(\Omega,\calF),\,\beta\mapsto\langle\blank\rangle_{G,\beta}^+$
        is a right-continuous function.
        \item The points of discontinuity form a countable subset of $[0,\infty)$.
    \end{itemize}
\end{exercise}

\begin{definition}[Magnetisation and critical temperature]
    Let $G$ denote a locally finite graph and $u$ some distinguished reference vertex.
    The function
    \[
        m=m_G:[0,\infty)\to\R,\,\beta\mapsto\langle\sigma_u\rangle_{G,\beta}^+
    \]
    is called the \emph{magnetisation}.

    The \emph{critical (inverse) temperature} is defined via
    \[
        \beta_c:=\inf\{\beta\in[0,\infty):m(\beta)>0\}.
    \]
\end{definition}

\begin{remark}
    The exercise above proves that $m$ is a non-decreasing right-continuous function.
    We have already seen that:
    \begin{itemize}
        \item If $G$ has max-degree $d$,
        then $m(\beta)=0$ for $\beta<1/d$ (Exercise~\ref{exercise:currents_peierls}),
        \item If $G$ is the graph $\Z^d$ for $d\geq 2$,
        then $\lim_{\beta\to\infty}m(\beta)>0$ (Exercise~\ref{exo:peierls_general}).
    \end{itemize}
    This implies in particular that on the square lattice graph $\Z^d$ in dimension
    $d\geq 2$,
    the critical inverse temperature $\beta_c$ is a strictly positive real number.
    A key objective of our field is to understand the behaviour of the Ising model
    at $\beta=\beta_c$ and at $\beta\approx\beta_c$.
    As a very first question we can ask:
    is the function $m$ continuous?
    We will see the answer in Theorem~\ref{}\todo{Add ref to theorem}.
\end{remark}
