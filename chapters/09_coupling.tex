\newpage 

\section{Coupling from the past}

Consider the following practical objective:
use a computer to sample a configuration of the Ising model on a finite graph.
Computers generally have access to a source of i.i.d.\ randomness,
but cannot immediately sample from complicated distributions.
We must therefore transform the i.i.d.\ randomness
into a sample from the Ising model.

Let us first consider \emph{Glauber dynamics},
which is a standard strategy for reaching this objective.
One uses the i.i.d.\ randomness
to construct a Markov chain whose invariant distribution is the Ising model
(for example, via the Metropolis--Hastings algorithm).
To take a sample, one starts at a deterministic configuration, runs the algorithm for a deterministic number of steps (say $N$),
and then takes the final state as the sample.
This strategy has two drawbacks:
\begin{itemize}
    \item The Markov chain approximates its invariant distribution;
    no sample is ``perfect'',
    \item The number $N$ required for the desired precision is often hard to calculate.
\end{itemize}

\emph{Coupling from the past} circumvents these problems.
This section explains the basic algorithm,
and the second section discusses theoretical implications (such as ergodicity, mixing, and uniqueness of Gibbs measures).
Throughout this section, we consider the Ising model
on a finite graph $G$ at inverse temperature $\beta$.

\begin{definition}[Treshold value at a single spin]
    The distribution of $\sigma_x$ conditional on $\{\sigma_{V\setminus\{x\}}=\zeta\}$ is given by
    \[
        \P_{G,\beta}[\{\sigma_x=-\}|\{\sigma_{V\setminus\{x\}}=\zeta\}]
        =
        \tau_\beta({\textstyle\sum_{y\sim x}\zeta_x});
        \qquad
        \tau_\beta(a):=\frac{e^{-\beta a}}{2\cosh\beta a}.
    \]
    The value $\tau_\beta$ is called the \emph{treshold value}.
\end{definition}

The conditional value of $\sigma_x$ may thus be sampled as follows:
first draw $U\sim U([0,1])$ (the uniform distribution on the unit interval),
then set
\[
    \sigma_x:=\begin{cases}
        - &\text{if $U\leq \tau_\beta(\sum_{y\sim x}\sigma_x)$,}\\
        + &\text{if $U> \tau_\beta(\sum_{y\sim x}\sigma_x)$.}
    \end{cases}
\]
In order to rigorously implement this conditional resampling into a Glauber dynamic,
we first describe an explicit Glauber map.

\begin{definition}[Glauber map]
    Let $G$ denote any locally finite graph.
    For any $x\in V$ and $U\in[0,1]$, define the \emph{Glauber map}
    \[
        R_{x,U}:\Omega\to\Omega,\,\sigma\mapsto
        \left(
            z\mapsto \begin{cases}
                \sigma_z&\text{if $z\neq x$}\\
                - &\text{if $z=x$ and $U\leq \tau_\beta(\sum_{y\sim x}\sigma_y)$}\\
                + &\text{if $z=x$ and $U>\tau_\beta(\sum_{y\sim x}\sigma_y)$}
            \end{cases} 
        \right).
    \]
    This map potentially flips the spin at $x$, while leaving all other spins invariant.
\end{definition}

\newcommand\Piid{\P_{\operatorname{iid}}}

Let $\Piid$ denote a new probability measure in which
 $((x_k,U_k))_{k\in\Z}$ is an i.i.d.\ sequence of
uniform random variables in $V\times[0,1]$.
The idea is to use the i.i.d.\ randomness in this sequence to set up a Markov chain
mixing to $\langle\blank\rangle_{G,\beta}$ (as described above).
In this probability space, we write $R^{k,\ell}$ for the random map
\[
    R^{k,\ell}:=R_{x_{k+1},U_{k+1}}\circ R_{x_{k+2},U_{k+2}} \circ \cdots \circ R_{x_{\ell},U_{\ell}}
\]
for any $k,\ell\in\Z$ with $k\leq\ell$.

This setups is intrinsically linked to a Markov chain, described in the proposition below.
We leave its proof to the reader.

\begin{proposition}[Glauber dynamics]
    Let $G$ denote a finite graph and let $\beta\in[0,\infty)$.
    Consider the measure $\Piid$ introduced above.
    Define the stochastic $|\Omega|\times|\Omega|$-matrix $A$ via
    \[
        A_{\sigma',\sigma}:=\Piid[\{R_{x_0,U_0}(\sigma)=\sigma'\}].
    \]
    Then all of the following hold true.
    \begin{itemize}
        \item The matrix $A$ is the transition kernel of an irreducible aperiodic Markov chain.
        \item $\langle\blank\rangle_{G,\beta}$ solves the detailed balance equations for $A$.
        \item For any $\sigma\in\Omega$ and $k\in\Z$, the random sequence
        \(
            (R^{k,k+n}(\sigma))_{n\geq 0}
        \)
        in $\Piid$
        has the law of the Markov chain with transition kernel $A$, started from $\sigma$.
    \end{itemize}
\end{proposition}

\emph{Coupling from the past} relies on one crucial idea:
one should not consider the asymptotic ($n\to\infty$) distribution of $R^{0,n}(\sigma)$,
but rather of $R^{-n,0}(\sigma)$.

\begin{theorem}[Coupling from the past]
    \label{thm:coupling_from_the_past}
    Let $G$ denote a finite graph and let $\beta\in[0,\infty)$.
    Consider the measure $\Piid$ introduced above.
    Let $T\in\Z_{\geq 0}\cup\{\infty\}$ denote the random stopping time
    defined via
    \[
        T:=\inf\{n\in\Z_{\geq 0}:\text{the function $R^{-n,0}:\Omega\to\Omega$ is constant as a function on $\Omega$}\}.
    \]
    Suppose that $\P(\{T<\infty\})=1$.
    Then all of the following are true:
    \begin{itemize}
        \item For any $S\geq T$, the function $R^{-S,0}$ is also constant on $\Omega$, and $R^{-S,0}=R^{-T,0}$,
        \item The distribution of function $R^{-T,0}$ is $\langle\blank\rangle_{G,\beta}$.
    \end{itemize}
    Here we simply write $R^{-T,0}$ for $R^{-T,0}(\sigma)$ (with $\sigma\in\Omega$ arbitrary)
    whenever $R^{-T,0}$ is constant.
\end{theorem}

\begin{proof}
    Fix a configuration $\sigma\in\Omega$
    and a sequence $((x_k,U_k))_k$.
    If $R^{-n,0}$ is constant for some $n$, then
    \[
            R^{-(n+1),0}=R^{-n,0}\circ R_{x_{-n},U_{-n}}=R^{-n,0}.
    \]
    In that case, we simply have
    \[
        \lim_{m\to\infty}R^{-m,0}(\sigma)=R^{-n,0}(\sigma)\equiv R^{-n,0}.
    \]
    In particular, if $\Piid(\{T<\infty\})=1$,
    then $\Piid$-almost surely
    \[
        \lim_{m\to\infty} R^{-m,0}(\sigma)=R^{-T,0}.
    \]
    The distribution of $R^{-m,0}(\sigma)$ converges to $\langle\blank\rangle_{G,\beta}$
    as $m\to\infty$ (see the previous proposition),
    and therefore $R^{-T,0}\sim\langle\blank\rangle_{G,\beta}$.
\end{proof}

It is clearly important that $\Piid(\{T<\infty\})=1$.
We invite the reader to prove this in the following exercise.
Another proof is provided later.

\begin{exercise}[Coupling from the past: convergence (generic)]
    Show that the stopping time $T$ in the theorem above is almost surely finite.
    Hint: observe that $\Piid$-almost surely,
    there are $|\Lambda|$ consecutive entries in the sequence $((x_k,U_k))_{k\in\Z_{\leq 0}}$
    where $x_k$ enumerates $\Lambda$ and where $U_k\approx 0$.
\end{exercise}

\begin{remark}[Algorithm description]
    The algorithm can practically be implemented in a computer as follows.
\begin{enumerate}
    \item Fix a strictly increasing sequence $(n_k)_{k\geq 0}$ of integers with $n_0=0$.
    \item Set $k=0$, and repeat the following procedure as long as $R^{-n_k,0}$ is \emph{not} constant:
    \begin{enumerate}
        \item Add $1$ to the counter $k$,
        \item Draw $((x_i,U_i))_{-n_k< i\leq -n_{k-1}}$ from the independent source of randomness,
        \item Calculate the map $R^{-n_k,0}$.
    \end{enumerate}
    \item Output the constant value $R^{-n_k,0}$ as our sample from $\langle\blank\rangle_{G,\beta}$.
\end{enumerate}
\end{remark}

The algorithm has two drawbacks:
\begin{itemize}
    \item It requires memory to record $R^{-n_k,0}$ or $((x_i,U_i))_{-n_k< i\leq 0}$ between iterations,
    \item It requires time to determine if $R^{-n_k,0}$ is constant on $\Omega$ (which has cardinal $2^{|V|}$).
\end{itemize}

We now describe an important ingredient not mentioned so far: \emph{monotonicity}.
With monotonicity, it is much easier to see that \emph{coupling from the past}
converges (more precisely, that $\Piid(\{T<\infty\})=1$).
It is also much easier to check if the map $R^{-n,0}$ is constant or not.
The monotonicity is relative to the partial ordering $\leq$
on the sample space $\Omega=\{\pm\}^V$;
we say that $\sigma\leq\sigma'$ whenever $\sigma_x\leq\sigma'_x$ for all $x\in V$.
Everything follows from the basic observation that each map $R_{x_k,U_k}$ preserves
$\leq$, in the sense that
\[
    \sigma\leq\sigma' \qquad \implies \qquad R_{x_k,U_k}(\sigma)\leq R_{x_k,U_k}(\sigma').
\]

\begin{lemma}[Monotonicity of Glauber dynamics]
    Consider $((x_i,U_i))_{i\in\Z}$ fixed. Then:
    \begin{enumerate}
        \item Each map $R_{x_i,U_i}:\Omega\to\Omega$ preserves $\leq$,
        \item Each map $R^{i,j}:\Omega\to\Omega$ preserves $\leq$,
        \item $(R^{-n,0}(\sigma^{\max}))_n$ is decreasing in $n$, where $\sigma^{\max}\equiv+\in\Omega$ is the maximal element,
        \item $(R^{-n,0}(\sigma^{\min}))_n$ is increasing in $n$, where $\sigma^{\min}\equiv-\in\Omega$ is the minimal element,
        \item We have $R^{-\infty,0}(\sigma^{\min})\leq R^{-\infty,0}(\sigma^{\max})$,
        where $R^{-\infty,0}(\cdots)$ denotes the $n\to\infty$ limit,
        \item If  $R^{-n,0}(\sigma^{\min})= R^{-n,0}(\sigma^{\max})$, then the map $R^{-n,0}$ is constant.
    \end{enumerate}
\end{lemma}

\begin{proof}
    The first item can be proved by simply inspecting the definition of $R_{x_i,U_i}$.
    The second item then follows: indeed, a composition of increasing functions is increasing.
    For the third item (and similarly, the fourth item), observe that monotonicity of $R^{-n,0}$ implies:
    \[
        R^{-(n+1),0}(\sigma^{\max})
        =
        R^{-n,0}(R_{x_{-n},U_{-n}}(\sigma^{\max}))
        \leq
        R^{-n,0}(\sigma^{\max}).
    \]
    The last two items follow from monotonicity of the map $R^{-n,0}$ (the second item).
\end{proof}


\begin{lemma}[Coupling from the past: convergence]
    Recall the context of Theorem~\ref{thm:coupling_from_the_past}.
    Then we have
    \[
        T=\inf\{n\in\Z_{\geq 0}:R^{-n,0}(\sigma^{\max})=R^{-n,0}(\sigma^{\min})\},
    \]
    and this random variable is $\Piid$-almost surely finite.
\end{lemma}

\begin{proof}
    The reformulation of the stopping time is valid thanks to the previous lemma.
    It suffices to prove that it is almost surely finite.
    Notice that the distribution of both $R^{-\infty,0}(\sigma^{\max})$ and $R^{-\infty,0}(\sigma^{\min})$
    is given by $\langle\blank\rangle_{G,\beta}$.
    Since they have the same distribution and are almost surely $\leq$-ordered in $\Piid$,
    they must be almost surely equal.
    Since the state space $\Omega$ is discrete,
    this means that $\Piid$-almost surely
    there is some $n\in\Z_{\geq 0}$ such that
    \[
        R^{-n,0}(\sigma^{\max})=
        R^{-\infty,0}(\sigma^{\max})
        \qquad
        \text{and}
        \qquad
        R^{-n,0}(\sigma^{\min})=
        R^{-\infty,0}(\sigma^{\min}).
    \]
    This finishes the proof.
\end{proof}
